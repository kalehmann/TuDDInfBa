\documentclass{scrreprt}

\usepackage{aligned-overset}
\usepackage{amsmath}
\usepackage{amsthm}
\usepackage{amssymb}
\usepackage{bm}
\usepackage[inline,shortlabels]{enumitem}
\usepackage{hyperref}
\usepackage[utf8]{inputenc}
\usepackage{multicol}
\usepackage{mathtools}
\usepackage{pdflscape}
\usepackage{physics}
\usepackage{tabularx}
\usepackage[table]{xcolor}
\usepackage{titling}
\usepackage{fancyhdr}
\usepackage{xfrac}
\usepackage{pgfplots}

\pgfplotsset{compat = newest}
\usepgfplotslibrary{fillbetween}
\usetikzlibrary{calc}


\author{Karsten Lehmann}
\date{SoSe 2025}
\title{Übungsblatt 01\\INF-B-260, Informations- und Kodierungstheorie}

\newcommand{\ld}{\text{ld}}

\setlength{\parindent}{0pt}

\setlength{\headheight}{26pt}
\pagestyle{fancy}
\fancyhf{}
\lhead{\thetitle}
\rhead{\theauthor}
\lfoot{\thedate}
\rfoot{Seite \thepage}

\begin{document}
\paragraph{Ü 1.1 Diskrete Informationsquellen mit unabhängigen Ereignissen}
\begin{enumerate}[1.]
\item Berechnen Sie den mittleren Informationsgehalt einer diskreten Quelle mit
  6 Zeichen:

  \begin{enumerate}[(a)]
  \item
    \begin{multicols}{3}
      $p\qty\big(x_1) = 0.5$

      $p\qty\big(x_4) = 0.1$

      $p\qty\big(x_2) = 0.2$

      $p\qty\big(x_5) = 0.05$

      $p\qty\big(x_3) = 0.1$

      $p\qty\big(x_6) = 0.05$
    \end{multicols}

    \subparagraph{Lsg.} Es sind
    \begin{flalign*}
      H_1 &= \ld \frac{1}{p\qty\big(x_1)} = \ld \frac{1}{0.5} = 1 \\
      H_2 &= \ld \frac{1}{p\qty\big(x_2)} = \ld \frac{1}{0.2} \approx 2.32 \\
      H_3 &= \ld \frac{1}{p\qty\big(x_3)} = \ld \frac{1}{0.1} \approx 3.32 \\
      H_4 &= \ld \frac{1}{p\qty\big(x_4)} = \ld \frac{1}{0.1} \approx 3.32 \\
      H_5 &= \ld \frac{1}{p\qty\big(x_5)} = \ld \frac{1}{0.05} \approx 4.32 \\
      H_6 &= \ld \frac{1}{p\qty\big(x_6)} = \ld \frac{1}{0.05} \approx 4.32 \\
    \end{flalign*}

    Nun ist der mittlere Informationsgehalt
    \[
      H_m = \sum_{i = 1}^N p\qty\big(x_i)H_i \approx 2.06
    \]

  \item alle Zeichen sind gleich wahrscheinlich.

    \subparagraph{Lsg.} Es ist $p\qty\big(x_i) = \frac{1}{6}, i \in 1 \ldots 6$
    sowie $H_i = \ld \frac{1}{\sfrac{1}{6}} \approx 2.58$ und schließlich
    \[
      H_m = 6 \cdot \frac{1}{6} \cdot 2.58 = 2.58
    \]
  \end{enumerate}

\newpage
\item
  \begin{enumerate}[(a)]
  \item Bestimmen Sie die Entropien von Binärquellen, wenn die Wahrscheinlichkeiten
    $p_i \qty\big(1) = 0.1 \cdot i \qty\big(i = 0, 1, \ldots, 5)$ gegeben sind.

    \subparagraph{Lsg.} Es ist $H_{m_i} = p_i\qty\big(0) \ld\frac{1}{p_1\qty\big(0)} +
    p_i\qty\big(1) \ld\frac{1}{p_1\qty\big(1)}$.

    \begin{tabular}{c|c|c|c}
      $i$ & $p\qty\big(0)$ & $p\qty\big(1)$ & $H_{m_i}$ \\
      \hline
      0 & 1   & 0   & 0 \\
      1 & 0.9 & 0.1 & 0.47 \\
      2 & 0.8 & 0.2 & 0.72 \\
      3 & 0.7 & 0.3 & 0.88 \\
      4 & 0.6 & 0.4 & 0.97\\
      5 & 0.5 & 0.5 & 1 \\
    \end{tabular}

    \textbf{Achtung:} Wenn hier ein Ereignis eine Wahrscheinlichkeit von 100\%
    hat, dann wird in der Rechnung durch 0 geteilt.
    Allerdings wurde in der Vorlesung wohl mal gezeigt, warum dort trotzdem 0
    hin muss.

  \item Stellen Sie die Funktion $H\qty\big(p) = p \cdot \ld \frac{1}{p} +
    \qty\big(1 - p)\ld\frac{1}{1 - p}$ für
    $0 \leq p \leq 1$ grafisch dar!
    Beachten Sie die Symmetrie!

    \subparagraph{Lsg.} Es ist

    \begin{tikzpicture}[scale=1.5]
      \begin{axis}[
        axis equal image,
        axis x line=center,
        axis y line=center,
        grid=both,
        samples=500,
        xmin=-0.1,
        xmax=1.1,
        xtick distance=1,
        ymin=-0.1,
        ymax=1.1,
        ytick distance=1,
      ]
        \addplot[domain=0:1, smooth] { \x * log2(1 / \x) + (1 - \x) * log2(1 / (1 - \x)) };
      \end{axis}
    \end{tikzpicture}
  \end{enumerate}

\newpage
\item Bestimmen Sie den Informationsgehalt einer Buchseite (40 Zeilen,
  65 Zeichen pro Zeile).
  Für die Berechnung sind 45 unabhängige Zeichen anzunehmen.

  \subparagraph{Lsg.} Es ist
  \begin{flalign*}
    H_{\text{Zeichen}} &= \ld 45 \\
    H_{\text{Seite}} &= \frac{\text{Zeichen}}{\text{Seite}} \cdot H_{\text{Zeichen}} \\
                       &= 65 \frac{\text{Zeichen}}{\text{Zeile}} \cdot 40 \frac{\text{Zeilen}}{\text{Seite}} \cdot \ld 45 = 14.3 \cdot 10^3 \frac{\text{bit}}{\text{Seite}}
  \end{flalign*}

\item Eine automatische Waage umfasst den Messbereich von $0 \ldots 100g$ mit
  der Schrittweite von $1g$.
  \begin{enumerate}[(a)]
  \item Bestimmen Sie den mittleren Informationsgehalt je Messwert.

    \subparagraph{Lsg.} Die Waage kann 101 Messwerte anzeigen, es wird davon
    ausgegangen, dass diese gleichverteilt mit einer Wahrscheinlichkeit von
    $p\qty\big(x_i) = \frac{1}{101}$ sowie
    $H_i = \ld \frac{1}{\sfrac{1}{101}} = \ld 101 \approx 6.66 \text{Bits}$
    und schließlich
    \[
      H_m = \sum{i = 1}^{101} p\qty\big(x_i) H_i = \frac{101}{101} H_i \approx 6.66 \frac{\text{Bits}}{\text{Messwert}}
    \]

  \item Wie groß wird der mittlere Informationsgehalt je Messwert bei einer
    Schrittweite von $0.1g$?

    \subparagraph{Lsg.} Analog zu Teilaufgabe (a)
    \[
      H_m = \ldots = \frac{1001}{1001} H_i = \ld 1001 \approx 9.97 \frac{\text{Bits}}{\text{Messwert}}
    \]
  \end{enumerate}

\newpage
\setcounter{enumi}{5}
\item Ein kontinuierliches Signal mit exponentieller Verteilungsdichte soll
  quantisiert werden.
  Der Amplitudenbereich wird dazu in 7 Intervalle aufgeteilt, in denen die
  Amplitudenwerte mit folgenden Wahrscheinlichkeiten auftreten:

  \begin{tabular}{c|c}
    Intervall $i$ & Auftrittswahrscheinlichkeit $p\qty\big(x_i)$ \\
    \hline
    1 & 0.47 \\
    2 & 0.25 \\
    3 & 0.13 \\
    4 & 0.07 \\
    5 & 0.04 \\
    6 & 0.02 \\
    7 & 0.02 \\
  \end{tabular}

  \begin{enumerate}[(a)]
  \item Wie groß ist der mittlere Informationsgehalt eines Amplitudenwertes
    (Intervalls)?

    \subparagraph{Lsg.} Es ist

    \begin{tabular}{c|c}
      Intervall $i$ & $H_i = \ld\frac{1}{p\qty\big(x_i)}$ \\
      \hline
      1 & $\approx 0.76$ \\
      2 & $\approx 1.39$ \\
      3 & $\approx 2.04$ \\
      4 & $\approx 2.66$ \\
      5 & $\approx 3.22$ \\
      6 & $\approx 3.91$ \\
      7 & $\approx 3.91$ \\
    \end{tabular}

    und schließlich
    \[
      H_m = \sum_{i = 1}^7 p\qty\big(x_i) H_i \approx 2.07
    \]

  \newpage
  \item Wie groß wird der mittlere Informationsgehalt, wenn jedes Intervall
    zusätzlich in 16 Teilintervalle zerlegt wird?
    (Gleiche Auftrittswahrscheinlichkeiten in den Teilintervallen angenommen!)

    \subparagraph{Lsg.} Sei zuerst $M_i = 16$ die Intervallgröße.
    Die Auftrittswahrscheinlichkeit jedes Teilintervalls ist dann
    $\overline{p\qty\big(x_i)} = \frac{p\qty\big(x_i)}{M_i}$ und
    \begin{flalign*}
      H_m &= \sum_{i = 1}^{M_i \cdot N} \frac{p\qty\big(x_i)}{M_i} \ld \frac{1}{\sfrac{p\qty\big(x_i)}{M_i}} \\
          &= \sum_{i = 1}^N M_i \cdot \frac{p\qty\big(x_i)}{M_i} \ld \frac{1}{\sfrac{p\qty\big(x_i)}{M_i}} \\
          &= \sum_{i = 1}^N p\qty\big(x_1) \ld\qty(\frac{1}{p\qty\big(x_i) \cdot M_i}) \\
          &= \sum_{i = 1}^N p\qty\big(x_1) \qty(\ld\qty(\frac{1}{p\qty\big(x_i)}) + \ld\qty\big(M_i)) \\
          &\approx 6.07 \frac{\text{Bit}}{\text{Teilintervall}}
    \end{flalign*}
  \end{enumerate}
\end{enumerate}

\end{document}
