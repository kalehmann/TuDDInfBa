\documentclass{scrreprt}

\usepackage{aligned-overset}
\usepackage{amsmath}
\usepackage{amsthm}
\usepackage{amssymb}
\usepackage{bm}
\usepackage[inline,shortlabels]{enumitem}
\usepackage{framed}
\usepackage{hyperref}
\usepackage[utf8]{inputenc}
\usepackage{multicol}
\usepackage{mathtools}
\usepackage{pdflscape}
\usepackage{physics}
\usepackage{polynom}
\usepackage{tabularx}
\usepackage[table]{xcolor}
\usepackage{titling}
\usepackage{fancyhdr}
\usepackage{xfrac}
\usepackage{pgfplots}

\pgfplotsset{compat = newest}
\usepgfplotslibrary{fillbetween}
\usetikzlibrary{calc}


\author{Karsten Lehmann}
\date{WiSe 2024/25}
\title{Übungsblatt 13\\INF-B-110, Lineare Algebra}

\setlength{\parindent}{0pt}

\setlength{\headheight}{26pt}
\pagestyle{fancy}
\fancyhf{}
\lhead{\thetitle}
\rhead{\theauthor}
\lfoot{\thedate}
\rfoot{Seite \thepage}

\begin{document}
\paragraph{Ü 13.3 Orthogonalprojektion}

Verifizieren Sie, dass die beiden Vektoren
$u_1 = \qty\big(-7, 1, 4)^T$ und $u_2 = \qty\big(-1, 1, -2)^T$
orthogonal sind.
Nutzen Sie dies aus, um die Orthogonalprojektion von $v = \qty\big(-9, 1, 6)^T$
auf die durch $u_1$ und $u_2$ erzeugte Ebene zu berechnen.

\subparagraph{Lsg.} Es ist
\[
  \begin{pmatrix}
    -7 \\
    1  \\
    4  \\
  \end{pmatrix} \bullet \begin{pmatrix}
    -1 \\
    1  \\
    -2 \\
  \end{pmatrix} =
  \qty\big(-7) \cdot \qty\big(-1) + 1 \cdot 1 + 4 \cdot \qty\big(-2)
  = 7 + 1 - 8
  = 0
\]
Somit sind $u_1$ und $u_2$ tatsächlich orthogonal zueinander und bilden
eine Orthogonalbasis von $U$.

Somit ist
\[
  \hat{v}
  = \frac{v \bullet u_1}{u_1 \bullet u_1}u_1 + \frac{v \bullet u_2}{u_2 \bullet u_2}u_2
  = \frac{4}{3}u_1 + \frac{11}{3}u_2
  = \begin{pmatrix}
    -\frac{28}{3} \\
    \frac{4}{3}   \\
    \frac{16}{3}  \\
  \end{pmatrix} + \begin{pmatrix}
    -\frac{11}{3} \\
    \frac{11}{3}  \\
    -\frac{22}{3} \\
  \end{pmatrix} = \begin{pmatrix}
    -13 \\
    5   \\
    -2  \\
  \end{pmatrix}
\]
und schließlich $w = v - \hat{v} = \qty\big(4, -4, 8)^T$.

\paragraph{Ü 13.4 Orthogonalprojektion zur Abstandsberechnung}

Bestimmen Sie den Abstand des Punktes $v = \qty\big(2, -2, 4)^T$ von derjenigen
Geraden $g$, die durch die Punkte $w_1 = \qty\big(1, 0, 1)^T$ und
$w_2 = \qty\big(2, 2, 3)^T$ verläuft.

\subparagraph{Lsg.} Es ist $g = w_1 + \text{Span}\qty\big{w_2 - w_1}
= \qty\big(1, 0, 1)^T + \text{Span}\qty{\qty\big(1, 2, 2)^T}$.
Nun verläuft $g$ leider nicht durch den Ursprung und stellt damit keinen
Untervektorraum von $\mathbb{R}^3$ dar.
Um $g$ zu einen Untervektorraum zu machen verschieben wir $g$ (und dann auch
$v$) um $-w_1$.

Nun ist
\[
  g' =  \text{Span}\qty{\qty\big(1, 2, 2)^T}
  \text{ und }
  v' = v - w_1 = \qty\big(1, -2, 3)^T
\]
Anschließend ist
\[
  \hat{v'} = \frac{v' \bullet \qty\big(1, 2, 2)^T}{\qty\big(1, 2, 2)^T \bullet \qty\big(1, 2, 2)^T}\qty\big(1, 2, 2)^T
  = \frac{1}{3}\qty\big(1, 2, 2)^T
  = \qty(\frac{1}{3}, \frac{2}{3}, \frac{2}{3})^T
\]
und
\[
  w = v' - \hat{v'} = \qty\big(\frac{2}{3}, -\frac{8}{3}, \frac{7}{3})
\]
Schließlich
\[
  \hat{v} = \hat{v'} + w_1 = \qty(\frac{4}{3}, \frac{2}{3}, \frac{5}{3})^T
\]

\newpage
\textbf{Probe:} Ist $\hat{v} \in g$?
\[
  \hat{v} = \qty(\frac{4}{3}, \frac{2}{3}, \frac{5}{3})^T = \qty\big(1, 0, 1)^T + \frac{1}{3} \cdot \qty\big(1, 2, 2)^T
\]
$\Rightarrow \hat{v} \in g$

Ist $w$ orthogonal zu $g$?
\[
  w \bullet \qty\big(w_2 - w_1) =
  \qty\big(\frac{2}{3}, -\frac{8}{3}, \frac{7}{3})^T \bullet \qty\big(1, 2, 2)^T
  = \frac{2}{3} \cdot 1 - \frac{8}{3} \cdot 2 + \frac{7}{3} \cdot 2
  = \frac{2}{3} - \frac{16}{3} + \frac{14}{3} = 0
\]
$\Rightarrow g \bot w$

Ist $v = \hat{v} + w$?
\[
  \hat{v} + w =
  \qty(\frac{4}{3}, \frac{2}{3}, \frac{5}{3})^T + \qty\big(\frac{2}{3}, -\frac{8}{3}, \frac{7}{3})^T
  = \qty\big(2, -2, 4)^T = v
\]

Und nun ist der Abstand
\[
  \norm{w} = \norm{\begin{pmatrix}
    \frac{2}{3}  \\
    -\frac{8}{3} \\
    \frac{7}{3}  \\
  \end{pmatrix}} = \sqrt{
    \qty(\frac{2}{3})^2 + \qty(-\frac{8}{3})^2 + \qty(\frac{7}{3})^2
  } = \sqrt{
    \frac{4}{9} + \frac{64}{9} + \frac{49}{9}
  } = \sqrt{\frac{117}{9}} = \sqrt{13}
\]

\paragraph{Ü 13.5 Gram-Schmidt-Verfahren} Bestimmen Sie eine Orthonormalbasis $B$
mit dem Gram-Schmidtschen Orthonormalisierungsverfahren für den von den Vektoren
\[
  v_1 = \qty\big(3, 1, -1, 3)^T, v_2 = \qty\big(-5, 1, 5, -7)^T
  \text{ und } v_3 = \qty\big(4, 4, 2, 2)^T
\]
aufgespannten Untervektorraum
$U \coloneqq \text{Span}\qty\big{v_1, v_2, v_3} \subseteq \mathbb{R}^4$.

Bestimmen Sie die Koordinaten von $w = \qty\big(-4, 4, 8, -8)^T \in U$
bezüglich der Basis $B$.

Berechnen Sie den Orthogonalraum $U^{\bot} \subseteq \mathbb{R}^4$

\subparagraph{Lsg.} Es ist
\begin{flalign*}
  \bar{b_1} &= v_1 \\
  b_1 &= \frac{1}{\norm{\bar{b_1}}} \cdot b_1 = \frac{1}{\norm{\qty\big(3, 1, -1, 3)^T}} \cdot \qty\big(3, 1, -1, 3)^T \\
      &= \frac{1}{\sqrt{20}} \qty\big(3, 1, -1, 3)^T = \qty(\frac{3}{\sqrt{20}}, \frac{1}{\sqrt{20}}, -\frac{1}{\sqrt{20}}, \frac{3}{\sqrt{20}})^T
\end{flalign*}
\begin{flalign*}
  \bar{b_2} &= v_2 - \qty\big(v_2 \bullet b_1) b_1
  = \begin{pmatrix}
    -5 \\
    1  \\
    5  \\
    -7 \\
  \end{pmatrix} - \qty(\begin{pmatrix}
    -5 \\
    1  \\
    5  \\
    -7 \\
  \end{pmatrix} \bullet \begin{pmatrix}
    \frac{3}{\sqrt{20}}  \\
    \frac{1}{\sqrt{20}}  \\
    -\frac{1}{\sqrt{20}} \\
    \frac{3}{\sqrt{20}}  \\
  \end{pmatrix}) \cdot \begin{pmatrix}
    \frac{3}{\sqrt{20}}  \\
    \frac{1}{\sqrt{20}}  \\
    -\frac{1}{\sqrt{20}} \\
    \frac{3}{\sqrt{20}}  \\
  \end{pmatrix} \\
  &= \begin{pmatrix}
    -5 \\
    1  \\
    5  \\
    -7 \\
  \end{pmatrix} + \frac{40}{\sqrt{20}}\begin{pmatrix}
    \frac{3}{\sqrt{20}}  \\
    \frac{1}{\sqrt{20}}  \\
    -\frac{1}{\sqrt{20}} \\
    \frac{3}{\sqrt{20}}  \\
  \end{pmatrix} \\
  &= \begin{pmatrix}
    -5 \\
    1  \\
    5  \\
    -7 \\
  \end{pmatrix} + \begin{pmatrix}
    6  \\
    2  \\
    -2 \\
    6  \\
  \end{pmatrix}
  = \begin{pmatrix}
    1  \\
    3  \\
    3  \\
    -1 \\
  \end{pmatrix} \\
  b_2 &= \frac{1}{\norm{\bar{b_2}}} \cdot b_2
       = \frac{1}{\norm{\begin{pmatrix}
         1  \\
         3  \\
         3  \\
         -1 \\
       \end{pmatrix}}}\begin{pmatrix}
         1  \\
         3  \\
         3  \\
         -1 \\
       \end{pmatrix}
       = \frac{1}{\sqrt{20}} \begin{pmatrix}
         1  \\
         3  \\
         3  \\
         -1 \\
       \end{pmatrix} \\
  \bar{b_3} &= v_3 - \qty\big(v_3 \bullet b_1)b_1 - \qty\big(v_3 \bullet b_2)b_2 \\
            &= \begin{pmatrix}
              4 \\
              4 \\
              2 \\
              2 \\
            \end{pmatrix} - \qty(
              \begin{pmatrix}
                4 \\
                4 \\
                2 \\
                2 \\
              \end{pmatrix} \bullet \frac{1}{\sqrt{20}} \begin{pmatrix}
                3  \\
                1  \\
                -1 \\
                3  \\
              \end{pmatrix}
            ) \frac{1}{\sqrt{20}} \begin{pmatrix}
              3  \\
              1  \\
              -1 \\
              3  \\
            \end{pmatrix} - \qty(
              \begin{pmatrix}
                4 \\
                4 \\
                2 \\
                2 \\
              \end{pmatrix} \bullet \frac{1}{\sqrt{20}} \begin{pmatrix}
                1  \\
                3  \\
                3  \\
                -1 \\
              \end{pmatrix}
            ) \frac{1}{\sqrt{20}} \begin{pmatrix}
              1  \\
              3  \\
              3  \\
              -1 \\
            \end{pmatrix} \\
            &= \begin{pmatrix}
              4 \\
              4 \\
              2 \\
              2 \\
            \end{pmatrix} - \frac{20}{\sqrt{20}}\frac{1}{\sqrt{20}} \begin{pmatrix}
              3  \\
              1  \\
              -1 \\
              3  \\
            \end{pmatrix} - \frac{20}{\sqrt{20}}\frac{1}{\sqrt{20}} \begin{pmatrix}
              1  \\
              3  \\
              3  \\
              -1 \\
            \end{pmatrix} \\
            &= \begin{pmatrix}
              4 \\
              4 \\
              2 \\
              2 \\
            \end{pmatrix} - \begin{pmatrix}
              3  \\
              1  \\
              -1 \\
              3  \\
            \end{pmatrix} - \begin{pmatrix}
              1  \\
              3  \\
              3  \\
              -1 \\
            \end{pmatrix} = \begin{pmatrix}
              0 \\
              0 \\
              0 \\
              0 \\
            \end{pmatrix}
\end{flalign*}
Hierbei fällt jetzt auch auf, dass schon $v_1, v_2$ und $v_3$ linear abhängig
sind mit
\[
  3 \cdot v_1 + v_2 = 3 \cdot \begin{pmatrix}
    3  \\
    1  \\
    -1 \\
    3  \\
  \end{pmatrix} + \begin{pmatrix}
    -5 \\
    1  \\
    5  \\
    -7 \\
  \end{pmatrix} = \begin{pmatrix}
    4 \\
    4 \\
    2 \\
    2 \\
  \end{pmatrix} = v_3
\]
Anyway, $B = \qty\big(b_1, b_2)$.
Weiter ist (das $\frac{1}{\sqrt{20}}$ wurde in der Matrix weggelassen um
Schreibaufwand zu sparen.
Beim Ergebnis ist es unbedingt wieder einzufügen)
\begin{flalign*}
  \qty(\begin{array}{cc|c}
    3  & 1  & -4 \\
    1  & 3  & 4  \\
    -1 & 3  & 8  \\
    3  & -1 & -8 \\
  \end{array})
  \overset{Z_4 = -\frac{1}{2} \cdot \qty\big(Z_4 - Z_1)}&\leadsto
  \qty(\begin{array}{cc|c}
    3  & 1 & -4 \\
    1  & 3 & 4  \\
    -1 & 3 & 8  \\
    0  & 1 & 2  \\
  \end{array}) \\
  \overset{Z_2 = Z_2 - 3 \cdot Z_4}&\leadsto
  \qty(\begin{array}{cc|c}
    3  & 1 & -4 \\
    1  & 0 & -2 \\
    -1 & 3 & 8  \\
    0  & 1 & 2  \\
  \end{array})  \\
\end{flalign*}
Und somit ist $w = -2\sqrt{20} \cdot b_1 + 2\sqrt{20} \cdot b_2$ und
$w_B = \sqrt{20}\qty\big(-2, 2)^T$.

Zum Schluss (das $\frac{1}{\sqrt{20}}$ ist hier wieder egal)
\begin{flalign*}
  U^{\bot} &= \text{Ker}\begin{pmatrix}
    3 & 1 & -1 & 3  \\
    1 & 3 & 3  & -1 \\
  \end{pmatrix} \\
  \overset{Z_1 = -\frac{1}{8}\qty\big(Z_1 - 3 \cdot Z_2)}&= \text{Ker}\begin{pmatrix}
    0 & 1  & \frac{5}{4} & -\frac{3}{4}  \\
    1 & 3  & 3           & -1            \\
  \end{pmatrix} \\
  \overset{Z_2 = Z_2 - 3 \cdot Z_1}&= \text{Ker}\begin{pmatrix}
    0 & 1  & \frac{5}{4}  & -\frac{3}{4}  \\
    1 & 0  & -\frac{3}{4} & \frac{5}{4}   \\
  \end{pmatrix} \\
  &= \qty{
    \begin{pmatrix}
      \frac{3}{4}x - \frac{5}{4}y  \\
      -\frac{5}{4}x + \frac{3}{4}y \\
      x                            \\
      y                            \\
    \end{pmatrix}
    \:\middle|\:
    x, y \in \mathbb{R}
  } \\
  &= \qty{
    x \cdot \begin{pmatrix}
      \frac{3}{4}  \\
      -\frac{5}{4} \\
      1            \\
      0            \\
    \end{pmatrix} + y \cdot \begin{pmatrix}
      -\frac{5}{4} \\
      \frac{3}{4}  \\
      0            \\
      1            \\
    \end{pmatrix}
    \:\middle|\:
    x, y \in \mathbb{R}
  } \\
  &= \text{Span}\qty{
    \begin{pmatrix}
      3  \\
      -5 \\
      4  \\
      0  \\
    \end{pmatrix}, \begin{pmatrix}
      -5 \\
      3  \\
      0  \\
      4  \\
    \end{pmatrix}
  }
\end{flalign*}

\newpage
\paragraph{Ü 13.6 Gram-Schmidt-Verfahren für einen Polynom-VR}

Die Menge aller reellen Polynomfunktionen $p\qty\big(x)$ vom Grad höchstens 2
bildet zusammen mit der Addition von Polynomen und der Skalarmultiplikation
bekanntermaßen den $\mathbb{R}$-Vektorraum $\mathbb{P}_{\leq 2}$.
\begin{enumerate}[(a)]
\item Das Tripel $\qty\big(f_1, f_2, f_3)$ mit $f_1\qty\big(x) = 1$,
  $f_2\qty\big(x) = x$, $f_3\qty\big(x) = x^2$ ist eine Basis von
  $\mathbb{P}_{\leq 2}$.
  Bestimmen Sie aus dieser Basis eine Orthonormalbasis $B$ mit Hilfe des
  Gram-Schmidt-Verfahres bezüglich des Skalarprodukts
  \[
    f \bullet g = \int_{-1}^1 f\qty\big(x)g\qty\big(x)dx
  \]

  \subparagraph{Lsg.} Es ist
  \begin{flalign*}
    \bar{b_1} &= f_1 = 1 \\
    b_1 &= \frac{1}{\norm{\bar{b_1}}}\bar{b_1}
         = \frac{1}{\sqrt{\int_{-1}^1 1 dx}}
         = \frac{1}{\sqrt{\qty[x]_{-1}^{1}}}
         = \frac{1}{\sqrt{2}} \\
    \bar{b_2} &= f_2 - \qty\big(f_2 \bullet b_1)b_1
               = x - x \bullet \frac{1}{\sqrt{2}}
               = x - \int_{-1}^1 \frac{x}{\sqrt{2}} dx \\
              &= x - \qty[\frac{1}{2}x^2]_{-1}^1
               = x - 0 = x \\
    b_2 &= \frac{1}{\norm{\bar{b_2}}}\bar{b_2}
         = \frac{1}{\sqrt{\int_{-1}^1 x^2 dx}}x
         = \frac{1}{\sqrt{\qty[\frac{x^3}{3}]_{-1}^1}}x
         = \frac{\sqrt{3}}{\sqrt{2}}x  \\
    \bar{b_3} &= f_3 - \qty\big(f_3 \bullet b_1)b_1 - \qty\big(f_3 \bullet b_2)b_2
               = x^2 - \qty(x^2 \bullet \frac{1}{\sqrt{2}})\frac{1}{\sqrt{2}} - \qty(x^2 \bullet \frac{\sqrt{3}}{\sqrt{2}}x) \frac{\sqrt{3}}{\sqrt{2}}x \\
              &= x^2 - \qty(\int_{-1}^1 \frac{x^2}{\sqrt{2}} dx)\frac{1}{\sqrt{2}} -
                 \qty(\int_{-1}^1\frac{\sqrt{3}}{\sqrt{2}}x^3 dx)\frac{\sqrt{3}}{\sqrt{2}}x \\
              &= x^2 - \qty[\frac{x^3}{3\sqrt{2}}]_{-1}^1\frac{1}{\sqrt{2}} -
                \qty[\frac{\sqrt{3}}{4\sqrt{2}}x^4]_{-1}^1\frac{\sqrt{3}}{\sqrt{2}}x \\
              &= x^2 - \frac{1}{3} \\
  \end{flalign*}
  \begin{flalign*}
    b_3 &= \frac{1}{\norm{\bar{b_3}}}\bar{b_3}
         = \frac{1}{\sqrt{\int_{-1}^1 x^4 - \frac{2}{3}x^2 + \frac{1}{9} dx}}\qty(x^2 - \frac{1}{3}) \\
        &= \frac{1}{\sqrt{\qty[\frac{1}{5}x^5 - \frac{2}{9}x^3 + \frac{1}{9}x]_{-1}^1}} \qty(x^2 - \frac{1}{3}) \\
        &= \frac{1}{\sqrt{\frac{8}{45}}}\qty(x^2 - \frac{1}{3}) \\
        &= \frac{3\sqrt{5}}{2\sqrt{2}}\qty(x^2 - \frac{1}{3}) \\
        &= \frac{3\sqrt(5)}{2\sqrt(2)}x^2 - \frac{\sqrt{5}}{2\sqrt{2}}
  \end{flalign*}
  $B = \qty\big(b_1, b_2, b_3)$

\item Bestimmen Sie die Koordinaten von $g\qty\big(x) = x^2 - x$ bezüglich der
  orthonormierten Basis $B$.

  \subparagraph{Lsg.} Es ist
  \[
    g\qty\big(x) = x^2 - x =
    \frac{2\sqrt{2}}{3\sqrt{5}}\frac{3\sqrt(5)}{2\sqrt(2)}x^2 - \frac{2\sqrt{2}}{3\sqrt{5}}\frac{\sqrt{5}}{2\sqrt{2}}
    + \frac{\sqrt{2}}{3}\frac{1}{\sqrt{2}} - \frac{\sqrt{2}}{\sqrt{3}}\frac{\sqrt{3}}{\sqrt{2}}x
  \]
  und $g_B = \qty(\frac{\sqrt{2}}{3}, -\frac{\sqrt{2}}{\sqrt{3}}, \frac{2\sqrt{2}}{3\sqrt{5}})^T$.
\end{enumerate}
\end{document}
