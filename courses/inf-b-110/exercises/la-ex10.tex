\documentclass{scrreprt}

\usepackage{aligned-overset}
\usepackage{amsmath}
\usepackage{amsthm}
\usepackage{amssymb}
\usepackage{bm}
\usepackage[inline,shortlabels]{enumitem}
\usepackage{framed}
\usepackage{hyperref}
\usepackage[utf8]{inputenc}
\usepackage{multicol}
\usepackage{mathtools}
\usepackage{pdflscape}
\usepackage{physics}
\usepackage{polynom}
\usepackage{tabularx}
\usepackage[table]{xcolor}
\usepackage{titling}
\usepackage{fancyhdr}
\usepackage{xfrac}
\usepackage{pgfplots}

\pgfplotsset{compat = newest}
\usepgfplotslibrary{fillbetween}
\usetikzlibrary{calc}


\author{Karsten Lehmann}
\date{WiSe 2024/25}
\title{Übungsblatt 10\\INF-B-110, Lineare Algebra}

\setlength{\parindent}{0pt}

\setlength{\headheight}{26pt}
\pagestyle{fancy}
\fancyhf{}
\lhead{\thetitle}
\rhead{\theauthor}
\lfoot{\thedate}
\rfoot{Seite \thepage}

\begin{document}
\paragraph{Ü10.3 Berechnung von Eigenwerten und Eigenräumen}
\begin{enumerate}[(a)]
\item Es sei $M = \begin{pmatrix}
    3 & 2 \\
    2 & 3 \\
  \end{pmatrix} \in \mathbb{R}^{2 \times 2}$.
  Zeigen Sie, dass $v_1 = \begin{pmatrix} 1 \\ 1 \end{pmatrix}$ ein Eigenvektor
  von $M$ ist und bestimmen Sie den zugehörigen Eigenwert $\lambda_1$.
  Zeigen Sie, dass $\lambda_2 = 1$ ein Eigenwert von $M$ ist und bestimmen Sie
  den zugehörigen Eigenraum $\text{Eig}\qty\big(M, \lambda_2)$.

  \subparagraph{Lsg.} Es ist
  \[
    \begin{pmatrix}
      3 & 2 \\
      2 & 3 \\
    \end{pmatrix} \cdot \begin{pmatrix}
      1 \\
      1 \\
    \end{pmatrix} = \begin{pmatrix}
      5 \\
      5 \\
    \end{pmatrix} = 5 \cdot v_1
  \]
  Somit ist $v_1$ ein Eigenvektor mit dem Eigenwert $\lambda_1 = 5$.
  Weiter ist
  \begin{flalign*}
    \det\qty\big(M - \lambda \cdot E_2)
    &= \det\qty(\begin{pmatrix}
      3 - \lambda & 2           \\
      2           & 3 - \lambda \\
    \end{pmatrix}) \\
    &= \qty\big(3 - \lambda)^2 - 4 \\
    &= \lambda^2 - 6\lambda  + 5 \\
    &= \qty\big(\lambda - 5)\qty\big(\lambda - 1)
  \end{flalign*}
  Damit ist $\lambda_1 = 5$ und $\lambda_2 = 1$ die einzigen Eigenwerte von $M$.

  \textbf{Alternativ nach der Übung:} Einfach $\text{Eig}\qty\big(M, 1)$
  aufstellen und schauen, dass es nicht der Nullraum ist.

  Nun ist $\text{Eig}\qty\big(M, \lambda_2)
  = \text{Ker}\qty\big(M - \lambda_2 \cdot E_2)$ und
  \begin{flalign*}
    M - 1 \cdot E_2
    &= \begin{pmatrix}
      3 - 1 & 2     \\
      2     & 3 - 1 \\
    \end{pmatrix} \\
    &= \begin{pmatrix}
      2 & 2 \\
      2 & 2 \\
    \end{pmatrix} \\
    \overset{Z_2 = Z_2 - Z_1}&\leadsto \begin{pmatrix}
      2 & 2 \\
      0 & 0 \\
    \end{pmatrix} \\
    \overset{Z_1 = \frac{1}{2} \cdot Z_1}&\leadsto \begin{pmatrix}
      1 & 1 \\
      0 & 0 \\
    \end{pmatrix}
  \end{flalign*}
  Somit ist $x_1 + x_2 = 0$ oder $x_2 = -x_1$ und
  \[
    \text{Eig}\qty\big(M, \lambda_2) =
    \text{Ker}\qty\big(M - \lambda_2 \cdot E_2) =
    \qty{
      \begin{pmatrix}
        x_1  \\
        -x_1 \\
      \end{pmatrix}
      \:\middle|\:
      x_1 \in \mathbb{R}
    } =
    \text{Span}\qty(\begin{pmatrix} 1 \\ -1 \end{pmatrix})
  \]

\newpage
\item Bestimmen Sie für die gegebenen Matrizen $A, B \in \mathbb{R}^{3 \times 3}$
  und $C \in \mathbb{R}^{4 \times 4}$ jeweils das charakteristische Polynom, die
  Eigenwerte und eine Basis für jeden Eigenraum.
  \[
    A = \begin{pmatrix}
      4  & 0 & 1 \\
      -2 & 1 & 0 \\
      -2 & 0 & 1 \\
    \end{pmatrix}, \qquad
    B = \begin{pmatrix}
      2 & 3 & 3 \\
      3 & 2 & 3 \\
      3 & 3 & 2 \\
    \end{pmatrix}, \qquad
    C = \begin{pmatrix}
      0  & 0 & 1 & 1 \\
      -2 & 2 & 1 & 1 \\
      -3 & 0 & 4 & 1 \\
      -3 & 0 & 1 & 4 \\
    \end{pmatrix}
  \]

  \subparagraph{Lsg.} Es ist
  \begin{flalign*}
    \det\qty\big(A - \lambda \cdot E_3)
    &= \det\qty(\begin{pmatrix}
      4 - \lambda & 0           & 1           \\
      -2          & 1 - \lambda & 0           \\
      -2          & 0           & 1 - \lambda \\
    \end{pmatrix}) \\
    \overset{\text{Entwicklung nach der 2. Spalte}}&=
    \qty\big(1 - \lambda) \cdot \det\qty(\begin{pmatrix}
      4 - \lambda & 1           \\
      -2          & 1 - \lambda \\
    \end{pmatrix}) \\
    &= \qty\big(1 - \lambda)\qty\big(\qty\big(4 - \lambda)\qty\big(1 - \lambda) + 2) \\
    &= \qty\big(1 - \lambda)\qty\big(\lambda^2 - 5\lambda + 6) \\
    &= \qty\big(1 - \lambda)\qty\big(\lambda - 2)\qty\big(\lambda - 3)
  \end{flalign*}
  Somit sind $\lambda_1 = 1$, $\lambda_2 = 2$ und $\lambda_3 = 3$ die Eigenwerte
  von $A$.
  \begin{enumerate}[($\lambda_1$)]
  \item
    \begin{flalign*}
      A - \lambda_1 \cdot E_3
      &= \begin{pmatrix}
        3  & 0 & 1 \\
        -2 & 0 & 0 \\
        -2 & 0 & 0 \\
      \end{pmatrix}
      \overset{Z_3 = Z_3 - Z_2}\leadsto \begin{pmatrix}
        3  & 0 & 1 \\
        -2 & 0 & 0 \\
        0  & 0 & 0 \\
      \end{pmatrix} \\
      \overset{Z_2 = -\frac{1}{2} \cdot Z_2}&\leadsto \begin{pmatrix}
        3 & 0 & 1 \\
        1 & 0 & 0 \\
        0 & 0 & 0 \\
      \end{pmatrix}
      \overset{Z_1 = Z_1 - 3 \cdot Z_2}\leadsto \begin{pmatrix}
        0 & 0 & 1 \\
        1 & 0 & 0 \\
        0 & 0 & 0 \\
      \end{pmatrix}
    \end{flalign*}
    Somit sind $x_1 = 0$ und $x_3 = 0$.
    \[
      \text{Eig}\qty\big(A, \lambda_1) =
      \text{Ker}\qty\big(A - \lambda_1 \cdot E_3) =
      \qty{
        \begin{pmatrix}
          0   \\
          x_2 \\
          0   \\
        \end{pmatrix}
        \:\middle|\:
        x_2 \in \mathbb{R}
      } = \text{Span}\qty(\begin{pmatrix} 0 \\ 1 \\ 0 \end{pmatrix})
    \]

  \item
    \[
      A - \lambda_2 \cdot E_3
      = \begin{pmatrix}
        2  & 0  & 1  \\
        -2 & -1 & 0  \\
        -2 & 0  & -1 \\
      \end{pmatrix}
      \overset{Z_3 = Z_3 + Z_1}\leadsto \begin{pmatrix}
        2  & 0  & 1 \\
        -2 & -1 & 0 \\
        0  & 0  & 0 \\
      \end{pmatrix}
      \overset{Z_2 = Z_2 + Z_1}\leadsto \begin{pmatrix}
        2 & 0  & 1 \\
        0 & -1 & 1 \\
        0 & 0  & 0 \\
      \end{pmatrix}
    \]
    Somit sind $-\frac{1}{2}x_1 = x_3$ und $x_2 = x_3$.
    \[
      \text{Eig}\qty\big(A, \lambda_2) =
      \text{Ker}\qty\big(A - \lambda_2 \cdot E_3) =
      \qty{
        \begin{pmatrix}
          -\frac{1}{2}x_3 \\
          x_3             \\
          x_3             \\
        \end{pmatrix}
        \:\middle|\:
        x_3 \in \mathbb{R}
      } = \text{Span}\qty(\begin{pmatrix} -1 \\ 2 \\ 2 \end{pmatrix})
    \]

  \item
    \[
      A - \lambda_3 \cdot E_3
      = \begin{pmatrix}
        1  & 0  & 1  \\
        -2 & -2 & 0  \\
        -2 & 0  & -2 \\
      \end{pmatrix}
      \overset{Z_3 = Z_3 + 2 \cdot Z_1}\leadsto \begin{pmatrix}
        1  & 0  & 1  \\
        -2 & -2 & 0  \\
        0  & 0  & 0 \\
      \end{pmatrix}
      \overset{Z_2 = -\frac{1}{2}\qty(Z_2 + 2 \cdot Z_1)}\leadsto \begin{pmatrix}
        1 & 0  & 1  \\
        0 & 1  & -1 \\
        0 & 0  & 0  \\
      \end{pmatrix}
    \]
    Somit sind $x_1 = -x_3$ und $x_2 = x_3$.
    \[
      \text{Eig}\qty\big(A, \lambda_3) =
      \text{Ker}\qty\big(A - \lambda_3 \cdot E_3) =
      \qty{
        \begin{pmatrix}
          -x_3 \\
          x_3  \\
          x_3  \\
        \end{pmatrix}
        \:\middle|\:
        x_3 \in \mathbb{R}
      } = \text{Span}\qty(\begin{pmatrix} -1 \\ 1 \\ 1 \end{pmatrix})
    \]
  \end{enumerate}

  \newpage
  \begin{landscape}
  Weiter ist
  \begin{flalign*}
    \det\qty\big(B - \lambda \cdot E_3)
    &= \det\qty(\begin{pmatrix}
      2 - \lambda & 3           & 3           \\
      3           & 2 - \lambda & 3           \\
      3           & 3           & 2 - \lambda \\
    \end{pmatrix}) \\
    \overset{\text{Entwicklung nach 1. Spalte}}&=
      \qty\big(2 - \lambda)\det\qty(\begin{pmatrix}
        2 - \lambda & 3           \\
        3           & 2 - \lambda \\
      \end{pmatrix}) - 3 \det\qty(\begin{pmatrix}
        3 & 3           \\
        3 & 2 - \lambda \\
      \end{pmatrix}) + 3 \det\qty(\begin{pmatrix}
        3           & 3 \\
        2 - \lambda & 3 \\
      \end{pmatrix}) \\
    &= \qty\big(2 - \lambda)\qty(\qty\big(2 - \lambda)^2 - 9) - 3\qty\big(2\qty\big(2 - \lambda) - 9) + 3\qty\big(9 - 3\qty\big(2 - \lambda)) \\
    &= \qty\big(2 - \lambda)\qty(\qty\big(2 - \lambda)^2 - 9) + \qty\big(9 + 9\lambda) + \qty\big(9 + 9\lambda) \\
    &= \qty\big(2 - \lambda)\qty(\lambda^2 - 4\lambda - 5) + \qty\big(9 + 9\lambda) + \qty\big(9 + 9\lambda) \\
    &= 2\lambda^2 - 8\lambda - 10 - \lambda^3 + 4\lambda^2 + 5\lambda + 9 + 9\lambda + 9 + 9\lambda \\
    &= -\lambda^3 + 6\lambda^2 + 15 \lambda + 8
  \end{flalign*}
  Hier sieht man direkt $\lambda_1 = -1$.
  Es verbleiben
  \[
    \qty\big(\lambda + 1)\qty\big(-\lambda^2 + 7\lambda + 8) =
    \qty\big(\lambda + 1)\qty\big(-\lambda + 8)\qty\big(\lambda + 1)
  \]
  Somit sind $\lambda_1 = -1$ und $\lambda_2 = 8$ die Eigenwerte von $B$.

  \textbf{Alternativ nach der Übung:}
  \begin{flalign*}
    \begin{pmatrix}
      2 - \lambda & 3           & 3           \\
      3           & 2 - \lambda & 3           \\
      3           & 3           & 2 - \lambda \\
    \end{pmatrix}
    \overset{S_2 = S_2 - S_1}&\leadsto
    \begin{pmatrix}
      2 - \lambda & 1 + \lambda  & 3           \\
      3           & -1 - \lambda & 3           \\
      3           & 0            & 2 - \lambda \\
    \end{pmatrix}
    \overset{S_3 = S_3 - S_1}\leadsto
    \begin{pmatrix}
      2 - \lambda & 1 + \lambda  & 1 + \lambda  \\
      3           & -1 - \lambda & 0            \\
      3           & 0            & -1 - \lambda \\
    \end{pmatrix} \\
    \overset{Z_1 = Z_1 + Z_2 + Z_3}&\leadsto
    \begin{pmatrix}
      8 - \lambda & 0            & 0            \\
      3           & -1 - \lambda & 0            \\
      3           & 0            & -1 - \lambda \\
    \end{pmatrix}
  \end{flalign*}
  Und dann ist die Determinante der oberen Dreiecksmatrix trivial.
  \end{landscape}
  \begin{enumerate}[($\lambda_1$)]
  \item
    \begin{flalign*}
      B - \lambda_1 \cdot E_3
      &= \begin{pmatrix}
        3 & 3 & 3 \\
        3 & 3 & 3 \\
        3 & 3 & 3 \\
      \end{pmatrix}
      \overset{Z_3 = Z_3 - Z_1}\leadsto \begin{pmatrix}
        3 & 3 & 3 \\
        3 & 3 & 3 \\
        0 & 0 & 0 \\
      \end{pmatrix} \\
      \overset{Z_2 = Z_2 - Z_1}&\leadsto \begin{pmatrix}
        3 & 3 & 3 \\
        0 & 0 & 0 \\
        0 & 0 & 0 \\
      \end{pmatrix}
      \overset{Z_1 = \frac{1}{3} \cdot Z_1}\leadsto \begin{pmatrix}
        1 & 1 & 1 \\
        0 & 0 & 0 \\
        0 & 0 & 0 \\
      \end{pmatrix}
    \end{flalign*}
    Somit ist $x_1 = -x_2 - x_3$ und
    \begin{flalign*}
      \text{Eig}\qty\big(B, \lambda_1) &=
      \text{Ker}\qty\big(B - \lambda_1 \cdot E_3) =
      \qty{
        \begin{pmatrix}
          -x_2 \\
          x_2  \\
          0    \\
        \end{pmatrix}, \begin{pmatrix}
          -x_3 \\
          0    \\
          x_3  \\
        \end{pmatrix}
        \:\middle|\:
        x_2, x_3 \in \mathbb{R}
      }\\
      &= \text{Span}\qty(
        \begin{pmatrix} -1 \\ 1 \\ 0 \end{pmatrix},
        \begin{pmatrix} -1 \\ 0 \\ 1 \end{pmatrix}
      )
    \end{flalign*}

  \item
    \begin{flalign*}
      B - \lambda_2 \cdot E_3
      = \begin{pmatrix}
        -6 & 3  & 3  \\
        3  & -6 & 3  \\
        3  & 3  & -6 \\
      \end{pmatrix}
      \overset{Z_3 = Z_3 - Z_2}&\leadsto \begin{pmatrix}
        -6 & 3  & 3  \\
        3  & -6 & 3  \\
        0  & 9  & -9 \\
      \end{pmatrix}
      \overset{Z_2 = 2 \cdot Z_2 + \cdot Z_1}\leadsto \begin{pmatrix}
        -6 & 3  & 3  \\
        0  & -9 & 9 \\
        0  & 9  & -9 \\
      \end{pmatrix} \\
      \overset{Z_3 = Z_3 + Z_2}&\leadsto \begin{pmatrix}
        -6 & 3  & 3 \\
        0  & -9 & 9 \\
        0  & 0  & 0 \\
      \end{pmatrix}
      \overset{Z_2 = -\frac{1}{9} \cdot Z_2}\leadsto \begin{pmatrix}
        -6 & 3 & 3  \\
        0  & 1 & -1 \\
        0  & 0 & 0  \\
      \end{pmatrix} \\
      \overset{Z_1 = -\frac{1}{6} \cdot Z_1}&\leadsto \begin{pmatrix}
        1 & -\frac{1}{2} & -\frac{1}{2} \\
        0 & 1            & -1           \\
        0 & 0            & 0            \\
      \end{pmatrix}
      \overset{Z_1 = Z_1 + \frac{1}{2} \cdot Z_2}\leadsto \begin{pmatrix}
        1 & 0 & -1 \\
        0 & 1 & -1 \\
        0 & 0 & 0  \\
      \end{pmatrix}
    \end{flalign*}
    Somit sind $x_1 = x_3$ und $x_2 = x_3$
    \[
      \text{Eig}\qty\big(B, \lambda_2) =
      \text{Ker}\qty\big(B - \lambda_2 \cdot E_3) =
      \qty{
        \begin{pmatrix}
          x_3 \\
          x_3 \\
          x_3 \\
        \end{pmatrix}
        \:\middle|\:
        x_3 \in \mathbb{R}
      } = \text{Span}\qty(\begin{pmatrix} 1 \\ 1 \\ 1 \end{pmatrix})
    \]

  \end{enumerate}
  \newpage
  \begin{landscape}
  Schließlich ist noch
  \begin{flalign*}
    \det\qty\big(C - \lambda \cdot E_4)
      &= \det\qty(\begin{pmatrix}
        -\lambda & 0           & 1           & 1           \\
        -2       & 2 - \lambda & 1           & 1           \\
        -3       & 0           & 4 - \lambda & 1           \\
        -3       & 0           & 1           & 4 - \lambda \\
      \end{pmatrix}) \\
    \overset{\text{Entwicklung nach 2. Spalte}}&=
      \qty\big(2 - \lambda)\det\qty(\begin{pmatrix}
        -\lambda & 1           & 1           \\
        -3       & 4 - \lambda & 1           \\
        -3       & 1           & 4 - \lambda \\
      \end{pmatrix})  \\
    \overset{\text{Entwicklung nach 1. Zeile}}&=
      \qty\big(2 - \lambda)\qty(-\lambda \det\begin{pmatrix}
        4 - \lambda & 1           \\
        1           & 4 - \lambda \\
      \end{pmatrix} - \det\begin{pmatrix}
        -3 & 1           \\
        -3 & 4 - \lambda \\
      \end{pmatrix} + \det\begin{pmatrix}
        -3       & 4 - \lambda \\
        -3       & 1           \\
      \end{pmatrix})  \\
    &= \qty\big(2 - \lambda)\qty\Big(
        -\lambda\qty(\qty\big(4 - \lambda)^2 - 1) -
        \qty(-3\qty\big(4 - \lambda) + 3) +
        \qty\big(-3 + 3\qty\big(4 - \lambda))
      ) \\
    &= \qty\big(2 - \lambda)\qty\Big(
        -\lambda\qty(\lambda^2 - 8\lambda + 17) -
        \qty\big(-0 + 3\lambda) +
        \qty\big(9 - 3\lambda)
      ) \\
    &= \qty\big(2 - \lambda)\qty\Big(
        -\lambda^3 + 8\lambda^2 - 17\lambda +
        9 - 3\lambda + 9 - 3\lambda
      ) \\
    &= \qty\big(2 - \lambda)\qty\big(
        -\lambda^3 + 8\lambda^2 - 23\lambda + 18
      )
  \end{flalign*}
  Mit etwas Glück erkennt man in der zweiten Klammer das
  $\lambda_1 = \lambda_2 = 2$.
  Weiter geht es mit
  \[
    \qty\big(2 - \lambda)^2\qty\big(\lambda^2 - 6\lambda + 9) =
    \qty\big(2 - \lambda)^2\qty\big(3 - \lambda)^2
  \]
  Somit sind $\lambda_1 = \lambda_2 = 2$ und $\lambda_3 = \lambda_4 = 3$ die
  Eigenwerte von $C$
  \end{landscape}
  \begin{enumerate}
  \item[($\lambda_{1|2}$)]
    \begin{flalign*}
      C - \lambda_{1|2} \cdot E_4
      &= \begin{pmatrix}
        -2 & 0 & 1 & 1 \\
        -2 & 0 & 1 & 1 \\
        -3 & 0 & 2 & 1 \\
        -3 & 0 & 1 & 2 \\
      \end{pmatrix}
      \overset{Z_2 = Z_2 - Z_1}\leadsto \begin{pmatrix}
        -2 & 0 & 1 & 1 \\
        0  & 0 & 0 & 0 \\
        -3 & 0 & 2 & 1 \\
        -3 & 0 & 1 & 2 \\
      \end{pmatrix} \\
      \overset{Z_4 = Z_4 - Z_3}&\leadsto \begin{pmatrix}
        -2 & 0 & 1  & 1 \\
        0  & 0 & 0  & 0 \\
        -3 & 0 & 2  & 1 \\
        0  & 0 & -1 & 1 \\
      \end{pmatrix}
      \overset{Z_1 = -\frac{1}{2} \cdot \qty\big(Z_1 + Z_4)}\leadsto \begin{pmatrix}
        1  & 0 & 0  & -1 \\
        0  & 0 & 0  & 0  \\
        -3 & 0 & 2  & 1  \\
        0  & 0 & -1 & 1  \\
      \end{pmatrix} \\
      \overset{Z_3 = \frac{1}{2} \cdot \qty\big(Z_3 + 3 \cdot Z_1)}&\leadsto \begin{pmatrix}
        1 & 0 & 0  & -1 \\
        0 & 0 & 0  & 0  \\
        0 & 0 & 1  & -1 \\
        0 & 0 & -1 & 1  \\
      \end{pmatrix}
      \overset{Z_4 = Z_4 + Z_3)}\leadsto \begin{pmatrix}
        1 & 0 & 0  & -1 \\
        0 & 0 & 0  & 0  \\
        0 & 0 & 1  & -1 \\
        0 & 0 & 0  & 0  \\
      \end{pmatrix}
    \end{flalign*}
    Somit ist $x_1 = x_4$ und $x_3 = x_4$
    \begin{flalign*}
      \text{Eig}\qty\big(C, \lambda_{1|2}) &=
      \text{Ker}\qty\big(C - \lambda_{1|2} \cdot E_4) =
      \qty{
        \begin{pmatrix}
          x_4 \\
          0   \\
          x_4 \\
          x_4 \\
        \end{pmatrix}, \begin{pmatrix}
          0   \\
          x_2 \\
          0   \\
          0   \\
        \end{pmatrix}
        \:\middle|\:
        x_2, x_4 \in \mathbb{R}
      }\\
      &= \text{Span}\qty(
        \begin{pmatrix} 1 \\ 0 \\ 1 \\ 1 \end{pmatrix},
        \begin{pmatrix} 0 \\ 1 \\ 0 \\ 0 \end{pmatrix}
      )
    \end{flalign*}

  \item[($\lambda_{3|4}$)]
    \begin{flalign*}
      C - \lambda_{3|4} \cdot E_4
      = \begin{pmatrix}
        -3 & 0  & 1 & 1 \\
        -2 & -1 & 1 & 1 \\
        -3 & 0  & 1 & 1 \\
        -3 & 0  & 1 & 1 \\
      \end{pmatrix}
      \overset{Z_4 = Z_4 - Z_3}&\leadsto \begin{pmatrix}
        -3 & 0  & 1 & 1 \\
        -2 & -1 & 1 & 1 \\
        -3 & 0  & 1 & 1 \\
        0  & 0  & 0 & 0 \\
      \end{pmatrix}
      \overset{Z_3 = Z_3 - Z_1}\leadsto \begin{pmatrix}
        -3 & 0  & 1 & 1 \\
        -2 & -1 & 1 & 1 \\
        0  & 0  & 0 & 0 \\
        0  & 0  & 0 & 0 \\
      \end{pmatrix} \\
      \overset{Z_2 = \frac{1}{3}\qty\big(-3 \cdot Z_2 + 2 \cdot Z_1)}&\leadsto \begin{pmatrix}
        -3 & 0 & 1            & 1            \\
        0  & 1 & -\frac{1}{3} & -\frac{1}{3} \\
        0  & 0 & 0            & 0            \\
        0  & 0 & 0            & 0            \\
      \end{pmatrix} \\
      \overset{Z_1 = -\frac{1}{3} \cdot Z_1 - Z_2}&\leadsto \begin{pmatrix}
        1 & -1 & 0            & 0            \\
        0 & 1  & -\frac{1}{3} & -\frac{1}{3} \\
        0 & 0  & 0            & 0            \\
        0 & 0  & 0            & 0            \\
      \end{pmatrix}
    \end{flalign*}
    Somit sind $x_1 = x_2$ und $x_3 = 3x_2 - x_4$
    \[
      \text{Eig}\qty\big(C, \lambda_{3|4}) =
      \text{Ker}\qty\big(C - \lambda_{3|4} \cdot E_4) =
      \qty{
        \begin{pmatrix}
          x_2  \\
          x_2  \\
          3x_2 \\
          0    \\
        \end{pmatrix}, \begin{pmatrix}
          0    \\
          0    \\
          -x_4 \\
          x_4  \\
        \end{pmatrix}
        \:\middle|\:
        x_3 \in \mathbb{R}
      } = \text{Span}\qty(
        \begin{pmatrix} 1 \\ 1 \\ 3  \\ 0 \end{pmatrix},
        \begin{pmatrix} 0 \\ 0 \\ -1 \\ 1 \end{pmatrix}
      )
    \]

  \end{enumerate}
\end{enumerate}

\newpage
\paragraph{Ü10.4 Eigenschaften von Eigenwerten und Potenzen einer Matrix}
\begin{enumerate}[(a)]
\item Es seien $n \in \mathbb{N} \setminus \qty\big{0}, K$ ein Körper und
  $A \in K^{n \times n}$.
  Zeigen Sie (dafür bietet sich vollständige Induktion an), dass für jedes
  $m \in \mathbb{N}$ und jeden Eigenwert $\lambda$ von $A$ gilt:
  $\lambda^m$ ist ein Eigenwert von $A^m$.

  \subparagraph{Lsg.} Seien $\lambda_1, \ldots, \lambda_t$ die Eigenwerte von
  $A$.
  Sei weiter die Behauptung
  \[
    P\qty\big(m) \colon \lambda_1^m, \ldots, \lambda_t^m
    \text{ sind Eigenwerte von } A^m
  \]

  \textbf{Induktionsanfang:} Es ist $P\qty\big(1)$ trivial.

  \textbf{Alternativ in der Übung:} $P\qty\big(0) \colon A^0 \cdot v
  = E_n \cdot v = v = 1 \cdot v = \lambda^0 \cdot v$.

  Sei nun die Behauptung $P\qty\big(m)$ für ein beliebiges $m \in \mathbb{N}$
  wahr (\textbf{Induktionsvoraussetzung (IV)}).

  \textbf{Induktionsschritt:} Sei $\lambda_s$, $1 \leq s \leq t$ ein beliebiger
  Eigenwert von $A$ und außerdem $v \in \text{Eig}\qty\big(A, \lambda_s)$ ein
  beliebiger Eigenvektor zu $\lambda_s$.

  Nun ist für ein beliebiges $m > 0$
  \[
    P\qty\big(m + 1) \colon A^{m + 1} \cdot v = A \cdot A^m \cdot v
    \overset{\text{(IV)}}= A \cdot \lambda_s^m \cdot v
  \]
  Da $\text{Eig}\qty\big(A, \lambda_s)$ ein Unterraum von $K^n$ ist, ist auch
  $\mu \cdot v \in \text{Eig}\qty\big(A, \lambda_s)$ für jedes $\mu \in K$.
  Insbesondere sind $\lambda_s, \lambda_s^m \in K$ und
  $\lambda_s \cdot v, \lambda_s^m \cdot v \in \text{Eig}\qty\big(A, \lambda_s)$.

  Folglich
  \[
    A \cdot \lambda_s^m \cdot v = \lambda \cdot \lambda_s^m \cdot v
    = \lambda^{m + 1} \cdot v
  \]
  Und aus dem Satz über die vollständige Induktion folgt $P\qty\big(m)$ für alle
  $m \in \mathbb{N}$.

  \textbf{Alternativ nach der Übung:} Es ist
  \[
    A^{m + 1} \cdot v = A \cdot A^m \cdot v
    \overset{\text{Induktionsvoraussetzung}}= A \cdot \lambda^m \cdot v
    \overset{\text{Linear}}= \lambda^m \cdot A \cdot v
    \overset{\text{Eigenvektor}}= \lambda^m \cdot \lambda \cdot v
    = \lambda^{m + 1} \cdot v
  \]

\item Zeigen Sie, dass jeder Eigenvektor von $A$ auch ein Eigenvektor von $A^m$
  ist.

  \subparagraph{Lsg.} Sei $\lambda$ ein beliebiger Eigenwert von $A$.
  Nun wurde in Teilaufgabe (a) bereits gezeigt, dass
  $A^m \cdot v = \lambda^m \cdot v$ und somit ist
  $v \in \text{Eig}\qty\big(A^m, \lambda^m)$.

\newpage
\item In (b) wurde bewiesen, dass
  $\text{Eig}\qty\big(A, \lambda) \subseteq \text{Eig}\qty\big(A^m, \lambda^m)$
  gilt.
  Zeigen Sie, dass die Eigenräume i.A. nicht gleich sind.
  Dazu können Sie z.B. die Eigenräume von $A = \begin{pmatrix}
    0 & 1 \\
    1 & 0 \\
  \end{pmatrix}$ und $m = 2$ betrachten (Sie können den Beweis aber auch anders
  führen).

  \subparagraph{Lsg.} Es ist
  $\det\qty(A - \lambda \cdot E_2) = \qty\big(\lambda + 1)\qty\big(\lambda - 1)$.
  Somit besitzt die Matrix die beiden Eigenwerte $\lambda_1 = -1$ und
  $\lambda_2 = 1$ mit den Eigenräumen
  $\text{Eig}\qty\big(A, \lambda_1) = \text{Span}\qty(\qty\big(1, 1)^T)$ und
  $\text{Eig}\qty\big(A, \lambda_2) = \text{Span}\qty(\qty\big(-1, 1)^T)$.

  Nun ist $A^2 = E_2$ mit dem einzigen Eigenwert 1 und dem Eigenraum
  $\text{Eig}\qty\big(A^2, 1) = \mathbb{R}^2$.
  Dabei gilt wie bereits gezeigt
  $\text{Eig}\qty\big(A, \lambda_1) \subseteq \text{Eig}\qty\big(A^2, \lambda_1^2)$
  und
  $\text{Eig}\qty\big(A, \lambda_2) \subseteq \text{Eig}\qty\big(A^2, \lambda_2^2)$,
  allerdings nicht
  $\text{Eig}\qty\big(A, \lambda_1) \supseteq \text{Eig}\qty\big(A^2, \lambda_1^2)$.
\end{enumerate}

\paragraph{Ü 10.5 Determinante und Eigenwerte}

Es sei $n \in \mathbb{N} \setminus \qty\big{0}, K$ ein Körper und
$A \in K^{n \times n}$.
Beweisen Sie folgende Aussagen:

\fcolorbox{black}{gray!20}{
  \begin{minipage}{\textwidth}
    Im Folgenden wird das charakteristische Polynom von $A$ als
    $\det\qty\big(\lambda \cdot E_n - A)$ definiert.
    Damit ist das Polynom normiert und anders machen die Aufgaben keinen Sinn.
  \end{minipage}
}

\begin{enumerate}[(a)]
\item $A$ ist genau dann invertierbar, wenn $0$ kein Eigenwert von $A$ ist.

  \subparagraph{Lsg.} Angenommen 0 wäre ein Eigenwert von $A$, dann ist $0$ auch
  eine Nullstelle des charakteristischen Polynoms von $A$.

  Nun ist das charakteristische Polynom von $A$ definiert als
  $\det\qty\big(\lambda \cdot E_n - A)$ und für $\lambda = 0$ also
  $\det\qty\big(0 \cdot E_n - A) = \qty\big(-1)^n \det\qty\big(A)$.

  Folglich wäre mit 0 als Nullstelle $\det\qty\big(A) = 0$ und in der Vorlesung
  wurde bereits gezeigt, dass eine Matrix genau dann invertierbar ist, wenn ihre
  Determinante von 0 verschieden ist.

\item Hat $A$ das charakteristische Polynom
  $\chi_A\qty\big(X) = a_nX^n + \ldots + a_1X + a_0$, dann gilt
  $a_0 = \qty\big(-1)^n\det\qty\big(A)$.

  \subparagraph{Lsg.} Das Absolutglied $a_0$ ist gleich
  \[
    \chi_A\qty\big(0) = \det\qty\big(0 \cdot E_n - A)
    = \det\qty\big(-A) = \qty\big(-1)^n \cdot \det\qty\big(A)
  \]

\item Hat die Matrix $A \in K^{n \times n}$ die Eigenwerte
  $\lambda_1, \ldots, \lambda_n$, dann gilt
  $\det\qty\big(A) = \prod_{i = 1}^n \lambda_i$.

  \subparagraph{Lsg.} Jedes Polynom ist durch seine Nullstellen eindeutig
  definiert.
  Im gegebenen Fall wäre
  $\chi_A\qty\big(X) = \prod_{i = 1}^n\qty\big(\lambda_i - X)$.
  Nun ist
  \[
    \det\qty\big(A) = \qty\big(-1)^n \det\qty\big(0 \cdot E_n - A) =
    \qty\big(-1)^n \chi_A\qty\big(0) = \qty\big(-1)^n \prod_{i = 1}^n \lambda_i
  \]
  Außerdem ist $\det\qty\big(-A) = \qty\big(-1)^n \det\qty\big(A)$.

  $\Rightarrow \det\qty\big(A) = \prod_{i = 1}^n \lambda_i$
\end{enumerate}
\end{document}
