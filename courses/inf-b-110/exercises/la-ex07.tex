\documentclass{scrreprt}

\usepackage{aligned-overset}
\usepackage{amsmath}
\usepackage{amsthm}
\usepackage{amssymb}
\usepackage{bm}
\usepackage[shortlabels]{enumitem}
\usepackage{framed}
\usepackage{hyperref}
\usepackage[utf8]{inputenc}
\usepackage{multicol}
\usepackage{mathtools}
\usepackage{physics}
\usepackage{polynom}
\usepackage{tabularx}
\usepackage[table]{xcolor}
\usepackage{titling}
\usepackage{fancyhdr}
\usepackage{xfrac}
\usepackage{pgfplots}

\pgfplotsset{compat = newest}
\usepgfplotslibrary{fillbetween}
\usetikzlibrary{patterns}
\usetikzlibrary{through}


\author{Karsten Lehmann}
\date{WiSe 2024/25}
\title{Übungsblatt 7\\INF-B-110, Lineare Algebra}

\setlength{\headheight}{26pt}
\pagestyle{fancy}
\fancyhf{}
\lhead{\thetitle}
\rhead{\theauthor}
\lfoot{\thedate}
\rfoot{Seite \thepage}

\begin{document}
\paragraph{Ü7.3}
\begin{enumerate}[(a)]
\item \textbf{Lineare Abbildungen}

  Welche der folgenden Abbildungen $f_i$ auf den gegebenen $K$-Vektorräumen sind
  linear?

  \begin{enumerate}[(i)]
  \item $K$, $V$ beliebig, $f_1 \colon V \times V \to V$ mit
    $f_1\qty\big(x_1, x_2) \coloneqq x_1$.

  \item $K = \mathbb{C}$, $f_2 \colon \mathbb{C} \to \mathbb{C}^2$ mit
    $f_2\qty\big(x) \coloneqq \qty\big(x - 2, x + 1)^T$

  \item $K = \mathbb{R}$,
    $f_3 \colon \text{Abb}\qty\big(\mathbb{R}, \mathbb{R}) \to \mathbb{R}$ mit
    $f_3\qty\big(g) \coloneqq g\qty\big(1)$.

  \item $K = \mathbb{R}$, $f_4 \colon \mathbb{R}^3 \to \mathbb{R}^3$ mit
    $f_4 \qty\big(x_1, x_2, x_3) \coloneqq
    \qty\big(2x_1 + 2x_2, 3x_2 - 2x_3, 2x_3 - x_1)^T$
  \end{enumerate}

  \subparagraph{Lsg.}
  \begin{enumerate}[(i)]
  \item Sein $a, b \in V \times V$, $\lambda \in K$ beliebig mit
    $a \coloneqq \qty\big(a_1, a_2)^T$ sowie $b \coloneqq \qty\big(b_1, b_2)^T$.
    Dann ist
    \begin{flalign*}
      f_1 \qty\big(a + \lambda \cdot b)
      &= f_1 \qty(\qty\big(a_1, a_2)^T + \lambda \cdot \qty\big(b_1, b_2)^T) \\
      &= f_1 \qty(\qty\big(a_1, a_2)^T + \qty\big(\lambda b_1, \lambda b_2)^T) \\
      &= f_1 \qty(\qty\big(a_1 + \lambda b_1, a_2 + \lambda b_2)^T) \\
      &= a_1 + \lambda b_1 \\
      &= f_1\qty\big(a) + \lambda f_1\qty\big(b)
    \end{flalign*}
    $\Rightarrow f_1$ ist linear.

  \item Sei $x = 1$, $\lambda = 2$.
    Dann ist
    \begin{flalign*}
      f_2\qty\big(\lambda \cdot x) = f_2\qty\big(2)
      &= \qty\big(2 - 2, 2 + 1)^T = \qty\big(0, 3)^T \\
      & \ne \qty\big(-2, 4)^T = \qty\big(2 \cdot (-2), 2 \cdot 2)^T \\
      &= 2 \cdot \qty\big(1 - 2, 1 + 1)^T \\
      &= 2 \cdot f_2\qty\big(1) = \lambda \cdot f_2\qty\big(x)
    \end{flalign*}
    $\Rightarrow f_2$ ist nicht linear

  \item Sei $g \coloneqq \mathbb{R} \to \mathbb{R}$,
    $h \colon \mathbb{R} \to \mathbb{R}$ und $\lambda \in \mathbb{R}$
    beliebig.

    Seien weiter $x_1 = g\qty\big(1)$ und $x_2 = h\qty\big(1)$.
    Nun ist
    \begin{flalign*}
      f_3\qty\big(g + \lambda \cdot h)
      &= \qty\big(g + \lambda \cdot h)\qty\big(1) \\
      &= g\qty\big(1) + \qty\big(\lambda \cdot h)\qty\big(1) \\
      &= g\qty\big(1) + \lambda h\qty\big(1) \\
      &= x_1 + \lambda x_2 \\
      &= f_3\qty\big(g) + \lambda f_3\qty\big(h)
    \end{flalign*}
    $\Rightarrow f_3$ ist linear.

  \item Seien $a, b \in \mathbb{R}^3$, $\lambda \in \mathbb{R}$ beliebig mit
    $a \coloneqq \qty\big(a_1, a_2, a_3)^T$ sowie
    $b \coloneqq \qty\big(b_1, b_2, b_3)^T$.
    Dann ist
    \begin{flalign*}
      f_4 \qty\big(a + \lambda \cdot b)
      &= f_4 \qty(\qty\big(a_1, a_2, a_3)^T + \lambda \cdot \qty\big(b_1, b_2, b_3)^T) \\
      &= f_4 \qty(\qty\big(a_1, a_2, a_3)^T + \qty\big(\lambda b_1, \lambda b_2, \lambda b_3)^T) \\
      &= f_4 \qty(\qty\big(a_1 + \lambda b_2, a_2 + \lambda b_2, a_3 + \lambda b_3)^T) \\
      &= \begin{pmatrix}
        2 \cdot \qty\big(a_1 + \lambda b_1) + 2 \cdot \qty\big(a_2 + \lambda b_2) \\
        3 \cdot \qty\big(a_2 + \lambda b_2) - 2 \cdot \qty\big(a_3 + \lambda b_3) \\
        2 \cdot \qty\big(a_3 + \lambda b_3) - \qty\big(a_1 + \lambda b_1) \\
      \end{pmatrix}\\
      &= \begin{pmatrix}
        2a_1 + 2 \lambda b_1 + 2a_2 + 2 \lambda b_2 \\
        3 a_2 + 3 \lambda b_2 - 2 a_3 - 2 \lambda b_3 \\
        2 a_3 + 2 \lambda b_3 - a_1 - \lambda b_1 \\
      \end{pmatrix}\\
      &= \begin{pmatrix}
        2a_1 + 2a_2 + 2 \lambda b_1 + 2 \lambda b_2 \\
        3 a_2 - 2 a_3 + 3 \lambda b_2 - 2 \lambda b_3 \\
        2 a_3 - a_1 +  2 \lambda b_3 - \lambda b_1 \\
      \end{pmatrix}\\
      &= \begin{pmatrix}
        2a_1 + 2a_2 \\
        3 a_2 - 2 a_3 \\
        2 a_3 - a_1  \\
      \end{pmatrix} + \begin{pmatrix}
        2 \lambda b_1 + 2 \lambda b_2 \\
        3 \lambda b_2 - 2 \lambda b_3 \\
        2 \lambda b_3 - \lambda b_1
      \end{pmatrix} \\
      &= \begin{pmatrix}
        2a_1 + 2a_2 \\
        3 a_2 - 2 a_3 \\
        2 a_3 - a_1  \\
      \end{pmatrix} + \lambda \cdot \begin{pmatrix}
        2 b_1 + 2 b_2 \\
        3 b_2 - 2 b_3 \\
        2 b_3 - b_1
      \end{pmatrix} \\
      &= f_4\qty\big(a) + \lambda \cdot f_4\qty\big(b)
    \end{flalign*}
    $\Rightarrow f_4$ ist linear.

    \textbf{Alternativ nach der Übung:} $f_4$ lässt sich auch durch die Matrix
      $\begin{pmatrix}
        2  & 2 & 0  \\
        0  & 3 & -1 \\
        -1 & 0 & 2  \\
      \end{pmatrix}$ darstellen und ist damit automatisch linear.
  \end{enumerate}

\item \textbf{Spezielle lineare Abbildungen}

  Es sei $V$ ein $K$-Vektorraum über einem Körper $K$.

  \begin{enumerate}[(i)]
  \item Beweisen Sie: Die identische Abbildung
    $\text{id}_V \colon V \to V, v \mapsto v$ ist linear.

    \subparagraph{Lsg.} Seien $a, b \in V$, $\lambda \in K$ beliebig.
    Dann ist $\text{id}\qty\big(a + \lambda \cdot b) = a + \lambda \cdot b$.
    Außerdem ist auch
    $\text{id}\qty\big(a) + \lambda \cdot \text{id}\qty\big(b)
    = a + \lambda \cdot b$.

  \item Beweisen Sie: Die Nullabbildung
    $c_0 \colon V \to V, v \mapsto 0_V$ ist linear.

    \subparagraph{Lsg.} Seien $a, b \in V$, $\lambda \in K$ beliebig.
    Dann ist $\text{id}\qty\big(a + \lambda \cdot b) = 0_V$.
    Außerdem ist auch
    $\text{id}\qty\big(a) + \lambda \cdot \text{id}\qty\big(b)
    = 0_V + \lambda \cdot 0_V$ und mit den üblichen Rechenregel für $0_V$ ist
    $0_V + \lambda \cdot 0_V = 0_V$.
  \end{enumerate}
\end{enumerate}

\paragraph{7.4 Kern und Bild linearer Abbildungen}\phantom{\null}

Wir untersuchen die durch $f(\begin{pmatrix} x_1 \\ x_2 \\ x_3 \end{pmatrix}) =
\begin{pmatrix}
  1  & 2 & -1 \\
  -3 & 0 & 1  \\
\end{pmatrix} \begin{pmatrix} x_1 \\ x_2 \\ x_3 \end{pmatrix}$ gegebene lineare
Abbildung $f \colon \mathbb{R}^3 \to \mathbb{R}^2$.
\begin{enumerate}[(a)]
\item \label{7_4_a} Bestimmen Sie eine Basis des Bildes $\text{Im}\qty\big(f)$
  und des Kerns $\text{Ker}\qty\big(f)$ von $f$.

  \subparagraph{Lsg.} Es ist
  \begin{flalign*}
    \qty(\begin{array}{ccc|c}
      1  & 2 & -1 & 0 \\
      -3 & 0 & 1  & 0 \\
    \end{array})
    \overset{Z_2 = \frac{1}{6}\qty\big(Z_2 + 3 \cdot Z_1)}&\leadsto
    \qty(\begin{array}{ccc|c}
      1 & 2 & -1           & 0 \\
      0 & 1 & -\frac{1}{3} & 0 \\
    \end{array}) \\
    \overset{Z_1 = Z_1 - 2 \cdot Z_2}&\leadsto
    \qty(\begin{array}{ccc|c}
      1 & 0 & -\frac{1}{3}  & 0 \\
      0 & 1 & -\frac{1}{3} & 0 \\
    \end{array})
  \end{flalign*}
  Somit ist $x_1 = \frac{1}{3}x_3$ sowie $x_2 = \frac{1}{3}x_3$.
  Also ist
  \[
    \text{Kern}\qty\big(f) = \qty{
      \begin{pmatrix}
        \frac{1}{3}x_3 \\
        \frac{1}{3}x_3 \\
        x_3 \\
      \end{pmatrix}
      \:\middle|\:
      x_3 \in \mathbb{R}
    } = \text{Span}\qty{\begin{pmatrix}
        1 \\
        1 \\
        3 \\
      \end{pmatrix}}
  \]
  mit $\qty{\qty\big(1, 1, 3)^T}$ als Basis des Kerns von $f$.

  Nun ist das Bild einer linearen Abbildung der Aufspann der Spalten der
  Abbildungsmatrix.
  Für eine Basis des Bildes müssen somit die linear unabhängigen
  Spaltenvektoren der Abbildungsmatrix bestimmt werden.

  Nun ist
  \begin{flalign*}
    \begin{pmatrix}
      1  & -3 \\
      2  & 0  \\
      -1 & 1  \\
    \end{pmatrix}
    \overset{Z_3 = 2 \cdot Z_3 + Z_2}&\leadsto
    \begin{pmatrix}
      1 & -3 \\
      2 & 0  \\
      0 & 2  \\
    \end{pmatrix} \\
    \overset{Z_1 = 2 \cdot Z_1 - Z_2 + 3 \cdot Z_3}&\leadsto
    \begin{pmatrix}
      0 & 0 \\
      2 & 0 \\
      0 & 2 \\
    \end{pmatrix}
  \end{flalign*}
  $\Rightarrow$ die zweite und dritte Spalte der Abbildungsmatrix
  sind linear unabhängig und
  $\qty(\begin{pmatrix}2 \\ 0 \end{pmatrix},
  \begin{pmatrix} -1 \\ 1 \end{pmatrix})$ ist eine angeordnete Basis des Bildes
  von $f$.

  \textbf{Alternativ nach der Übung:} Man sieht auch schon direkt aus der
  reduzierten Zeilenstufenform für die Berechnung des Kerns, dass die dritte
  Spalte von den ersten beiden linear abhängig ist, mit
  \[
    -\frac{1}{3} \begin{pmatrix} 1 \\ -3 \end{pmatrix}
    - \frac{1}{3} \begin{pmatrix} 2 \\ o \end{pmatrix}
    = \begin{pmatrix} -1 \\ 1 \end{pmatrix}
  \]

\newpage
\item Ist $f$ injektiv, surjektiv oder bijektiv?
  Begründen Sie Ihre Antworten!

  \subparagraph{Lsg.} Es sind
  \begin{itemize}
  \item $f\qty(\qty\big(1, 0, 0)^T) = \qty\big(1, -3)^T$
  \item $f\qty(\qty\big(0, 1, 0)^T) = \qty\big(2, 0)^T$
  \item $f\qty(\qty\big(0, 0, 1)^T) = \qty\big(-1, 1)^T$
  \end{itemize}
  Nun sind die 3 Vektoren nach \hyperref[7_4_a]{Teilaufgabe (a)} linear
  abhängig.
  Somit ist $f$ nicht injektiv.
  Allerdings ist $\qty(\begin{pmatrix}2 \\ 0 \end{pmatrix},
  \begin{pmatrix} -1 \\ 1 \end{pmatrix})$ eine angeordnete Basis von
  $\mathbb{R}^2$ und $f$ somit surjektiv.
\end{enumerate}

\paragraph{Ü 7.5 Äquivalente Bedingungen für die Existenz inverser Matrizen}

\label{7_5}
Es sei $A \in K^{n \times n}$ eine beliebige Matrix über einem Körper $K$.
Zeigen Sie, dass folgende Aussagen äquivalent sind:
\begin{enumerate}[(1)]
\item $A$ ist invertierbar
\item $\text{Kern}\qty\big(A) = \qty{0_{K^n}}$
\item $\text{rg}\qty\big(A) = n$
\end{enumerate}

\subparagraph{Lsg.}
\begin{itemize}
\item[$(1) \Rightarrow (3)$]  Sei $A$ invertierbar.
  Dann existiert eine Matrix $A^{-1} \in K^{n \times n}$ mit
  $A \cdot A^{-1} = E_n$.
  Seien nun $Z_1, \ldots, Z_n$ die Zeilenvektoren von $A$ und
  $S_1, \ldots, S_n$ die Spaltenvektoren von $A^{-1}$.

  Angenommen es wären $Z_1, \ldots, Z_n$ linear abhängig, dann existiert
  $Z_i$, $1 \leq i \leq n$ mit $Z_i = \sum_{j = 1, j \ne i}^n \lambda_j \cdot Z_j$.
  Nun ist $Z_i \cdot S_i = E_{(i,i)} = 1$.
  Da $Z_i$ linear abhängig ist
  \[
    1 = \qty(\sum_{j = 1, j \ne i}^n \lambda_j \cdot Z_j) \cdot S_i
    = \sum_{j = 1, j \ne i}^n \lambda_j \cdot \qty\big(Z_j \cdot S_i)
  \]
  Dafür muss mindestens ein $Z_j \cdot S_i \ne 0$ sein.
  Folglich hat $A \cdot S_i$ mindestens zwei Elemente ungleich Null, ein
  Widerspruch zu $A \cdot A^{-1} = A \cdot \qty\big(S_1, \ldots, S_n) = E_n$,
  da in jeder Spalte nur ein einziges Element ungleich Null ist.

  $\Rightarrow Z_1, \ldots, Z_n$ sind linear unabhängig.

  Somit ``\emph{$A$ ist invertierbar'' $\Rightarrow$} ``\emph{$\text{rg}\qty\big(A) = n$}''

\item[$(3) \Rightarrow (2)$] Sei $A \in K^{n \times n}$ beliebig mit
  $\text{rg}\qty\big(A) = n$.
  Nach der Dimensionsformel ist
  \[
    \text{dim}\qty(\text{Kern}\qty\big(A)) = n - \text{rg}\qty\big(A) = 4 - 4 = 0
  \]
  Also ist $\text{Kern}\qty\big(A)$ ein Untervektorraum von $K^n$ mit der
  Dimension $0$ und tatsächlich hat $K^n$ genau einen solchen Untervektorraum,
  nämlich $\qty{0_{K^n}}$.

\item[$(2) \Rightarrow (1)$] Sei $A \in K^{n \times n}$ beliebig mit
  $\text{Kern}\qty\big(A) = \qty{0_{K^n}}$.

  Aus der Dimensionsformel folgt $\text{rg}\qty\big(A) = n$ und somit sind
  alle Spalten $S_1, \ldots, S_n$ von $A$ linear unabhängig.
  Somit sind $\qty\big(S_1, \ldots S_n)$ eine angeordnete Basis des Vektorraum
  $K^n$ und alle Elemente aus $K^n$ sind durch Linearkombinationen der
  $S_1, \ldots, S_n$ darstellbar, insbesondere die Zeilen $Z_1, \ldots, Z_n$ von
  $E_n$.

  Es ist jedes $Z_i = \sum_{k = 1}^n \lambda_{k, i} \cdot S_k
  = \sum_{k = 1}^n S_k \cdot \lambda_{k, i}$ und somit
  \[
    A \cdot \begin{pmatrix}
      \lambda_{1, i} \\
      \vdots \\
      \lambda_{n, i}
    \end{pmatrix} = Z_i
  \]
  und folglich
  \[
    A \cdot \begin{pmatrix}
      \lambda_{1, 1} & \ldots & \lambda_{1, n}\\
      \vdots & \ddots & \vdots \\
      \lambda_{n, 1} & \ldots & \lambda_{n, n} \\
    \end{pmatrix} = E_n
  \]
  $\Rightarrow A$ ist invertierbar.
\end{itemize}

\paragraph{Ü 7.6 Matrixinvertierung mit dem Gauß-Verfahren}

\begin{enumerate}[(a)]
\item Sind die folgenden Matrizen über $\mathbb{C}$ bzw. $\mathbb{R}$
  invertierbar?
  Geben Sie die inverse Matrix an, falls möglich.
  \[
    A_1 = \begin{pmatrix}
      1     & i - 1 \\
      i + 1 & 1     \\
    \end{pmatrix}, \quad
    A_2 = \begin{pmatrix}
      1 & 3  & 4 \\
      2 & -1 & 5 \\
      1 & 4  & 3 \\
    \end{pmatrix}, \quad
    A_3 = \begin{pmatrix}
      6 & 3 & 4 & 5 \\
      1 & 2 & 2 & 1 \\
      2 & 4 & 3 & 2 \\
      3 & 3 & 4 & 2 \\
    \end{pmatrix}, \quad
    A_4 = \begin{pmatrix}
      0 & 0 & 0 & 1 \\
      0 & 0 & 1 & 0 \\
      0 & 1 & 0 & 0 \\
      1 & 0 & 0 & 0 \\
    \end{pmatrix}
  \]

  \subparagraph{Lsg.} Es ist
  \begin{flalign*}
    \qty(\begin{array}{cc|cc}
      1     & i - 1 & 1 & 0 \\
      i + 1 & 1     & 0 & 1 \\
    \end{array})
    \overset{Z_2 = Z_2 - \qty\big(i + 1) \cdot Z_1}&\leadsto
    \qty(\begin{array}{cc|cc}
      1 & i - 1 & 1      & 0 \\
      0 & 3     & -1 - i & 1 \\
    \end{array}) \\
    \overset{Z_2 = \frac{1}{3}Z_2}&\leadsto
    \qty(\begin{array}{cc|cc}
      1 & i - 1 & 1                          & 0           \\
      0 & 1     & -\frac{1}{3} - \frac{i}{3} & \frac{1}{3} \\
    \end{array}) \\
    \overset{Z_1 = Z_1 - \qty\big(i - 1) \cdot Z_2}&\leadsto
    \qty(\begin{array}{cc|cc}
      1 & 0 & \frac{1}{3}               & \frac{1}{3} - \frac{i}{3} \\
      0 & 1 & \frac{1}{3} - \frac{i}{3} & \frac{1}{3}               \\
    \end{array})
  \end{flalign*}
  Somit ist
  \[
    A_1^{-1} = \begin{pmatrix}
      \frac{1}{3}               & \frac{1}{3} - \frac{i}{3} \\
      \frac{1}{3} - \frac{i}{3} & \frac{1}{3}               \\
    \end{pmatrix}
  \]

  \newpage
  Weiter ist
  \begin{flalign*}
    \qty(\begin{array}{ccc|ccc}
      1 & 3  & 4 & 1 & 0 & 0 \\
      2 & -1 & 5 & 0 & 1 & 0 \\
      1 & 4  & 3 & 0 & 0 & 1 \\
    \end{array})
    \overset{Z_2 = Z_2 - 2 \cdot Z_1}&\leadsto
    \qty(\begin{array}{ccc|ccc}
      1 & 3  & 4  & 1  & 0 & 0 \\
      0 & -7 & -3 & -2 & 1 & 0 \\
      1 & 4  & 3  & 0  & 0 & 1 \\
    \end{array}) \\
    \overset{Z_3 = Z_3 - Z_1}&\leadsto
    \qty(\begin{array}{ccc|ccc}
      1 & 3  & 4  & 1  & 0 & 0 \\
      0 & -7 & -3 & -2 & 1 & 0 \\
      0 & 1  & -1 & -1 & 0 & 1 \\
    \end{array}) \\
    \overset{Z_3 = 7 \cdot Z_3 + Z_2}&\leadsto
    \qty(\begin{array}{ccc|ccc}
      1 & 3  & 4   & 1  & 0 & 0 \\
      0 & -7 & -3  & -2 & 1 & 0 \\
      0 & 0  & -10 & -9 & 1 & 7 \\
    \end{array}) \\
    \overset{Z_2 = -\frac{1}{7}\qty\big(10 \cdot Z_2 - 3 Z_3)}&\leadsto
    \qty(\begin{array}{ccc|ccc}
      1 & 3  & 4   & 1  & 0  & 0 \\
      0 & 10 & 0   & -1 & -1 & 3 \\
      0 & 0  & -10 & -9 & 1  & 7 \\
    \end{array}) \\
    \overset{Z_1 = 10 \cdot Z_1 - 3 Z_2}&\leadsto
    \qty(\begin{array}{ccc|ccc}
      10 & 0  & 40  & 13 & 3  & -9 \\
      0  & 10 & 0   & -1 & -1 & 3  \\
      0  & 0  & -10 & -9 & 1  & 7  \\
    \end{array}) \\
    \overset{Z_1 = \frac{1}{10}\qty\big(Z_1 + 4 Z_3)}&\leadsto
    \qty(\begin{array}{ccc|ccc}
      1 & 0  & 0   & -\frac{23}{10} & \frac{7}{10} & \frac{19}{10} \\
      0 & 10 & 0   & -1             & -1           & 3             \\
      0 & 0  & -10 & -9             & 1            & 7             \\
    \end{array}) \\
    \overset{Z_2 = \frac{1}{10}Z_2}&\leadsto
    \qty(\begin{array}{ccc|ccc}
      1 & 0 & 0   & -\frac{23}{10} & \frac{7}{10}  & \frac{19}{10} \\
      0 & 1 & 0   & -\frac{1}{10}  & -\frac{1}{10} & \frac{3}{10}  \\
      0 & 0 & -10 & -9             & 1             & 7             \\
    \end{array}) \\
    \overset{Z_3 = -\frac{1}{10}Z_3}&\leadsto
    \qty(\begin{array}{ccc|ccc}
      1 & 0 & 0 & -\frac{23}{10} & \frac{7}{10}  & \frac{19}{10} \\
      0 & 1 & 0 & -\frac{1}{10}  & -\frac{1}{10} & \frac{3}{10}  \\
      0 & 0 & 1 & \frac{9}{10}   & -\frac{1}{10} & -\frac{7}{10} \\
    \end{array}) \\
  \end{flalign*}
  Somit ist
  \[
    A_2^{-1} = \begin{pmatrix}
      -\frac{23}{10} & \frac{7}{10}  & \frac{19}{10} \\
      -\frac{1}{10}  & -\frac{1}{10} & \frac{3}{10}  \\
      \frac{9}{10}   & -\frac{1}{10} & -\frac{7}{10} \\
    \end{pmatrix}
  \]

  \newpage
  Für die Matrix $A_3$:
  \begin{flalign*}
    \qty(\begin{array}{cccc|cccc}
      6 & 3 & 4 & 5 & 1 & 0 & 0 & 0 \\
      1 & 2 & 2 & 1 & 0 & 1 & 0 & 0 \\
      2 & 4 & 3 & 2 & 0 & 0 & 1 & 0 \\
      3 & 3 & 4 & 2 & 0 & 0 & 0 & 1 \\
    \end{array})
    \overset{Z_2 = 2 \cdot Z_2 - \cdot Z_3}&\leadsto
    \qty(\begin{array}{cccc|cccc}
      6 & 3 & 4 & 5 & 1 & 0 & 0  & 0 \\
      0 & 0 & 1 & 0 & 0 & 2 & -1 & 0 \\
      2 & 4 & 3 & 2 & 0 & 0 & 1  & 0 \\
      3 & 3 & 4 & 2 & 0 & 0 & 0  & 1 \\
    \end{array}) \\
    \overset{Z_2 \leftrightarrow Z_3}&\leadsto
    \qty(\begin{array}{cccc|cccc}
      6 & 3 & 4 & 5 & 1 & 0 & 0  & 0 \\
      2 & 4 & 3 & 2 & 0 & 0 & 1  & 0 \\
      0 & 0 & 1 & 0 & 0 & 2 & -1 & 0 \\
      3 & 3 & 4 & 2 & 0 & 0 & 0  & 1 \\
    \end{array}) \\
    \overset{Z_2 = 3 \cdot Z_2 - Z_1}&\leadsto
    \qty(\begin{array}{cccc|cccc}
      6 & 3 & 4 & 5 & 1  & 0 & 0  & 0 \\
      0 & 9 & 5 & 1 & -1 & 0 & 3  & 0 \\
      0 & 0 & 1 & 0 & 0  & 2 & -1 & 0 \\
      3 & 3 & 4 & 2 & 0  & 0 & 0  & 1 \\
    \end{array}) \\
    \overset{Z_4 = 2 \cdot Z_4 - Z_1}&\leadsto
    \qty(\begin{array}{cccc|cccc}
      6 & 3 & 4 & 5  & 1  & 0 & 0  & 0 \\
      0 & 9 & 5 & 1  & -1 & 0 & 3  & 0 \\
      0 & 0 & 1 & 0  & 0  & 2 & -1 & 0 \\
      0 & 3 & 4 & -1 & -1 & 0 & 0  & 2 \\
    \end{array}) \\
    \overset{Z_4 = 3 \cdot Z_4 - Z_2}&\leadsto
    \qty(\begin{array}{cccc|cccc}
      6 & 3 & 4 & 5  & 1  & 0 & 0  & 0 \\
      0 & 9 & 5 & 1  & -1 & 0 & 3  & 0 \\
      0 & 0 & 1 & 0  & 0  & 2 & -1 & 0 \\
      0 & 0 & 7 & -4 & -2 & 0 & -3 & 6 \\
    \end{array}) \\
    \overset{Z_4 = Z_4 - 7 \cdot Z_3}&\leadsto
    \qty(\begin{array}{cccc|cccc}
      6 & 3 & 4 & 5  & 1  & 0   & 0  & 0 \\
      0 & 9 & 5 & 1  & -1 & 0   & 3  & 0 \\
      0 & 0 & 1 & 0  & 0  & 2   & -1 & 0 \\
      0 & 0 & 0 & -4 & -2 & -14 & 4  & 6 \\
    \end{array}) \\
    \overset{Z_4 = -\frac{1}{4}Z_4}&\leadsto
    \qty(\begin{array}{cccc|cccc}
      6 & 3 & 4 & 5 & 1           & 0           & 0  & 0            \\
      0 & 9 & 5 & 1 & -1          & 0           & 3  & 0            \\
      0 & 0 & 1 & 0 & 0           & 2           & -1 & 0            \\
      0 & 0 & 0 & 1 & \frac{1}{2} & \frac{7}{2} & -1 & -\frac{3}{2} \\
    \end{array}) \\
    \overset{Z_2 = Z_2 - 5 \cdot Z_3}&\leadsto
    \qty(\begin{array}{cccc|cccc}
      6 & 3 & 4 & 5 & 1           & 0           & 0  & 0            \\
      0 & 9 & 0 & 1 & -1          & -10         & 8  & 0            \\
      0 & 0 & 1 & 0 & 0           & 2           & -1 & 0            \\
      0 & 0 & 0 & 1 & \frac{1}{2} & \frac{7}{2} & -1 & -\frac{3}{2} \\
    \end{array}) \\
    \overset{Z_2 = Z_2 - Z_4}&\leadsto
    \qty(\begin{array}{cccc|cccc}
      6 & 3 & 4 & 5 & 1            & 0             & 0  & 0            \\
      0 & 9 & 0 & 0 & -\frac{3}{2} & -\frac{27}{2} & 9  & \frac{3}{2}  \\
      0 & 0 & 1 & 0 & 0            & 2             & -1 & 0            \\
      0 & 0 & 0 & 1 & \frac{1}{2}  & \frac{7}{2}   & -1 & -\frac{3}{2} \\
    \end{array}) \\
    \overset{Z_2 = \frac{1}{9} \cdot Z_2}&\leadsto
    \qty(\begin{array}{cccc|cccc}
      6 & 3 & 4 & 5 & 1            & 0            & 0  & 0            \\
      0 & 1 & 0 & 0 & -\frac{1}{6} & -\frac{3}{2} & 1  & \frac{1}{6}  \\
      0 & 0 & 1 & 0 & 0            & 2            & -1 & 0            \\
      0 & 0 & 0 & 1 & \frac{1}{2}  & \frac{7}{2}  & -1 & -\frac{3}{2} \\
    \end{array}) \\
  \end{flalign*}
  \begin{flalign*}
    \qty(\begin{array}{cccc|cccc}
      6 & 3 & 4 & 5 & 1            & 0            & 0  & 0            \\
      0 & 1 & 0 & 0 & -\frac{1}{6} & -\frac{3}{2} & 1  & \frac{1}{6}  \\
      0 & 0 & 1 & 0 & 0            & 2            & -1 & 0            \\
      0 & 0 & 0 & 1 & \frac{1}{2}  & \frac{7}{2}  & -1 & -\frac{3}{2} \\
    \end{array})
    \overset{Z_1 = Z_1 - 3 \cdot Z_2}&\leadsto
    \qty(\begin{array}{cccc|cccc}
      6 & 0 & 4 & 5 & \frac{3}{2}  & \frac{9}{2}  & -3 & -\frac{1}{2} \\
      0 & 1 & 0 & 0 & -\frac{1}{6} & -\frac{3}{2} & 1  & \frac{1}{6}  \\
      0 & 0 & 1 & 0 & 0            & 2            & -1 & 0            \\
      0 & 0 & 0 & 1 & \frac{1}{2}  & \frac{7}{2}  & -1 & -\frac{3}{2} \\
    \end{array}) \\
    \overset{Z_1 = Z_1 - 4 \cdot Z_3}&\leadsto
    \qty(\begin{array}{cccc|cccc}
      6 & 0 & 0 & 5 & \frac{3}{2}  & -\frac{7}{2}  & 1  & -\frac{1}{2} \\
      0 & 1 & 0 & 0 & -\frac{1}{6} & -\frac{3}{2} & 1  & \frac{1}{6}  \\
      0 & 0 & 1 & 0 & 0            & 2            & -1 & 0            \\
      0 & 0 & 0 & 1 & \frac{1}{2}  & \frac{7}{2}  & -1 & -\frac{3}{2} \\
    \end{array}) \\
    \overset{Z_1 = Z_1 - 5 \cdot Z_4}&\leadsto
    \qty(\begin{array}{cccc|cccc}
      6 & 0 & 0 & 0 & -1           & -21          & 6  & 7            \\
      0 & 1 & 0 & 0 & -\frac{1}{6} & -\frac{3}{2} & 1  & \frac{1}{6}  \\
      0 & 0 & 1 & 0 & 0            & 2            & -1 & 0            \\
      0 & 0 & 0 & 1 & \frac{1}{2}  & \frac{7}{2}  & -1 & -\frac{3}{2} \\
    \end{array}) \\
    \overset{Z_1 = \frac{1}{6} \cdot Z_1}&\leadsto
    \qty(\begin{array}{cccc|cccc}
      1 & 0 & 0 & 0 & -\frac{1}{6} & -\frac{7}{2} & 1  & \frac{7}{6}  \\
      0 & 1 & 0 & 0 & -\frac{1}{6} & -\frac{3}{2} & 1  & \frac{1}{6}  \\
      0 & 0 & 1 & 0 & 0            & 2            & -1 & 0            \\
      0 & 0 & 0 & 1 & \frac{1}{2}  & \frac{7}{2}  & -1 & -\frac{3}{2} \\
    \end{array})
  \end{flalign*}
  Somit ist
  \[
    A_3^{-1} = \begin{pmatrix}
      -\frac{1}{6} & -\frac{7}{2} & 1  & \frac{7}{6}  \\
      -\frac{1}{6} & -\frac{3}{2} & 1  & \frac{1}{6}  \\
      0            & 2            & -1 & 0            \\
      \frac{1}{2}  & \frac{7}{2}  & -1 & -\frac{3}{2} \\
    \end{pmatrix}
  \]

  Und für die Matrix $A_4$:
  \begin{flalign*}
    \qty(\begin{array}{cccc|cccc}
      0 & 0 & 0 & 1 & 1 & 0 & 0 & 0 \\
      0 & 0 & 1 & 0 & 0 & 1 & 0 & 0 \\
      0 & 1 & 0 & 0 & 0 & 0 & 1 & 0 \\
      1 & 0 & 0 & 0 & 0 & 0 & 0 & 1 \\
    \end{array})
    \overset{Z_1 \leftrightarrow Z_4}&\leadsto
    \qty(\begin{array}{cccc|cccc}
      1 & 0 & 0 & 0 & 0 & 0 & 0 & 1 \\
      0 & 0 & 1 & 0 & 0 & 1 & 0 & 0 \\
      0 & 1 & 0 & 0 & 0 & 0 & 1 & 0 \\
      0 & 0 & 0 & 1 & 1 & 0 & 0 & 0 \\
    \end{array}) \\
    \overset{Z_2 \leftrightarrow Z_3}&\leadsto
    \qty(\begin{array}{cccc|cccc}
      1 & 0 & 0 & 0 & 0 & 0 & 0 & 1 \\
      0 & 1 & 0 & 0 & 0 & 0 & 1 & 0 \\
      0 & 0 & 1 & 0 & 0 & 1 & 0 & 0 \\
      0 & 0 & 0 & 1 & 1 & 0 & 0 & 0 \\
    \end{array})
  \end{flalign*}
  Somit ist $A_4^{-1} = A_4$.

\newpage
\item Für welche Werte $a, b \in \mathbb{R}$ ist die Matrix $C = \begin{pmatrix}
  a & b  & 1 \\
  0 & a  & 1 \\
  1 & -1 & 0 \\
\end{pmatrix} \in \mathbb{R}^{3 \times 3}$ invertierbar?
Geben Sie die inverse Matrix an.

\subparagraph{Lsg.} Nach \hyperref[7_5]{Ü 7.5} ist eine Matrix genau dann
invertierbar, falls sie vollen Rang hat.
Somit ist $C$ invertierbar, falls die 3 Zeilen linear unabhängig sind.

Nun ist
\begin{flalign*}
  \qty(\begin{array}{ccc|ccc}
    a & b  & 1 & 1 & 0 & 0 \\
    0 & a  & 1 & 0 & 1 & 0 \\
    1 & -1 & 0 & 0 & 0 & 1 \\
  \end{array})
  \overset{Z_1 \leftrightarrow Z_3}&\leadsto
  \qty(\begin{array}{ccc|ccc}
    1 & -1 & 0 & 0 & 0 & 1 \\
    0 & a  & 1 & 0 & 1 & 0 \\
    a & b  & 1 & 1 & 0 & 0 \\
  \end{array}) \\
  \overset{Z_3 = Z_3 - a \cdot Z_1}&\leadsto
  \qty(\begin{array}{ccc|ccc}
    1 & -1    & 0 & 0 & 0 & 1  \\
    0 & a     & 1 & 0 & 1 & 0  \\
    0 & b + a & 1 & 1 & 0 & -a \\
  \end{array}) \\
  \overset{Z_3 = Z_3 - Z_2}&\leadsto
  \qty(\begin{array}{ccc|ccc}
    1 & -1 & 0 & 0 & 0  & 1  \\
    0 & a  & 1 & 0 & 1  & 0  \\
    0 & b  & 0 & 1 & -1 & -a \\
  \end{array}) \\
  \overset{Z_2 \leftrightarrow Z_3}&\leadsto
  \qty(\begin{array}{ccc|ccc}
    1 & -1 & 0 & 0 & 0  & 1  \\
    0 & b  & 0 & 1 & -1 & -a \\
    0 & a  & 1 & 0 & 1  & 0  \\
  \end{array}) \\
  \overset{Z_3 = b \cdot Z_3 - a \cdot Z_2}&\leadsto
  \qty(\begin{array}{ccc|ccc}
    1 & -1 & 0 & 0  & 0     & 1   \\
    0 & b  & 0 & 1  & -1    & -a  \\
    0 & 0  & b & -a & b + a & a^2 \\
  \end{array}) \\
  \overset{Z_1 = b \cdot Z_1 + Z_2}&\leadsto
  \qty(\begin{array}{ccc|ccc}
    b & 0 & 0 & 1  & -1    & b - a \\
    0 & b & 0 & 1  & -1    & -a    \\
    0 & 0 & b & -a & b + a & a^2   \\
  \end{array})
\end{flalign*}
Somit hat die Matrix vollen Rang und ist invertierbar, solange $b \ne 0$ und
die inverse Matrix ist
\[
  C^{-1} = \frac{1}{b} \cdot \begin{pmatrix}
    1  & -1    & b - a \\
    1  & -1    & -a    \\
    -a & b + a & a^2   \\
  \end{pmatrix}
\]
\end{enumerate}
\end{document}
