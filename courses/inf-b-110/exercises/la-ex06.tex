\documentclass{scrreprt}

\usepackage{aligned-overset}
\usepackage{amsmath}
\usepackage{amsthm}
\usepackage{amssymb}
\usepackage{bm}
\usepackage[shortlabels]{enumitem}
\usepackage{framed}
\usepackage{hyperref}
\usepackage[utf8]{inputenc}
\usepackage{multicol}
\usepackage{mathtools}
\usepackage{physics}
\usepackage{polynom}
\usepackage{tabularx}
\usepackage[table]{xcolor}
\usepackage{titling}
\usepackage{fancyhdr}
\usepackage{xfrac}
\usepackage{pgfplots}

\pgfplotsset{compat = newest}
\usepgfplotslibrary{fillbetween}
\usetikzlibrary{patterns}
\usetikzlibrary{through}


\author{Karsten Lehmann}
\date{WiSe 2024/25}
\title{Übungsblatt 6\\INF-B-110, Lineare Algebra}

\setlength{\headheight}{26pt}
\pagestyle{fancy}
\fancyhf{}
\lhead{\thetitle}
\rhead{\theauthor}
\lfoot{\thedate}
\rfoot{Seite \thepage}

\begin{document}
\paragraph{Ü6.3 Homogene und inhomogene Gleichungssysteme}
\begin{enumerate}[(a)]
\item Bestimmen Sie zur unten stehenden Matrix $A \in \mathbb{R}^{4 \times 5}$
  die Lösungsmenge $L^*$ des zugehörigen homogenen Gleichungssystems.
  \[
    A \coloneqq \begin{pmatrix}
      1 & 2 & 3  & 4 & 5  \\
      0 & 0 & 1  & 1 & 4  \\
      0 & 0 & -1 & 1 & -2 \\
      0 & 0 & 0  & 2 & 2  \\
    \end{pmatrix}
  \]

  \subparagraph{Lsg.} Es ist
  \begin{flalign*}
    \qty(\begin{array}{ccccc|c}
      1 & 2 & 3  & 4 & 5  & 0 \\
      0 & 0 & 1  & 1 & 4  & 0 \\
      0 & 0 & -1 & 1 & -2 & 0 \\
      0 & 0 & 0  & 2 & 2  & 0 \\
    \end{array})
    \overset{Z_3 = Z_3 + Z_2}&\leadsto
    \qty(\begin{array}{ccccc|c}
      1 & 2 & 3 & 4 & 5 & 0 \\
      0 & 0 & 1 & 1 & 4 & 0 \\
      0 & 0 & 0 & 2 & 2 & 0 \\
      0 & 0 & 0 & 2 & 2 & 0 \\
    \end{array}) \\
    \overset{Z_4 = Z_4 - Z_3}&\leadsto
    \qty(\begin{array}{ccccc|c}
      1 & 2 & 3 & 4 & 5 & 0 \\
      0 & 0 & 1 & 1 & 4 & 0 \\
      0 & 0 & 0 & 2 & 2 & 0 \\
      0 & 0 & 0 & 0 & 0 & 0 \\
    \end{array}) \\
    \overset{Z_3 = \frac{1}{2}Z_3}&\leadsto
    \qty(\begin{array}{ccccc|c}
      1 & 2 & 3 & 4 & 5 & 0 \\
      0 & 0 & 1 & 1 & 4 & 0 \\
      0 & 0 & 0 & 1 & 1 & 0 \\
      0 & 0 & 0 & 0 & 0 & 0 \\
    \end{array}) \\
    \overset{Z_3 = Z_3 - Z_4}&\leadsto
    \qty(\begin{array}{ccccc|c}
      1 & 2 & 3 & 4 & 5 & 0 \\
      0 & 0 & 1 & 0 & 3 & 0 \\
      0 & 0 & 0 & 1 & 1 & 0 \\
      0 & 0 & 0 & 0 & 0 & 0 \\
    \end{array}) \\
    \overset{Z_1 = Z_1 - 3 \cdot Z_2 - 4 \cdot Z_3}&\leadsto
    \qty(\begin{array}{ccccc|c}
      1 & 2 & 0 & 0 & -8 & 0 \\
      0 & 0 & 1 & 0 & 3  & 0 \\
      0 & 0 & 0 & 1 & 1  & 0 \\
      0 & 0 & 0 & 0 & 0  & 0 \\
    \end{array})
  \end{flalign*}
  Nun ist $x_1 = 8x_5 - 2x_2$, $x_3 = -3x_5$ und  $x_4 = -x_5$
  \begin{flalign*}
    \text{Kern}\qty\big(A) = L^* &= \qty{
      \begin{pmatrix}
        8x_5 - 2x_2 \\
        x_2 \\
        -3x_5 \\
        -x_5 \\
        x_5 \\
      \end{pmatrix}
      \:\middle|\:
      x_2, x_5 \in \mathbb{R}
    } = \qty{
      \begin{pmatrix}
        - 2x_2 \\
        x_2 \\
        0 \\
        0 \\
        0 \\
      \end{pmatrix} + \begin{pmatrix}
        8x_5 \\
        0 \\
        -3x_5 \\
        -x_5 \\
        x_5 \\
      \end{pmatrix}
      \:\middle|\:
      x_2, x_5 \in \mathbb{R}
    } \\
    &= \text{Span}\qty(\qty{
      \begin{pmatrix} -2 \\ 1 \\ 0 \\ 0 \\ 0 \end{pmatrix},
      \begin{pmatrix} 8 \\ 0 \\ -3 \\ -1 \\ 1 \end{pmatrix}
    })
  \end{flalign*}

\newpage
\item Bestimmen Sie des Lösungsmenge $L$ des Systems für
  $b \coloneqq \qty\big(5, 1, 1, 1)^T$.

  \subparagraph{Lsg.} Es ist
  \begin{flalign*}
    \qty(\begin{array}{ccccc|c}
      1 & 2 & 3  & 4 & 5  & 5 \\
      0 & 0 & 1  & 1 & 4  & 1 \\
      0 & 0 & -1 & 1 & -2 & 1 \\
      0 & 0 & 0  & 2 & 2  & 1 \\
    \end{array})
    \overset{Z_3 = Z_3 + Z_2}&\leadsto
    \qty(\begin{array}{ccccc|c}
      1 & 2 & 3 & 4 & 5 & 5 \\
      0 & 0 & 1 & 1 & 4 & 1 \\
      0 & 0 & 0 & \cellcolor{red!20} 2 & \cellcolor{red!20} 2 & \cellcolor{red!20} 2 \\
      0 & 0 & 0 & \cellcolor{red!20} 2 & \cellcolor{red!20} 2 & \cellcolor{red!20} 1 \\
    \end{array})
  \end{flalign*}
  $\Rightarrow$ ein Widerspruch, $L = \emptyset$

\item Bestimmen Sie die Lösungsmenge $L$ des Systems für
  $c \coloneqq \qty\big(5, -1, 3, r)^T$ in Abhängigkeit von dem Parameter
  $r \in \mathbb{R}$.

  \subparagraph{Lsg.} Es ist
  \begin{flalign*}
    \qty(\begin{array}{ccccc|c}
      1 & 2 & 3  & 4 & 5  & 5  \\
      0 & 0 & 1  & 1 & 4  & -1 \\
      0 & 0 & -1 & 1 & -2 & 3  \\
      0 & 0 & 0  & 2 & 2  & r  \\
    \end{array})
    \overset{Z_3 = Z_3 + Z_2}&\leadsto
    \qty(\begin{array}{ccccc|c}
      1 & 2 & 3 & 4 & 5 & 5  \\
      0 & 0 & 1 & 1 & 4 & -1 \\
      0 & 0 & 0 & 2 & 2 & 2  \\
      0 & 0 & 0 & 2 & 2 & r  \\
    \end{array})
  \end{flalign*}
  Nun entsteht für $r \ne 2$ ein Widerspruch und das LGS hätte eine leere
  Lösungsmenge.

  Sei im Folgenden $r = 2$:
  \begin{flalign*}
    \qty(\begin{array}{ccccc|c}
      1 & 2 & 3 & 4 & 5 & 5  \\
      0 & 0 & 1 & 1 & 4 & -1 \\
      0 & 0 & 0 & 2 & 2 & 2  \\
      0 & 0 & 0 & 2 & 2 & 2  \\
    \end{array})
    \overset{Z_4 = Z_4 - Z_3}&\leadsto
    \qty(\begin{array}{ccccc|c}
      1 & 2 & 3 & 4 & 5 & 5  \\
      0 & 0 & 1 & 1 & 4 & -1 \\
      0 & 0 & 0 & 2 & 2 & 2  \\
      0 & 0 & 0 & 0 & 0 & 0  \\
    \end{array}) \\
    \overset{Z_3 = \frac{1}{2}Z_3}&\leadsto
    \qty(\begin{array}{ccccc|c}
      1 & 2 & 3 & 4 & 5 & 5  \\
      0 & 0 & 1 & 1 & 4 & -1 \\
      0 & 0 & 0 & 1 & 1 & 1  \\
      0 & 0 & 0 & 0 & 0 & 0  \\
    \end{array}) \\
    \overset{Z_2 = Z_2 - Z_3}&\leadsto
    \qty(\begin{array}{ccccc|c}
      1 & 2 & 3 & 4 & 5 & 5  \\
      0 & 0 & 1 & 0 & 3 & -2 \\
      0 & 0 & 0 & 1 & 1 & 1  \\
      0 & 0 & 0 & 0 & 0 & 0  \\
    \end{array}) \\
    \overset{Z_1 = Z_1 - 3 \cdot Z_2 - 4 \cdot Z_3}&\leadsto
    \qty(\begin{array}{ccccc|c}
      1 & 2 & 0 & 0 & -8 & 7 \\
      0 & 0 & 1 & 0 & 3  & -2 \\
      0 & 0 & 0 & 1 & 1  & 1  \\
      0 & 0 & 0 & 0 & 0  & 0  \\
    \end{array})
  \end{flalign*}
  Nun ist $x_1 = 7 - 2x_2 + 8x_5$, $x_3 = -2 - 3x_5$ und $x_4 = 1 - x_5$ und
  damit
  \begin{flalign*}
    L &= \qty{\begin{pmatrix}
      7 - 2 \lambda + 8 \mu \\
      \lambda \\
      -2 - 3 \mu \\
      1 - \mu \\
      \mu \\
    \end{pmatrix} \:\middle|\: \lambda, \mu \in \mathbb{R}} \\
    &= \qty{\begin{pmatrix}
      7  \\
      0  \\
      -2 \\
      1  \\
      0  \\
    \end{pmatrix} + \begin{pmatrix}
      -2\lambda + 8\mu \\
      \lambda \\
      -3\mu \\
      -\mu \\
      \mu \\
    \end{pmatrix} \:\middle|\: \lambda, \mu \in \mathbb{R}} \\
    &= \qty{\begin{pmatrix}
      7  \\
      0  \\
      -2 \\
      1  \\
      0  \\
    \end{pmatrix} + \begin{pmatrix}
      -2\lambda \\
      \lambda \\
      0 \\
      0 \\
      0 \\
    \end{pmatrix}  + \begin{pmatrix}
      8 \mu \\
      0 \\
      -3 \mu \\
      -\mu \\
      \mu \\
    \end{pmatrix} \:\middle|\: \lambda, \mu \in \mathbb{R}} \\
    &= \qty{\begin{pmatrix}
      7  \\
      0  \\
      -2 \\
      1  \\
      0  \\
    \end{pmatrix} + \lambda \cdot \begin{pmatrix}
      -2 \\
      1  \\
      0  \\
      0  \\
      0  \\
    \end{pmatrix} + \mu \cdot \begin{pmatrix}
      8  \\
      0  \\
      -3 \\
      -1 \\
      1  \\
    \end{pmatrix}
    \:\middle|\:
    \lambda, \mu \in \mathbb{R}}
  \end{flalign*}

\item Wie hängen die Lösungsmengen von homogenen und inhomogenen Gleichungssystem
  zusammen?

  \subparagraph{Lsg.} Es ist $L = L^* + \qty\big(7, 0, -2, 1, 0)^T$.
  Das heißt, die beiden Lösungsmengen sind zueinander parallele Ebenen.
\end{enumerate}

\newpage
\paragraph{Ü 6.4 Rang und Kern einer Matrix}
\begin{enumerate}[(a)]
\item Bestimmen Sie für die Matrizen $A, B \in \mathbb{R}^{4 \times 4}$ den Rang
  und den Kern:
  \[
    A \coloneqq \begin{pmatrix}
      5 & 2 & -7 & 1 \\
      0 & 1 & -4 & 0 \\
      0 & 0 & 1  & 1 \\
      0 & 0 & 2  & 2 \\
    \end{pmatrix}
    \qquad
    B \coloneqq \begin{pmatrix}
      1 & 1 & 1 & 1 \\
      1 & 1 & 1 & 1 \\
      1 & 1 & 1 & 1 \\
      1 & 1 & 1 & 1 \\
    \end{pmatrix}
  \]
  Geben Sie jeweils eine Basis des Kerns an und stellen Sie sämtliche Elemente
  des Kerns als Linearkombination der Basisvektoren dar.
  Bestimmen Sie die Dimension des Spaltenraums $\text{Col}\qty\big(A)$.
  Liegt der Vektor $a \coloneqq \qty\big(0, 2, -5, -10)^T$ im Spaltenraum
  $\text{Col}\qty\big(A)$?
  Liegt der Vektor $b \coloneqq \qty\big(2, -2, 2, -2)^T$ im Spaltenraum
  $\text{Col}\qty\big(B)$?

  \subparagraph{Lsg.} Es ist
  \begin{flalign*}
    \begin{pmatrix}
      5 & 2 & -7 & 1 \\
      0 & 1 & -4 & 0 \\
      0 & 0 & 1  & 1 \\
      0 & 0 & 2  & 2 \\
    \end{pmatrix}
    \overset{Z_4 = Z_4 - 2 \cdot Z_3}&\leadsto
    \begin{pmatrix}
      5 & 2 & -7 & 1 \\
      0 & 1 & -4 & 0 \\
      0 & 0 & 1  & 1 \\
      0 & 0 & 0  & 0 \\
    \end{pmatrix} \\
    \overset{Z_2 = Z_2 + 4 \cdot Z_3}&\leadsto
    \begin{pmatrix}
      5 & 2 & -7 & 1 \\
      0 & 1 & 0  & 4 \\
      0 & 0 & 1  & 1 \\
      0 & 0 & 0  & 0 \\
    \end{pmatrix} \\
    \overset{Z_1 = \frac{1}{5}\qty\big(Z_1 - 2 \cdot Z_2 + 7 \cdot Z_3)}&\leadsto
    \begin{pmatrix}
      1 & 0 & 0 & 0 \\
      0 & 1 & 0 & 4 \\
      0 & 0 & 1 & 1 \\
      0 & 0 & 0 & 0 \\
    \end{pmatrix}
  \end{flalign*}
  Somit ist $\text{Rang}\qty\big(A) = 3$.
  Außerdem ist $x_1 = 0$, $x_2 = -4x_4$ und $x_3 = -x_4$.
  Somit ist
  \[
    \text{Kern}\qty\big(A) = \qty{
      \begin{pmatrix}
        0 \\
        -4x_4 \\
        -x_4 \\
        x_4 \\
      \end{pmatrix}
      \:\middle|\:
      x_4 \in \mathbb{R}
    } = \text{Span}\qty(\qty{
      \begin{pmatrix}
        0  \\
        -4 \\
        -1 \\
        1  \\
      \end{pmatrix}
    })
  \]

  Es ist offensichtlich $\text{Rang}\qty\big(B) = 1$.
  Außerdem ist $x_1 = -x_2 - x_3 - x_4$.
  Es folgt
  \begin{flalign*}
    \text{Kern}\qty\big(B) &= \qty{
      \begin{pmatrix}
        -x_2 - x_3 - x_4 \\
        x_2 \\
        x_3 \\
        x_4 \\
      \end{pmatrix}
      \:\middle|\:
      x_2, x_3, x_4 \in \mathbb{R}
    } \\
    &= \qty{
      \begin{pmatrix}
        -x_2 \\
        x_2  \\
        0    \\
        0    \\
      \end{pmatrix} + \begin{pmatrix}
        -x_3 \\
        0    \\
        x_3  \\
        0    \\
      \end{pmatrix} + \begin{pmatrix}
        -x_4 \\
        0    \\
        0    \\
        x_4  \\
      \end{pmatrix}
      \:\middle|\:
      x_2, x_3, x_4 \in \mathbb{R}
    } \\
    &= \text{Span}\qty(\qty{
      \begin{pmatrix}
        -1 \\
        1 \\
        0 \\
        0 \\
      \end{pmatrix},
      \begin{pmatrix}
        -1 \\
        0 \\
        1 \\
        0 \\
      \end{pmatrix},
      \begin{pmatrix}
        -1 \\
        0 \\
        0 \\
        1 \\
      \end{pmatrix}
    })
  \end{flalign*}

  Nun ist $\text{Col}\qty\big(A) = \qty{
    \begin{pmatrix}
      5 \\
      0 \\
      0 \\
      0 \\
    \end{pmatrix},
    \begin{pmatrix}
      2 \\
      1 \\
      0 \\
      0 \\
    \end{pmatrix},
    \begin{pmatrix}
      -7 \\
      -4 \\
      1  \\
      2  \\
    \end{pmatrix},
    \begin{pmatrix}
      1 \\
      0 \\
      1 \\
      2 \\
    \end{pmatrix}
  }$ und offensichtlich
  \[
    \begin{pmatrix}
      -7 \\
      -4 \\
      1  \\
      2  \\
    \end{pmatrix} + 4 \cdot \begin{pmatrix}
      2 \\
      1 \\
      0 \\
      0 \\
    \end{pmatrix} = \begin{pmatrix}
      1 \\
      0 \\
      1 \\
      2 \\
    \end{pmatrix}
  \]
  Somit sind die 4 Vektoren nicht linear unabhängig.
  Allerdings sind $\qty\big(5, 0, 0, 0)^T, \qty\big(2, 1, 0, 0)^T$ und
  $\qty\big(-7, -4, 1, 2)^T$ offensichtlich linear unabhängig.
  (Betrachte dazu die letzten beiden Felder).

  $\Rightarrow \dim\qty(\text{Col}\qty\big(A)) = 3$.

  Nun ist für den Vektor $a$
  \begin{flalign*}
    \qty(\begin{array}{cccc|c}
      5 & 2 & -7 & 0   & 0 \\
      0 & 1 & -4 & 2   & 0 \\
      0 & 0 & 1  & -5  & 0 \\
      0 & 0 & 2  & -10 & 0 \\
    \end{array})
    \overset{Z_4 = Z_4 - 2 \cdot Z_3}&\leadsto
    \qty(\begin{array}{cccc|c}
      5 & 2 & -7 & 0  & 0 \\
      0 & 1 & -4 & 2  & 0 \\
      0 & 0 & 1  & -5 & 0 \\
      0 & 0 & 0  & 0  & 0 \\
    \end{array}) \\
    \overset{Z_2 = Z_2 + 4 \cdot Z_3}&\leadsto
    \qty(\begin{array}{cccc|c}
      5 & 2 & -7 & 0   & 0 \\
      0 & 1 & 0  & -18 & 0 \\
      0 & 0 & 1  & -5  & 0 \\
      0 & 0 & 0  & 0   & 0 \\
    \end{array}) \\
    \overset{Z_1 = \frac{1}{5}\qty\big(Z_1 - 2 \cdot Z_2 + 7 \cdot Z_3)}&\leadsto
    \qty(\begin{array}{cccc|c}
      1 & 0 & 0 & \frac{1}{5} & 0 \\
      0 & 1 & 0 & -18         & 0 \\
      0 & 0 & 1 & -5          & 0 \\
      0 & 0 & 0 & 0           & 0 \\
    \end{array}) \\
  \end{flalign*}
  Somit ist $a$ von 3 weiteren Spaltenvektoren aus $A$ linear abhängig mit zum
  Beispiel
  \[
    \frac{1}{5} \cdot \begin{pmatrix}
      5 \\
      0 \\
      0 \\
      0 \\
    \end{pmatrix} - 18 \cdot \begin{pmatrix}
      2 \\
      1 \\
      0 \\
      0 \\
    \end{pmatrix} - 5 \cdot \begin{pmatrix}
      -7 \\
      -4 \\
      1  \\
      2  \\
    \end{pmatrix} = \begin{pmatrix}
      0   \\
      2   \\
      -5  \\
      -10 \\
    \end{pmatrix}
  \]
  $\Rightarrow a$ liegt im Spaltenraum $\text{Col}\qty\big(A)$.

  Nun ist der Spaltenraum von $B$ offensichtlich
  \[
    \text{Col}\qty\big(B) = \text{Span}\qty(\qty{
      \begin{pmatrix} 1 \\ 1 \\ 1 \\ 1 \end{pmatrix}
    })
  \]
  und $b$ liegt nicht im Spaltenraum von $B$.

\item Zählen Sie alle Vektoren aus $\text{GF}\qty\big(2)^7$ auf, die im Kern der
  Matrix $C \in \text{GF}\qty\big(2)^{4 \times 7}$ liegen.
  Welche Dimension hat der Spaltenraum von $C$?

  \[
    C \coloneqq \begin{pmatrix}
      0 & 0 & 0 & 1 & 1 & 1 & 1 \\
      0 & 1 & 1 & 0 & 0 & 1 & 1 \\
      1 & 0 & 1 & 0 & 1 & 0 & 1 \\
      1 & 1 & 0 & 1 & 0 & 0 & 0 \\
    \end{pmatrix}
  \]

  \subparagraph{Lsg.} Es ist
  \begin{flalign*}
    \qty(\begin{array}{ccccccc|c}
      0 & 0 & 0 & 1 & 1 & 1 & 1 & 0 \\
      0 & 1 & 1 & 0 & 0 & 1 & 1 & 0 \\
      1 & 0 & 1 & 0 & 1 & 0 & 1 & 0 \\
      1 & 1 & 0 & 1 & 0 & 0 & 0 & 0 \\
    \end{array})
    \overset{Z_1 \leftrightarrow Z_3}&\leadsto
    \qty(\begin{array}{ccccccc|c}
      1 & 1 & 0 & 1 & 0 & 0 & 0 & 0 \\
      0 & 1 & 1 & 0 & 0 & 1 & 1 & 0 \\
      1 & 0 & 1 & 0 & 1 & 0 & 1 & 0 \\
      0 & 0 & 0 & 1 & 1 & 1 & 1 & 0 \\
    \end{array}) \\
    \overset{Z_3 = Z_3 + Z_1 + Z_2}&\leadsto
    \qty(\begin{array}{ccccccc|c}
      1 & 1 & 0 & 1 & 0 & 0 & 0 & 0 \\
      0 & 1 & 1 & 0 & 0 & 1 & 1 & 0 \\
      0 & 0 & 0 & 1 & 1 & 1 & 0 & 0 \\
      0 & 0 & 0 & 1 & 1 & 1 & 1 & 0 \\
    \end{array}) \\
    \overset{Z_4 = Z_4 + Z_3}&\leadsto
    \qty(\begin{array}{ccccccc|c}
      1 & 1 & 0 & 1 & 0 & 0 & 0 & 0 \\
      0 & 1 & 1 & 0 & 0 & 1 & 1 & 0 \\
      0 & 0 & 0 & 1 & 1 & 1 & 0 & 0 \\
      0 & 0 & 0 & 0 & 0 & 0 & 1 & 0 \\
    \end{array}) \\
    \overset{Z_2 = Z_2 + Z_3}&\leadsto
    \qty(\begin{array}{ccccccc|c}
      1 & 1 & 0 & 1 & 0 & 0 & 0 & 0 \\
      0 & 1 & 1 & 0 & 0 & 1 & 0 & 0 \\
      0 & 0 & 0 & 1 & 1 & 1 & 0 & 0 \\
      0 & 0 & 0 & 0 & 0 & 0 & 1 & 0 \\
    \end{array})
  \end{flalign*}
  Nun ist $x_7 = 0$, $x_1 = x_2 + x_4$, $x_3 = x_2 + x_6$ und
  $x_5 = x_4 + x_6$.
  Folglich ist
  \begin{flalign*}
    \text{Kern}\qty\big(C) &= \qty{
      \begin{pmatrix}
        x_2 + x_4 \\
        x_2 \\
        x_2 + x_6 \\
        x_4 \\
        x_4 + x_6 \\
        x_6 \\
        0 \\
      \end{pmatrix}
      \:\middle|\:
      x_2, x_4, x_6 \in \text{GF}\qty\big(2)
    } \\
    &= \qty{
      \begin{pmatrix}
        x_2 \\
        x_2 \\
        x_2 \\
        0   \\
        0   \\
        0   \\
        0   \\
      \end{pmatrix} + \begin{pmatrix}
        x_4 \\
        0   \\
        0   \\
        x_4 \\
        x_4 \\
        0   \\
        0   \\
      \end{pmatrix} + \begin{pmatrix}
        0   \\
        0   \\
        x_6 \\
        0   \\
        x_6 \\
        x_6 \\
        0   \\
      \end{pmatrix}
      \:\middle|\:
      x_2, x_4, x_6 \in \text{GF}\qty\big(2)
    } \\
    &= \text{Span}\qty(\qty{
      \begin{pmatrix}
        1 \\
        1 \\
        1 \\
        0 \\
        0 \\
        0 \\
        0 \\
      \end{pmatrix},
      \begin{pmatrix}
        1 \\
        0 \\
        0 \\
        1 \\
        1 \\
        0 \\
        0 \\
      \end{pmatrix},
      \begin{pmatrix}
        0 \\
        0 \\
        1 \\
        0 \\
        1 \\
        1 \\
        0 \\
      \end{pmatrix}
    })
  \end{flalign*}
  Mit $\text{Dim}\qty(\text{Kern}\qty\big(C)) = 2$.
  Konkret aufgezählt sind die Elemente alle möglichen Summen der drei
  Basisvektoren (denn in $\text{GF}\qty\big(2)$ können diese 0 oder 1 mal
  auftreten):
  \[
    \qty{
      \begin{pmatrix}
        0 \\
        0 \\
        0 \\
        0 \\
        0 \\
        0 \\
        0 \\
      \end{pmatrix},
      \begin{pmatrix}
        1 \\
        1 \\
        1 \\
        0 \\
        0 \\
        0 \\
        0 \\
      \end{pmatrix},
      \begin{pmatrix}
        1 \\
        0 \\
        0 \\
        1 \\
        1 \\
        0 \\
        0 \\
      \end{pmatrix},
      \begin{pmatrix}
        0 \\
        0 \\
        1 \\
        0 \\
        1 \\
        1 \\
        0 \\
      \end{pmatrix},
      \begin{pmatrix}
        0 \\
        1 \\
        1 \\
        1 \\
        1 \\
        0 \\
        0 \\
      \end{pmatrix},
      \begin{pmatrix}
        1 \\
        1 \\
        0 \\
        0 \\
        1 \\
        1 \\
        0 \\
      \end{pmatrix},
      \begin{pmatrix}
        1 \\
        0 \\
        1 \\
        1 \\
        0 \\
        1 \\
        0 \\
      \end{pmatrix},
      \begin{pmatrix}
        0 \\
        1 \\
        0 \\
        1 \\
        0 \\
        1 \\
        0 \\
      \end{pmatrix}
    }
  \]
\end{enumerate}

\newpage
\paragraph{Ü 6.5 Lösbarkeit linearer Gleichungssysteme}
Es seien $A = \begin{pmatrix}
  1  & 0 & 0 \\
  -2 & 0 & 1 \\
  4  & a & b \\
\end{pmatrix} \in \mathbb{R}^{3 \times 3}$
und $u = \begin{pmatrix} 1 \\ 2 \\ 7 \end{pmatrix} \in \mathbb{R}^3$.
Finden Sie alle Werte für $a, b \in \mathbb{R}$ für die das lineare
Gleichungssystem $Ax = u$
\begin{enumerate}[(a)]
\item keine Lösung
\item genau eine Lösung
\item mehrere Lösungen
\end{enumerate}
hat.
Geben Sie die Lösungsmengen von (b) und (c) an.

\subparagraph{Lsg.} Eine lineares Gleichungssystem hat
\begin{enumerate}[(a)]
\item keine Lösung, falls zwei Gleichungen in Widerspruch zueinander stehen, dass
  heißt einer Nullzeile in $A$ ein von Null verschiedener Wert in $b$ zugeordnet
  ist, oder zwei gleiche Zeilen in $A$ unterschiedliche Werte in $b$ besitzen.

  Anders ausgedrückt $\text{Rang}\qty\big(A) < \text{Rang}\qty\big{A|b}$.

\item genau eine Lösung, falls
  $\text{Rang}\qty\big(A) = \text{Rang}\qty\big{A|b}$ und gleich der Anzahl der
  Variablen ist.

\item mehrere Lösungen, falls $\text{Rang}\qty\big(A) = \text{Rang}\qty\big{A|b}$
  und kleiner der Anzahl der Variablen ist.
\end{enumerate}

Nun ist
\begin{enumerate}[(a)]
\item $\text{Rang}\qty\big(A) < \text{Rang}\qty\big{A|b}$, wenn eine oder
  mehrere Zeilen in $A$ linear abhängig sind, dafür allerdings in
  $\text{Rang}\qty\big(A|b)$ auf einmal linear unabhängig wären.
  Dabei sind die erste und die zweite Zeile bereits offensichtlich linear
  unabhängig.

  Allerdings wären
  \begin{itemize}
  \item die erste und die dritte Zeile linear abhängig, falls $a = b = 0$
  \item die zweite und die dritte Zeile linear abhängig, falls $a = 0$ und
    $b = -2$.
  \item die dritte von den ersten beiden Zeilen linear abhängig, falls
    $a = 0$ und $b$ eine Lösung für $-2 \cdot b + x = 4$ mit $x \in \mathbb{R}$
    ist.
    Nun gibt es für jedes $x \in \mathbb{R}$ ein entsprechendes $b$, damit ist
    die dritte Zeile für $a = 0$ immer linear abhängig.
  \end{itemize}
  Nun ist
  \[
    \qty(\begin{array}{ccc|c}
      1  & 0 & 0 & 1 \\
      -2 & 0 & 1 & 2 \\
      4  & 0 & b & 7 \\
    \end{array})
    \overset{Z_2 = Z_2 + 2 \cdot Z_1}\leadsto
    \qty(\begin{array}{ccc|c}
      1 & 0 & 0 & 1 \\
      0 & 0 & 1 & 4 \\
      4 & 0 & b & 7 \\
    \end{array})
    \overset{Z_3 = Z_3 - 4 \cdot Z_1 - \frac{3}{4} \cdot Z_2}\leadsto
    \qty(\begin{array}{ccc|c}
      1 & 0 & 0               & 1 \\
      0 & 0 & 1               & 4 \\
      0 & 0 & b - \frac{3}{4} & 0 \\
    \end{array})
  \]
  und das die zweite und dritte Zeile ständen für $b \ne \frac{3}{4}$ im
  Widerspruch.

  $\Rightarrow$ das LGS hat für $a = 0$ und $b \ne \frac{3}{4}$ keine Lösung.

\item Das LGS hat 3 unbekannte.
  Somit muss für genau eine Lösung
  $\text{Rang}\qty\big(A) = \text{Rang}\qty\big(A|b) = 3$ sein.
  Nun wurde in (a) bereits gezeigt, dass $\text{Rang}\qty\big(A) = 3$, falls
  $a \ne 0$.
  Nun ist
  \[
    \qty(\begin{array}{ccc|c}
      1  & 0 & 0 & 1 \\
      -2 & 0 & 1 & 2 \\
      4  & a & b & 7 \\
    \end{array})
    \overset{Z_2 = Z_2 + 2 \cdot Z_1}\leadsto
    \qty(\begin{array}{ccc|c}
      1 & 0 & 0 & 1 \\
      0 & 0 & 1 & 4 \\
      4 & a & b & 7 \\
    \end{array})
    \overset{Z_3 = Z_3 - 4 \cdot Z_1 - \frac{3}{4} \cdot Z_2}\leadsto
    \qty(\begin{array}{ccc|c}
      1 & 0 & 0               & 1 \\
      0 & 0 & 1               & 4 \\
      0 & a & b - \frac{3}{4} & 0 \\
    \end{array})
  \]
  und $x_1 = 1$, $x_3 = 4$ sowie
  \begin{flalign*}
    a \cdot x_2 &= x_3\qty\big(\frac{3}{4} - b) \\
    \overset{x_3 = 4}&= 3 - 4 \cdot b && {\Big |} \cdot \frac{1}{a}, a \ne 0 \\
    x_2 &= \frac{3 - 4b}{a}
  \end{flalign*}
  Dementsprechend ist
  \[
    L = \qty{\begin{pmatrix}
      1 \\
      \frac{3 - 4b}{a} \\
      4 \\
    \end{pmatrix} \:\middle|\: a \ne 0, b \in \mathbb{R}}
  \]

\item $\text{Rang}\qty\big(A) = \text{Rang}\qty\big{A|b} < 3$, falls
  die dritte Zeile der erweiterten Koeffizientenmatrix linear abhängig ist.
  In (a) wurde bereits ermittelt, dass dies für $a = 0$ und $b = \frac{3}{4}$
  der Fall ist.

  Nun ist
  \[
    \qty(\begin{array}{ccc|c}
      1  & 0 & 0           & 1 \\
      -2 & 0 & 1           & 2 \\
      4  & 0 & \frac{3}{4} & 7 \\
    \end{array})
    \overset{Z_2 = Z_2 + 2 \cdot Z_1}\leadsto
    \qty(\begin{array}{ccc|c}
      1 & 0 & 0           & 1 \\
      0 & 0 & 1           & 4 \\
      4 & 0 & \frac{3}{4} & 7 \\
    \end{array})
    \overset{Z_3 = Z_3 - 4 \cdot Z_1 - \frac{3}{4} \cdot Z_2}\leadsto
    \qty(\begin{array}{ccc|c}
      1 & 0 & 0 & 1 \\
      0 & 0 & 1 & 4 \\
      0 & 0 & 0 & 0 \\
    \end{array})
  \]
  und
  \[
    L = \qty{\begin{pmatrix}
      1 \\
      \lambda \\
      4 \\
    \end{pmatrix} \:\middle|\: \lambda \in \mathbb{R}}
  \]
\end{enumerate}

\newpage
\paragraph{Ü 6.6 Zeilenumformungen ändern nicht den Rang}
Es seien $A$ und $B$ zwei $\qty\big(m \times n)$-Matrizen über einem Körper $K$
und $u_1, \ldots, u_m \in K^n$ bzw. $z_1, \ldots, z_m \in K^m$ seine die Zeilen von
$A$ bzw. $B$.
Beweisen Sie: Entsteht $B$ aus $A$ durch eine elementare Zeilenumformung, dann
ist
$\text{Span}\qty\big{u_1, \ldots, u_m} = \text{Span}\qty\big{z_1, \ldots, z_m}$.

\subparagraph{Lsg.} Elementare Zeilenumformungen sind
\begin{enumerate}[(i)]
\item Vertauschen zweier Zeilen
\item Multiplizieren einer Zeile mit einem $k \ne 0 \in K$
\item Addieren des $k$-fachen ($k \in K$) einer Zeile zu einer anderen Zeile.
\end{enumerate}

\noindent
Sein nun $B$ durch elementare Zeilenumformungen aus $A$ entstanden, dann lässt
sich jedes $z_i$, $1 \leq i \leq m$ als
\[
  z_i = \lambda \cdot u_j + \sum_{\underset{l \ne j}{l = 1}}^m k_l \cdot u_l,
  \qquad
  \lambda \ne 0, k_l \in K
\]
beschreiben.
Somit ist $z_i$ eine Linearkombination der $u_1, \ldots, u_m$.

$\Rightarrow$ alle Zeilenvektoren in $B$ sind von den Zeilenvektoren in $A$
linear Abhängig und $\text{Rang}\qty\big(B) = \dim\qty(\text{Row}\qty\big(B))
\leq \text{dim}\qty(\text{Row}\qty\big(A)) = \text{Rang}\qty\big(A)$.
Außerdem ist jedes $z_i \in \text{Span}\qty\big(u_1, \ldots, u_m)$.
\\

\noindent
Nun lässt sich jede einzelne elementare Zeilenumformung durch eine andere
elementare Zeilenumformung umkehren:
\begin{enumerate}[(i)]
\item Zwei Zeilen lassen sich zurück vertauschen.
\item Da $K$ ein Körper, existiert auf $\frac{1}{k} \in K \setminus \qty\big{0}$,
  mit dem die Zeile wieder multipliziert werden kann.
\item Da $K$ ein Körper, existiert $-k \in K$ und die Zeile kann mit dem
  $-k$-fachen einer anderen Zeile addiert werden.
\end{enumerate}
Damit lässt entsteht auch wieder $A$ aus $B$ durch elementare Zeilenumformungen
und analog zu oben

$\Rightarrow$ alle Zeilenvektoren in $A$ sind von den Zeilenvektoren in $B$
linear Abhängig und $\text{Rang}\qty\big(A) = \dim\qty(\text{Row}\qty\big(A))
\leq \text{dim}\qty(\text{Row}\qty\big(B)) = \text{Rang}\qty\big(B)$.
Außerdem ist jedes $u_i \in \text{Span}\qty\big(z_1, \ldots, z_m)$

$\Rightarrow \text{Rang}\qty\big(A) = \text{Rang}\qty\big(B)$ und
$\text{Span}\qty\big{u_1, \ldots, u_m} = \text{Span}\qty\big{z_1, \ldots, z_m}$.

\end{document}
