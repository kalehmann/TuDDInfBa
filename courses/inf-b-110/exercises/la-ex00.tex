\documentclass{scrreprt}

\usepackage{amsmath}
\usepackage{amsthm}
\usepackage{amssymb}
\usepackage{bm}
\usepackage[shortlabels]{enumitem}
\usepackage{hyperref}
\usepackage[utf8]{inputenc}
\usepackage{multicol}
\usepackage{mathtools}
\usepackage{physics}
\usepackage{polynom}
\usepackage{tabularx}
\usepackage[table]{xcolor}
\usepackage{titling}
\usepackage{fancyhdr}
\usepackage{xfrac}

\author{Karsten Lehmann}
\date{WiSe 2024/25}
\title{Übungsblatt 0\\INF-B-110, Lineare Algebra}

\setlength{\headheight}{26pt}
\pagestyle{fancy}
\fancyhf{}
\lhead{\thetitle}
\rhead{\theauthor}
\lfoot{\thedate}
\rfoot{Seite \thepage}

% See https://tex.stackexchange.com/a/79254
\newcommand{\dropsign}[1]{\smash{\llap{\raisebox{-.5\normalbaselineskip}{$#1$\hspace{2\arraycolsep}}}}}%


\begin{document}
\paragraph{Ü0.2 Polynome und Nullstellen}
\begin{enumerate}[(a)]
\item Geben Sie alle reellen Lösungen der Gleichung
  $x\qty\big(2x-\sqrt{3})\sqrt{3x+4} = 0$ an.

  \subparagraph{Lsg.} $x_1 = 0$, $x_2 = \frac{\sqrt{3}}{2}$, $x_3 = -\frac{4}{3}$

\item Finden Sie alle reellen Nullstellen der Polynomfunktion
  $p\qty\big(x) = \qty\big(x^2+3x-10)\qty\big(x^2-1)$

  \subparagraph{Lsg.} Es ist
  \begin{flalign*}
    p\qty\big(x) &= \qty\big(x^2+3x-10)\qty\big(x^2-1) &\\
                 &= \qty\big(x+5)\qty\big(x-2)\qty\big(x^2-1)
  \end{flalign*}
  $\Rightarrow$ $x_1 = -5$, $x_2 = 2$, $x_3 = -1$, $x_4 = 1$

\item Zeigen Sie, dass $x = 1$ eine Nullstelle von
  $p\qty\big(x) = x^3 - 6x^2 + 11x - 6$ ist.
  Berechnen Sie $q\qty\big(x)$ so, dass
  $p\qty\big(x) = \qty\big(x - 1)q\qty\big(x)$ gilt.
  Wenden Sie dazu Polynomdivision an.
  Bestimmen Sie anschließend die restlichen Nullstellen von $p\qty\big(x)$

  \subparagraph{Lsg.} Es ist
  \begin{flalign*}
    p\qty\big(1) &= 1^3 - 6 \cdot 1^2 + 11 \cdot 1 - 6 &\\
                 &=1 - 6 + 11 - 6 \\
                 &= 0
  \end{flalign*}
  Weiter ist
  \[
    \begin{array}{lllllr}
      \dropsign{-} \big(x^3 - & 6x^2 + &11x - &6\big) &: \qty\big(x - 1) &= x^2 - 5x + 6 \\
      \big(x^3 - &\phantom{6} x^2\big) \\
      \cline{1-2}
      & \dropsign{-} \phantom{\big(} -5x^2 + &11x \\
      & \big(-5x^2 + &\phantom{2}5x\big) \\
      \cline{2-3}
      && \dropsign{-} \phantom{1}6x - &6 \\
      && \big(6x - &6\big) \\
      \cline{3-4}
      &&& 0
    \end{array}
  \]
  Somit ist $q\qty\big(x) = x^2 - 5x + 6$.
  Aus $q\qty\big(x) = x^2 - 5x + 6 = \qty\big(x - 2)\qty\big(x - 3)$ folgt
  $x_2 = 2$ und $x_3 = 3$.
\end{enumerate}
\newpage

\paragraph{Ü0.3 analytische Geometrie im $\mathbb{R}^2$}

Sei $g \coloneqq \qty{\begin{pmatrix}x\\y\end{pmatrix} \in \mathbb{R}^2 \:\middle|\: y = 3x + 1}$
eine Gerade in $\mathbb{R}^2$.

\begin{enumerate}[(a)]
\item Bestimmen Sie den Schnittpunkt mit der Geraden
  $h \coloneqq \qty{\begin{pmatrix}x\\y\end{pmatrix} \in \mathbb{R}^2 \:\middle|\: y = -x - 3}$

  \subparagraph{Lsg.} Es ist
  \begin{flalign*}
    3x + 1 &= -x - 3 & {\big |} +3 \\
    3x + 4 &= -x & {\big |} +x \\
    4x + 4 &= 0 \\
    \Rightarrow x = -1
  \end{flalign*}
  Nun ist $y = 3 \cdot \qty\big(-1) + 1 = -2$.
  Es folgt der Schnittpunkt $\begin{pmatrix}-1\\-2\end{pmatrix}$.

\item Geben Sie $g$ in Parameterdarstellung an.

  \subparagraph{Lsg.} Bestimme zwei Punkte auf $g$, zum Beispiel
  $\begin{pmatrix}0\\1\end{pmatrix}$ für $x_0 = 0$ und
  $\begin{pmatrix}1\\4\end{pmatrix}$ für $x_1 = 1$.
  Nun ist \[
    g = \qty{
      \begin{pmatrix}0\\1\end{pmatrix} + \lambda \cdot
      \qty(\begin{pmatrix}1\\4\end{pmatrix} - \begin{pmatrix}0\\1\end{pmatrix})
      \:\middle|\: \lambda
      \in \mathbb{R}
    } = \qty{
      \begin{pmatrix}0\\1\end{pmatrix} + \lambda \cdot
      \begin{pmatrix}1\\3\end{pmatrix}
      \:\middle|\: \lambda
      \in \mathbb{R}
    }
  \]

\item Berechnen Sie den Abstand des Punktes
  $q = \begin{pmatrix}3\\0\end{pmatrix}$ zum Schnittpunkt der Geraden $g$ und
  $h$.

  \subparagraph{Lsg.} Es ist
  \[
    \norm{\begin{pmatrix}3\\0\end{pmatrix} - \begin{pmatrix}-1\\-2\end{pmatrix}}_2
    = \sqrt{4^2 + \qty\big(-2)^2} = \sqrt{20}
  \]

\newpage
\item Berechnen Sie den Abstand des Punktes $q$ zu der Geraden $g$

  \subparagraph{Lsg.} Es ist $\begin{pmatrix}1\\3\end{pmatrix}$ ein
  Richtungsvektor von $g$.
  Somit ist $\begin{pmatrix}-3\\1\end{pmatrix}$ ein Normalenvektor von $g$.
  Sein nun $g' = \qty{q + \mu \cdot \begin{pmatrix}-3\\1\end{pmatrix} \:\middle|\: \mu \in \mathbb{R}}$.
  Anschließend setzt man $g$ mit $g'$ gleich:
  \[
    \begin{pmatrix}0\\1\end{pmatrix} + \lambda \cdot
    \begin{pmatrix}1\\3\end{pmatrix}
    =
    \begin{pmatrix}3\\0\end{pmatrix} + \mu \cdot
    \begin{pmatrix}-3\\1\end{pmatrix}
  \]
  Entweder löst man das Gleichungssystem normal, oder sieht direkt die Lösung
  mit $\lambda = 0$ und $\mu = 1$.
  Es folgt der Schnittpunkt $\begin{pmatrix}0\\1\end{pmatrix}$.
  Schließlich ist
  \[
    \norm{\begin{pmatrix}3\\0\end{pmatrix} - \begin{pmatrix}0\\1\end{pmatrix}}_2
    = \sqrt{3^2 + 1^2} = \sqrt{10}
  \]
  
\end{enumerate}

\end{document}
