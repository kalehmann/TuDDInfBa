\documentclass{scrreprt}

\usepackage{aligned-overset}
\usepackage{amsmath}
\usepackage{amsthm}
\usepackage{amssymb}
\usepackage{bm}
\usepackage[inline,shortlabels]{enumitem}
\usepackage{framed}
\usepackage{hyperref}
\usepackage[utf8]{inputenc}
\usepackage{multicol}
\usepackage{mathtools}
\usepackage{physics}
\usepackage{polynom}
\usepackage{tabularx}
\usepackage[table]{xcolor}
\usepackage{titling}
\usepackage{fancyhdr}
\usepackage{xfrac}
\usepackage{pgfplots}

\pgfplotsset{compat = newest}
\usepgfplotslibrary{fillbetween}
\usetikzlibrary{calc}


\author{Karsten Lehmann}
\date{WiSe 2024/25}
\title{Übungsblatt 9\\INF-B-110, Lineare Algebra}

\setlength{\parindent}{0pt}

\setlength{\headheight}{26pt}
\pagestyle{fancy}
\fancyhf{}
\lhead{\thetitle}
\rhead{\theauthor}
\lfoot{\thedate}
\rfoot{Seite \thepage}

\begin{document}
\paragraph{Ü9.3 Determinantenbestimmung mit Sarrus, Gauß, Laplace}

Berechnen Sie die Determinante der Matrix
\[
  A = \begin{pmatrix}
    1 & 2  & 2  \\
    1 & -2 & 3  \\
    3 & 1  & -4 \\
  \end{pmatrix} \in \mathbb{R}^{3 \times 3}
\]
\begin{enumerate}[(a)]
\item Durch Überführung in Zeilenstufenform.

  \subparagraph{Lsg.} Es ist
   \begin{flalign*}
    \det\begin{pmatrix}
      1 & 2  & 2  \\
      1 & -2 & 3  \\
      3 & 1  & -4 \\
    \end{pmatrix}
    \overset{Z_3 = Z_3 - 3 Z_1}&\leadsto
    \det\begin{pmatrix}
      1 & 2  & 2   \\
      1 & -2 & 3   \\
      0 & -5 & -10 \\
    \end{pmatrix} \\
    \overset{Z_2 = Z_2 - Z_1}&\leadsto
    \det\begin{pmatrix}
      1 & 2  & 2   \\
      0 & -4 & 1   \\
      0 & -5 & -10 \\
    \end{pmatrix} \\
    \overset{Z_3 = -4 \cdot Z_3}&\leadsto
    -\frac{1}{4} \cdot \det\begin{pmatrix}
      1 & 2  & 2  \\
      0 & -4 & 1  \\
      0 & 20 & 40 \\
    \end{pmatrix} \\
    \overset{Z_3 = Z_3 + 5 \cdot Z_2}&\leadsto
    -\frac{1}{4} \cdot \det\begin{pmatrix}
      1 & 2  & 2  \\
      0 & -4 & 1  \\
      0 & 0  & 45 \\
    \end{pmatrix} \\
    &= -\frac{1}{4} \cdot 1 \cdot \qty\big(-4) \cdot 45 = 45
  \end{flalign*}
\item mit der Regel von Sarrus.

  \subparagraph{Lsg.} Es ist
  \begin{flalign*}
    \det\begin{pmatrix}
      \cellcolor{red!30}1    & \cellcolor{blue!20}2   & \cellcolor{yellow!30}2   \\
      \cellcolor{orange!30}1 & \cellcolor{green!30}-2 & \cellcolor{teal!20}3     \\
      \cellcolor{lime!40}3   & \cellcolor{cyan!20}1   & \cellcolor{magenta!20}-4 \\
    \end{pmatrix}
    = &\colorbox{red!30}{$1$} \cdot \colorbox{green!30}{$\qty\big(-2)$} \cdot \colorbox{magenta!20}{$\qty\big(-4)$} +
       \colorbox{blue!20}{$2$} \cdot \colorbox{teal!20}{$3$} \cdot \colorbox{lime!40}{$3$} +
       \colorbox{yellow!30}{$2$} \cdot \colorbox{orange!30}{$1$} \cdot \colorbox{cyan!20}{$1$} \\
      &- \colorbox{lime!40}{$3$} \cdot \colorbox{green!30}{$\qty\big(-2)$} \cdot \colorbox{yellow!30}{$2$}
       - \colorbox{cyan!20}{$1$} \cdot \colorbox{teal!20}{$3$} \cdot \colorbox{red!30}{$1$}
       - \colorbox{magenta!20}{$\qty\big(-4)$} \cdot \colorbox{orange!30}{$1$} \cdot \colorbox{blue!20}{$2$} \\
    = &45
  \end{flalign*}

\newpage
\item mit Hilfe des Entwicklungssatzes.

  \subparagraph{Lsg.} Es ist
  \begin{flalign*}
    \det\begin{pmatrix}
      \cellcolor{red!30}1    & 2  & 2  \\
      \cellcolor{blue!20}1   & -2 & 3  \\
      \cellcolor{yellow!30}3 & 1  & -4 \\
    \end{pmatrix}
    &= \colorbox{red!30}{$1$} \cdot \det\begin{pmatrix}
      -2 & 3  \\
      1  & -4 \\
    \end{pmatrix} - \colorbox{blue!20}{$1$} \cdot \det\begin{pmatrix}
      2 & 2  \\
      1 & -4 \\
    \end{pmatrix} + \colorbox{yellow!30}{$3$} \cdot \det\begin{pmatrix}
      2  & 2 \\
      -2 & 3 \\
    \end{pmatrix} \\
    &= \qty\big(\qty\big(-2) \cdot \qty\big(-4) - 3 \cdot 1)
       - \qty\big(2 \cdot \qty\big(-4) - 2 \cdot 1)
       + \colorbox{yellow!30}{$3$} \cdot \qty\big(2 \cdot 3 - 2 \cdot \qty\big(-2)) \\
    &= 5 + 10 + \colorbox{yellow!30}{$3$} \cdot 10 \\
    &= 45
  \end{flalign*}
\end{enumerate}

\paragraph{Ü 9.4 Determinante einer Vandermonde-Matrix}

Seien $x_1, x_2, \ldots, x_n \in \mathbb{C}$.
Zeigen Sie, dass für die \emph{Vandermonde-Matrix}
\[
  V\qty\big(x_1, x_2, x_3) \coloneqq \begin{pmatrix}
    1 & x_1 & x_1^2 \\
    1 & x_2 & x_2^2 \\
    1 & x_3 & x_3^2 \\
  \end{pmatrix} \in \mathbb{C}^{3 \times 3}
\]
die Gleichung $\det\qty(V\qty\big(x_1, x_2, x_3)) = \qty\big(x_3 - x_1) \cdot
\qty\big(x_3 - x_2) \cdot \qty\big(x_2 - x_1)$ gilt.

\subparagraph{Lsg.} Es ist
\begin{flalign*}
  \det \begin{pmatrix}
    1 & x_1 & x_1^2 \\
    1 & x_2 & x_2^2 \\
    1 & x_3 & x_3^2 \\
  \end{pmatrix}
  \overset{Z_2 = Z_2 - Z_1}&=
  \det \begin{pmatrix}
    1 & x_1 & x_1^2 \\
    0 & x_2 - x_1 & x_2^2 - x_1^2 \\
    1 & x_3 & x_3^2 \\
  \end{pmatrix} \\
  \overset{Z_3 = Z_3 - Z_1}&=
  \det \begin{pmatrix}
    1 & x_1 & x_1^2 \\
    0 & x_2 - x_1 & x_2^2 - x_1^2 \\
    0 & x_3 - x_1 & x_3^2 - x_1^2 \\
  \end{pmatrix} \\
  &= \det \begin{pmatrix}
    x_2 - x_1 & x_2^2 - x_1^2 \\
    x_3 - x_1 & x_3^2 - x_1^2 \\
  \end{pmatrix} \\
  &= \qty\big(x_2 - x_1) \cdot \qty\big(x_3^2 - x_1^2)
    - \qty\big(x_2^2 - x_1^2) \cdot \qty\big(x_3 - x_1) \\
  &= \qty\big(x_2 - x_1) \cdot \qty\big(x_3^2 - x_1^2)
    - \qty\big(x_2^2x_3 - x_2^2x_1 - x_1^2x_3 - x_1^3)\\
  &= \qty\big(x_2 - x_1) \cdot \qty\big(x_3^2 - x_1^2)
    + \qty\big(x_2^2x_3 + x_2^2x_1 + x_1^2x_3 + x_1^3)\\
  &= \qty\big(x_2 - x_1) \cdot \qty\big(x_3^2 - x_1^2)
    + \qty\big(x_2 - x_1) \cdot \qty\big(x_1x_2 - x_1x_3 - x_2x_3 + x_1^2)\\
  &= \qty\big(x_2 - x_1) \cdot \qty\big(x_3^2 - x_1^2 + x_1x_2 - x_1x_3 - x_2x_3 + x_1^2) \\
  &= \qty\big(x_2 - x_1) \cdot \qty\big(x_3^2 - x_2x_3 - x_1x_3 + x_1x_2) \\
  &= \qty\big(x_2 - x_1) \cdot \qty\big(x_3 - x_1) \cdot \qty\big(x_3 - x_2)
\end{flalign*}

\textbf{Alternativ nach der Übung:}
\begin{flalign*}
  \det\begin{pmatrix}
    1 & x_1 & x_1^2 \\
    1 & x_2 & x_2^2 \\
    1 & x_3 & x_3^2 \\
  \end{pmatrix}
  \overset{\substack{Z_2 = Z_2 - Z_1 \\ Z_3 = Z_3 - Z_1}}&=
  \det\begin{pmatrix}
    1 & x_1 & x_1^2 \\
    0 & x_2 - x_1 & x_2^2 - x_1^2 \\
    0 & x_3 - x_1 & x_3^2 - x_1^2\\
  \end{pmatrix} \\
  \overset{\substack{Z_2 = \frac{Z_2}{x_2 - x_1} \\ Z_2 = \frac{Z_3}{x_3 - x_1}}}&=
  \qty\big(x_2 - x_1)\qty\big(x_3 - x_1)\det\begin{pmatrix}
    1 & x_1 & x_1^2 \\
    0 & 1   & x_2 + x_1 \\
    0 & 1   & x_3 + x_1\\
  \end{pmatrix} && \qty\Big(*) \\
  \overset{Z_3 = Z_3 - Z_2}&=
  \qty\big(x_2 - x_1)\qty\big(x_3 - x_1)\det\begin{pmatrix}
    1 & x_1 & x_1^2 \\
    0 & 1   & x_2 + x_1 \\
    0 & 0   & x_3 + x_1 - \qty\big(x_2 + x_1) \\
  \end{pmatrix} \\
  &=
  \qty\big(x_2 - x_1)\qty\big(x_3 - x_1)\det\begin{pmatrix}
    1 & x_1 & x_1^2 \\
    0 & 1   & x_2 + x_1 \\
    0 & 0   & x_3 - x_2 \\
  \end{pmatrix} \\
  &= \qty\big(x_2 - x_1)\qty\big(x_3 - x_1)\qty\big(x_3 - x_2)
\end{flalign*}
$\qty\Big(*) x_2 - x_1 \ne 0, x_3 - x_1 \ne 0$.
Ansonsten ist $\det\qty\big(\ldots) = 0$ trivial.

\newpage
\paragraph{Ü 9.5 Determinanten und lineare Unabhängigkeit}
\begin{enumerate}[(a)]
\item Für welche $\lambda \in \mathbb{R}$ ist die Determinante der Matrix
  $A \in \mathbb{R}^{4 \times 4}$ von Null verschieden?
  \[
    A = \begin{pmatrix}
      0  & \lambda & -1 & 2       \\
      -1 & 2       & 0  & 1       \\
      1  & 3       & 2  & \lambda \\
      2  & -1      & 1  & -1      \\
    \end{pmatrix}
  \]

  \subparagraph{Lsg.} Es ist
  \begin{flalign*}
    \det\begin{pmatrix}
      0  & \lambda & -1 & 2       \\
      -1 & 2       & 0  & 1       \\
      1  & 3       & 2  & \lambda \\
      2  & -1      & 1  & -1      \\
    \end{pmatrix}
    \overset{\substack{Z_1 \leftrightarrow Z_2 \\ Z_3 \leftrightarrow Z_4}}&=
    \det\begin{pmatrix}
      -1 & 2       & 0  & 1       \\
      0  & \lambda & -1 & 2       \\
      2  & -1      & 1  & -1      \\
      1  & 3       & 2  & \lambda \\
    \end{pmatrix} \\
    \overset{\substack{Z_3 = Z_3 + 2Z_1 \\ Z_4 = Z_4 + Z_1}}&=
    \det\begin{pmatrix}
      -1 & 2       & 0  & 1           \\
      0  & \lambda & -1 & 2           \\
      0  & 3       & 1  & 1           \\
      0  & 5       & 2  & \lambda + 1 \\
    \end{pmatrix} \\
    \overset{Z_4 = 3 \cdot Z_4 - 5 \cdot Z_3}&=
    \frac{1}{3}\det\begin{pmatrix}
      -1 & 2       & 0  & 1            \\
      0  & \lambda & -1 & 2            \\
      0  & 3       & 1  & 1            \\
      0  & 0       & 1  & 3\lambda - 2 \\
    \end{pmatrix} \\
    &=
    -\frac{1}{3}\det\begin{pmatrix}
      \lambda & -1 & 2            \\
      3       & 1  & 1            \\
      0       & 1  & 3\lambda - 2 \\
    \end{pmatrix} \\
    \overset{\substack{Z_1 = Z_1 + Z_2 \\ Z_3 = Z_3 - Z_2}}&=
    -\frac{1}{3}\det\begin{pmatrix}
      \lambda + 3 & 0 & 3            \\
      3           & 1 & 1            \\
      -3          & 0 & 3\lambda - 3 \\
    \end{pmatrix} \\
    &= -\frac{1}{3}\det\begin{pmatrix}
      \lambda + 3 & 3            \\
      -3          & 3\lambda - 3 \\
    \end{pmatrix} \\
    &= -\frac{1}{3} \qty(\qty\big(\lambda + 3)\qty\big(3\lambda - 3) + 9) \\
    &= -\qty(\qty\big(\lambda + 3)\qty\big(\lambda - 1) + 3) \\
    &= -\lambda^2 - 2\lambda  = -\lambda\qty\big(\lambda + 2)
  \end{flalign*}
  Nun ist die Determinante gleich Null für $\lambda \in \qty\big{-2, 0}$

\item Für welche reellen Zahlen $\lambda$ sind die Vektoren
  \[
    v_1 = \qty\big(0, -1, 1, 2)^T,
    v_2 = \qty\big(\lambda, 2, 3, -1)^T,
    v_3 = \qty\big(-1, 0, 2, 1)^T,
    v_4 = \qty\big(2, 1, \lambda, -1)^T
  \]
  linear unabhängig?

  \subparagraph{Lsg.} Das sind die Spaltenvektoren der Matrix aus der
  Teilaufgabe (a).
  Wenn die Vektoren linear Abhängig sind, dann lässt sich einer der Vektoren als
  Linearkombination der 3 anderen Vektoren darstellen.
  Somit lässt sich durch elementare Spaltenumformungen die Matrix $A$ in die
  Matrix $B$ mit einer Nullspalte überführen.
  Nun wurde in der Vorlesung bereits gezeigt, dass die Addition des vielfachen
  einer Spalte zu einer anderen Spalte in der Matrix die Determinante nicht
  ändert.
  Folglich ist $\det A = \det B$ und durch Entwicklung nach der Nullspalte von
  $B$ folgt $\det A = \det B = 0$.

  Schließlich ist aus Teilaufgabe (a) bereits bekannt, dass $\det A = 0$ für
  $\lambda \in \qty\big{-2, 0}$.
  Als sind die Vektoren $v_1, v_2, v_3, v_4$ für $\lambda = -2$ und $\lambda = 0$
  linear abhängig.

\item Für welche reellen Zahlen sind die reellen Polynomfunktionen
  \begin{flalign*}
    p_1\qty\big(x) &\coloneqq -x^3 + x^2 +2x,
    p_2\qty\big(x) \coloneqq 2x^3 + 3x^2 - x + \lambda, \\
    p_3\qty\big(x) &\coloneqq 2x^2 + x - 1,
    p_4\qty\big(x) \coloneqq x^3 + \lambda x^2 - x + 2
   \end{flalign*}
  linear unabhängig?

  \subparagraph{Lsg.} Es ist
  \begin{flalign*}
    \det\begin{pmatrix}
      0  & \lambda & -1 & 2       \\
      -1 & 2       & 0  & 1       \\
      1  & 3       & 2  & \lambda \\
      2  & -1      & 1  & -1      \\
    \end{pmatrix}
    \overset{Z_1 \leftrightarrow Z_2}&=
    -\det\begin{pmatrix}
      -1 & 2       & 0  & 1       \\
      0  & \lambda & -1 & 2       \\
      1  & 3       & 2  & \lambda \\
      2  & -1      & 1  & -1      \\
    \end{pmatrix} \\
    \overset{Z_2 \leftrightarrow Z_3}&=
    \det\begin{pmatrix}
      -1 & 2       & 0  & 1       \\
      1  & 3       & 2  & \lambda \\
      0  & \lambda & -1 & 2       \\
      2  & -1      & 1  & -1      \\
    \end{pmatrix} \\
    \overset{Z_3 \leftrightarrow Z_4}&=
    -\det\begin{pmatrix}
      -1 & 2       & 0  & 1       \\
      1  & 3       & 2  & \lambda \\
      2  & -1      & 1  & -1      \\
      0  & \lambda & -1 & 2       \\
    \end{pmatrix}
  \end{flalign*}
  Sei nun
  \[
    B = \begin{pmatrix}
      -1 & 2       & 0  & 1       \\
      1  & 3       & 2  & \lambda \\
      2  & -1      & 1  & -1      \\
      0  & \lambda & -1 & 2       \\
    \end{pmatrix}
  \]
  Dann entsprechen $p_1, p_2, p_3, p_4$ den Spalten von $B$.
  In Teilaufgabe (b) wurde bereits ermittelt, dass die Spaltenvektoren von $A$
  für $\lambda \in \qty\big{-2, 0}$ linear abhängig sind, also sind auch
  $p_1, p_2, p_3, p_4$ für $\lambda \in \qty\big{-2, 0}$ linear abhängig.

  \textbf{Alternativ nach der Übung:} Verwende $B = \qty\big(1, x^3, x^2, x)$
  als Basis.
\end{enumerate}

\newpage
\paragraph{Ü 9.6 Determinanten oberer Dreiecksmatrizen}
\begin{enumerate}[(a)]
\item Beweisen Sie: Ist $A = \qty\big(a_{ij}) \in K^{n \times n}$ eine obere
  Dreiecksmatrix (d.h. $a_{ij} = 0$ gilt für alle $i > j$), dann gilt
  $\det\qty\big(A) = \prod_{i = 1}^n a_{ii}$.

  \subparagraph{Lsg.} Beweis durch vollständige Induktion.

  \textbf{Behauptung}: Sei $P\qty\big(n)$ für alle
  $n \in \mathbb{N} \setminus \qty\big{0}$ wahr mit
  \[
    P\qty\big(n) \colon \det\qty\big(A) = \prod_{i = 1}^n a_{ii}
    \text{ für jede obere Dreiecksmatrix } A \in K^{n \times n}
  \]

  \textbf{Induktionsanfang}: Für $n = 1$ gilt $P\qty\big(n)$, da jede Matrix
  $A = \qty(a_{11}) \in K^{1 \times 1}$ eine obere Dreiecksmatrix ist und die
  Determinante $\det\qty\big(A) = a_{11} = \prod_{i = 1}^1 a_{ii}$.

  \textbf{Induktionsschritt:} Sei $P\qty\big(n)$ für ein beliebiges
  $n \in \mathbb{N}$ wahr (Induktionsvorraussetzung (IV)).
  Dann gilt für eine beliebige obere Dreiecksmatrix
  $A = \qty(a_{ij}) \in K^{n + 1 \times n + 1}$ nach dem Entwicklungssatz
  \[
    \det\qty\big(A)
    = \sum_{j = 1}^m \qty\big(-1)^{1 + j} \cdot a_{1j} \cdot \det\qty\big(A_{1j})
  \]

  Nun wissen wir bereits, dass $a_{ij} = 0$ für $i > j$, und insbesondere
  $a_{1j} = 0$ für alle $j > 1$.
  Somit ist
  \[
    \det\qty\big(A) = a_{11} \cdot \det\qty(A_{11})
  \]
  und nach der Induktionsvorraussetzung ist
  $\det\qty(A_{11}) = \prod_{i = 2}^1 a_{ii}$.
  Es folgt $\det\qty(A) = \prod_{i = 1}^1 a_{ii}$ und aus dem Satz über die
  vollständige Induktion die Behauptung $P\qty\big(n)$.

\item Gibt es Matrixen $A, B, C, D \in \mathbb{R}^{2 \times 2}$ mit

  \begin{enumerate*}[(a)]
  \item $AA = \begin{pmatrix}
    0 & 1 \\
    1 & 0 \\
  \end{pmatrix}$
  \item $BB^T = \begin{pmatrix}
    1 & 2 \\
    2 & 1 \\
  \end{pmatrix}$
  \item $CC^T = \begin{pmatrix}
    -1 & 0  \\
    0  & -1 \\
  \end{pmatrix}$
  \item $DD^T = \begin{pmatrix}
    1 & -1 \\
    1 & 1  \\
  \end{pmatrix}$
  \end{enumerate*}

  Geben Sie Beispiele an bzw. begründen Sie, warum derartige Matrizen nicht
  existieren.
  Nutzen Sie dazu u.a. die Determinantenfunktion und deren Eigenschaften.

  \subparagraph{Lsg.}
  \begin{enumerate}[(a)]
  \item Es ist
    \[
      \det\begin{pmatrix}
        0 & 1 \\
        1 & 0 \\
      \end{pmatrix} = -1
    \]
    und nach dem Multiplikationssatz $-1 = \det\qty\big(A)^2$.
    Nun existiert eine Matrix $A$, welche diese Gleichung erfüllt in
    $\mathbb{R}^{2 \times 2}$ nicht.

  \item Es ist
    \[
      \det\begin{pmatrix}
        1 & 2 \\
        2 & 1 \\
      \end{pmatrix} = -3
    \]
    und $\det\qty\big(B^T) = \det\qty\big(B)$.
    Somit auch $-3 = \det\qty\big(B)^2$ und auch eine Matrix $B$, welche diese
    Gleichung erfüllt existiert in $\mathbb{R}^{2 \times 2}$ nicht.

  \item Sei
    \[
      C = \begin{pmatrix}
        c_{11} & c_{12} \\
        c_{21} & c_{22} \\
      \end{pmatrix},
      C^T = \begin{pmatrix}
        c_{11} & c_{21} \\
        c_{12} & c_{22} \\
      \end{pmatrix}
    \]
    Dann ist
    \[
      CC^T = \begin{pmatrix}
        c_{11} & c_{12} \\
        c_{21} & c_{22} \\
      \end{pmatrix}\begin{pmatrix}
        c_{11} & c_{21} \\
        c_{12} & c_{22} \\
      \end{pmatrix} = \begin{pmatrix}
        c_{11}c_{11} + c_{12}c_{12} & c_{11}c_{21} + c_{12}c_{22} \\
        c_{21}c_{11} + c_{22}c_{12} & c_{21}c_{21} + c_{22}c_{22} \\
      \end{pmatrix} = \begin{pmatrix}
        -1 & 0  \\
        0  & -1 \\
      \end{pmatrix}
    \]
    Und $c_{11}, c_{12}$ mit $c_{11}^2 + c_{12}^2 = -1$ existieren in
    $\mathbb{R}$ nicht.

  \item Sei
    \[
      D = \begin{pmatrix}
        d_{11} & d_{12} \\
        d_{21} & d_{22} \\
      \end{pmatrix},
      D^T = \begin{pmatrix}
        d_{11} & d_{21} \\
        d_{12} & d_{22} \\
      \end{pmatrix}
    \]
    Dann ist
    \[
      DD^T = \begin{pmatrix}
        d_{11} & d_{12} \\
        d_{21} & d_{22} \\
      \end{pmatrix}\begin{pmatrix}
        d_{11} & d_{21} \\
        d_{12} & d_{22} \\
      \end{pmatrix} = \begin{pmatrix}
        d_{11}d_{11} + d_{12}c_{12} & d_{11}d_{21} + d_{12}d_{22} \\
        d_{21}d_{11} + d_{22}d_{12} & d_{21}d_{21} + d_{22}d_{22} \\
      \end{pmatrix} = \begin{pmatrix}
        1 & -1 \\
        1 & 1  \\
      \end{pmatrix}
    \]
    Nun soll $d_{21}d_{11} + d_{22}d_{12}$ gleichzeitig 1 und -1 entsprechen, ein
    Widerspruch.
  \end{enumerate}
\end{enumerate}
\end{document}
