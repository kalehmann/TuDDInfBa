\documentclass{scrreprt}

\usepackage{amsmath}
\usepackage{amsthm}
\usepackage{amssymb}
\usepackage{bm}
\usepackage[shortlabels]{enumitem}
\usepackage{hyperref}
\usepackage[utf8]{inputenc}
\usepackage{multicol}
\usepackage{mathtools}
\usepackage{pdflscape}
\usepackage{physics}
\usepackage{polynom}
\usepackage{tabularx}
\usepackage[table]{xcolor}
\usepackage{titling}
\usepackage{fancyhdr}
\usepackage{xfrac}
\usepackage{pgfplots}

\pgfplotsset{compat = newest}
\usepgfplotslibrary{fillbetween}
\usetikzlibrary{arrows, arrows.meta}
\usetikzlibrary{calc}
\usetikzlibrary{patterns}

% See https://tex.stackexchange.com/a/118217
\DeclarePairedDelimiter\ceil{\lceil}{\rceil}


\author{Karsten Lehmann}
\date{WiSe 2024/25}
\title{Übungsblatt 2\\INF-B-110, Diskrete Strukturen}

\setlength{\headheight}{26pt}
\pagestyle{fancy}
\fancyhf{}
\lhead{\thetitle}
\rhead{\theauthor}
\lfoot{\thedate}
\rfoot{Seite \thepage}

\begin{document}

\paragraph{Ü 2.1} Es sei $f \colon S \to M$ eine Abbildung von einer Menge $S$ in
eine Menge $M$.
Zeigen Sie, dass für alle $A, B \subseteq S$ gilt:
$f\qty\big[A \cap B] \subseteq f\qty\big[A] \cap f\qty\big[B]$.

\subparagraph{Lsg.} Gemäß der Definition des \emph{Schnittes} von $A$ und $B$,
$A \cap B \coloneqq \qty\big{e \:{\big |}\: e \in A \text{ und } e \in B}$
folgen $A \cap B \subseteq A$ und $A \cap B \subseteq B$.
Dementsprechend gelten $f\qty\big[A \cap B] \subseteq f\qty\big[A]$ und
$f\qty\big[A \cap B] \subseteq f\qty\big[B]$.

Durch erneute Anwendung der Definition des \emph{Schnittes} von $A$ und $B$ folgt
die Behauptung.

\paragraph{Ü 2.2}
\begin{enumerate}[(a)]
\item Untersuchen Sie die folgenden Abbildungen $f$ darauf, ob Sie injektiv,
  surjektiv oder bijektiv sind, und bestimmen Sie die Umkehrabbildung $f^{-1}$,
  falls sie existiert:
  \begin{enumerate}[(1)]
  \item $f \colon \qty\big(0, \infty) \to \mathbb{R}, f\qty\big(x) = 1 + \frac{1}{2} \ln x$

    \subparagraph{Lsg.} Aus der Schulmathematik ist der Logarithmus bereits als
    bijektive Funktion mit der Umkehrfunktion $e^x$ bekannt.

    Seien nun $x_1, x_2 \in \qty\big(0, \infty)$, so dass
    $f\qty\big(x_1) = f\qty\big(x_2)$.
    Es folgt
    \begin{flalign*}
      1 + \frac{1}{2} \ln x_1 &= 1 + \frac{1}{2} \ln x_2 && {\Big |} -1 \\
      \frac{1}{2} \ln x_1 &= \frac{1}{2} \ln x_2 && {\Big |} \cdot 2 \\
      \ln x_1 &= \ln x_2 && {\Big |} e^{\qty\big(\ldots)}\\
      e^{\ln x_1} &= e^{\ln x_2} \\
      x_1 &= x_2
    \end{flalign*}

    $\Rightarrow f$ ist \textbf{injektiv}.

    Sei nun $y \in \mathbb{R}$ beliebig.
    Sei weiter $x \in \qty\big(0, \infty)$ mit
    \begin{flalign*}
      x &= e^{2y - 2} && {\Big |} \ln\qty\big(\ldots) \\
      \ln x &= 2y - 2 && {\Big |} \cdot \frac{1}{2} \\
      \frac{1}{2} \ln x &= y - 1 && {\Big |} + 1 \\
      1 + \frac{1}{2} \ln x &= y
    \end{flalign*}

    $\Rightarrow$ da $y$ beliebig, ist $f$ \textbf{injektiv}.

    Die Umkehrfunktion ist $f^{-1}\qty\big(y) = e^{2y - 2}$

  \newpage
  \item $f \colon \qty\big{0, 1} \times \qty\big{0, 1} \to \qty\big{-1, 0, 1}$
    mit $f\qty\big(x_1, x_2) = x_1 - x_2$

    \subparagraph{Lsg.} Seien $\qty\big(0, 0), \qty\big(1, 1) \in \qty\big{0, 1} \times \qty\big{0, 1}$.
    Dann ist $f\qty\big((0, 0)) = 0 = f\qty\big((1, 1))$.

    $\Rightarrow f$ ist \textbf{nicht injektiv}.

    Alternativ ist
    $\abs{\qty\big{0, 1} \times \qty\big{0, 1}} > \abs{\qty\big{-1, 0, 1}}$.
    Gemäß der Vorlesung ist für die Existenz einer Injektion Voraussetzung, dass
    $\abs{\qty\big{0, 1} \times \qty\big{0, 1}} \leq \abs{\qty\big{-1, 0, 1}}$.
    Somit kann $f$ keine Injektion sein.

    Weiter ist  $f\qty\big(0, 1) = -1, f\qty\big(0, 0) = 0, f\qty\big(1, 0) = 1$.

    $\Rightarrow$ es existiert für jedes $y \in \qty\big{-1, 0, 1}$ ein
    $x \in \qty\big{0, 1} \times \qty\big{0, 1}$, so dass $f\qty\big(x) = y$.

    $\Rightarrow f$ ist \textbf{surjektiv}.

  \item die Abbildung
    $f \colon \mathcal{P}\qty\big(N) \setminus \qty\big{\emptyset} \to \mathbb{N}$,
    die jeder nichtleeren Teilmenge $A \subseteq \mathbb{N}$ das kleinste
    Element in $A$ zuordnet.

    \subparagraph{Lsg.} Seien $\qty\big{1, 2}, \qty\big{1, 3} \in \mathcal{P}\qty\big(N) \setminus \qty\big{\emptyset}$.
    Dann ist $f\qty\big(\qty{1, 2}) = 1 = f\qty\big(\qty{1, 3})$.

    $\Rightarrow f$ ist \textbf{nicht injektiv}.

    Sei nun $n \in \mathbb{N}$ beliebig.
    Dann existiert
    $\qty\big{n} \in \mathcal{P}\qty\big(N) \setminus \qty\big{\emptyset}$
    mit $f\qty\big(\qty{n}) = n$.

    $\Rightarrow f$ ist \textbf{surjektiv}.

  \end{enumerate}
\item Es seien $A, B, C$ beliebige, nichtleere Mengen und
  $f \colon A \to B, g \colon B \to C$ zwei Abbildungen.
  Beweisen Sie:
  \begin{itemize}
  \item Wenn $g \circ f$ injektiv ist, dann ist auch $f$ injektiv.

    \subparagraph{Lsg.} Angenommen $f$ wäre nicht injektiv.
    Dann existieren $x_1 \ne x_2 \in A$ mit
    $f\qty\big(x_1) = f\qty\big(x_2)$.
    Damit wäre
    \[
      \qty\big(g \circ f)\qty\big(x_1) =
      g\qty\big(f(x_1)) =
      g\qty\big(f(x_2)) =
      \qty\big(g \circ f)\qty\big(x_2)
    \]
    Ein Widerspruch zur Voraussetzung $g \circ f$ ist injektiv.

    $\Rightarrow f$ ist ebenfalls injektiv.

  \item Wenn $g \circ f$ surjektiv ist, dann ist auch $g$ surjektiv.

    \subparagraph{Lsg.} Sei $y \in C$ beliebig.
    Da $g \circ f$ surjektiv, existiert $x \in A$ mit
    $\qty\big(g \circ f)\qty\big(x) = g\qty\big(f(x)) = y$.

    Sei weiter $b \in B$ mit $b \coloneqq f\qty\big(x)$.
    ($b$ muss existieren sonst Widerspruch zu $g \circ f$ surjektiv).
    Somit ist $g\qty\big(b) = y$.
    Da $y$ beliebig folgt die Behauptung.
  \end{itemize}
\end{enumerate}

\newpage
\paragraph{Ü 2.3} Es sei $U$ die Menge der ungeraden natürlichen Zahlen.
Zeigen Sie, dass $\abs{\mathbb{Z}} = \abs{U}$ gilt.

\subparagraph{Lsg.} Nach der Vorlesung existiert bereits eine Bijektion
$\mathbb{Z} \to \mathbb{N}$.
Somit reicht es eine Bijektion $\mathbb{N} \to U$ zu finden.
Sei $f \colon \mathbb{Z} \to U, x \mapsto 2x - 1$.
Seien weiter $x_1, x_2 \in \mathbb{Z}$ beliebig mit
$f\qty\big(x_1) = f\qty\big(x_2)$.
Somit ist
\begin{flalign*}
  2x_1 - 1 &= 2x_2 - 1 && {\Big |} + 1 &&&& \\
  2x_1 &= 2x_2 && {\Big |} \cdot \frac{1}{2} \\
  x_1 &= x_2
\end{flalign*}
$\Rightarrow f$ ist injektiv.

\noindent
Sei nun $y \in U$ beliebig.
Dann ist $y + 1$ eine gerade Zahl und somit $\frac{y + 1}{2} \in \mathbb{N}$.
Nun ist $f\qty(\frac{y + 1}{2}) = 2 \cdot \frac{y + 1}{2} - 1 = y$.

\noindent
$\Rightarrow$ für jedes $y \in U$ existiert
$x \coloneqq \frac{y + 1}{2} \in \mathbb{N}$ mit $f\qty\big(x) = y$.

\noindent
$\Rightarrow f$ ist surjektiv.

\noindent
$\Rightarrow f$ ist bijektiv.

\noindent
Gemäß dem Kapitel \emph{``Größenvergleich von Mengen''} aus der Vorlesung
sind zwei Mengen gleich mächtig, wenn eine Bijektion zwischen den Mengen
existiert.

\noindent
$\Rightarrow \abs{\mathbb{N}} = \abs{U}$
$\Rightarrow \abs{\mathbb{Z}} = \abs{U}$

\paragraph{Ü 2.4}
\begin{enumerate}[(a)]
\item Betrachtet werden folgende Permutationen der Menge
  $X \coloneqq \qty\big{1, 2, 3, 4, 5, 6, 7, 8, 9}$:
  \[
    \alpha \coloneqq \begin{pmatrix}
      1 & 2 & 3 & 4 & 5 & 6 & 7 & 8 & 9 \\
      2 & 5 & 8 & 7 & 1 & 6 & 4 & 3 & 9
    \end{pmatrix}, \qquad \beta \coloneqq
    \begin{pmatrix} 5 & 6 \end{pmatrix} \circ
    \begin{pmatrix} 3 & 5 & 6 & 9 \end{pmatrix} \circ
    \begin{pmatrix} 5 & 6 \end{pmatrix}
  \]
  \begin{itemize}
  \item Geben Sie $\alpha$ und $\beta$ und ihre Umkehrabbildung in
    Zyklenschreibweise an.

    \subparagraph{Lsg.} Es ist
    \[
      \alpha = \begin{pmatrix} 2 & 5 & 1 \end{pmatrix}
        \begin{pmatrix} 6 \end{pmatrix}
        \begin{pmatrix} 3 & 8 \end{pmatrix}
        \begin{pmatrix} 9 \end{pmatrix}
        \begin{pmatrix} 7 & 4 \end{pmatrix} =
        \begin{pmatrix} 2 & 5 & 1 \end{pmatrix}
        \begin{pmatrix} 3 & 8 \end{pmatrix}
        \begin{pmatrix} 7 & 4 \end{pmatrix}
    \]
    Weiter ist
    \begin{flalign*}
      \beta = \begin{pmatrix} 5 & 6 \end{pmatrix} \circ
      \begin{pmatrix} 3 & 5 & 6 & 9 \end{pmatrix} \circ
      \begin{pmatrix} 5 & 6 \end{pmatrix}
      &= \begin{pmatrix} 5 & 6 \end{pmatrix} \circ
      \begin{pmatrix} 5 & 6 & 9 & 3\end{pmatrix} \circ
      \begin{pmatrix} 5 & 6 \end{pmatrix} & \\
      &= \underset{=\text{id}}{\underbrace{\begin{pmatrix} 5 & 6 \end{pmatrix} \circ
      \begin{pmatrix} 5 & 6 \end{pmatrix}}}
      \begin{pmatrix} 6 & 9 \end{pmatrix}
      \begin{pmatrix} 9 & 3\end{pmatrix} \circ
      \begin{pmatrix} 5 & 6 \end{pmatrix} & \\
      &= \begin{pmatrix} 9 & 3 & 6 & 5 \end{pmatrix}
    \end{flalign*}

    Für die Umkehrabbildungen sind dann
    \[
      \alpha^{-1} = \begin{pmatrix} 4 & 7 \end{pmatrix}
      \begin{pmatrix} 8 & 3 \end{pmatrix}
      \begin{pmatrix} 1 & 5 & 2 \end{pmatrix}, \quad
      \beta^{-1} = \begin{pmatrix} 5 & 6 & 3 & 9 \end{pmatrix}
    \]

  \newpage
  \item Bestimmen Sie $\alpha \circ \beta$, $\alpha^2$, $\alpha^3$ und $\alpha^4$
    in Zyklenschreibweise.

    \subparagraph{Lsg.} Für $\alpha \circ \beta = \begin{pmatrix} 2 & 5 & 1 \end{pmatrix}
      \begin{pmatrix} 3 & 8 \end{pmatrix}
      \begin{pmatrix} 7 & 4 \end{pmatrix} \circ \begin{pmatrix} 9 & 3 & 6 & 5 \end{pmatrix}$ ist
    \begin{flalign*}
      1 &\mapsto 2 \\
      2 &\mapsto 5 \\
      5 &\mapsto 9 \\
      9 &\mapsto 3 \mapsto 8 \\
      8 &\mapsto 3 \\
      3 &\mapsto 6 \\
      6 &\mapsto 5 \mapsto 1 \\
      4 &\mapsto 7 \\
      7 &\mapsto 4
    \end{flalign*}
    Somit ist $\alpha \circ \beta = \begin{pmatrix}1 & 2 & 5 & 9 & 8 & 3 & 6\end{pmatrix}
    \begin{pmatrix} 4 & 7 \end{pmatrix}$.

    Außerdem ist
    \[
      \alpha^2 = \begin{pmatrix} 2 & 5 & 1 \end{pmatrix}^2
      \underset{=\text{id}}{\underbrace{\begin{pmatrix} 3 & 8 \end{pmatrix}^2}}
      \underset{=\text{id}}{\underbrace{\begin{pmatrix} 7 & 4 \end{pmatrix}^2}}
      = \begin{pmatrix} 2 & 1 & 5\end{pmatrix}
    \]
    Alternativ
    \[
      \alpha^2 = \begin{pmatrix} 2 & 5 & 1 \end{pmatrix}^{3 - 1}
      \underset{=\text{id}}{\underbrace{\begin{pmatrix} 3 & 8 \end{pmatrix}^2}}
      \underset{=\text{id}}{\underbrace{\begin{pmatrix} 7 & 4 \end{pmatrix}^2}}
      = \underset{=\text{id}}{\underbrace{\begin{pmatrix} 2 & 1 & 5 \end{pmatrix}^3}}
      \begin{pmatrix} 2 & 5 & 1 \end{pmatrix}^{-1}
      = \begin{pmatrix} 2 & 1 & 5\end{pmatrix}
    \]
    \[
      \alpha^3 = \underset{=\text{id}}{\underbrace{\begin{pmatrix} 2 & 5 & 1 \end{pmatrix}^3}}
      \begin{pmatrix} 3 & 8 \end{pmatrix}^{2 + 1}
      \begin{pmatrix} 7 & 4 \end{pmatrix}^{2 + 1}
      = \underset{=\text{id}}{\underbrace{\begin{pmatrix} 3 & 8 \end{pmatrix}^2}}
      \begin{pmatrix} 3 & 8 \end{pmatrix}
      \underset{=\text{id}}{\underbrace{\begin{pmatrix} 7 & 4 \end{pmatrix}^2}}
      \begin{pmatrix} 7 & 4 \end{pmatrix}
      = \begin{pmatrix} 3 & 8 \end{pmatrix} \begin{pmatrix} 7 & 4 \end{pmatrix}
    \]
    \begin{flalign*}
      \alpha^4 &= \begin{pmatrix} 2 & 5 & 1 \end{pmatrix}^{3 + 1}
      \begin{pmatrix} 3 & 8 \end{pmatrix}^{2 + 2}
      \begin{pmatrix} 7 & 4 \end{pmatrix}^{2 + 2} & \\
      &= \underset{=\text{id}}{\underbrace{\begin{pmatrix} 2 & 5 & 1 \end{pmatrix}^3}}
      \begin{pmatrix} 2 & 5 & 1 \end{pmatrix}
      \underset{=\text{id}}{\underbrace{\begin{pmatrix} 3 & 8 \end{pmatrix}^2}}
      \underset{=\text{id}}{\underbrace{\begin{pmatrix} 3 & 8 \end{pmatrix}^2}}
      \underset{=\text{id}}{\underbrace{\begin{pmatrix} 7 & 4 \end{pmatrix}^2}}
      \underset{=\text{id}}{\underbrace{\begin{pmatrix} 7 & 4 \end{pmatrix}^2}} \\
      &= \begin{pmatrix} 2 & 5 & 1 \end{pmatrix}
    \end{flalign*}

  \item Wie kann man $\alpha^{2024}$ effizient berechnen?

    \subparagraph{Lsg.} Es ist
    \begin{flalign*}
      \alpha^{2024} &= \begin{pmatrix} 2 & 5 & 1 \end{pmatrix}^{2025 - 1}
      \begin{pmatrix} 3 & 8 \end{pmatrix}^{2024}
      \begin{pmatrix} 7 & 4 \end{pmatrix}^{2024} & \\
      &= \underset{=\text{id}}{\underbrace{\begin{pmatrix} 2 & 5 & 1 \end{pmatrix}^3}}^k
      \begin{pmatrix} 2 & 5 & 1 \end{pmatrix}^{-1}
      \underset{=\text{id}}{\underbrace{\begin{pmatrix} 3 & 8 \end{pmatrix}^2}}^l
      \underset{=\text{id}}{\underbrace{\begin{pmatrix} 7 & 4 \end{pmatrix}^2}}^l \\
      &= \begin{pmatrix} 2 & 1 & 5 \end{pmatrix}
    \end{flalign*}

  \newpage
  \item Stellen Sie die Permutation $\beta$ als Komposition von Transpositionen
    dar.

    \subparagraph{Lsg.} Es ist
    \[
      \beta = \qty\big(3\;6) \circ \qty\big(6\;5) \circ \qty\big(5\;9)
    \]
  \end{itemize}

\item Beschreiben Sie die Symmetrieabbildungen, die ein gleichseitiges Dreieck
  auf sich selbst abbilden, durch Permutationen der Eckpunkte in
  Zyklenschreibweise.
  Dabei seien die Eckpunkte entgegen dem Uhrzeigersinn mit den Ziffern 1, 2 und
  3 bezeichnet.

  \subparagraph{Lsg.} Das gleichseitige Dreieck ist sowohl punkt- als auch
  Achsensymmetrisch.
  Für die einzelnen Achsen sind die Abbildungen:

  \begin{tikzpicture}[scale=2]
    \node[label = above:{1}] (A) at (0, 0) {};
    \node[label = below left:{2}] (B) at (-0.5, -0.866) {};
    \node[label = below right:{3}] (C) at (0.5, -0.866) {};
    \draw (A.center) -- (B.center) -- (C.center) -- (A.center);
    \node[label = {[blue] above right:{$\begin{pmatrix} 1 & 3 \end{pmatrix}$}}] (D) at (1, 0) {};
    \node[label = {[blue] above left:{$\begin{pmatrix} 1 & 2 \end{pmatrix}$}}] (E) at (-1, 0) {};
    \node[label = {[blue] below:{$\begin{pmatrix} 2 & 3 \end{pmatrix}$}}] (F) at (0, -1.722) {};
    \draw[dashed, blue] (A) -- (F);
    \draw[dashed, blue] (B) -- (D);
    \draw[dashed, blue] (C) -- (E);
  \end{tikzpicture}

  Die beiden Punktsymmetrischen Abbildungen sind
  $\begin{pmatrix} 1 & 2 & 3 \end{pmatrix}$ für die Rotation gegen den
  Uhrzeigersinn und
  $\begin{pmatrix} 1 & 3 & 2 \end{pmatrix}$ für die Rotation im Uhrzeigersinn.

  Schließlich verbleibt noch die Abbildung $\text{id}$, welche das Dreieck auf
  sich selbst abbildet.
\end{enumerate}

\end{document}
