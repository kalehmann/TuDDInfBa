\documentclass{scrreprt}

\usepackage{aligned-overset}
\usepackage{amsmath}
\usepackage{amsthm}
\usepackage{amssymb}
\usepackage{bm}
\usepackage[inline, shortlabels]{enumitem}
\usepackage{hyperref}
\usepackage[utf8]{inputenc}
\usepackage{listings}
\usepackage{multicol}
\usepackage{mathtools}
\usepackage{pdflscape}
\usepackage{physics}
\usepackage{polynom}
\usepackage{tabularx}
\usepackage[table]{xcolor}
\usepackage{titling}
\usepackage{fancyhdr}
\usepackage{xfrac}
\usepackage{pgfplots}

\pgfplotsset{compat = newest}
\usepgfplotslibrary{fillbetween}
\usetikzlibrary{arrows, arrows.meta}
\usetikzlibrary{calc}
\usetikzlibrary{patterns}

\author{Karsten Lehmann}
\date{WiSe 2024/25}
\title{Übungsblatt 8\\INF-B-110, Diskrete Strukturen}

\setlength{\parindent}{0pt}

\setlength{\headheight}{26pt}
\pagestyle{fancy}
\fancyhf{}
\lhead{\thetitle}
\rhead{\theauthor}
\lfoot{\thedate}
\rfoot{Seite \thepage}

\newcommand{\ggT}[0]{\text{ggT}}
\DeclarePairedDelimiter{\floor}{\lfloor}{\rfloor}

\begin{document}

\paragraph{Ü 8.1} Betrachtet wird die Einheitengruppe
$\qty\big(\mathbb{Z}_{14}^*; \cdot)$
\begin{enumerate}[(a)]
\item Stellen Sie die Verknüpfungstafel für $\qty\big(\mathbb{Z}_{14}^*; \cdot)$
  auf.

  \subparagraph{Lsg.}\phantom{\null}

  \begin{tabular}{|c|cccccc|}
    \hline
    $\cdot$ & 1  & 3  & 5  & 9  & 11 & 13 \\
    \hline
    1       & 1  & 3  & 5  & 9  & 11 & 13 \\
    3       & 3  & 9  & 1  & 13 & 5  & 11 \\
    5       & 5  & 1  & 11 & 3  & 13 & 9  \\
    9       & 9  & 13 & 3  & 11 & 1  & 5  \\
    11      & 11 & 5  & 13 & 1  & 9  & 3  \\
    13      & 13 & 11 & 9  & 5  & 3  & 1  \\
    \hline
  \end{tabular}

\item Für welche $k \in \mathbb{N}$ kann $\qty\big(\mathbb{Z}_{14}^*; \cdot)$
  nach dem Satz von Lagrange Untergruppen der Ordnung $k$ besitzen?

  \subparagraph{Lsg.} Es ist $\abs{\qty\big(\mathbb{Z}_{14}^*; \cdot)} = 6$ und
  $6 = 2 \cdot 3$.
  Somit sind die Teiler von 6 die Zahlen 1, 2, 3, 6 und
  $\qty\big(\mathbb{Z}_{14}^*; \cdot)$ kann Untergruppen der Ordnung 1, 2, 3
  und 6 enthalten.

\item Bestimmen Sie die Untergruppe $\big\langle 3 \big\rangle$ von
  $\qty\big(\mathbb{Z}_{14}^*; \cdot)$.
  Was können Sie schlussfolgern?

  \subparagraph{Lsg.} Es sind
  \begin{itemize}
  \item $3^1 = 3 \qty\big(\mod 14)$
  \item $3^2 = 9 \qty\big(\mod 14)$
  \item $3^3 = 13 \qty\big(\mod 14)$
  \item $3^4 = 11 \qty\big(\mod 14)$
  \item $3^5 = 5 \qty\big(\mod 14)$
  \item $3^6 = 1 \qty\big(\mod 14)$
  \end{itemize}

  Somit ist die 3 eine Primitivwurzel / ein Erzeuger der Gruppe
  $\qty\big(\mathbb{Z}_{14}^*; \cdot)$ und die Gruppe ist zyklisch.

\item Finden Sie alle weiteren Untergruppen von
  $\qty\big(\mathbb{Z}_{14}^*; \cdot)$.

  \subparagraph{Lsg.} Es sind Untergruppen:
  \begin{itemize}
  \item $\qty\big{1}$
  \item $\qty\big{1, 13}$
  \item $\qty\big{1, 9, 11}$
  \item $\qty\big(\mathbb{Z}_{14}^*; \cdot)$
  \end{itemize}

  \textbf{Alternativ nach der Übung:} Nach Teilaufgabe (b) ist bereits bekannt,
  dass Untergruppen nur die Ordnungen 1, 2,3 und 6 haben können.
  Dabei sind die Untergruppen mit der Ordnung 1 und 6 trivial zu finden.
  Nun gilt für Untergruppen der Ordnung 2 und 3 nach dem Satz von Lagrange
  wieder, dass die Ordnung jedes Elementes wieder ein Teiler der Kardinalität der
  Gruppe ist.

  Somit enthält eine Untergruppe der Kardinalität 2 nur die Elemente der
  Ordnungen 1 und 2.
  Außerdem enthält eine Untergruppe der Kardinalität 2 nur die Elemente der
  Ordnungen 1 und 3.
  Diese Elemente finden sich nun leicht in der Verknüpfungstafel.

\item Bestimmen Sie die Ordnung jedes Elementes aus
  $\qty\big(\mathbb{Z}_{14}^*; \cdot)$.

  \subparagraph{Lsg.} Die Ordnung eines Elementes
  $g \in \qty\big(\mathbb{Z}_{14}^*; \cdot)$ ist die kleinste natürliche Zahl
  $a$ mit $g^a = 1 \qty\big(\mod 14)$.

  Nun sind
  \begin{itemize}
  \item $1^1 = 1$
  \item $3^6 = 1$
  \item $5^6 = 1$
  \item $9^3 = 1$
  \item $11^3 = 1$
  \item $13^2 = 1$
  \end{itemize}

\item Bestimmen Sie für jede Untergruppe von $\qty\big(\mathbb{Z}_{14}^*; \cdot)$
  alle ihre Linksnebenklassen.

  \subparagraph{Lsg.}\phantom{\null}
  \begin{enumerate}[(1)]
  \item Für $\qty\big{1}$:
    \begin{itemize}
    \item $1 \cdot \qty\big{1} = \qty\big{1}$
    \item $3 \cdot \qty\big{1} = \qty\big{3}$
    \item $5 \cdot \qty\big{1} = \qty\big{5}$
    \item $9 \cdot \qty\big{1} = \qty\big{9}$
    \item $11 \cdot \qty\big{1} = \qty\big{11}$
    \item $13 \cdot \qty\big{1} = \qty\big{13}$
    \end{itemize}

    $\Rightarrow$ Menge der Linksnebenklassen von $\qty\big{1}$ ist
    \[
      \qty\Big{
        \qty\big{1},
        \qty\big{3},
        \qty\big{5},
        \qty\big{9},
        \qty\big{11},
        \qty\big{13}
      }
    \]
  \newpage
  \item Für $\qty\big{1, 13}$:
    \begin{itemize}
    \item $1 \cdot \qty\big{1, 13} = \qty\big{1, 13}$
    \item $3 \cdot \qty\big{1, 13} = \qty\big{3, 11}$
    \item $5 \cdot \qty\big{1, 13} = \qty\big{5, 9}$
    \item $9 \cdot \qty\big{1, 13} = \qty\big{5, 9}$
    \item $11 \cdot \qty\big{1, 13} = \qty\big{3, 11}$
    \item $13 \cdot \qty\big{1, 13} = \qty\big{1, 13}$
    \end{itemize}

    $\Rightarrow$ Menge der Linksnebenklassen von $\qty\big{1, 13}$ ist
    \[
      \qty\Big{
        \qty\big{1, 13},
        \qty\big{3, 11},
        \qty\big{5, 9}
      }
    \]

  \item Für $\qty\big{1, 9, 11}$:
    \begin{itemize}
    \item $1 \cdot \qty\big{1, 9, 11} = \qty\big{1, 9, 11}$
    \item $3 \cdot \qty\big{1, 9, 11} = \qty\big{3, 5, 13}$
    \item $5 \cdot \qty\big{1, 9, 11} = \qty\big{3, 5, 13}$
    \item $9 \cdot \qty\big{1, 9, 11} = \qty\big{1, 9, 11}$
    \item $11 \cdot \qty\big{1, 9, 11} = \qty\big{1, 9, 11}$
    \item $13 \cdot \qty\big{1, 9, 11} = \qty\big{3, 5, 13}$
    \end{itemize}

    $\Rightarrow$ Menge der Linksnebenklassen von $\qty\big{1, 9, 11}$ ist
    \[
      \qty\Big{
        \qty\big{1, 9, 11},
        \qty\big{3, 5, 13}
      }
    \]

  \item Für $g \in \qty\big(\mathbb{Z}_{14}^*; \cdot)$ ist
    $g \cdot \qty\big(\mathbb{Z}_{14}^*; \cdot) = \qty\big(\mathbb{Z}_{14}^*; \cdot)$

    $\Rightarrow$ Menge der Linksnebenklassen von $\mathbb{Z}_{14}^*$ ist
    \[
      \qty\Big{\mathbb{Z}_{14}^*}
    \]

  \end{enumerate}

  \textbf{Aus der Übung:} Es ist $\qty\big[G, H] = \frac{\abs{G}}{\abs{H}}$
  der Index von $G$ und $H$ gleich der Menge von Linksnebenklassen von $H$
  unter $G$.
\end{enumerate}

\newpage
\paragraph{Ü 8.2} Bestimmen Sie für die Einheitengruppe
$\qty\big(\mathbb{Z}_{24}^*; \cdot)$ eine Untergruppe $U$ mit
$7 \in U$ und $\abs{U} = 4$.
Wie viele Linksnebenklassen hat $U$?
Ist $\qty\big(U; \cdot)$ zu $\qty\big(\mathbb{Z}_4; +)$ isomorph?

\subparagraph{Lsg.} Es sind
\begin{itemize}
\item $1 \cdot 1 \equiv 1 \qty\big(\mod 24)$
\item $7 \cdot 1 \equiv 7 \qty\big(\mod 24)$
\item $7 \cdot 7 \equiv 1 \qty\big(\mod 24)$
\end{itemize}
Für eine Untergruppe mit vier Elementen werden nun noch zwei weitere Elemente
benötigt.
Nun lassen sich alle Elemente von $\qty\big(\mathbb{Z}_{24}^*; \cdot)$
ermitteln durch
\begin{lstlisting}
>>> [i for i in range(24) for j in range(24) if i*j%24==1]
[1, 5, 7, 11, 13, 17, 19, 23]
\end{lstlisting}

Daraus folgen als mögliche Untergruppen mit 4 Elementen inklusive 1 und 7:

\begin{minipage}{.3\textwidth}
  \begin{tabular}{|c|cccc|}
    \hline
    $\cdot$ & 1  & 5  & 7  & 11 \\
    \hline
    1       & 1  & 5  & 7  & 11 \\
    5       & 5  & 1  & 11 & 7 \\
    7       & 7  & 11 & 1  & 5 \\
    11      & 11 & 7  & 5  & 1 \\
    \hline
  \end{tabular}
\end{minipage}
\begin{minipage}{.3\textwidth}
  \begin{tabular}{|c|cccc|}
    \hline
    $\cdot$ & 1  & 7  & 13 & 19 \\
    \hline
    1       & 1  & 7  & 13 & 19 \\
    7       & 7  & 1  & 19 & 13 \\
    13      & 13 & 19 & 1  & 7 \\
    19      & 19 & 13 & 7  & 1 \\
    \hline
  \end{tabular}
\end{minipage}
\begin{minipage}{.3\textwidth}
  \begin{tabular}{|c|cccc|}
    \hline
    $\cdot$ & 1  & 7  & 17 & 23 \\
    \hline
    1       & 1  & 7  & 17 & 23 \\
    7       & 7  & 1  & 23 & 17 \\
    17      & 17 & 23 & 1  & 7 \\
    23      & 23 & 17 & 7  & 1 \\
    \hline
  \end{tabular}
\end{minipage}

Nun besitzt die erste Gruppe die folgenden beiden Linksnebenklassen:
\begin{enumerate}[(1)]
\item $1 \cdot \qty\big{1, 5, 7, 11}
  = 5 \cdot \qty\big{1, 5, 7, 11}
  = 7 \cdot \qty\big{1, 5, 7, 11}
  = 11 \cdot \qty\big{1, 5, 7, 11}
  = \qty\big{1, 5, 7, 11}$

\item $13 \cdot \qty\big{1, 5, 7, 11} =
  17 \cdot \qty\big{1, 5, 7, 11} =
  19 \cdot \qty\big{1, 5, 7, 11} =
  23 \cdot \qty\big{1, 5, 7, 11} =
  \qty\big{13, 17, 19, 23}$
\end{enumerate}

Weiter ist $1$ eine Primitivwurzel von $\qty\big(\mathbb{Z}_4; +)$.
Angenommen $U$ wäre nun zu $\qty\big(\mathbb{Z}_4; +)$ isomorph, dann gäbe es
einen Isomorphismus $f \colon \qty\big(\mathbb{Z}_4; +) \to \qty\big(U; \cdot)$.
Sei nun $f\qty\big(1) = u$ mit $u \in U$.
Dann wären

\newlist{inline}{enumerate*}{1}
\setlist[inline]{itemjoin = \hspace{1cm}, label=$\bullet$}

\begin{inline}
\item $f\qty\big(1) = u$
\item $f\qty\big(1 + 1) = u^2$
\item $f\qty\big(1 + 1 + 1) = u^3$
\item $f\qty\big(1 + 1 + 1 + 1) = u^4$
\end{inline}

Somit wäre $u$ ebenfalls eine Primitivwurzel in $U$, aus den obigen
Verknüpfungstafeln ist allerdings ersichtlich, dass $U$ keinen Erzeuger hat.
Ein Widerspruch und die Gruppen sind nicht isomorph.

\paragraph{Ü 8.3} Es wird die Teilmenge $U = \qty\big{1, i, -1, -i}$ der
komplexen Zahlen $\mathbb{C}$ betrachtet.
\begin{enumerate}[(a)]
\item Zeigen Sie, dass $U$ bzgl. der üblichen Multiplikation in den komplexen
  Zahlen eine Untergruppe von $\qty\big(\mathbb{C} \setminus \qty\big{0}; \cdot)$
  bildet.

  \subparagraph{Lsg.} Es ist $1$ das neutrale Element der Multiplikation in
  $\qty\big(\mathbb{C} \setminus \qty\big{0}; \cdot)$ und in $U$ enthalten.

  \begin{tabular}{|c|cccc|}
    \hline
    $\cdot$ & 1    & $i$  & $-1$ & $-i$ \\
    \hline
    1       & 1    & $i$  & $-1$ & $-i$ \\
    $i$     & $i$  & $-1$ & $-i$ & $-1$ \\
    $-1$    & $-1$ & $-i$ & 1    & $i$  \\
    $-i$    & $-i$ & 1    & $i$  & $-1$ \\
    \hline
  \end{tabular}

  Somit ist $U$ bezüglich der Multiplikation und $^{-1}$ abgeschlossen.

  $\Rightarrow U$ ist eine Untergruppe von
  $\qty\big(\mathbb{C} \setminus \qty\big{0}; \cdot)$.

\item Geben Sie für das Element $1 + i$ die Linksnebenklasse von $U$ in
  $\qty\big(\mathbb{C} \setminus \qty\big{0}; \cdot)$ an und skizzieren Sie diese
  Menge in der Gausschen Zahlenebene.

  \subparagraph{Lsg.} Es ist $1 + i \in \mathbb{C}$ und
  \begin{flalign*}
    \qty\big{1 + i, -1 + i, -1 - i, 1 - i}
    &= \qty\big(1 + i) \cdot \qty\big{1, i, -1, -i} \\
    &= \qty\big(1 - i) \cdot \qty\big{1, i, -1, -i} \\
    &= \qty\big(-1 - i) \cdot \qty\big{1, i, -1, -i} \\
    &= \qty\big(-1 + i) \cdot \qty\big{1, i, -1, -i}
  \end{flalign*}

  \begin{tikzpicture}[scale=1]
    \begin{axis}[
      axis equal image,
      axis x line=center,
      axis y line=center,
      grid=both,
      xmin=-1.2,
      xmax=1.2,
      xtick distance=1,
      ymin=-1.2,
      ymax=1.2,
      ytick={-1, 1},
      yticklabels={$-i$,$i$},
      ytick distance=1,
    ]
      \node[circle,fill=black,inner sep=0pt,minimum size=3pt,label=below:{$1 + i$}] at (1, 1) {};
      \node[circle,fill=black,inner sep=0pt,minimum size=3pt,label=below:{$-1 + i$}] at (-1, 1) {};
      \node[circle,fill=black,inner sep=0pt,minimum size=3pt,label=below:{$-1 - i$}] at (-1, -1) {};
      \node[circle,fill=black,inner sep=0pt,minimum size=3pt,label=below:{$1 - i$}] at (1, -1) {};
    \end{axis}
  \end{tikzpicture}

\item Welche Ordnungen haben die Elemente von $U$.

  \subparagraph{Lsg.} Es sind
  \begin{itemize}
  \item $1^1 = 1$
  \item $i^4 = 1$
  \item $\qty\big(-1)^2 = 1$
  \item $\qty\big(-i)^4 = 1$
  \end{itemize}

\item Ist $\qty\big(U; \cdot)$ zu $\qty\big(\mathbb{Z}_4; +)$ isomorph?

  \subparagraph{Lsg.} Sei $f \colon U \to \mathbb{Z}_4$
  mit $f\qty\big(1) = i$ linear.
  Da $i$ mit der Ordnung 4 ein Erzeuger von $U$ ist, ist $f$ eine Bijektion
  und ein Gruppenisomorphismus.
\end{enumerate}

\newpage
\paragraph{Ü 8.4} Verwenden Sie bei den folgenden Aufgaben, wenn möglich, den
Satz von Euler-Fermat:
\begin{enumerate}[(a)]
\item Berechnen Sie:
  \[
    42^{555} \mod 11, \qquad
    2^{25} \mod 12, \qquad
    13^{842} \mod 1225
  \]

  \subparagraph{Lsg.}
  \begin{enumerate}[(1)]
  \item Es ist 11 eine Primzahl und 42 nicht durch 11 teilbar.
    Also gilt nach dem kleinen Fermat, dass $42^{10} \mod 11 = 1$.
    Folglich ist
    \[
      42^{555} = \qty(42^{10})^{55} \cdot 42^5 \equiv 42^5
      \equiv 1 \qty\big(\mod 11)
    \]

  \item Da 12 weder eine Primzahl ist, noch 2 teilerfremd zu 12 ist, lässt sich
    der Satz von Euler-Fermat hier nicht anwenden.
    Allerdings ist $12 \cdot 21 = 252$ und somit
    $2^8 = 256 \equiv 4 \qty\big(\mod 12)$.
    Es folgt
    \[
      2^{25} = \qty\big(2^8)^3 \cdot 2
      \equiv 4 \cdot 4 \cdot 4 \cdot 2
      \equiv 4 \cdot 8
      \equiv 8 \qty\big(\mod 12)
    \]

    \textbf{Alternativ nach der Übung:} Chinesischer Restsatz.
    \begin{flalign*}
      12 &= 4 \cdot 3 \\
      2^{25} &\equiv 0 \mod 4 && \text{(Trivial)}\\
      2^{25} &\equiv 2 \mod 3 && \text{(Erler-Fermat)}
    \end{flalign*}
    $\overset{\text{Chin. Restsatz}}\Rightarrow 2^{25} \equiv 8 \mod 12$

    (Es gilt $2 = 0 \mod 4$ und $2 = 2 \mod 3$ und $2^{25} = 0 \mod 4$ und
    $2^25 = 2 \mod 3$.
    Nach dem Chinesischen Restsatz existiert ein $x \in \mathbb{Z}_{12}$ mit
    $x = 2^{25} \mod 4$ und $x = 2^{25} \mod 3$.
    Somit ist $x = 8$)

  \item Es ist $1225 = 5^2 \cdot 7^2$.
    Somit sind 1225 und 13 teilerfremd.
    Weiter ist $\phi\qty\big(1225)
    = \qty\big(5 - 1) \cdot 5^{2 - 1} \cdot \qty\big(7 - 1) \cdot 7^{2 - 1}
    = 4 \cdot 5 \cdot 6 \cdot 7 = 840$.
    Nun ist nach Euler-Fermat $13^{840} = 1$ und somit
    \[
      13^{842} = 13^{840} \cdot 13^2 \equiv 13^2 \equiv 169 \qty\big(\mod 1225)
    \]
  \end{enumerate}

\newpage
\item Finden Sie alle $x \in \mathbb{Z}_{40}$, welche die Gleichung
  $123^{321} \cdot x \equiv 4 \qty\big(\mod 40)$ erfüllen.

  \subparagraph{Lsg.} Es ist $40 = 2^3 \cdot 5$ und $123 = 3 \cdot 41$.
  Somit sind $40$ und $41$ zueinander teilerfremd.
  Weiter ist $\phi\qty\big(40)
  = \qty\big(2 - 1) \cdot 2^{3 - 1} \cdot \qty\big(5 - 1) \cdot 5^{1 - 1}
  = 2^2 \cdot 4  = 16$ und auch $320 = 16 \cdot 20$.
  Schließlich ist
  \[
    123^{321} = \qty(123^{16})^{20} \cdot 123 \equiv 123 \equiv 3 \qty\big(\mod 40)
  \]
  Nun bleibt die Frage für welches $x \in \mathbb{Z}_{40}$ ist
  $3 \cdot x \equiv 4 \qty\big(\mod 40)$?

  Offensichtlich ist $3 \cdot \qty\big(x - 1) \equiv 1 \qty\big(\mod 40)$
  und da 3 zu 40 teilerfremd ist $3^{16} \equiv 1 \qty\big(\mod 40)$.
  Es folgt $3 \cdot 3^{15} \equiv 1 \qty\big(\mod 40)$.

  $\Rightarrow 3^{15}$ ist Inverses zu 3 in $\mathbb{Z}_{40}$.

  Nun ist $3^{15} = \qty\big(3^{4})^3 \cdot 3^3
  \equiv 1 \cdot 1 \cdot 1 \cdot 3^3
  \equiv 27 \qty\big(\mod 40)$ und somit $3 \cdot 27 \equiv 1 \qty\big(\mod 40)$
  und endlich $3 \cdot 28 \equiv 4 \qty\big(\mod 40)$.

  Da $\qty\big(\mathbb{Z}_{40}; \cdot)$ eine Gruppe und somit das Inverse von 3
  eindeutig ist, folgt $x = 28$ als einzige Lösung.

  \textbf{Alternativ nach der Übung:} $123 \equiv 3 \mod 40$ und
  $\ggT\qty\big(3, 40) = 1$.
  Somit lässt sich Euler-Fermat anwenden:
  \[
    123^{321} \equiv 3^{321} \equiv \qty(3^{16})^{20} \cdot 3 \equiv 3 \qty\big(\mod 40)
  \]
  Das heißt wir suchen ein $x \in \mathbb{Z}_{40}$ mit
  $3 \cdot x \equiv 4 \qty\big(\mod 40)$.

\item Zeigen Sie für alle $m \in \mathbb{N}$, dass $m^{13}$ und $m$ im
  Dezimalsystem die gleiche Endziffer haben.

  \subparagraph{Lsg.} Die Endziffer erhält man durch die Operation $\mod 10$.
  Somit ist zu zeigen, dass $m^{13} \equiv m \qty\big(\mod 10)$.
  Dabei besteht allerdings das Problem, dass $\ggT\qty\big(m, 10) = 1$ nicht
  gewährleistet werden kann.

  Zur Lösung benötigt man den Chinesischen Restsatz.
  Es ist $10 = 2 \cdot 5$ und somit genügt es zu zeigen, dass für alle
  $m \in \mathbb{N}$ gilt, dass
  \begin{itemize}
    \item $m^{13} \equiv m \mod 2$ und
    \item $m^{13} \equiv m \mod 5$
  \end{itemize}
  Dann ist nach dem chinesischen Restsatz auch $m^{13} \equiv m \mod 10$
  \begin{flalign*}
    m^{13} &\equiv \begin{cases}
      0 & m \text{ ist gerade} \\
      1 & m \text{ ist ungerade}
    \end{cases}
    \equiv m \qty\big(\mod 2) \\
    m^{13} &\equiv \begin{cases}
      0 & 5|m \\
      m & \ggT\qty\big(m, 5) = 1
    \end{cases}
    \equiv m \qty\big(\mod 5)
  \end{flalign*}

\end{enumerate}
\end{document}
