\documentclass{scrreprt}

\usepackage{aligned-overset}
\usepackage{amsmath}
\usepackage{amsthm}
\usepackage{amssymb}
\usepackage{bm}
\usepackage[inline,shortlabels]{enumitem}
\usepackage{framed}
\usepackage{hyperref}
\usepackage[utf8]{inputenc}
\usepackage{multicol}
\usepackage{mathtools}
\usepackage{pdflscape}
\usepackage{physics}
\usepackage{polynom}
\usepackage{tabularx}
\usepackage[table]{xcolor}
\usepackage{titling}
\usepackage{fancyhdr}
\usepackage{xfrac}
\usepackage{pgfplots}

\pgfplotsset{compat = newest}
\usepgfplotslibrary{fillbetween}
\usetikzlibrary{calc}


\author{Karsten Lehmann}
\date{WiSe 2024/25}
\title{Übungsblatt 14\\INF-B-110, Lineare Algebra}

\setlength{\parindent}{0pt}

\setlength{\headheight}{26pt}
\pagestyle{fancy}
\fancyhf{}
\lhead{\thetitle}
\rhead{\theauthor}
\lfoot{\thedate}
\rfoot{Seite \thepage}

\begin{document}
\paragraph{Ü 14.2 Bestapproximation, Normalgleichungssystem}

Das lineare Gleichungssystem $Ax = b$ mit
\[
  A = \begin{pmatrix}
    4 & 0 \\
    0 & 2 \\
    1 & 1 \\
  \end{pmatrix} \in \mathbb{R}^{3 \times 2}
  \text{ und }
  b = \begin{pmatrix}
    2  \\
    0  \\
    11 \\
  \end{pmatrix} \in \mathbb{R}^3
\]
besitzt keine reelle Lösung.
Gesucht ist der Vektor $\hat{x} \in \mathbb{R}^2$ für den bezüglich der
Standardbasis im $\mathbb{R}^2$ gilt
\[
  \norm{b - A\hat{x}} = \min_{x \in \mathbb{R}^2}\norm{b - Ax}
\]
\begin{enumerate}[(a)]
\item Berechnen Sie $\hat{x}$ durch Bestapproximation.

  \subparagraph{Lsg.} Es ist $\qty\big(4, 0, 1)^T \bullet \qty\big(0, 2, 1) = 1$
  und damit sind die Spalten der Matrix $A$ nicht orthogonal zueinander.
  im ersten Schritt muss eine Orthogonalbasis für den Spaltenraum der Matrix
  $A$ bestimmt werden.
  Nun ist
  \begin{flalign*}
    b_1 &= \qty\big(0, 2, 1)^T \\
    b_2 &= \qty\big(4, 0, 1)^T - \frac{\qty\big(4, 0, 1)^T \bullet \qty\big(0, 2, 1)^T}{\qty\big(0, 2, 1)^T \bullet \qty\big(0, 2, 1)^T}\qty\big(0, 2, 1)^T \\
        &= \qty\big(4, 0, 1)^T - \frac{1}{5}\qty\big(0, 2, 1)^T \\
        &= \frac{1}{5}\qty\big(20, -2, 4)^T
  \end{flalign*}
  und $B = \qty\big(b_1, b_2)$ eine Orthogonalbasis des Spaltenraumes von $A$.
  Weiter ist
  \begin{flalign*}
    \hat{b} &= \text{proj}_{\text{Col}\qty\big(A)}b \\
            &= \frac{\qty\big(2, 0, 11)^T \bullet \qty\big(0, 2, 1)^T}{\qty\big(0, 2, 1)^T \bullet \qty\big(0, 2, 1)^T}\qty\big(0, 2, 1)^T
              + \frac{\qty\big(2, 0, 11)^T \bullet \frac{1}{5}\qty\big(20, -2, 4)^T}{\frac{1}{5}\qty\big(20, -2, 4)^T \bullet \frac{1}{5}\qty\big(20, -2, 4)^T}\frac{1}{5}\qty\big(20, -2, 4)^T \\
            &= \frac{11}{5}\qty\big(0, 2, 1)^T + \frac{\frac{84}{5}}{\frac{84}{5}}\frac{1}{5}\qty\big(20, -2, 4)^T \\
            &= \frac{1}{5}\qty\big(20, 20, 15)^T = \qty\big(4, 4, 3)^T
  \end{flalign*}
  Schließlich ist eine Lösung
  \[
    \begin{pmatrix}
      4 & 0 \\
      0 & 2 \\
      1 & 1 \\
    \end{pmatrix} \cdot \begin{pmatrix}
      x \\
      y \\
    \end{pmatrix} =  \begin{pmatrix}
      4 \\
      4 \\
      3 \\
    \end{pmatrix}
  \]
  gesucht und diese ist offensichtlich $\hat{x} = \qty\big(1, 2)^T$.

\newpage
\item Berechnen Sie $\hat{x}$ durch Lösung der Normalgleichung.

  \subparagraph{Lsg.} Gesucht ist eine Lösung des Normalengleichungssystems
  $\qty(A^TA)\hat{x} = A^Tb$.
  Nun ist
  \begin{flalign*}
    \qty(A^TA)\hat{x}
    &= A^Tb \\
    \qty(\begin{pmatrix}
      4 & 0 & 1 \\
      0 & 2 & 1 \\
    \end{pmatrix}\begin{pmatrix}
      4 & 0 \\
      0 & 2 \\
      1 & 1 \\
    \end{pmatrix})\hat{x}
    &= \begin{pmatrix}
      4 & 0 & 1 \\
      0 & 2 & 1 \\
    \end{pmatrix}\begin{pmatrix}
    2  \\
    0  \\
    11 \\
    \end{pmatrix} \\
    \begin{pmatrix}
      17 & 1 \\
      1  & 5 \\
    \end{pmatrix}\hat{x}
    &= \begin{pmatrix}
      19 \\
      11 \\
    \end{pmatrix}
  \end{flalign*}
  und $\hat{x} = \qty\big(1, 2)^T$
\end{enumerate}

\paragraph{Ü 14.3 Ein Normalgleichungssystem} \phantom{\null}

\begin{minipage}{.8\textwidth}
  Ein Physiker will durch ein Experiment den Zusammenhang dreier Größen $x$, $y$
  und $z$ untersuchen.
  Er erhält die angegebenen Messwerte.
  Bestimmen Sie drei Zahlen $a_0, a_1, a_2 \in \mathbb{R}$, so dass die Funktion
  $f\qty\big(x, y) = a_0 + a_1x + a_2y$ den Zusammenhang $z = f\qty\big(x, y)$
  zwischen $x$, $y$ und $z$ möglichst gut beschreibt.
\end{minipage}
\begin{minipage}{.2\textwidth}
  \begin{tabular}{|c|ccc|}
    \hline
    $x$ & 1 & 3 & 4  \\
    \hline
    $y$ & 4 & 8 & 10 \\
    \hline
    $z$ & 1 & 2 & 3  \\
    \hline
  \end{tabular}
\end{minipage}

\subparagraph{Lsg.} Sei $A = \begin{pmatrix}
  1 & 1 & 4  \\
  1 & 3 & 8  \\
  1 & 4 & 10 \\
\end{pmatrix}$.
Dann ist eine bestmögliche Lösung für
\[
  \begin{pmatrix}
    1 & 1 & 4  \\
    1 & 3 & 8  \\
    1 & 4 & 10 \\
  \end{pmatrix} \cdot \begin{pmatrix}
    a_0 \\
    a_1 \\
    a_2 \\
  \end{pmatrix} = \begin{pmatrix}
    1 \\
    2 \\
    3 \\
  \end{pmatrix}
\]
gesucht.
Dafür muss das Normalgleichungssystem $\qty\big(A^TA)x = A^T\qty\big(1, 2, 3)$
gelöst werden.
Nun ist
\begin{flalign*}
  \qty(
    \begin{pmatrix}
      1 & 1 & 1  \\
      1 & 3 & 4  \\
      4 & 8 & 10 \\
    \end{pmatrix}\begin{pmatrix}
      1 & 1 & 4  \\
      1 & 3 & 8  \\
      1 & 4 & 10 \\
    \end{pmatrix}
  ) \cdot \begin{pmatrix}
    a_0 \\
    a_1 \\
    a_2 \\
  \end{pmatrix}
  &= \begin{pmatrix}
    1 & 1 & 1  \\
    1 & 3 & 4  \\
    4 & 8 & 10 \\
  \end{pmatrix}\begin{pmatrix}
    1 \\
    2 \\
    3 \\
  \end{pmatrix} \\
  \begin{pmatrix}
    3  & 8  & 22  \\
    8  & 26 & 68  \\
    22 & 68 & 180 \\
  \end{pmatrix}\begin{pmatrix}
    a_0 \\
    a_1 \\
    a_2 \\
  \end{pmatrix}
  &= \begin{pmatrix}
    6  \\
    19 \\
    50 \\
  \end{pmatrix}
\end{flalign*}
\newpage
\begin{flalign*}
  \qty(
    \begin{array}{ccc|c}
      3  & 8  & 22  & 6  \\
      8  & 26 & 68  & 19 \\
      22 & 68 & 180 & 50 \\
    \end{array}
  )
  \overset{Z_3 = Z_3 - 2 \cdot Z_2 - 2 \cdot Z_1}&\leadsto
  \qty(
    \begin{array}{ccc|c}
      3  & 8  & 22  & 6  \\
      8  & 26 & 68  & 19 \\
      0  & 0  & 0   & 0  \\
    \end{array}
  ) \\
  \overset{Z_2 = 3 \cdot Z_2 - 8 \cdot Z_1}&\leadsto
  \qty(
    \begin{array}{ccc|c}
      3  & 8  & 22  & 6 \\
      0  & 14 & 28  & 9 \\
      0  & 0  & 0   & 0 \\
    \end{array}
  ) \\
  \overset{Z_2 = \frac{1}{14} \cdot Z_2}&\leadsto
  \qty(
    \begin{array}{ccc|c}
      3  & 8 & 22 & 6            \\
      0  & 1 & 2  & \frac{9}{14} \\
      0  & 0 & 0  & 0            \\
    \end{array}
  ) \\
  \overset{Z_1 = Z_1 - 8 \cdot Z_2}&\leadsto
  \qty(
    \begin{array}{ccc|c}
      3  & 0 & 6  & \frac{6}{7}  \\
      0  & 1 & 2  & \frac{9}{14} \\
      0  & 0 & 0  & 0            \\
    \end{array}
  ) \\
  \overset{Z_1 = \frac{1}{3} \cdot Z_1}&\leadsto
  \qty(
    \begin{array}{ccc|c}
      1  & 0 & 2  & \frac{2}{7}  \\
      0  & 1 & 2  & \frac{9}{14} \\
      0  & 0 & 0  & 0            \\
    \end{array}
  ) \\
\end{flalign*}
\[
  L^* = \qty{
    \begin{pmatrix}
      \frac{2}{7}  - 2a_3 \\
      \frac{9}{14} - 2a_3 \\
      a_3                 \\
    \end{pmatrix}
    \:\middle|\:
    a_3 \in \mathbb{R}
  } = \frac{1}{14}\begin{pmatrix}
    4 \\
    9 \\
    0 \\
  \end{pmatrix} + \text{Span}\qty{
    \begin{pmatrix}
      -2 \\
      -2 \\
      1  \\
    \end{pmatrix}
  }
\]
\textbf{Probe:} Sei $a_0 = \frac{2}{7}$, $a_1 = \frac{9}{14}$ und $a_2 = 0$.
Dann ist
\begin{flalign*}
  \frac{2}{7} + \frac{9}{14} \cdot 1 + 0 \cdot 4  &= \frac{13}{14} \approx 1 \\
  \frac{2}{7} + \frac{9}{14} \cdot 3 + 0 \cdot 8  &= \frac{31}{14} \approx 2 \\
  \frac{2}{7} + \frac{9}{14} \cdot 4 + 0 \cdot 10 &= \frac{40}{14} \approx 3 \\
\end{flalign*}
 Sei $a_0 = -\frac{12}{7}$, $a_1 = -\frac{19}{14}$ und $a_2 = 1$.
Dann ist
\begin{flalign*}
  -\frac{12}{7} - \frac{19}{14} \cdot 1 + 1 \cdot 4  &= \frac{13}{14} \approx 1 \\
  -\frac{12}{7} - \frac{19}{14} \cdot 3 + 1 \cdot 8  &= \frac{31}{14} \approx 2 \\
  -\frac{12}{7} - \frac{19}{14} \cdot 4 + 1 \cdot 10 &= \frac{40}{14} \approx 3 \\
\end{flalign*}
Sieht passend aus.

\newpage
\paragraph{Ü 14.4 Eigenschaften orthogonaler Matrizen}
\begin{enumerate}[(a)]
\item Beweisen oder widerlegen Sie: Das Produkt und die Inversen orthogonaler
  Matrizen sind wieder orthogonal.

  \subparagraph{Lsg.} Sei $A, B \in \mathbb{R}^{n \times n}$ orthogonal.
  Nun ist die Multiplikation quadratischer Matrizen assoziativ und damit
  \begin{flalign*}
    \qty\big(A \cdot B) \cdot \qty\big(A \cdot B)^T
    &=  \qty\big(A \cdot B) \cdot \qty\big(B^T \cdot A^T) & \\
    &=  A \cdot \qty\big(B \cdot B^T) \cdot A^T \\
    &=  A \cdot E^n \cdot A^T \\
    &=  A \cdot A^T \\
    &=  E^n
  \end{flalign*}

  Weiter folgt aus $A$ ist orthogonal, dass $A \cdot A^T = E_n$.
  Da das Inverse einer Matrix eindeutig bestimmt ist, folgt $A^{-1} = A^T$.
  Schließlich ist
  \[
    A^{-1} \cdot \qty(A^{-1})^T = A^T \cdot \qty(A^T)^T = A^T \cdot A = E_n
  \]

\item Beweisen Sie: Ist $A \in \mathbb{R}^{n \times n}$ orthogonal, dann ist die
  lineare Abbildung $f \colon \mathbb{R}^n \times \mathbb{R}^n, v \mapsto Av$
  längentreu und winkeltreu.

  \subparagraph{Lsg.} Seien $z_1, \ldots, z_n$ die Zeilen von $A$ und
  $u, v \in \mathbb{R}^n$.
  Dann ist
  \begin{flalign*}
    f\qty\big(u) \bullet f\qty\big(v)
    &= \qty\big(Au) \bullet \qty\big(Av) \\
    &= \qty\big(u_1z_1 + \ldots + u_nz_n) \bullet \qty\big(v_1z_1 + \ldots + v_nz_n) \\
    \overset{\text{Bilinear}}&= u_1z_1 \bullet \qty\big(v_1z_1 + \ldots + v_nz_n) + \ldots + u_nz_n \bullet \qty\big(v_1z_1 + \ldots + v_nz_n) \\
    \overset{\text{Bilinear}}&= u_1z_1 \bullet v_1z_1 + \ldots + v_1z_1 \bullet u_nz_n + \ldots + u_nz_n \bullet v_nz_1 + \ldots + u_nz_n \bullet v_nz_n \\
    \overset{\text{Bilinear}}&= u_1v_1 \underset{1}{\underbrace{\qty(z_1 \bullet z_1)}} + \ldots +
                                v_1u_n \underset{0}{\underbrace{\qty\big(z_1 \bullet z_n)}} + \ldots +
                                v_vu_n \underset{0}{\underbrace{\qty\big(z_1 \bullet z_n)}} + \ldots +
                                v_nu_n \underset{1}{\underbrace{\qty\big(z_n \bullet z_n)}} \\
    &= u_1v_1 + \ldots + u_nv_n \\
    &= u \bullet v
  \end{flalign*}
  und somit
  \[
    \norm{v} = \sqrt{v \bullet v} = \sqrt{f\qty\big(v) \bullet f\qty\big(v)} = \norm{f\qty\big(v)}
  \]
  als auch
  \[
    \cos\sphericalangle u, v = \frac{u \bullet v}{\norm{u} \bullet \norm{v}}
    = \frac{f\qty\big(u) \bullet f\qty\big(v)}{\norm{f\qty\big(u)} \bullet \norm{f\qty\big(v)}}
    = \cos\sphericalangle f\qty\big(u), f\qty\big(v)
  \]
\end{enumerate}

\newpage
\paragraph{Ü 14.5 Die orthogonale Gruppe $O\qty\big(2)$}
In dieser Aufgabe wollen wir überlegen, welche orthogonalen Abbildungen es im
$\mathbb{R}^2$ gibt.
Zeigen Sie dazu folgende Aussagen:
\begin{enumerate}[(a)]
\item Jede Spiegelung $s_{\alpha} \colon \mathbb{R}^2 \to \mathbb{R}^2$
  bezüglich einer Spiegelgeraden mit Anstiegswinkel $\frac{\alpha}{2}$
  ist eine orthogonale Abbildung.

  \subparagraph{Lsg.} Die Spiegelungsmatrix um den Winkel $\frac{\alpha}{2}$ ist
  $\begin{pmatrix}
    \cos\alpha & \sin\alpha  \\
    \sin\alpha & -\cos\alpha \\
  \end{pmatrix}$ und
  \begin{flalign*}
    \begin{pmatrix}
      \cos\alpha & \sin\alpha  \\
      \sin\alpha & -\cos\alpha \\
    \end{pmatrix} \cdot \begin{pmatrix}
      \cos\alpha & \sin\alpha \\
      \sin\alpha & -\cos\alpha  \\
    \end{pmatrix} &= \begin{pmatrix}
      \cos^2\alpha + \sin^2\alpha                  & \cos\alpha\sin\alpha -\cos\alpha\sin\alpha \\
      \cos\alpha\sin\alpha - \cos\alpha\sin\alpha  & \sin^2\alpha+\cos^2\alpha                  \\
    \end{pmatrix} \\
                 &= \begin{pmatrix}
                   1 & 0 \\
                   0 & 1 \\
                 \end{pmatrix}
  \end{flalign*}

\item Jede Drehung $d_{\alpha} \colon \mathbb{R}^2 \to \mathbb{R}^2$ um den
  Winkel $\alpha$ ist eine orthogonale Abbildung.

  \subparagraph{Lsg.} Eine Drehung um einen Winkel $\alpha$ wird durch die Matrix
  $\begin{pmatrix}
    \cos\alpha & -\sin\alpha \\
    \sin\alpha & \cos\alpha  \\
  \end{pmatrix}$ beschrieben.
  Nun ist
  \begin{flalign*}
    \begin{pmatrix}
      \cos\alpha & -\sin\alpha \\
      \sin\alpha & \cos\alpha  \\
    \end{pmatrix} \cdot \begin{pmatrix}
      \cos\alpha & \sin\alpha \\
      -\sin\alpha & \cos\alpha  \\
    \end{pmatrix} &= \begin{pmatrix}
      \cos^2\alpha + \sin^2\alpha                  & \cos\alpha\sin\alpha -\cos\alpha\sin\alpha \\
      \cos\alpha\sin\alpha - \cos\alpha\sin\alpha  & \sin^2\alpha+\cos^2\alpha                  \\
    \end{pmatrix} \\
                 &= \begin{pmatrix}
                   1 & 0 \\
                   0 & 1 \\
                 \end{pmatrix}
  \end{flalign*}

\item Jede orthogonale Abbildung $f \colon \mathbb{R}^2 \to \mathbb{R}^2$ ist
  eine Drehung oder Spiegelung.

  \subparagraph{Lsg.} Eine lineare Abbildung mit Abbildungsmatrix
  $\begin{pmatrix}
    a & b \\
    c & d \\
  \end{pmatrix} \in \mathbb{R}^2$ ist orthogonal genau dann wenn
  \[
    \begin{pmatrix}
      a & b \\
      c & d \\
    \end{pmatrix} \begin{pmatrix}
      a & c \\
      b & d \\
    \end{pmatrix} = \begin{pmatrix}
      a^2 + b^2 & ac + bd   \\
      ac + bd   & c^2 + d^2 \\
    \end{pmatrix} = \begin{pmatrix}
      1 & 0 \\
      0 & 1 \\
    \end{pmatrix}
  \]
  Weiter wurde in der Vorlesung gezeigt, dass
  $\abs{\det A} = 1$, somit in $\mathbb{R}$ gilt
  $\det A = \pm 1$.

  Somit hat man zweimal vier Gleichungen mit 4 Unbekannten:
  \begin{flalign*}
    a^2 + b^2 &= 1 \\
    c^2 + d^2 &= 1 \\
    ac + db   &= 0 \\
    ad - cb   &\pm 1 \\
  \end{flalign*}
  Durch Umstellen der 1. Gleichung erhält man $b = \sqrt{1 - a^2}$
  und durch Umstellen der 3. Gleichung $c = -\frac{db}{a}$.
  Setzt man dies in die 4. Gleichung ein, erhält man
  \begin{flalign*}
    \pm 1 &= ad - cb = ad + \frac{db^2}{a} = ad - \frac{d\qty\big(1 - a^2)}{a} && {\Big |} \cdot a && && \\
    \pm a &= a^2d + d - a^2d = d
  \end{flalign*}
  Nun der Rest für

  \begin{minipage}{.45\textwidth}
    $a = d$:

    \begin{flalign*}
      0 &= ac + db \\
      \overset{a = d}&= dc + db \\
        &= d \cdot (b + c)
    \end{flalign*}
    $\Rightarrow b = \pm\sqrt{1 - a^2}$

    $\Rightarrow c = \mp\sqrt{1 - a^2}$
  \end{minipage}
  \hspace{.05\textwidth}
  \vrule
  \hspace{.05\textwidth}
  \begin{minipage}{.45\textwidth}
    $a = -d$:

    \begin{flalign*}
      0 &= ac + db \\
      \overset{a = -d}&= -dc + db \\
        &= d \cdot (b - c)
    \end{flalign*}
    $\Rightarrow b = \pm\sqrt{1 - a^2}$

    $\Rightarrow c = \pm\sqrt{1 - a^2}$
  \end{minipage}

  Somit hat eine orthogonale Matrix im $\mathbb{R}^{2 \times 2}$ eine der beiden
  Formen
  \[
    \begin{pmatrix}
      a                    & \pm\sqrt{1 - a^2} \\
      \mp\sqrt{1 - a^2} & a                    \\
    \end{pmatrix}, \begin{pmatrix}
      a                 & \pm\sqrt{1 - a^2}  \\
      \pm\sqrt{1 - a^2} & -a                 \\
    \end{pmatrix}
  \]

  Sieht man nun die beiden Matrizen aus den Aufgaben zuvor, dann kann man
  $a = \cos\qty\big(\alpha)$ setzen und
  \begin{flalign*}
    b &= \sqrt{1 - \cos^2\qty\big(\alpha)} \\
    \overset{Trigonometrischer Pythagoras}&= \sqrt{1 - \qty(1 - \sin^2\qty\big(\alpha))} \\
      &= \sqrt{\sin^2\qty\big(\alpha)} \\
      &= \pm \sin\qty\big(\alpha)
  \end{flalign*}
  Schließlich folgt aus $a, b \in \mathbb{R}$ und $a^b + b^2 = 1$, dass
  $a \in \qty\big[-1, 1]$ und somit findet sich für jedes $a$ auch ein $\alpha$
  mit $\cos\qty\big(\alpha) = a$.

  Folglich hat jede orthogonale Matrix im $\mathbb{R}^{2 \times 2}$ die Form
  \[
    \begin{pmatrix}
      \cos\qty\big(\alpha)    & \pm\sin\qty\big(\alpha) \\
      \mp\sin\qty\big(\alpha) & \cos\qty\big(\alpha)    \\
    \end{pmatrix} \text{ oder } \begin{pmatrix}
      \cos\qty\big(\alpha)    & \pm\sin\qty\big(\alpha) \\
      \pm\sin\qty\big(\alpha) & -\cos\qty\big(\alpha)   \\
    \end{pmatrix}
  \]
  und das sind genau die Matrizen für Drehungen und Spiegelungen.
\end{enumerate}
\end{document}
