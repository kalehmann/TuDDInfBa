\documentclass{scrreprt}

\usepackage{aligned-overset}
\usepackage{amsmath}
\usepackage{amsthm}
\usepackage{amssymb}
\usepackage{bm}
\usepackage[shortlabels]{enumitem}
\usepackage{framed}
\usepackage{hyperref}
\usepackage[utf8]{inputenc}
\usepackage{multicol}
\usepackage{mathtools}
\usepackage{physics}
\usepackage{polynom}
\usepackage{tabularx}
\usepackage[table]{xcolor}
\usepackage{titling}
\usepackage{fancyhdr}
\usepackage{xfrac}
\usepackage{pgfplots}

\pgfplotsset{compat = newest}
\usepgfplotslibrary{fillbetween}
\usetikzlibrary{patterns}
\usetikzlibrary{through}


\author{Karsten Lehmann}
\date{WiSe 2024/25}
\title{Übungsblatt 5\\INF-B-110, Lineare Algebra}

\setlength{\headheight}{26pt}
\pagestyle{fancy}
\fancyhf{}
\lhead{\thetitle}
\rhead{\theauthor}
\lfoot{\thedate}
\rfoot{Seite \thepage}

\begin{document}
\paragraph{Ü5.3 lineare Unabhängigkeit - Basis - Erzeugendensystem}

\begin{enumerate}[(a)]
\item Zeigen Sie, dass für
  \[
    v_1 \coloneqq \qty\big(1, 4, 3)^T, \quad
    v_2 \coloneqq \qty\big(1, 3, 4)^T, \quad
    v_3 \coloneqq \qty\big(2, 9, 5)^T, \quad
    v_4 \coloneqq \qty\big(2, 6, 7)^T, \quad
    v_5 \coloneqq \qty\big(0, 2, 2)^T
  \]
  die Menge $E \coloneqq \qty\big{v_1, v_2, \ldots, v_5} \subseteq \mathbb{R}^3$
  ein Erzeugendensystem von $\mathbb{R}^3$ ist.
  Geben Sie eine Basis $B$ von $\mathbb{R}^3$ mit $B \subseteq E$ an.

  \subparagraph{Lsg.} Es sind
  \[
    \frac{1}{3} \cdot \qty\big(v_1 + v_4 - 5 \cdot v_5) =
    \frac{1}{3} \cdot \qty(
    \begin{pmatrix}
      1 \\
      4 \\
      3 \\
    \end{pmatrix} +
    \begin{pmatrix}
      2 \\
      6 \\
      7 \\
    \end{pmatrix} - 5 \cdot
    \begin{pmatrix}
      0 \\
      2 \\
      2 \\
    \end{pmatrix})
    = \begin{pmatrix}
      1 \\
      0 \\
      0 \\
    \end{pmatrix} = e_1
  \]
  \[
    \frac{1}{4}\qty(2 \cdot \qty\big(v_1 - v_2) + v_5) = \frac{1}{4} \cdot \qty(2 \cdot \qty(
    \begin{pmatrix}
      1 \\
      4 \\
      3 \\
    \end{pmatrix} -
    \begin{pmatrix}
      1 \\
      3 \\
      4 \\
    \end{pmatrix}
    )
    +
    \begin{pmatrix}
      0 \\
      2 \\
      2 \\
    \end{pmatrix}
    ) = \begin{pmatrix}
      0 \\
      1 \\
      0 \\
    \end{pmatrix} = e_2
  \]
  \[
    2 \cdot v_2 - v_4 = 2 \cdot
    \begin{pmatrix}
      1 \\
      3 \\
      4 \\
    \end{pmatrix} -
    \begin{pmatrix}
      2 \\
      6 \\
      7 \\
    \end{pmatrix}
    = \begin{pmatrix}
      0 \\
      0 \\
      1 \\
    \end{pmatrix} = e_3
  \]
  Somit lässt sich die Standardbasis $\qty\big(e_1, e_2, e_3)$ von $\mathbb{R}^3$
  als Linearkombination der Vektoren $v_1, v_2, v_3, v_4, v_5 \in \mathbb{R}^3$
  darstellen.
  Es folgt
  $\text{Span}\qty\big(v_1, v_2, v_3, v_4, v_5) = \mathbb{R}^3$.

  Für eine Basis $B$ von $\mathbb{R}^3$ mit $B \subseteq E$ müssen nun 3 linear
  unabhängige Element aus $v_1, v_2, v_3, v_4, v_5$ ermittelt werden.

  Dabei gilt $v_1, v_2, v_3$ sind linear unabhängig falls
  \[
    \lambda_1 \cdot v_1 + \lambda_2 \cdot v_2 + \lambda_3 \cdot v_3 = 0 \iff \lambda_1 = \lambda_2 = \lambda_3 = 0
  \]
  \begin{flalign*}
    \qty(\begin{array}{ccc|c}
      1 & 1 & 2 & 0 \\
      4 & 3 & 9 & 0 \\
      3 & 4 & 5 & 0
    \end{array})
    \overset{Z_3 + Z_2 - 7 \cdot Z_1}&\leadsto
    \qty(\begin{array}{ccc|c}
      1 & 1 & 2 & 0 \\
      4 & 3 & 9 & 0 \\
      \cellcolor{yellow} 0 & \cellcolor{yellow} 0 & \cellcolor{yellow} 0 & \cellcolor{yellow} 0
    \end{array})
  \end{flalign*}

  \colorbox{yellow}{$\Rightarrow$} Es gibt Lösungen mit
  $\lambda_1 \ne 0 \lor \lambda_2 \ne 0 \lor \lambda_3 \ne 0$ - zum Beispiel
  $3 \cdot v_1 - v_2 - v_3 = 0$ - und die Vektoren sind linear abhängig.

  \newpage
  Angenommen es wären nun  $v_1, v_2, v_4$ linear unabhängig, dann gelte
  \[
    \lambda_1 \cdot v_1 + \lambda_2 \cdot v_2 + \lambda_3 \cdot v_4 = 0 \iff \lambda_1 = \lambda_2 = \lambda_3 = 0
  \]
  \begin{flalign*}
    \qty(\begin{array}{ccc|c}
      1 & 1 & 2 & 0 \\
      4 & 3 & 6 & 0 \\
      3 & 4 & 7 & 0
    \end{array})
    \overset{-1 \cdot \qty\big(Z_3 + Z_2 - 7 \cdot Z_1)}&\leadsto
    \qty(\begin{array}{ccc|c}
      1 & 1 & 2 & 0 \\
      4 & 3 & 6 & 0 \\
      0 & 0 & 1 & 0
    \end{array}) \\
    \overset{-1 \cdot \qty\big(Z_2 - 4 \cdot Z_1 + 2 \cdot Z_3)}&\leadsto
    \qty(\begin{array}{ccc|c}
      1 & 1 & 2 & 0 \\
      0 & 1 & 0 & 0 \\
      0 & 0 & 1 & 0
    \end{array}) \\
    \overset{Z_1 - Z_2 - 2 \cdot Z_3}&\leadsto
    \qty(\begin{array}{ccc|c}
      1 & 0 & 0 & 0 \\
      0 & 1 & 0 & 0 \\
      0 & 0 & 1 & 0
    \end{array})
  \end{flalign*}

  $\Rightarrow \lambda_1 = \lambda_2 = \lambda_3 = 0$ und somit sind
  $v_1, v_2. v_4$ linear unabhängig.

  $\Rightarrow B = \qty\big(v_1, v_2, v_4)$ ist eine Basis von $\mathbb{R}^3$.

\item Zeigen Sie, dass für
  \[
    u_1 \coloneqq \qty\big(1, -2, 3, -4)^T, \quad
    u_2 \coloneqq \qty\big(2, -3, 6, -11)^T, \quad
    u_3 \coloneqq \qty\big(-1, 3, -2, 6)^T
  \]
  die Menge $F \coloneqq \qty\big{u_1, u_2, u_3} \subseteq \mathbb{R}^4$ linear
  unabhängig ist.
  Geben Sie eine Basis $B$ von $\mathbb{R}^4$ mit $F \subseteq B$ an.

  \subparagraph{Lsg.} Es ist zu prüfen, ob
  \[
    \lambda_1 \cdot u_1 + \lambda_2 \cdot u_2 + \lambda_3 \cdot u_4 = 0 \Rightarrow \lambda_1 = \lambda_2 = \lambda_3 = 0
  \]
  Nun ist
  \begin{flalign*}
    \qty(\begin{array}{ccc|c}
      1  & 2   & -1 & 0 \\
      -2 & -3  & 3  & 0 \\
      3  & 6   & -2 & 0 \\
      -4 & -11 & 6  & 0
    \end{array})
    \overset{Z_2 = Z_2 + 2Z_1}&\leadsto
    \qty(\begin{array}{ccc|c}
      1  & 2   & -1 & 0 \\
      0  & 1   & 1  & 0 \\
      3  & 6   & -2 & 0 \\
      -4 & -11 & 6  & 0
    \end{array})
    \overset{Z_3 = -\frac{1}{5}\qty\big(Z_3 - 3Z_1)}\leadsto
    \qty(\begin{array}{ccc|c}
      1  & 2   & -1 & 0 \\
      0  & 1   & 1  & 0 \\
      0  & 0   & 1  & 0 \\
      -4 & -11 & 6  & 0
    \end{array})  \\
    \overset{Z_4 = Z_4 + 4Z_1 + 3Z_2 - 5Z_3}&\leadsto
    \qty(\begin{array}{ccc|c}
      1 & 2 & -1 & 0 \\
      0 & 1 & 1  & 0 \\
      0 & 0 & 1  & 0 \\
      0 & 0 & 0  & 0
    \end{array})
    \overset{Z_2 = Z_2 - Z_3} \leadsto
    \qty(\begin{array}{ccc|c}
      1 & 2 & -1 & 0 \\
      0 & 1 & 0  & 0 \\
      0 & 0 & 1  & 0 \\
      0 & 0 & 0  & 0
    \end{array}) \\
    \overset{Z_1 = Z_1 - 2 Z_2 + Z_3}&\leadsto
    \qty(\begin{array}{ccc|c}
      1 & 0 & 0 & 0 \\
      0 & 1 & 0 & 0 \\
      0 & 0 & 1 & 0 \\
      0 & 0 & 0 & 0
    \end{array})
  \end{flalign*}
  $\Rightarrow \lambda_1 = \lambda_2 = \lambda_3 = 0$ und die Vektoren
  $u_1, u_2, u_3$ sind linear unabhängig.

  \newpage
  Um nun $F$ zu einer Basis von $\mathbb{R}^4$ zu erweitern, muss noch ein
  weiteres Element, welches von $u_1, u_2$ und $u_3$ linear unabhängig ist
  hinzugefügt werden.

  Dazu kann von der Einheitsmatrix $I_4$ ausgegangen werden und vorherigen
  Umformungen rückwärts angewendet werden
  \begin{flalign*}
    \qty(\begin{array}{cccc|c}
      1 & 0 & 0 & 0 & 0 \\
      0 & 1 & 0 & 0 & 0 \\
      0 & 0 & 1 & 0 & 0 \\
      0 & 0 & 0 & 1 & 0
    \end{array})
    \overset{Z_1 = Z_1 + 2Z_2 - Z_3}&\leadsto
    \qty(\begin{array}{cccc|c}
      1 & 2 & -1 & 0 & 0 \\
      0 & 1 & 0  & 0 & 0 \\
      0 & 0 & 1  & 0 & 0 \\
      0 & 0 & 0  & 1 & 0
    \end{array})
    \overset{Z_2 = Z_2 + Z_3}\leadsto
    \qty(\begin{array}{cccc|c}
      1 & 2 & -1 & 0 & 0 \\
      0 & 1 & 1  & 0 & 0 \\
      0 & 0 & 1  & 0 & 0 \\
      0 & 0 & 0  & 1 & 0
    \end{array}) \\
    \overset{Z_4 = Z_4 - 4Z_1 - 3Z_2 + 5Z_3}&\leadsto
    \qty(\begin{array}{cccc|c}
      1  & 2   & -1 & 0 & 0 \\
      0  & 1   & 1  & 0 & 0 \\
      0  & 0   & 1  & 0 & 0 \\
      -4 & -11 & 6  & 1 & 0
    \end{array})
    \overset{Z_3 = -5Z_3 + 3Z_1)}\leadsto
    \qty(\begin{array}{cccc|c}
      1  & 2   & -1 & 0 & 0 \\
      0  & 1   & 1  & 0 & 0 \\
      3  & 6   & -2 & 0 & 0 \\
      -4 & -11 & 6  & 1 & 0
    \end{array}) \\
     \overset{Z_2 = Z_2 - 2Z_1}&\leadsto
    \qty(\begin{array}{cccc|c}
      1  & 2   & -1 & 0 & 0 \\
      -2 & -3   & 3 & 0 & 0 \\
      3  & 6   & -2 & 0 & 0 \\
      -4 & -11 & 6  & 1 & 0
    \end{array})
  \end{flalign*}
  Aus diesem Gleichungssystem lassen sich nun die Vektoren
  $u_1, u_2, u_3$ sowie $\qty\big(0, 0, 0, 1)^T$ entnehmen und es ist bekannt,
  dass sich das Gleichungssystem durch äquivalente Umformungen in $I_4$
  überführen lässt.
  Somit folgt aus $\lambda_1 \cdot u_1 + \lambda_2 \cdot u_2 + \lambda_3 \cdot u_3 + \lambda_4 \cdot \qty\big(0, 0, 0, 1)^T$,
  dass $\lambda_1 = \lambda_2 = \lambda_3 = \lambda_4 = 0$.

  $\Rightarrow u_1, u_2, u_3, \qty\big(0, 0, 0, 1)^T$

  $\Rightarrow B = \qty\big{u_1, u_2, u_3, \qty\big(0, 0, 0, 1)^T}$ ist eine Basis von
  $\mathbb{R}^4$.
\end{enumerate}

\paragraph{Ü 5.4 Linear unabhängige Erweiterung linear unabhängiger Vektoren}

Es sei $V$ ein $K$-Vektorraum und $\qty\big(b_1, \ldots, b_n)$ ein Tupel von
Vektoren aus $V$.
Beweisen Sie: Ist $\qty\big(b_1, \ldots, b_n)$ linear unabhängig und
$x \in V \setminus \text{Span}\qty\big{b_1, \ldots, b_n}$, dann ist
$\qty\big(b_1, \ldots, b_n, x)$ linear unabhängig.

\subparagraph{Lsg.} Es ist
\[
  \text{Span}\qty\big(b_1, \ldots, b_n) = \qty{
    \lambda_1 \cdot b_1 + \ldots + \lambda_n \cdot b_n
    \:{\big |}\:
    \lambda_1, \ldots, \lambda_n \in \mathbb{K}
  }
\]
Sei nun $x \in V \setminus \text{Span}\qty\big{b_1, \ldots, b_n}$.
Dann ist nach Definition der Differenz zweier Mengen $x$ nicht in
$\text{Span}\qty\big{b_1, \ldots, b_n}$ enthalten.

Also gibt es keine $\lambda_1, \ldots, \lambda_n \in K$ mit
$\lambda_1 \cdot b_1 + \ldots + \lambda_n \cdot b_n = x$.
Oder anders formuliert
\[
  \lambda_1 \cdot b_1 + \ldots + \lambda_n \cdot b_n  + \mu \cdot x = 0
  \iff
  \lambda_1 = \ldots = \lambda_n = \mu = 0
\]
$\Rightarrow b_1, \ldots, b_n, x$ sind linear unabhängig.

\textbf{Alternativ (nach der Übung):} Sei
\[
  \sum_{i = 1}^n \mu_i \cdot b_i + \lambda x = 0
\]
Angenommen $\lambda \ne 0$, dann ist
\begin{flalign*}
  0 &= \sum_{i = 1}^n \mu_i \cdot b_i + \lambda x && {\Big |} \cdot \frac{1}{\lambda} \\
  0 &= \sum_{i = 1}^n \frac{\mu_i}{\lambda} \cdot b_i + x && {\Big |} -x \\
  x &= \sum_{i = 1}^n \frac{\mu_i}{\lambda} \cdot b_i
\end{flalign*}
und somit wäre $x$ als Linearkombination der $b_1, \ldots, b_n$ darstellbar und
in $\text{Span}\qty\big{b_1, \ldots, b_n}$ enthalten, ein Widerspruch.
Somit folgt $\lambda = 0$ und ebenso $0 = \sum_{i = 1}^n \mu_i \cdot b_i$.

Da $b_1, \ldots, b_n$ linear unabhängig, folgt $\mu_1, \ldots, \mu_n = 0$.
\begin{flalign*}
  &\Rightarrow \lambda_1 \cdot b_1 + \ldots + \lambda_n \cdot b_n  + \mu \cdot x = 0
  \iff
  \lambda_1 = \ldots = \lambda_n = \mu = 0 &
\end{flalign*}

\paragraph{Ü 5.5 Koordinatenvektoren}

Gegeben sind die Vektoren aus $\mathbb{R}^3$
\[
  v_1 \coloneqq \qty\big(1, -1, 6)^T, \quad
  v_2 \coloneqq \qty\big(1, 0, -2)^T, \quad
  v_3 \coloneqq \qty\big(5, -2, 6)^T, \quad
  w_1 \coloneqq \qty\big(4, -1, 0)^T, \quad
  w_2 \coloneqq \qty\big(3, 0, 2)^T
 \]
sowie der Untervektorraum $U = \text{Span}\qty\big{v_1, v_2, v_3}$.
\begin{enumerate}[(a)]
\item Ist die Menge $\qty\big{v_1, v_2, v_3}$ linear unabhängig?
  Bildet sie ein Erzeugendensystem des $\mathbb{R}^3$?

  \subparagraph{Lsg.} Es ist
  \begin{flalign*}
    \qty(\begin{array}{ccc|c}
      1  & 1  & 5  & 0 \\
      -1 & 0  & -2 & 0 \\
      6  & -2 & 6  & 0 \\
    \end{array})
    \overset{Z_3 = Z_3 + 8 \cdot Z_2 + 2 \cdot Z_1}&\leadsto
    \qty(\begin{array}{ccc|c}
      1  & 1 & 5  & 0 \\
      -1 & 0 & -2 & 0 \\
      0  & 0 & 0  & 0 \\
    \end{array})
  \end{flalign*}
  $\Rightarrow$ die Vektoren sind linear abhängig und bilden somit kein
  Erzeugendensystem des $\mathbb{R}^3$.

\item Bestimmen Sie alle Teilmengen der Menge $\qty\big{v_1, v_2, v_3}$, die eine
  Basis von $U$ bilden.
  Welche Dimension hat $U$?

  \subparagraph{Lsg.} Es existieren offensichtlich keine
  $\lambda_1, \lambda_2, \lambda_3 \in \mathbb{R}$ mit
  \[
    \lambda_1 \cdot \begin{pmatrix}
      1  \\
      -1 \\
      6  \\
    \end{pmatrix} = \begin{pmatrix}
      1  \\
      0  \\
      -2 \\
    \end{pmatrix},
    \lambda_2 \cdot \begin{pmatrix}
      1  \\
      -1 \\
      6  \\
    \end{pmatrix} = \begin{pmatrix}
      5  \\
      -2 \\
      6  \\
    \end{pmatrix},
    \lambda_3 \cdot \begin{pmatrix}
      1  \\
      0  \\
      -2 \\
    \end{pmatrix} = \begin{pmatrix}
      5  \\
      -2 \\
      6  \\
    \end{pmatrix}
  \]
  $\Rightarrow v_1, v_2$ sowie $v_1, v_3$ als auch $v_2, v_3$ sind linear
  unabhängig.

  $\Rightarrow$ Es sind
  $\qty\big(v_1, v_2), \qty\big(v_1, v_3), \qty\big(v_2, v_3)$
  Basen von $U$.

  $\Rightarrow \text{dim}\qty\big(U) = 2$

\item Liegen die Vektoren $w_1$ und $w_2$ in $U$?.
  Berechnen Sie gegebenenfalls die zugehörigen Koordinatenvektoren
  bzgl. der in (b) gefundenen Basen von $U$.

  \subparagraph{Lsg.} Angenommen $w_1, w_2 \in U$, dann sind die beiden Vektoren
  zu jeder beliebigen Basis aus $U$ linear abhängig.
  Also
  \begin{flalign*}
    \qty(\begin{array}{ccc|c}
      1  & 1  & 4  & 0 \\
      -1 & 0  & -1 & 0 \\
      6  & -2 & 0  & 0 \\
    \end{array})
    \overset{Z_3 = Z_3 + 8 \cdot Z_2 + 2 \cdot Z_1}&\leadsto
    \qty(\begin{array}{ccc|c}
      1  & 1  & 4  & 0 \\
      -1 & 0  & -1 & 0 \\
      \cellcolor{yellow} 0  & \cellcolor{yellow} 0  & \cellcolor{yellow} 0  & \cellcolor{yellow} 0 \\
    \end{array})
  \end{flalign*}

  \colorbox{yellow}{$\Rightarrow$} $w_1$ ist von $v_1$ und $v_2$ linear
  abhängig mit $w_1 = v_1 + 3 \cdot v_2$.

  $\Rightarrow w_1$ liegt in $U$

  \newpage
  Nun ist für $w_2$
  \begin{flalign*}
    \qty(\begin{array}{ccc|c}
      1  & 1  & 3 & 0 \\
      -1 & 0  & 0 & 0 \\
      6  & -2 & 2 & 0 \\
    \end{array})
    \overset{Z_1 \leftrightarrow Z_2}&\leadsto
    \qty(\begin{array}{ccc|c}
      -1 & 0  & 0 & 0 \\
      1  & 1  & 3 & 0 \\
      6  & -2 & 2 & 0 \\
    \end{array}) \\
    \overset{Z_1 = -1 \cdot Z_1}&\leadsto
    \qty(\begin{array}{ccc|c}
      1 & 0  & 0 & 0 \\
      1 & 1  & 3 & 0 \\
      6 & -2 & 2 & 0 \\
    \end{array}) \\
    \overset{Z_3 = \frac{1}{8}\qty\big(Z_3 + 2 \cdot Z_2 - 8 \cdot Z_1)}&\leadsto
    \qty(\begin{array}{ccc|c}
      1 & 0 & 0 & 0 \\
      1 & 1 & 3 & 0 \\
      0 & 0 & 1 & 0 \\
    \end{array}) \\
    \overset{Z_2 = Z_2 - Z_1 - 3 \cdot Z_3}&\leadsto
    \qty(\begin{array}{ccc|c}
      1 & 0 & 0 & 0 \\
      0 & 1 & 0 & 0 \\
      0 & 0 & 1 & 0 \\
    \end{array})
  \end{flalign*}
  $\Rightarrow w_2$ ist von der Basis $\qty\big(v_1, v_2)$ linear unabhängig.

  $\Rightarrow w_2$ ist nicht in $U$ enthalten.
\end{enumerate}

\paragraph{Ü 5.6 Basis und Dimension von Spannräumen}
\begin{enumerate}[(a)]
\item Untersuchen Sie untenstehende Teilmengen $U_i$ der
  $K_i$-Vektorräume $V_i$ in folgender Weise:
  \begin{itemize}
  \item Prüfen Sie, ob $U_i$ ein Untervektorraum von $V_i$ ist.
  \item Geben Sie ggf. eine Basis $B_i$ von $U_i$ an und bestimmen Sie
    $\text{dim}_{K_i}\qty\big(U_i)$.
  \item Vervollständigen Sie ggf. $B_i$ zu einer Basis $C_i$ von $V_i$.
  \end{itemize}

  \begin{enumerate}[(i)]
  \item $K_1 = \mathbb{R}$, $V_1 = \mathbb{R}^2$,
    $U_1 = \qty{\qty\big{x_1, x_2}^T \in \mathbb{R}^2 \:\middle|\: x_2 = -3x_1}$

    \subparagraph{Lsg.} Es ist $\qty\big(0, 0)^T$ das Nullelement in
    $\mathbb{R}^2$.
    Nun ist $\qty\big(0, 0)^T \in U_1$, da $0 = -3 \cdot 0$.

    Seien nun $x, y \in U_1 \lambda \in K_1$ beliebig.
    Dann ist
    \[
      x + y = \qty\big(x_1 + y_1, -3x_1 + \qty(-3y_1))^T
      = \qty\big(x_1 + y_1, -3\qty(x_1 + y_1))^T \in U_1
    \]
    und auch
    \[
      \lambda \cdot x = \qty\big(\lambda x_1, \lambda \cdot \qty(-3x_1))^T
      = \qty\big(\lambda x_1, -3 \qty(\lambda x_1))^T \in U_1
    \]
    $\Rightarrow U_1$ ist ein Untervektorraum von $V_1$.

    Eine Basis von $U_1$ ist $B_1 \coloneqq \qty\big(\qty(1, -3)^T)$,
    $\text{dim}_{K_1}\qty\big(U_1) = 1$ und
    $B_1$ lässt sich zur Basis $C_1$ von $\mathbb{R}^2$ vervollständigen mit
    zum Beispiel $C_1 = \qty\big(\qty(1, -3)^T, \qty\big(0, 1)^T)$.

  \newpage
  \item $K_2 = \mathbb{C}$, $V_2 = \mathbb{C}^3$,
    $U_2 = \qty{\qty\big{x_1, x_2, x_3}^T \in \mathbb{C}^3 \:\middle|\: x_3 = i \cdot x_1}$

    \subparagraph{Lsg.} Es ist $\qty\big(0, 0, 0)^T$ das Nullelement in
    $\mathbb{C}^3$.
    Nun ist $\qty\big(0, 0, 0)^T \in U_2$, da $0 = i \cdot 0$.

    Seien nun $x, y \in U_2 \lambda \in K_2$ beliebig.
    Dann ist
    \[
      x + y = \qty\big(x_1 + y_1, x_2 + y_2, \qty(i \cdot x_1) + \qty(i \cdot y_1))^T
      = \qty\big(x_1 + y_1, x_2 + y_2, i \cdot \qty(x_1 + y_1))^T \in U_2
    \]
    und auch
    \[
      \lambda \cdot x = \qty\big(\lambda x_1, \lambda x_2, \lambda \cdot \qty(i \cdot x_1))^T
      = \qty\big(\lambda x_1, \lambda x_2, i \cdot \qty(\lambda \cdot x_1))^T \in U_2
    \]
    $\Rightarrow U_2$ ist ein Untervektorraum von $V_2$.

    Eine Basis von $U_2$ ist
    $B_2 \coloneqq \qty(\qty\big(0, 1, 0)^T, \qty\big(1, 0, i)^T)$,
    $\text{dim}_{K_2}\qty\big(U_2) = 2$ und
    $B_2$ lässt sich zur Basis $C_2$ von $\mathbb{C}^3$ vervollständigen mit
    zum Beispiel
    $C_2 = \qty(\qty\big(1, 0, 0)^T, \qty\big(0, 1, 0)^T, \qty\big(1, 0, i)^T)$.

  \item $K_3 = \mathbb{R}$, $V_3 = \mathbb{R}^4$,
    $U_3 = \qty{\qty\big{x_1, x_2, x_3, x_4}^T \in \mathbb{R}^4 \:\middle|\: x_1 - x_2 - x_3 = 0 = x_2 + x_3 + x_4}$

    \subparagraph{Lsg.} Es ist $\qty\big(0, 0, 0, 0)^T$ das Nullelement in
    $\mathbb{R}^4$.
    Nun ist $\qty\big(0, 0, 0, 0)^T \in U_3$, da $0 - 0 - 0 = 0 = 0 + 0 + 0$.

    Seien nun $x, y \in U_3 \lambda \in K_3$ beliebig und $z = x + y$.
    Dann ist
    \[
      z = x + y = \qty\big(x_1 + y_1, x_2 + y_2, x_3 + y_3, x_4 + y_4)^T
    \]
    und $z \in U_3$, da
    \[
      \qty\big(x_1 + y_1) - \qty\big(x_2 + y_2) - \qty\big(x_3 + y_3)
      = x_1 + y_1 - x_2 - y_2 - x_3 - y_3
      = \underset{= 0\text{, da } x \in U_3}{\underbrace{x_1 - x_2 - x_3}} + \underset{= 0\text{, da } y \in U_3}{\underbrace{y_1 - y_2 - y_3}}
      = 0
    \]
    als auch
    \[
      \qty\big(x_2 + y_2) + \qty\big(x_3 + y_3) + \qty\big(x_4 + y_4)
      = x_2 + y_2 + x_3 + y_3 + x_4 + y_4
      = \underset{= 0\text{, da } x \in U_3}{\underbrace{x_2 + x_3 + x_4}} + \underset{= 0\text{, da } y \in U_3}{\underbrace{y_2 + y_3 + y_4}}
      = 0
    \]
    Schließlich ist $\qty\big(\lambda \cdot x) \in U_3$, da
    \[
      \lambda \cdot x_1 - \lambda \cdot x_2 - \lambda \cdot x_3 =
      \lambda \cdot \underset{= 0\text{, da } x \in U_3}{\underbrace{\qty(x_1 - x_2 - x_3)}}
      = 0
    \]
    als auch
    \[
      \lambda \cdot x_2 + \lambda \cdot x_3 + \lambda \cdot x_4 =
      \lambda \cdot \underset{= 0\text{, da } x \in U_3}{\underbrace{\qty(x_2 + x_3 + x_4)}}
      = 0
    \]
    $\Rightarrow U_3$ ist ein Untervektorraum von $V_3$.

    Eine Basis von $U_3$ ist
    $B_3 \coloneqq \qty(\qty\big(1, 1, 0, -1)^T, \qty\big(1, 0, 1, -1)^T)$,
    $\text{dim}_{K_3}\qty\big(U_3) = 2$ und
    $B_3$ lässt sich zur Basis $C_3$ von $\mathbb{R}^4$ vervollständigen mit
    zum Beispiel
    $C_3 = \qty(\qty\big(0, 1, 1, -1)^T, \qty\big(1, 1, 0, -1)^T, \qty\big(1, 0, 1, -1)^T, \qty\big(1, 1, 1, 0)^T)$.

  \item $K_4 = \mathbb{R}$, $V_4 = \mathbb{R}^3$,
    $U_4 = \qty{\qty\big{x_1, x_2, x_3}^T \in \mathbb{R}^3 \:\middle|\: x_3 = x_1 + 1, x_2 = x_1^2}$

    \subparagraph{Lsg.} Es ist $\qty\big(0, 0, 0)^T$ das Nullelement in
    $\mathbb{R}^3$.
    Nun ist $\qty\big(0, 0, 0)^T \notin U_4$, da $0 \ne 0 + 1$.

    $\Rightarrow U_4$ ist kein Untervektorraum von $V_4$.
  \end{enumerate}

\item Gemäß Ü 4.5 (e) ist die Menge $P_{\leq n}$ der reellen Polynomfunktionen
  vom Grad höchstens $n$ ein Untervektorraum von
  $\text{Abb}\qty\big(\mathbb{R}, \mathbb{R})$.
  Bestimmen Sie eine Basis von $P_{\leq n}$ sowie die
  Dimension dieses Vektorraumes.
  In $P_{\leq 2}$ ist $B \coloneqq \qty\big(1, 1 + x, 1 + x + x^2)$ eine
  angeordnete Basis (das können Sie im Selbststudium nachweisen).
  Geben Sie den Koordinatenvektor $p\qty\big(x)_B$ von
  $p\qty\big(x) \coloneqq 3x^2 + 2x + 4$ bezüglich $B$ an.

  \subparagraph{Lsg.} Seien $f_n \in \text{Abb}\qty\big(\mathbb{R}, \mathbb{R})$
  mit $f_n = x^n$.
  Nun hat jedes beliebige Polynom $g \in P_{\leq n}$ die Form
  \[
    g = \sum_{k = 0}^n a_k x^k, \text{ mit den Koeffezienten } a_0, \ldots, a_n
  \]
  und lässt sich als Linearkombination
  \[
    a_0 \cdot f_0 + \ldots + a_n \cdot f_n
  \]
  darstellen.

  $\Rightarrow \qty\big(f_0, \ldots, f_n)$ ist ein Erzeugendensystem von
  $P_{\leq n}$.

  Da alle Polynome $f_0, \ldots, f_n$ einen paarweise verschiedenen Grad haben,
  sind sie ebenfalls linear unabhängig.
  Somit ist $\qty\big(f_0, \ldots, f_n)$ eine Basis von $P_{\leq n}$ und
  $\text{dim}\qty(P_{\leq n}) = n + 1$.

  Schließlich ist
  \[
    p\qty\big(x) = 2 \cdot 1 - \qty\big(1 + x) + 3 \cdot \qty\big(1 + x + x^2)
  \]
  und
  \[
    p\qty\big(x)_B = \qty\big(2, -1, 3)^T
  \]
\end{enumerate}
\end{document}
