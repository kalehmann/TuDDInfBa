\documentclass{scrreprt}

\usepackage{amsmath}
\usepackage{amsthm}
\usepackage{amssymb}
\usepackage{bm}
\usepackage[shortlabels]{enumitem}
\usepackage{framed}
\usepackage{hyperref}
\usepackage[utf8]{inputenc}
\usepackage{multicol}
\usepackage{mathtools}
\usepackage{physics}
\usepackage{polynom}
\usepackage{tabularx}
\usepackage[table]{xcolor}
\usepackage{titling}
\usepackage{fancyhdr}
\usepackage{xfrac}
\usepackage{pgfplots}

\pgfplotsset{compat = newest}
\usepgfplotslibrary{fillbetween}
\usetikzlibrary{patterns}
\usetikzlibrary{through}


\author{Karsten Lehmann}
\date{WiSe 2024/25}
\title{Übungsblatt 2\\INF-B-110, Lineare Algebra}

\setlength{\headheight}{26pt}
\pagestyle{fancy}
\fancyhf{}
\lhead{\thetitle}
\rhead{\theauthor}
\lfoot{\thedate}
\rfoot{Seite \thepage}

\begin{document}
\paragraph{Ü2.3 Ein einfaches Lineares Gleichungssystem}
Peter ist doppelt so alt wie Max.
Max ist 10 Jahren jünger als Bert.
Zusammen zählen alle drei 86 Jahre.
Wie alt sind Peter, Max und Bert?

\subparagraph{Lsg.} Sei $p$ das Alter von Peter, $m$ das Alter von $Max$
und $b$ das Alter von Bert.
Dann ist
\begin{flalign*}
  p - 2 \cdot  m + 0 \cdot b &= 0  && \\
  0 \cdot p -  m + b &= 10 \\
  p + m + b &= 86
\end{flalign*}
Oder in der Matrix-Schreibweise für ein Lineares Gleichungssystem
\[
  \begin{pmatrix}
    1 & -2 & 0 \\
    0 & -1 & 1 \\
    1 &  1 & 1
  \end{pmatrix}
  \cdot
  \begin{pmatrix} p \\ m \\ b \end{pmatrix}
  =
  \begin{pmatrix} 0 \\ 10 \\ 86 \end{pmatrix}
\]
Die Matrix muss durch Elementare Zeilenumformungen in die
Zeilenstufenform gebracht werden.

\begin{flalign*}
  \qty(
    \begin{array}{ccc|c}
      1 & -2 & 0 & 0 \\
      0 & -1 & 1 & 10 \\
      1 &  1 & 1 & 86
    \end{array}
  )
  &\overset{\text{Zeile 3 - Zeile 1}}\leadsto
  \qty(
    \begin{array}{ccc|c}
      1 & -2 & 0 & 0 \\
      0 & -1 & 1 & 10 \\
      0 &  3 & 1 & 86
    \end{array}
  ) \\
  &\overset{\text{Zeile 3 + 3 $\cdot$ Zeile 2}}\leadsto
  \qty(
    \begin{array}{ccc|c}
      1 & -2 & 0 & 0 \\
      0 & -1 & 1 & 10 \\
      0 &  0 & 4 & 116
    \end{array}
  ) \\
  &\overset{\text{Zeile 3 : 4}}\leadsto
  \qty(
    \begin{array}{ccc|c}
      1 & -2 & 0 & 0 \\
      0 & -1 & 1 & 10 \\
      0 &  0 & 1 & 29
    \end{array}
  ) \\
\end{flalign*}

Jetzt ließe sich das LGS bereits durch Rückeinsetzen lösen.
Noch einfacher wird es allerdings mit der reduzierten Zeilenstufenform:

\begin{flalign*}
  \qty(
    \begin{array}{ccc|c}
      1 & -2 & 0 & 0 \\
      0 & -1 & 1 & 10 \\
      0 &  0 & 1 & 29
    \end{array}
  )
  &\overset{\text{Zeile 2 - Zeile 3}}\leadsto
  \qty(
    \begin{array}{ccc|c}
      1 & -2 & 0 & 0 \\
      0 & -1 & 0 & -19 \\
      0 &  0 & 1 & 29
    \end{array}
  ) \\
  &\overset{\text{Zeile 2 $\cdot (-1)$}}\leadsto
  \qty(
    \begin{array}{ccc|c}
      1 & -2 & 0 & 0 \\
      0 &  1 & 0 & 19 \\
      0 &  0 & 1 & 29
    \end{array}
  ) \\
  &\overset{\text{Zeile 1 + $2 \cdot$ Zeile 2}}\leadsto
  \qty(
    \begin{array}{ccc|c}
      1 &  0 & 0 & 38 \\
      0 &  1 & 0 & 19 \\
      0 &  0 & 1 & 29
    \end{array}
  ) \\
\end{flalign*}
Nun lässt sich die Lösungsmenge
$L = \qty{\begin{pmatrix}38 \\ 19 \\ 29\end{pmatrix}}$
einfach ablesen.
Somit ist Peter 38 Jahre alt, Max 19 Jahre und Bert 29 Jahre.

\paragraph{Ü2.4 Ein Lineares Gleichungssystem über dem Körper $\mathbb{C}$}

Bestimmen Sie die Lösungsmenge des folgenden linearen Gleichungssystems über
$\mathbb{C}$.

\[
  \begin{array}{rrl}
    \qty\big(1 - i) \cdot z_1 - &z_2 &= 2 - i  \\
    z_1 +& \qty\big(1 + i) \cdot z_2 &= 3 - i
  \end{array}
\]

\subparagraph{Lsg.}
Es ist
\[
  \begin{pmatrix}
    1 - i & -1 \\
    1     & 1 + i
  \end{pmatrix}
  \cdot
  \begin{pmatrix}
    z_1 \\
    z_2
  \end{pmatrix}
  =
  \begin{pmatrix}
    2 - i \\
    3 - i
  \end{pmatrix}
\]
Die Umformung in die reduzierte Zeilenstufenform:

\begin{flalign*}
  \qty(
    \begin{array}{cc|c}
      1 - i & -1    & 2 - i \\
      1     & 1 + i & 3 - i
    \end{array}
  )
  &\overset{\text{Zeile 1 $\cdot \qty\big(i + i)$}}\leadsto
  \qty(
    \begin{array}{cc|c}
      2 & -1 - i & 3 + i \\
      1 & 1 + i  & 3 - i
    \end{array}
  ) \\
  &\overset{2 \cdot \text{Zeile 2 - Zeile 1}}\leadsto
  \qty(
    \begin{array}{cc|c}
      2 & -1 - i & 3 + i \\
      0 & 3 + 3i  & 3 - 3i
    \end{array}
  ) \\
  &\overset{\text{Zeile 2 : 3}}\leadsto
  \qty(
    \begin{array}{cc|c}
      2 & -1 - i & 3 + i \\
      0 & 1 + i  & 1 - i
    \end{array}
  ) \\
  &\overset{\text{Zeile 1 + Zeile 2}}\leadsto
  \qty(
    \begin{array}{cc|c}
      2 & 0      & 4 \\
      0 & 1 + i  & 1 - i
    \end{array}
  ) \\
  &\overset{\text{Zeile 2} \cdot \qty(\frac{1}{2} - \frac{i}{2})}\leadsto
  \qty(
    \begin{array}{cc|c}
      2 & 0 & 4 \\
      0 & 1 & -i
    \end{array}
  ) \\
  &\overset{\text{Zeile 1 : 2}}\leadsto
  \qty(
    \begin{array}{cc|c}
      1 & 0 & 2 \\
      0 & 1 & -i
    \end{array}
  )
\end{flalign*}

Nun lässt sich die Lösungsmenge
$L = \qty{\begin{pmatrix}2 \\ -i\end{pmatrix}}$
einfach ablesen.

\newpage
\paragraph{Ü 2.5 Matrizen multiplizieren}
Gegeben sind die folgenden reellen Matrizen:
\[
  A =
  \begin{pmatrix}
    2  & 1  & -1 \\
    -1 & 0  & 3  \\
    0  & -2 & 1
  \end{pmatrix},
  B =
  \begin{pmatrix}
    5  & 0  & -2 \\
    -1 & 1  & 3  \\
    2  & -2 & 3
  \end{pmatrix},
  C =
  \begin{pmatrix}
    -1 & 3  & 1 \\
    0  & -1 & 2  \\
    0  & 0  & 2
  \end{pmatrix},
  D =
  \begin{pmatrix}
    2 & -1 & 0  \\
    1 & 1  & -2
  \end{pmatrix}
\]
Berechnen Sie, falls möglich, die folgenden Matrizenausdrücke
\[
  AB, BA, AD, A(B + C), AB + AC, A^TB^T, \qty\big(BA)^T, C^2, D^TB
\]
Berechnen Sie $A^8$ für die reelle Matrix
$E = \begin{pmatrix}
  1  & 2 & 0 \\
  -1 & 0 & 1 \\
  0  & 2 & 1
\end{pmatrix}$

\subparagraph{Lsg.} Es ist
\begin{flalign*}
  A \cdot B &=
  \begin{pmatrix}
    2 \cdot 5 + 1 \cdot (-1) + (-1) \cdot 2 & 2 \cdot 0 + 1 \cdot 1 + (-1) \cdot (-2) & 2 \cdot (-2) + 1 \cdot 3 + (-1) \cdot 3 \\
    (-1) \cdot 5 + 0 \cdot (-1) + 3 \cdot 2 & (-1) \cdot 0 + 0 \cdot 1 + 3 \cdot (-2) & (-1) \cdot (-2) + 0 \cdot 3 + 3 \cdot 3 \\
    0 \cdot 5 + (-2) \cdot (-1) + 1 \cdot 2 & 0 \cdot 0 + (-2) \cdot 1 + 1 \cdot (-2) & 0 \cdot (-2) + (-2) \cdot 3 + 1 \cdot 3 \\
  \end{pmatrix} \\
  &=
  \begin{pmatrix}
    7 & 3  & -4 \\
    1 & -6 & 11 \\
    4 & -4 & -3
  \end{pmatrix}\\
  B \cdot A &=
  \begin{pmatrix}
    5 \cdot 2 + 0 \cdot (-1) + (-2) \cdot 0 & 5 \cdot 1 + 0 \cdot 0 + (-2) \cdot (-2) & 5 \cdot (-1) + 0 \cdot 3 + (-2) \cdot 1 \\
    (-1) \cdot 2 + 1 \cdot (-1) + 3 \cdot 0 & (-1) \cdot 1 + 1 \cdot 0 + 3 \cdot (-2) & (-1) \cdot (-1) + 1 \cdot 3 + 3 \cdot 1 \\
    2 \cdot 2 + (-2) \cdot (-1) + 1 \cdot 0 & 2 \cdot 1 + (-2) \cdot 0 + 3 \cdot (-2) & 2 \cdot (-1) + (-2) \cdot 3 + 3 \cdot 1
  \end{pmatrix}\\
  &=
  \begin{pmatrix}
    10 & 9  & -7 \\
    -3 & -7 & 7 \\
    6  & -4 & -5
  \end{pmatrix}\\
  A \cdot D &\text{ existiert nicht} \\
  B + C &=
  \begin{pmatrix}
    5 + (-1) & 0 + 3    & (-2) + 1 \\
    (-1) + 0 & 1 + (-1) & 3 + 2    \\
    2 + 0    & (-2) + 0 & 3 + 2
  \end{pmatrix} =
  \begin{pmatrix}
    4  & 3  & -1 \\
    -1 & 0  & 5  \\
    2  & -2 & 5
  \end{pmatrix} \\
  A \cdot \qty\big(B + C) &=
  \begin{pmatrix}
    2 \cdot 4 + 1 \cdot (-1) + (-1) \cdot 2  & 2 \cdot 3 + 1 \cdot 0 + (-1) \cdot (-2)  & 2 \cdot (-1) + 1 \cdot 5 + (-1) \cdot 5 \\
    (-1) \cdot 4 + 0 \cdot (-1) + 3 \cdot 2  & (-1) \cdot 3 + 0 \cdot 0 + 3 \cdot (-2)  & (-1) \cdot (-1) + 0 \cdot 5 + 3 \cdot 5 \\
    0 \cdot 4 + (-2) \cdot (-1) + 1 \cdot 2  & 0 \cdot 3 + (-2) \cdot 0 + 1 \cdot (-2)  & 0 \cdot (-1) + (-2) \cdot 5 + 1 \cdot 5 \\
  \end{pmatrix} \\
  &=
  \begin{pmatrix}
    5 & 8  & -2 \\
    2 & -9 & 16 \\
    4 & -2 & -5
  \end{pmatrix} \\
  A \cdot C &=
  \begin{pmatrix}
    2 \cdot (-1) + 1 \cdot 0 + (-1) \cdot 0 & 2 \cdot 3 + 1 \cdot (-1) + (-1) \cdot 0 & 2 \cdot 1 + 1 \cdot 2 + (-1) \cdot 2 \\
    (-1) \cdot (-1) + 0 \cdot 0 + 3 \cdot 0 & (-1) \cdot 3 + 0 \cdot (-1) + 3 \cdot 0 & (-1) \cdot 1 + 0 \cdot 2 + 3 \cdot 2 \\
    0 \cdot (-1) + (-2) \cdot 0 + 1 \cdot 0 & 0 \cdot 3 + (-2) \cdot (-1) + 1 \cdot 0 & 0 \cdot 1 + (-2) \cdot 2 + 1 \cdot 2 \\
  \end{pmatrix} \\
  &=
  \begin{pmatrix}
    -2 & 5  & 2 \\
    1  & -3 & 5 \\
    0  & 2  & -2
  \end{pmatrix}
\end{flalign*}

\newpage
\begin{flalign*}
  A \cdot B + A \cdot C &=
  \begin{pmatrix}
    7 + (-2) & 3 + 5       & (-4) + 2    \\
    1 + 1    & (-6) + (-3) & 11 + 5      \\
    4 + 0    & (-4) + 2    & (-3) + (-2) \\
  \end{pmatrix} =
  \begin{pmatrix}
    5 & 8  & -2 \\
    2 & -9 & 16 \\
    4 & -2 & -5
  \end{pmatrix} \\
  A^T &=
  \begin{pmatrix}
    2  & -1 & 0 \\
    1  & 0  & 3 \\
    -1 & -2 & 1 \\
  \end{pmatrix} \\
  B^T &=
  \begin{pmatrix}
    5  & -1 & 2  \\
    0  & 1  & -2 \\
    -2 & 3  & 3  \\
  \end{pmatrix} \\
  A^T \cdot B^T &=
  \begin{pmatrix}
    2 \cdot 5 + (-1) \cdot 0 + 0 \cdot (-2) & 2 \cdot (-1) + (-1) \cdot 1 + 0 \cdot 3 & 2 \cdot 2 + (-1) \cdot (-2) + 0 \cdot 3 \\
    1 \cdot 5 + 0 \cdot 0 + (-2) \cdot (-2) & 1 \cdot (-1) + 0 \cdot 1 + (-2) \cdot 3 & 1 \cdot 2 + 0 \cdot (-2) + (-2) \cdot 3 \\
    (-1) \cdot 5 + 3 \cdot 0 + 1 \cdot (-2) & (-1) \cdot (-1) + 3 \cdot 1 + 1 \cdot 3 & (-1) \cdot 2 + 3 \cdot (-2) + 1 \cdot 3 \\
  \end{pmatrix} \\
  &=
  \begin{pmatrix}
    10 & -3 & 6  \\
    9  & -7 & -4 \\
    -7 & 7  & -5
  \end{pmatrix} \\
  \qty\big(B \cdot A)^T &=
  \begin{pmatrix}
    10 & -3 & 6  \\
    9  & -7 & -4 \\
    -7 & 7  & -5
  \end{pmatrix} \\
  C^2 &=
  \begin{pmatrix}
    (-1) \cdot (-1) + 3 \cdot 0 + 1 \cdot 0 & (-1) \cdot 3 + 3 \cdot -1 * 1 \cdot 0 & (-1) \cdot 1 + 3 \cdot 2 + 1 \cdot 2 \\
    0 \cdot (-1) + (-1) \cdot 0 + 2 \cdot 0 & 0 \cdot 3 + (-1) \cdot -1 * 2 \cdot 0 & 0 \cdot 1 + (-1) \cdot 2 + 2 \cdot 2 \\
    0 \cdot (-1) + 0 \cdot 0 + 2 \cdot 0    & 0 \cdot 3 + 0 \cdot -1 * 2 \cdot 0    & 0 \cdot 1 + 0 \cdot 2 + 2 \cdot 2 \\
  \end{pmatrix} \\
  &=
  \begin{pmatrix}
    1 & -6 & 7 \\
    0 & 1  & 2 \\
    0 & 0  & 4
  \end{pmatrix} \\
  D^T &=
  \begin{pmatrix}
    2  & 1  \\
    -1 & 1  \\
    0  & -2
  \end{pmatrix} \\
  D^T \cdot B &\text{ existiert nicht} \\
  E^2 &=
  \begin{pmatrix}
    1 \cdot 1 + 2 \cdot (-1) + 0 \cdot 0    & 1 \cdot 2 + 2 \cdot 0 + 0 \cdot 2    & 1 \cdot 0 + 2 \cdot 1 + 0 \cdot 1 \\
    (-1) \cdot 1 + 0 \cdot (-1) + 1 \cdot 0 & (-1) \cdot 2 + 0 \cdot 0 + 1 \cdot 2 & (-1) \cdot 0 + 0 \cdot 1 + 1 \cdot 1 \\
    0 \cdot 1 + 2 \cdot (-1) + 1 \cdot 0    & 0 \cdot 2 + 2 \cdot 0 + 1 \cdot 2    & 0 \cdot 0 + 2 \cdot 1 + 1 \cdot 1 \\
  \end{pmatrix} \\
  &=
  \begin{pmatrix}
    -1 & 2 & 2 \\
    -1 & 0 & 1 \\
    -2 & 2 & 3
  \end{pmatrix} \\
  E^4 = \begin{pmatrix}
    -1 & 2 & 2 \\
    -1 & 0 & 1 \\
    -2 & 2 & 3
  \end{pmatrix}^2 &=
  \begin{pmatrix}
    (-1) \cdot (-1) + 2 \cdot (-1) + 2 \cdot (-2) & (-1) \cdot 2 + 2 \cdot 0 + 2 \cdot 2 & (-1) \cdot 2 + 2 \cdot 1 + 2 \cdot 3 \\
    (-1) \cdot (-1) + 0 \cdot (-1) + 1 \cdot (-2) & (-1) \cdot 2 + 0 \cdot 0 + 1 \cdot 2 & (-1) \cdot 2 + 0 \cdot 1 + 1 \cdot 3 \\
    (-2) \cdot (-1) + 2 \cdot (-1) + 3 \cdot (-2) & (-2) \cdot 2 + 2 \cdot 0 + 3 \cdot 2 & (-2) \cdot 2 + 2 \cdot 1 + 3 \cdot 3 \\
  \end{pmatrix} \\
  &=
  \begin{pmatrix}
    -5 & 2  & 6 \\
    -1 & 0  & 1 \\
    -6 & 2  & 7
  \end{pmatrix} \\
\end{flalign*}

\newpage
\begin{flalign*}
  E^8 = \begin{pmatrix}
    -5 & 2  & 6 \\
    -1 & 0  & 1 \\
    -6 & 2  & 7
  \end{pmatrix}^2 &=
  \begin{pmatrix}
    (-5) \cdot (-5) + 2 \cdot (-1) + 6 \cdot (-6) & (-5) \cdot 2 + 2 \cdot 0 + 6 \cdot 2 & (-5) \cdot 6 + 2 \cdot 1 + 6 \cdot 7 \\
    (-1) \cdot (-5) + 0 \cdot (-1) + 1 \cdot (-6) & (-1) \cdot 2 + 0 \cdot 0 + 1 \cdot 2 & (-1) \cdot 6 + 0 \cdot 1 + 1 \cdot 7 \\
    (-6) \cdot (-5) + 2 \cdot (-1) + 7 \cdot (-6) & (-6) \cdot 2 + 2 \cdot 0 + 7 \cdot 2 & (-6) \cdot 6 + 2 \cdot 1 + 7 \cdot 7 \\
  \end{pmatrix} \\
  &=
  \begin{pmatrix}
    -13 & 2  & 14 \\
    -1  & 0  & 1 \\
    -14 & 2  & 15
  \end{pmatrix} \\
\end{flalign*}

\paragraph{Ü 2.6 Eigenschaften des Matrixprodukts}
Es seien $m, n, r > 0$ sowie
$A \in \mathbb{R}^{m \times r}, B \in \mathbb{R}^{r \times n}$
(die Produktmatrix $AB$ ist also definiert).

\begin{enumerate}[(a)]
\item Die dritte Spalte von $B$ sei gleich der Summe der beiden ersten Spalten.
  Was lässt sich über die dritte Spalte von $AB$ sagen?
  Warum?

  \subparagraph{Lsg.} Seien
  \[
    A \coloneqq \begin{pmatrix}
      a_{11} & \ldots & a_{1r} \\
      \vdots & \ddots & \vdots \\
      a_{m1} & \ldots & a_{mr}
    \end{pmatrix},
    B \coloneqq \begin{pmatrix}
      b_{11} & \ldots & b_{1n} \\
      \vdots & \ddots & \vdots \\
      b_{r1} & \ldots & b_{rn}
    \end{pmatrix},
    A \cdot B \coloneqq \begin{pmatrix}
      c_{11} & \ldots & c_{1n} \\
      \vdots & \ddots & \vdots \\
      c_{m1} & \ldots & c_{mn}
    \end{pmatrix}
  \]
  Gemäß der Vorlesung ist dann
  $c_{ij} = \sum_{k = 1}^r a_{ik} \cdot b_{kj}$.
  Nun gilt für $i \in 1, \ldots, r$, dass $b_{i3} = b_{i3} + b_{i3}$.
  Entsprechend folgt für $c_{i3}$, dass
  \[
    c_{3i} = \sum_{k = 1}^r a_{ik} \cdot b_{k3} =
    \sum_{k = 1}^r a_{ik} \cdot \qty(b_{k1} + b_{k2}) =
    \sum_{k = 1}^r a_{ik} \cdot b_{k1} + a_{ik} \cdot b_{k2} =
    \sum_{k = 1}^r a_{ik} \cdot b_{k1} + \sum_{k = 1}^r a_{ik} \cdot b_{k2} =
    c_{i1} + c_{i2}
  \]
  $\Rightarrow$ es ist auch die dritte Spalte von $AB$ die Summe der ersten
  beiden Spalten.

\item Die zweite Spalte von $B$ bestehe nur aus Nullen.
  Was lässt sich über die zweite Spalte von $AB$ sagen?
  Warum?

  \subparagraph{Lsg.} Es ist $b_{2i} = 0$ für $i \in 1, \ldots, r$.
  Entsprechend folgt für $c_{2i}$, dass
  \[
    c_{i2} = \sum_{k = 1}^r a_{ik} \cdot b_{k2} =
    \sum_{k = 1}^r a_{ik} 0 = 0
  \]
  $\Rightarrow$ es besteht auch die Zweite Spalte von $AB$ nur aus Nullen.
\end{enumerate}

\newpage
\paragraph{Ü 2.7 Distributivität der Matrixmultiplikation}
Beweises Sie die Distributivität der Matrixmultiplikation:
Für alle $A \in \mathbb{R}^{m \times n}$ und alle
$B, C \in \mathbb{R}^{n \times r}$ gilt
$A \cdot \qty\big(B + C) = A \cdot B + A \cdot C$.

\subparagraph{Lsg.} Seien
\[
  A \coloneqq \begin{pmatrix}
    a_{11} & \ldots & a_{1n} \\
    \vdots & \ddots & \vdots \\
    a_{m1} & \ldots & a_{mn}
  \end{pmatrix},
  B \coloneqq \begin{pmatrix}
    b_{11} & \ldots & b_{1r} \\
    \vdots & \ddots & \vdots \\
    b_{n1} & \ldots & b_{nr}
  \end{pmatrix},
  C \coloneqq \begin{pmatrix}
    c_{11} & \ldots & c_{1r} \\
    \vdots & \ddots & \vdots \\
    c_{n1} & \ldots & c_{nr}
  \end{pmatrix}
\]
Dann ist $\qty\big(B + C)_{ij} = b_{ij} + c_{ij}$ und
\begin{flalign*}
  \qty(A \cdot \qty\big(B + C))_{ij}
  &= \sum_{k = 1}^n a_{ik} \cdot \qty\big(B + C)_{kj} & \\
  &= \sum_{k = 1}^n a_{ik} \cdot \qty\big(b_{kj} + c_{kj}) \\
  &= \sum_{k = 1}^n a_{ik} \cdot b_{kj} + a_{ik} \cdot c_{kj} \\
  &= \sum_{k = 1}^n a_{ik} \cdot b_{kj} + \sum_{k = 1}^n a_{ik} \cdot c_{kj} \\
  &= \qty\big(A \cdot B)_{ij} + \qty\big(A \cdot C)_{ij} \\
\end{flalign*}
\end{document}
