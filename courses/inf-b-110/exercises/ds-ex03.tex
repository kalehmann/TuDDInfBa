\documentclass{scrreprt}

\usepackage{aligned-overset}
\usepackage{amsmath}
\usepackage{amsthm}
\usepackage{amssymb}
\usepackage{bm}
\usepackage[shortlabels]{enumitem}
\usepackage{hyperref}
\usepackage[utf8]{inputenc}
\usepackage{multicol}
\usepackage{mathtools}
\usepackage{pdflscape}
\usepackage{physics}
\usepackage{polynom}
\usepackage{tabularx}
\usepackage[table]{xcolor}
\usepackage{titling}
\usepackage{fancyhdr}
\usepackage{xfrac}
\usepackage{pgfplots}

\pgfplotsset{compat = newest}
\usepgfplotslibrary{fillbetween}
\usetikzlibrary{arrows, arrows.meta}
\usetikzlibrary{calc}
\usetikzlibrary{patterns}

% See https://tex.stackexchange.com/a/118217
\DeclarePairedDelimiter\ceil{\lceil}{\rceil}


\author{Karsten Lehmann}
\date{WiSe 2024/25}
\title{Übungsblatt 3\\INF-B-110, Diskrete Strukturen}

\setlength{\headheight}{26pt}
\pagestyle{fancy}
\fancyhf{}
\lhead{\thetitle}
\rhead{\theauthor}
\lfoot{\thedate}
\rfoot{Seite \thepage}

\begin{document}

\paragraph{Ü 3.1} Für eine Gewürzmischung sollen verschiedene Gewürze zu gleichen
Teilen vermischt werden.
Zur Auswahl stehen Anis, Basilikum, Chili und Dill.
Je nach Kombination der Gewürze können unterschiedliche Gewürzmischungen entstehen.
Dabei sind folgende Bedingungen sämtlich einzuhalten:
\begin{itemize}
\item Anis und Dill können, wenn überhaupt, nur gemeinsam verwendet werden.
\item Wenn Dill dabei ist, dann ist auch Chili dabei.
\item Eine Mischung, die Anis nicht enthält, muss Basilikum enthalten.
\item Basilikum und Dill schließen sich gegenseitig aus.
\end{itemize}

\begin{enumerate}
\item Stellen Sie jede der vier Bedingungen als aussagenlogischen Ausdruck dar.
  Verwenden Sie dabei die Bezeichnungen A, B, C und D für die Verwendung des
  entsprechenden Gewürzes mit diesem Anfangsbuchstaben.

  \subparagraph{Lsg.} Es sind
  \begin{enumerate}[(1)]
  \item $A \iff D$
  \item $D \Rightarrow C$
  \item $\neg A \Rightarrow B$
  \item $B \Rightarrow \neg D \land D \Rightarrow \neg B$
  \end{enumerate}

\item Ermitteln Sie alle möglichen Gewürzmischungen.

  \subparagraph{Lsg.} Mit den obigen Regeln sind die folgenden drei Mischungen
  möglich:

  \begin{tabular}{|c|c|c|c|}
    \hline
    A & B & C & D \\
    \hline
    x &  & x & x \\
      & x & x &  \\
      & x & & \\
    \hline
  \end{tabular}

  Alternativ sei $\beta \colon \qty\big{A, B, C, D} \to \qty\big{0, 1}$.

  Dann ist
  \begin{enumerate}[1. Fall:]
  \item $\beta(A) = 1 \Rightarrow \beta(D) = 1 \Rightarrow \beta(C) = 1 \Rightarrow \beta(B) = 0$.
  \item $\beta(A) = 0 \Rightarrow \beta(D) = 0 \Rightarrow \beta(B) = 1$
    \begin{enumerate}[(1)]
    \item $\beta(C) = 0$
    \item $\beta(C) = 1$
    \end{enumerate}
  \end{enumerate}

\end{enumerate}

\newpage
\paragraph{Ü 3.2}
\begin{enumerate}[(a)]
\item Disjunktion $\lor$ und Konjunktion $\land$ sind 2-stellige boolsche
  Funktionen.
  Verwenden Sie eine Wertetabelle, um die Distributivität der Konjunktion $\land$
  über der Disjunktion $\lor$ zu beweisen.

  \subparagraph{Lsg.}\:\\
  \begin{tabular}{|c|c|c|cc|}
    \hline
    A & B & C & $A \land \qty\big(B \lor C)$ & $\qty\big(A \land B) \lor \qty\big(A \land C)$ \\
    \hline
    - & - & - & - & - \\
    + & - & - & - & - \\
    + & + & - & + & + \\
    + & - & + & + & + \\
    + & + & + & + & + \\
    - & + & - & - & - \\
    - & - & + & - & - \\
    - & + & + & - & - \\
    \hline
  \end{tabular}

\item Zeigen Sie, dass $\qty\big((A \Rightarrow B) \land A) \Rightarrow B$ für
  beliebige Ausdrücke $A$, $B$ eine Tautologie ist.

  \subparagraph{Lsg.}\:\\
  \begin{tabular}{|c|c|ccc|}
    \hline
    $A$ & $B$ & $A \Rightarrow B$ & $\qty\big(A \Rightarrow B) \land A$ & $\qty\big((A \Rightarrow B) \land A) \Rightarrow B$ \\
    \hline
    - & - & + & - & + \\
    + & - & - & - & + \\
    - & + & + & - & + \\
    + & + & + & + & + \\
    \hline
  \end{tabular}

  Alternativ mit Umformen
  \begin{flalign*}
    \qty\big((A \Rightarrow B) \land A) \Rightarrow B
    \overset{Def. Implikation}&\iff
    \neg \qty\big((\neg A \lor B) \land A) \lor B \\
    \overset{Distri.}&\iff
    \neg \qty\big((\underset{Wahr}{\underbrace{(\neg A \land A)}} \lor (B \land A)) \land A) \lor B \\
    &\iff
    \neg \qty\big((B \land A) \land A) \lor B \\
    \overset{Asso.}&\iff
    \neg \qty\big(B \land (A \land A)) \lor B \\
    \overset{Idem.}&\iff
    \neg \qty\big(B \land A) \lor B \\
    \overset{De Morgan}&\iff
    \qty\big(\neg B \lor \neg A) \lor B \\
    \overset{Kommu.}&\iff
    \qty\big(\neg A \lor \neg B) \lor B \\
    \overset{Asso.}&\iff
    \neg A \lor \underset{Wahr}{\underbrace{\qty\big(\neg B \lor B)}} \\
    &\iff 1
  \end{flalign*}

\newpage
\item Zeigen Sie, dass das \emph{Prinzip der Kontraposition} gilt, d.h. für alle
  Ausdrücke $A$, $B$ ist $A \Rightarrow B$ äquivalent zu
  $\neg B \Rightarrow \neg A$.
  Verwenden Sie dazu die Rechenregeln aus der Vorlesung.

  \subparagraph{Lsg.}
  Gemäß der Vorlesung ist $A \Rightarrow B$ gleich $\neg A \lor B$.
  Also ist auch $\neg B \Rightarrow \neg A$ gleich
  $\neg \qty\big(\neg B) \lor \neg A$ oder $B \lor \neg A$.
  Schließlich ist die Konjuktion auch kommutativ, also ist
  $B \lor \neg A$ gleich $\neg A \lor B$.

\item Untersuchen Sie, ob die folgenden beiden aussagenlogischen Ausdrücke
  äquivalent sind:
  \begin{flalign*}
    A &\coloneqq x \Rightarrow y \\
    B &\coloneqq x \Rightarrow \neg \qty\big(y \Rightarrow \neg x)
  \end{flalign*}
  Verwenden Sie auch dazu die Rechenregeln aus der Vorlesung.

  \subparagraph{Lsg.} Es ist
  \begin{flalign*}
    x \Rightarrow \neg \qty\big(y \Rightarrow \neg x)
    \overset{Def. Implikation}&\iff
    \neg x \lor \neg \qty\big(\neg y \lor \neg x) \\
    \overset{De Morgan}&\iff
    \neg x \lor \qty(\neg \qty\big(\neg y) \land \neg\qty\big(\neg x)) \\
    \overset{\text{Doppelte Negation}}&\iff
    \neg x \lor \qty\big(y \land x) \\
    \overset{\text{Distributivität}}&\iff
    \qty\big(\neg x \lor y) \land \underset{Wahr}{\underbrace{\qty\big(\neg x \lor x)}} \\
    &\iff
    \neg x \lor y \\
    \overset{Def. Implikation}&\iff
    x \Rightarrow y
  \end{flalign*}

\item Impliziert der Ausdruck $D \coloneqq \neg x$ den Ausdruck $x \Rightarrow y$?

  \subparagraph{Lsg.} Es ist
  \begin{flalign*}
    D \Rightarrow \neg x \lor y
    \overset{Def. Implikation}&\iff
    \neg D \lor \qty\big(\neg x \lor y) \\
    \overset{Def. D}&\iff
    \neg \qty\big(\neg x) \lor \qty\big(\neg x \lor y) \\
    \overset{Idempotenz}&\iff
    x \lor \qty\big(\neg x \lor y) \\
    \overset{Idempotenz}&\iff
    x \lor \qty\big((\neg x \lor \neg x) \lor y) \\
    \overset{\text{Assoziativität}}&\iff
    \underset{Wahr}{\underbrace{\qty\big(x \lor \neg x)}} \lor \qty\big(\neg x \lor y) \\
    &\iff
    1
  \end{flalign*}

  $\Rightarrow$ die Ausdrücke sind nicht äquivalent
\end{enumerate}

\newpage
\paragraph{Ü 3.3} Durch untenstehende Wertetabelle ist eine boolsche Funktion
$f \colon \qty\big{0, 1}^3 \to \qty\big{0, 1}$ gegeben.

\begin{tabular}{|c|c|c|c|}
  \hline
  $x_1$ & $x_2$ & $x_3$ & $f\qty\big(x_1, x_2, x_3)$ \\
  \hline
  0 & 0 & 0 & 0 \\
  0 & 0 & 1 & 1 \\
  0 & 1 & 0 & 1 \\
  0 & 1 & 1 & 0 \\
  1 & 0 & 0 & 1 \\
  1 & 0 & 1 & 0 \\
  1 & 1 & 0 & 1 \\
  1 & 1 & 1 & 1 \\
  \hline
\end{tabular}

\begin{enumerate}[(a)]
\item Bestimmen Sie die Darstellung der Funktion in disjunktiver Normalform (DNF).
  \subparagraph{Lsg.}
  \[
    \qty\big(\neg x_1 \land \neg x_2 \land x_3) \lor
    \qty\big(\neg x_1 \land x_2 \land \neg x_3) \lor
    \qty\big(x_1 \land \neg x_2 \land \neg x_3) \lor
    \qty\big(x_1 \land x_2 \land \neg x_3) \lor
    \qty\big(x_1 \land x_2 \land x_3)
  \]
  Durch Anwendung gängiger Rechenregeln kürzt sich der Ausdruck zu
  \[
    \qty\big(\neg x_1 \land \neg x_2 \land x_3) \lor
    \qty\big(\neg x_1 \land x_2 \land \neg x_3) \lor
    \qty\big(x_1 \land \neg x_2 \land \neg x_3) \lor
    \qty\big(x_1 \land x_2)
  \]

\item Bestimmen Sie eine Darstellung der Funktion in konjunktiver Normalform (KNF).

  \subparagraph{Lsg.} Die Konjunktive Normalform lässt sich aus den Spalten mit
  0 als Funktionswert ablesen:
  \[
    \qty\big(x_1 \lor x_2 \lor x_3) \land
    \qty\big(x_1 \lor \neg x_2 \lor x_3) \land
    \qty\big(\neg x_1 \lor x_2 \neg \lor x_3)
  \]
\end{enumerate}

\newpage
\paragraph{Ü 3.4} Ein Ritter kommt an zwei große steinerne Tore, vor denen je
eine Wache steht.
Hinter einem der Tore führt der Weg zu Ruhm, ewigen Glück, hinter dem anderen
Tore in den sicheren Tod.
Der Ritter weiß, dass eine der Wachen stets die Wahrheit spricht, die andere
stets lügt.
Doch welche, das weiß er nicht.
Er darf nur einer der Wachen eine einzige Frage stellen.
Welche Frage sollte das sein.

\subparagraph{Lsg.} ``\emph{Wird mir die andere Wache sagen, dass ich durch das
  Tor hinter dir in den Tod trete?}''

Seien $W_1, W_2$ die Wachen und $T_1, T_2$ die Tore.
Ein + markiert, dass die Wache die Wahrheit sagt beziehungsweise das Tor in den
Tod führt.

\begin{tabular}{|ccccc|}
  \hline
  $W_1$ & $W_2$ & $T_1$ & $T_2$ & Die Antwort von $W_1$ \\
  + & - & + & - & Nein \\
  - & + & + & - & Nein \\
  + & - & - & + & Ja \\
  - & + & - & + & Ja \\
  \hline
\end{tabular}

Dementsprechend tritt man bei Ja in das Tor hinter der Wache, bei Nein in das
andere Tor.

\end{document}
