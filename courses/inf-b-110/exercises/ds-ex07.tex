\documentclass{scrreprt}

\usepackage{aligned-overset}
\usepackage{amsmath}
\usepackage{amsthm}
\usepackage{amssymb}
\usepackage{bm}
\usepackage[inline, shortlabels]{enumitem}
\usepackage{hyperref}
\usepackage[utf8]{inputenc}
\usepackage{listings}
\usepackage{multicol}
\usepackage{mathtools}
\usepackage{pdflscape}
\usepackage{physics}
\usepackage{polynom}
\usepackage{tabularx}
\usepackage[table]{xcolor}
\usepackage{titling}
\usepackage{fancyhdr}
\usepackage{xfrac}
\usepackage{pgfplots}

\pgfplotsset{compat = newest}
\usepgfplotslibrary{fillbetween}
\usetikzlibrary{arrows, arrows.meta}
\usetikzlibrary{calc}
\usetikzlibrary{patterns}

\author{Karsten Lehmann}
\date{WiSe 2024/25}
\title{Übungsblatt 7\\INF-B-110, Diskrete Strukturen}

\setlength{\headheight}{26pt}
\pagestyle{fancy}
\fancyhf{}
\lhead{\thetitle}
\rhead{\theauthor}
\lfoot{\thedate}
\rfoot{Seite \thepage}

\newcommand{\ggT}[0]{\text{ggT}}
\DeclarePairedDelimiter{\floor}{\lfloor}{\rfloor}

\begin{document}

\paragraph{Ü 7.1}
\begin{enumerate}[(a)]
\item Ist 21 ein Erzeuger der Gruppe $\qty\big(\mathbb{Z}_{450}; +)$?

  \subparagraph{Lsg.} Es ist $\ggT\qty\big(21, 450) = 3$.
  Somit wiederholt sich jede Aneinanderkettung der 21 in
  $\qty\big(\mathbb{Z}_{450}; +)$ nach $\frac{450}{3} = 150$ Verknüpfungen.

  Daher ist $\abs{<21>} = \frac{450}{3} = 150 < \abs{\mathbb{Z}_{450}}$.
  Somit kann 21 kein Erzeuger von $\qty\big(\mathbb{Z}_{450}; +)$ sein.

\item Wie viele erzeugende Elemente gibt es in den folgenden Gruppen?
  \begin{enumerate}[(1)]
  \item $\qty\big(\mathbb{Z}_{13}; +)$
  \item $\qty\big(\mathbb{Z}_{24}; +)$
  \item $\qty\big(\mathbb{Z}_{450}; +)$
  \end{enumerate}

  \subparagraph{Lsg.} Die Lösung wird mit der \emph{eulerschen-$\phi$-Funktion}
  ermittelt.
  \begin{enumerate}[(1)]
  \item 13 ist eine Primzahl, also ist $\phi\qty\big(13) = 13 - 1 = 12$.
  \item 24 ist keine Primzahl und hat die Primfaktorzerlegung $2^3 \cdot 3$.
    Somit ist $\phi\qty\big(24)
    = \qty\big(2 - 1) \cdot 2^{3 - 1} \cdot \qty\big(3 - 1) = 8$.
  \item 450 ist ebenfalls keine Primzahl und hat die Primfaktorzerlegung
    $2 \cdot 3^2 \cdot 5^2$.
    Somit ist
    \[
      \phi\qty\big(450) =
      \qty\big(2 - 1) \cdot \qty\big(3 - 1) \cdot 3^{2 - 1} \cdot \qty\big(5 - 1) \cdot 5^{2 - 1}
      = 1 \cdot 2 \cdot 3 \cdot 4 \cdot 5
      = 120
    \]
  \end{enumerate}
\end{enumerate}

\paragraph{Ü 7.2}
\begin{enumerate}[(a)]
\item Es wird die Einheitengruppe $\qty\big(\mathbb{Z}_{15}^*; \cdot)$
  betrachtet.
  \begin{enumerate}[(1)]
  \item Stellen Sie die Verknüpfungstafel dieser Gruppe auf.
  \item Finden Sie \colorbox{yellow}{$11^{-1}$} und \colorbox{blue!20}{$13^{-1}$}
    in $\qty\big(\mathbb{Z}_{15}^*; \cdot)$.
  \item Ist die Gruppe zyklisch?
  \item Ist die Gruppe zu $\qty\big(\mathbb{Z}_8; +)$ isomorph?
  \end{enumerate}

  \subparagraph{Lsg.} Es ist
  $\mathbb{Z}_{15}^* = \qty\big{1, 2, 4, 7, 8, 11, 13, 14}$.

  \begin{tabular}{|c|cccccccc|}
    \hline
    $\cdot$ & 1  & 2  & 4  & 7                    & 8  & 11                  & 13                   & 14 \\
    \hline
    1       & 1  & 2  & 4  & 7                    & 8  & 11                  & 13                   & 14 \\
    2       & 2  & 4  & 8  & 14                   & 1  & 7                   & 11                   & 13 \\
    4       & 4  & 8  & 1  & 13                   & 2  & 14                  & 7                    & 11 \\
    7       & 7  & 14 & 13 & 4                    & 11 & 2                   & \cellcolor{blue!20}1 & 8  \\
    8       & 8  & 1  & 2  & 11                   & 4  & 13                  & 14                   & 7  \\
    11      & 11 & 7  & 14 & 2                    & 13 & \cellcolor{yellow}1 & 8                    & 4  \\
    13      & 13 & 11 & 7  & \cellcolor{blue!20}1 & 14 & 8                   & 4                    & 2  \\
    14      & 14 & 13 & 11 & 8                    & 7  & 4                   & 2                    & 1  \\
    \hline
  \end{tabular}

  \begin{enumerate}[(1)]
  \setcounter{enumii}{2}
  \item \label{7_2_a_3} Nun ist die Gruppe aber nicht zyklisch, da die Ordnungen
    der einzelnen Elemente allesamt $\leq \abs{\mathbb{Z}_{15}^*}$ sind:
    \begin{itemize}
    \item Ordnung von 1 ist 1
    \item Ordnung von 2 ist 4
    \item Ordnung von 4 ist 2
    \item Ordnung von 8 ist 4
    \item Ordnung von 11 ist 2
    \item Ordnung von 13 ist 4
    \item Ordnung von 14 ist 2
    \end{itemize}

    \textbf{Alternativ nach der Übung:} Man sieht in der Diagonale der
    Verknüpfungstafel nur die Elemente 4 und 1.
    Nun ist $4^2 = 1$ und somit $n^4 = 1$ für alle $n \in \mathbb{Z}_{15}^*$.
    Daher kann es keinen Erzeuger geben.

  \item Angenommen $\qty\big(\mathbb{Z}_{15}^*; \cdot)$ wäre zu
    $\qty\big(\mathbb{Z}_8; +)$ isomorph, dann gäbe es eine Bijektion
    $f \colon \mathbb{Z}_{15}^* \to \mathbb{Z}_8$, so dass für alle
    $g_1, g_2 \in \mathbb{Z}_{15}^*$ gilt, dass
    $f\qty\big(g_1 \cdot g_2) = f\qty\big(g_1) + f\qty\big(g_2)$.

    Da $f$ ein Bijektion ist, gibt es $g \in \mathbb{Z}_{15}^*$ mit
    $f\qty\big(g) = 1$ und nach der Definition des Isomorphismus
    \begin{itemize}
    \item $f\qty\big(g \cdot g) = f\qty\big(g) + f\qty\big(g) = 2$
    \item $f\qty\big(g \cdot g \cdot g) = f\qty\big(g) + f\qty\big(g) + f\qty\big(g)= 3$
    \item $\vdots$
    \item $f\qty\big(g^7) = 7$
    \end{itemize}
    Durch die Injektivität von $f$ wäre $g$ nun ein Erzeuger von
    $\mathbb{Z}_{15}^*$, ein Widerspruch zu \hyperref[7_2_a_3]{Teilaufgabe (3)}.

  \end{enumerate}
\item Es wird nun die Einheitengruppe $\qty\big(\mathbb{Z}_{13}^*; \cdot)$
  betrachtet.
  \begin{enumerate}[(1)]
  \item Ist die Gruppe zyklisch?
  \item Wie viele Primitivwurzeln gibt es in
    $\qty\big(\mathbb{Z}_{13}^*; \cdot)$?
  \item Bestimmen Sie eine Primitivwurzel in $\mathbb{Z}_{13}^*$ und verwenden
    Sie diese, um die Elemente $m_1 = 9^{10} \cdot 11^{5}$ und $m_2 = 8^{-1}$ in
    $\mathbb{Z}_{13}^*$ zu berechnen.
  \item Zeigen Sie, dass $\qty\big(\mathbb{Z}_{13}^*; \cdot)$ zu
    $\qty\big(\mathbb{Z}_{12}; +)$ isomorph ist, indem Sie einen Isomorphismus
    angeben und nachweisen, dass er die geforderten Eigenschaften erfüllt.
  \end{enumerate}

  \subparagraph{Lsg.}
  \begin{enumerate}[(1)]
  \item Ja, nach Gauß (Satz 41) im Skript ist die Gruppe zyklisch, da 13 eine
    Primzahl ist.
  \item Die vier Primitivwurzeln von der Gruppe sind 2, 6, 7, 11
    (Seite 59 im Skript).

  \newpage
  \item In $\mathbb{Z}_{13}^*$ ist $9 = 2^8$, $11 = 2^7$ und $8 = 2^3$.
    Nun ist
    \[
      m_1 = 9^{10} \cdot 11^5 =\qty(2^8)^{10} \cdot \qty(2^7)^5
      = 2^{80} \cdot 2^{35} = 2^{115}
    \]
    Da die 2 als Primitivwurzel von $\mathbb{Z}_{13}^5$ aller 12
    Verknüpfungen wieder 2 ergibt, ist dieser Ausdruck äquivalent zu
    $2^{115 \mod 12} = 2^{7} = 11$.

    Außerdem ist $m_2 = \qty\big(2^3)^{-1} \equiv 2^9 = 5$ und tatsächlich ist
    $8 \cdot 5 = 40 \equiv 1 \mod 13$.

    \textbf{Alternativ wird die Primitivwurzel in der Übung bestimmt durch:}
    $\abs{\mathbb{Z}_{13}^*} = 12$ und die Teiler von 12 sind:

    \begin{tikzpicture}
      \node (12) at (0,0) {12};
      \node (4) at (-1,-1) {4};
      \node (2_1) at (-1.5,-2) {2};
      \node (2_2) at (-0.5,-2) {2};
      \node (6) at (1,-1) {6};
      \node (2_3) at (0.5,-2) {2};
      \node (3) at (1.5,-2) {3};

      \draw (12) -- (4);
      \draw (12) -- (6);
      \draw (4) -- (2_1);
      \draw (4) -- (2_2);
      \draw (6) -- (2_3);
      \draw (6) -- (3);
    \end{tikzpicture}

    Somit genügt es zu prüfen ob $n^4 \ne 1$ und $n^6 \ne 1$, da alle anderen
    Teiler von 12 auch 4 und 6 teilen.

    \textbf{Alternativ in der Übung auch mit dem kleinen Fermat:}

    Es ist nicht $2|13$ und damit nach dem Lemma von Euler-Fermat
    $2^{13 - 1} \mod 13 = 1$.
    Nun ist $2^{115} = 9 \cdot 2^{12} \cdot 2^7 = 9 \cdot 1 \cdot 2^7 = 2^7$.

  \item Es ist $\qty\big(\mathbb{Z}_{12}; +)$ zyklisch mit zum Beispiel dem
    Erzeuger 1.
    Sei nun $f \colon \mathbb{Z}_{13}^* \to \mathbb{Z}_{12}$ mit
    $f\qty\big(2) = 1$ und $f\qty(2^n) = 1 \cdot n$.

    Seien $g_1, g_2 \in \mathbb{Z}_{13}^*$ beliebig.
    Dann ist $g_1 = 2^j, g_2 = 2^k$ mit $1 \leq j,k \leq 12$.
    Nun ist
    \[
      f\qty\big(g_1 \cdot g_2) = f\qty(2^j \cdot 2^k) = f\qty(2^{j + k})
      = \qty\big(j + k) \cdot 1 = j + k = f\qty\big(g_1) + f\qty\big(g_2)
    \]
  \end{enumerate}

  \textbf{Alternativ nach der Übung:}
  Sei $f \colon \mathbb{Z}_{12} \to \mathbb{Z}_{13}^*$ mit $f\qty\big(x) = 2^x$.
  Dann ist $f$ bijektiv, da 2 eine Primitivwurzel in $\mathbb{Z}_{13}^*$ und
  $f\qty\big(x + y) = 2^{x + y} = 2^x \cdot 2^y = f\qty\big(x) \cdot f\qty\big(y)$.

  Dabei ist $2^{x + y} = \begin{cases}
    2^{x + y} & \text{Wenn } x+ y < 12 \\
    2^{x + y - 12} & \text{Ansonsten}
  \end{cases}$
\end{enumerate}

\newpage
\paragraph{Ü 7.3} Wir betrachten die Symmetriegruppe $\qty\big(S_3; \circ)$ des
regulären Dreiecks, dessen Eckpunkte mit den Zahlen 1, 2 und 3 gegen den
Uhrzeigersinn bezeichnet sind.
\begin{enumerate}[(a)]
\item Ist $\qty\big(S_3; \circ)$ zyklisch?

  \subparagraph{Lsg.} Das gleichseitige Dreieck ist sowohl punkt- als auch
  Achsensymmetrisch.
  Für die einzelnen Achsen sind die Abbildungen:

  \begin{tikzpicture}[scale=1.8]
    \node[label = above:{1}] (A) at (0, 0) {};
    \node[label = below left:{2}] (B) at (-0.5, -0.866) {};
    \node[label = below right:{3}] (C) at (0.5, -0.866) {};
    \draw (A.center) -- (B.center) -- (C.center) -- (A.center);
    \node[label = {[blue] above right:{$\begin{pmatrix} 1 & 3 \end{pmatrix}$}}] (D) at (1, 0) {};
    \node[label = {[blue] above left:{$\begin{pmatrix} 1 & 2 \end{pmatrix}$}}] (E) at (-1, 0) {};
    \node[label = {[blue] below:{$\begin{pmatrix} 2 & 3 \end{pmatrix}$}}] (F) at (0, -1.722) {};
    \draw[dashed, blue] (A) -- (F);
    \draw[dashed, blue] (B) -- (D);
    \draw[dashed, blue] (C) -- (E);
  \end{tikzpicture}

  Die beiden Punktsymmetrischen Abbildungen sind
  $\begin{pmatrix} 1 & 2 & 3 \end{pmatrix}$ für die Rotation gegen den
  Uhrzeigersinn und
  $\begin{pmatrix} 1 & 3 & 2 \end{pmatrix}$ für die Rotation im Uhrzeigersinn.

  Schließlich verbleibt noch die Abbildung $\text{id}$, welche das Dreieck auf
  sich selbst abbildet.

  Offensichtlich befindet sich das Dreieck nach zweifacher Spiegelung an einer
  Achse, dreifacher Hintereinanderausführung einer Drehung oder Anwendung der
  Identität wieder am Anfang.
  Somit ist die Gruppe nicht zyklisch.

  \textbf{Alternativ nach der Übung:} Jede zyklische Gruppe ist auch eine Abelsche
  Gruppe.
  Da $\qty\big(1\:2)\qty\big(2\:3)=\qty\big(1\:2\:3)$, aber
  $\qty\big(3\:2)\qty\big(1\:2)=\qty\big(3\:2\:1)$, ist $S_3$ nicht abelsch und
  kann somit auch nicht zyklisch sein.

  Ansonsten ist die Ordnung eines Zyklus immer seine Länge.
  Da kein Zyklus die Länge 6 hat, ist die Gruppe nicht zyklisch.

\item Die Menge $D$ aller Drehungen aus $S_3$ bildet mit der
  Hintereinanderausführung ebenfalls eine Gruppe.
  Zeigen Sie, dass $\qty\big(D; \circ)$ isomorph zu
  $\qty\big(\mathbb{Z}_3; +)$ ist.

  \subparagraph{Lsg.} Das Dreieck lässt sich um 0°, 120° im Uhrzeigersinn
  und 240° im Uhrzeigersinn drehen.
  Drehungen um 0°, 240°, 120° gegen den Uhrzeigersinn sind dabei identisch
  zu den Drehungen im Uhrzeigersinn.

  Nun ist die Gruppe zyklisch, da aus zweifacher Anwendung der Drehung um 120°
  die Drehung um 240° und aus dreifacher Anwendung die Drehung um 0° entsteht.

  Somit ist $\qty\big(D; \circ)$ isomorph zu $\qty\big(\mathbb{Z}_3; +)$ mit dem
  Isomorphismus $f \colon D \to \mathbb{Z}_3$ und $f\qty\big(120°) = 1$.

  \textbf{Alternativ nach der Übung:}
  $f \colon \mathbb{Z}_3 \to D, k \mapsto \qty\big(1\:2\:3)^k$
\end{enumerate}

\newpage
\paragraph{Ü 7.4} Alice und Bob wollen mit dem Diffie-Hellmann-Merkle-Verfahren
einen geheimen Schlüssel erzeugen.
Sie wollen das Verfahren anhand kleiner Zahlen austesten.
Beide einigen sich auf die Primzahl $p = 37$ und die Primitivwurzel 2 aus
$\mathbb{Z}_p^*$.

Alice schickt an Bob die Zahl $a' = 3$.
Bob verwendet selbst die Zahl $b = 27$.
\begin{enumerate}[(a)]
\item Wie lautet der geheime gemeinsame Schlüssel?

  \subparagraph{Lsg.} Zunächst ermittelt Bob die Zahl $b' \coloneqq 2^b \mod p
  = 2^{27} \mod 37$.
  Dabei kann $2^{27} \mod 37$ mit der Methode ``Quadrieren und Multiplizieren''
  ermittelt werden:

  $\text{bin}\qty\big(27) = 11011$, also ist $2^{27} = 2^4 + 2^3 + 2^1 + 2^0$ und
  folglich
  \[
    2^{27} = 2^{2^4} \cdot 2^{2^3} \cdot 2^{2^1} \cdot 2^{2^0}
    = \qty(\qty(2^2 \cdot 2)^{2\cdot2} \cdot 2)^2 \cdot 2
  \]
  Nun lassen sich die Zwischenergebnisse leicht modulo 37 rechnen:

  \begin{tabular}{|c|c|c|c|c|}
    \hline
    1 & 1 & 0 & 1 & 1 \\
    \hline
    $2 \mod 37$ & $\colorbox{yellow}{$2$}^2 \cdot 2 \mod 37$
      & $\colorbox{blue!20}{$8$}^2 \mod 37$
      & $\colorbox{orange!20}{$27$}^2 \cdot 2 \mod 37$
      & $\colorbox{green!20}{$15$}^2 \cdot 2 \mod 37$ \\
    $= \colorbox{yellow}{$2$}$ & $= \colorbox{blue!20}{$8$}$
      & $= \colorbox{orange!20}{$27$}$ & $= \colorbox{green!20}{$15$}$
      & $= \colorbox{red!20}{$6$}$  \\
    \hline
  \end{tabular}

  Somit ist $b' = \colorbox{red!20}{$6$}$ und Bob teilt Alice $b'$ mit.
  Nun berechnen Alice und Bob mit den jeweils ihnen bekannten Zahlen das
  Geheimnis $c \qty\big(b')^a \mod p = \qty\big(a')^b \mod p = 3^{27} \mod 37$.

  Analog zu oben lässt sich $3^{27} \mod 37$ wieder über die Methode ``Quadrieren
  und Multiplizieren'' berechnen mit
  \[
    3^{27} = 3^{2^4} \cdot 3^{2^3} \cdot 3^{2^1} \cdot 3^{2^0}
    = \qty(\qty(3^2 \cdot 3)^{2\cdot2} \cdot 3)^2 \cdot 3
  \]

  \begin{tabular}{|c|c|c|c|c|}
    \hline
    1 & 1 & 0 & 1 & 1 \\
    \hline
    $3 \mod 37$ & $\colorbox{yellow}{$3$}^2 \cdot 3 \mod 37$
      & $\colorbox{blue!20}{$27$}^2 \mod 37$
      & $\colorbox{orange!20}{$26$}^2 \cdot 3 \mod 37$
      & $\colorbox{green!20}{$30$}^2 \cdot 3 \mod 37$ \\
    $= \colorbox{yellow}{$3$}$ & $= \colorbox{blue!20}{$27$}$
      & $= \colorbox{orange!20}{$26$}$ & $= \colorbox{green!20}{$30$}$
      & $= \colorbox{red!20}{$36$}$  \\
    \hline
  \end{tabular}

  Somit ist $c = \colorbox{red!20}{$36$}$.

\item Eva hat den Schlüsselaustausch über den gemeinsamen Kommunikationskanal
  belauscht und $a'$ erfahren.
  Wie berechnet sie den geheimen Schlüssel von Alice?

  \paragraph{Lsg.} Eva hat unter anderem die Zahlen $a'$, $p$ und die
  Primitivwurzel 2 aus $\mathbb{Z}_p^*$ erfahren.
  Nun gibt es für Eva kein einfaches Verfahren um $a$ zu ermitteln,
  Sie muss für $n = 0, \ldots, 37$ solange $2^n \mod 37$ ermitteln, bis
  sie die 3 erhält.
  Dann ist $a = n$.

  Mittels eines Computers kann $a$ zum Beispiel in Python durch
\begin{lstlisting}
>>> min(filter(lambda n: 2**n % 37 == 3, range(37)))
26
\end{lstlisting}
  bestimmt werden.
  Dabei steigt der Aufwand für große $p$ ins Unendliche.
\end{enumerate}
\end{document}
