\documentclass{scrreprt}

\usepackage{aligned-overset}
\usepackage{amsmath}
\usepackage{amsthm}
\usepackage{amssymb}
\usepackage{bm}
\usepackage[shortlabels]{enumitem}
\usepackage{framed}
\usepackage{hyperref}
\usepackage[utf8]{inputenc}
\usepackage{multicol}
\usepackage{mathtools}
\usepackage{pdflscape}
\usepackage{physics}
\usepackage{polynom}
\usepackage{tabularx}
\usepackage[table]{xcolor}
\usepackage{titling}
\usepackage{fancyhdr}
\usepackage{xfrac}
\usepackage{pgfplots}

\pgfplotsset{compat = newest}
\usepgfplotslibrary{fillbetween}
\usetikzlibrary{patterns}
\usetikzlibrary{through}


\author{Karsten Lehmann \\ 4935758}
\date{WiSe 2024/25}
\title{Nachbereitungsaufgaben 6\\INF-B-110, Lineare Algebra}

\setlength{\headheight}{26pt}
\pagestyle{fancy}
\fancyhf{}
\lhead{\thetitle}
\rhead{\theauthor}
\lfoot{\thedate}
\rfoot{Seite \thepage}

\begin{document}
\paragraph{N 6.2} Es seien
\[
  m_1 \quad m_2 \quad m_3 \quad m_4 \quad m_5 \quad m_6 \quad m_7
\]
die Ziffern Ihrer Matrikelnummer.

Für beliebige $k \in \mathbb{R}$ ist die folgende Matrix $A_k$ gegeben:
\[
  A_k \coloneqq \begin{pmatrix}
    1 & -2 & -2 & k   \\
    1 & -1 & 2  & 2   \\
    0 & 1  & 4  & k^2 \\
  \end{pmatrix} \in \mathbb{R}^{3 \times 4}
\]
\begin{enumerate}[(a)]
\item \label{n_6_2_b} Bestimmen Sie (in Abhängigkeit von $k$) den Rang von $A_k$
  und die Dimension des Kerns von $A_k$.
  Verifizieren Sie die Dimensionsformel für Matrizen.

  \subparagraph{Lsg.} In Übung 6.6 wurde bereits gezeigt, dass elementare
  Zeilenumformungen nicht den Rang einer Matrix ändern.
  Nun ist
  \begin{flalign*}
    \begin{pmatrix}
      1 & -2 & -2 & k   \\
      1 & -1 & 2  & 2   \\
      0 & 1  & 4  & k^2 \\
    \end{pmatrix}
    \overset{Z_1 = -1 \cdot \qty\big(Z_1 - Z_2)}&\leadsto
    \begin{pmatrix}
      0 & 1  & 4  & 2 - k \\
      1 & -1 & 2  & 2     \\
      0 & 1  & 4  & k^2   \\
    \end{pmatrix} \\
    \overset{Z_3 = Z_3 - Z_1}&\leadsto
    \begin{pmatrix}
      0 & 1  & 4  & 2 - k \\
      1 & -1 & 2  & 2     \\
      0 & 0  & 0  & k^2 + k - 2 \\
    \end{pmatrix} \\
    \overset{Z_1 \leftrightarrow Z_2}&\leadsto
    \begin{pmatrix}
      1 & -1 & 2  & 2     \\
      0 & 1  & 4  & 2 - k \\
      0 & 0  & 0  & \qty\big(k + 2)\qty\big(k - 1) \\
    \end{pmatrix}
  \end{flalign*}

  Nun sind die ersten beiden Zeilen durch ${A_k}_{1, 1} = 1$ und
  ${A_k}_{2, 1} = 0$ offensichtlich linear unabhängig.
  Die dritte Zeile der Matrix wird ebenfalls von den ersten beiden Zeilen linear
  unabhängig, falls sie eine ``\emph{Nullzeile}'' ist, d.h. ${A_k}_{3, 4} = 0$
  für $k = -2$ oder $k  = 1$.

  $\Rightarrow \text{Rang}\qty\big(A_1) = \text{Rang}\qty\big(A_{-2}) = 2$ und
  für alle anderen $k \in \mathbb{R} \setminus \qty\big{-2, 1}$ ist
  $\text{Rang}\qty\big(A_k) = 3$.

  Für jeden Vektor $\qty\big(x_1, x_2, x_3, x_4)^T$ im Kern von $A_k$ ist
  nun $x_4 \in \mathbb{R}$, falls $k \in \qty\big{-2, 1}$ und ansonsten
  $x_4 = 0$.
  Weiter sind
  \begin{itemize}
  \item für $k \in \qty\big{-2, 1}$
    \begin{itemize}[\null]
    \item $x_4 \in \mathbb{R}$
    \item $x_3 \in \mathbb{R}$
    \item $x_2 = \qty\big(k - 2)x_4 - 4x_3$
    \item $x_1 = x_2 - 2 \cdot x_3 - 2 x_4 = \qty\big(k - 4) \cdot x_4 - 6 x_3$
    \end{itemize}
    \[
      \text{Kern}\qty\big(A_k) = \text{Span}\qty(\qty{
        \begin{pmatrix}
          -6 \\
          -4 \\
          1  \\
          0  \\
        \end{pmatrix},
        \begin{pmatrix}
          k - 4 \\
          k - 2 \\
          0     \\
          1     \\
        \end{pmatrix}
      }), \text{ mit }
      \text{Dim}\qty(\text{Kern}\qty\big(A_k)) = 2
    \]

  \item für $k \in \mathbb{R} \setminus \qty\big{-2, 1}$
    \begin{itemize}[\null]
    \item $x_4 = 0$
    \item $x_3 \in \mathbb{R}$
    \item $x_2 = -4x_3$
    \item $x_1 = x_2 - 2x_3 = -6x_3$
    \end{itemize}
    \[
      \text{Kern}\qty\big(A_k) = \text{Span}\qty(\qty{
        \begin{pmatrix}
          -6 \\
          -4 \\
          1  \\
          0  \\
        \end{pmatrix}
      }), \text{ mit }
      \text{Dim}\qty(\text{Kern}\qty\big(A_k)) = 1
    \]
  \end{itemize}
  Nun besagt die Dimensionsformel, dass für $A \in K^{m \times n}$ gilt
  \[
    \text{Rang}\qty\big(A) + \text{Dim}\qty(\text{Kern}\qty\big(A)) = n
  \]
  Und tatsächlich ist für die $\mathbb{R}^{3 \times 4}$ Matrizen $A_1$ und
  $A_{-2}$
  \[
    \text{Rang}\qty\big(A_1) + \text{Dim}\qty(\text{Kern}\qty\big(A_1)) = 2 + 2 = 4
  \]
  und
  \[
    \text{Rang}\qty\big(A_{-2}) + \text{Dim}\qty(\text{Kern}\qty\big(A_{-2})) = 2 + 2 = 4
  \]
  sowie für alle anderen $k \in \mathbb{R} \setminus \qty\big{-2, 1}$
  \[
    \text{Rang}\qty\big(A_k) + \text{Dim}\qty(\text{Kern}\qty\big(A_k)) = 3 + 1 = 4
  \]

\item \label{n_6_2_b} Bestimmen Sie eine Basis $B$ des Kerns von $A_1$ und eine
  Basis $C$ des Spaltenraums von $A_1$

  \subparagraph{Lsg.} In der \hyperref[n_6_2_a]{Teilaufgabe (a)} wurde für
  $k \in \qty\big{-2, 1}$ bereits
  \[
    \text{Kern}\qty\big(A_k) = \text{Span}\qty(\qty{
      \begin{pmatrix}
        -6 \\
        -4 \\
        1  \\
        0  \\
      \end{pmatrix},
      \begin{pmatrix}
        k - 4 \\
        k - 2 \\
        0     \\
        1     \\
      \end{pmatrix}
    })
  \]
  ermittelt.
  Somit ist
  \[
    \text{Kern}\qty\big(A_1) = \text{Span}\qty(\qty{
      \begin{pmatrix}
        -6 \\
        -4 \\
        1  \\
        0  \\
      \end{pmatrix},
      \begin{pmatrix}
        -3 \\
        -1 \\
        0  \\
        1  \\
      \end{pmatrix}
    })
  \]
  mit der Basis $B = \qty{\qty\big(-6, -4, 1, 0)^T, \qty\big(-3, -1, 0, 1)^T}$.

  Weiter ist $\text{Dim}\qty(\text{Row}\qty\big(A_1)) =
  \text{Dim}\qty(\text{Col}\qty\big(A_1)) = \text{Rang}\qty\big(A_1) = 2$.
  Somit enthält eine Basis $C$ des Spaltenraums von $A_1$ zwei Elemente und da
  die ersten beiden Spalten von $A_1$ offensichtlich linear unabhängig sind, ist
  $C = \qty(\qty\big(1, 1, 0)^T, \qty\big(-2, -1, 1)^T)$ eine angeordnete Basis
  des Spaltenraums von $A_1$.

\newpage
\item \label{n_6_2_c} Zeigen Sie, dass der Vektor
  $\omega \coloneqq \qty\big(m_6, m_6 + m_7, m_7)^T$ im Spaltenraum
  $\text{Col}\qty\big(A_1)$ liegt.
  Bestimmen Sie reelle Zahlen $r, t \in \mathbb{R}$ so, dass für den Vektor
  $v \coloneqq \qty\big(1, 1, t, r)^T$ gilt:
  $A_1v = \omega$.

  \subparagraph{Lsg.} Die einzelnen Ziffern der Matrikelnummer sind

  \begin{tabular}{|c|c|c|c|c|c|c|}
    \hline
    4 & 9 & 3 & 5 & 7 & 5 & 8 \\
    \hline
    $m_1$ & $m_2$ & $m_3$ & $m_4$ & $m_5$ & $m_6$ & $m_7$ \\
    \hline
  \end{tabular}

  Nun ist $\omega = \qty\big(5, 13, 8)^T$.
  Aus der \hyperref[n_6_2_a]{Teilaufgabe (a)} ist bereits
  $\text{Rang}\qty\big(A_1) = 2$ bekannt.
  Wenn nun ebenfalls
  $\text{Rang}\qty\big(A_1|\omega) = \text{Rang}\qty\big(A_1) =  2$, dann ändert
  die Erweiterung von $A_1$ um den Vektor $\omega$ den Spaltenrang nicht, und
  $\omega$ wäre im Spaltenraum $A_1$ enthalten.

  Nun ist
  \begin{flalign*}
    \qty(\begin{array}{cccc|c}
      1 & -2 & -2 & 1 & 5  \\
      1 & -1 & 2  & 2 & 13 \\
      0 & 1  & 4  & 1 & 8  \\
    \end{array})
    \overset{Z_2 = Z_2 - Z_1}&\leadsto
    \qty(\begin{array}{cccc|c}
      1 & -2 & -2 & 1 & 5  \\
      0 & 1  & 4  & 1 & 8 \\
      0 & 1  & 4  & 1 & 8  \\
    \end{array}) \\
    \overset{Z_3 = Z_3 - Z_2}&\leadsto
    \qty(\begin{array}{cccc|c}
      1 & -2 & -2 & 1 & 5 \\
      0 & 1  & 4  & 1 & 8 \\
      0 & 0  & 0  & 0 & 0 \\
    \end{array}) \\
    \overset{Z_1 = Z_1 - Z_2}&\leadsto
    \qty(\begin{array}{cccc|c}
      1 & -3 & -6 & 0 & -3 \\
      0 & 1  & 4  & 1 & 8 \\
      0 & 0  & 0  & 0 & 0 \\
    \end{array})
  \end{flalign*}
  Somit ist tatsächlich $\text{Rang}\qty\big(A_1|\omega) = 2$ und $\omega$ im
  Spaltenraum von $A_1$ enthalten.

  Die Werte für $t, r$ erhält man, indem man den gegebenen Vektor in die
  Einzelnen Zeilen von $A_1|\omega$ einsetzt.
  So ist für die erste Zeile
  \begin{flalign*}
    1 - 3 - 6 \cdot t &= -3 && {\Big |} + 2 \\
    -6 \cdot t &= -1 && {\Big |} \cdot -\frac{1}{6} \\
    t &= \frac{1}{6}
  \end{flalign*}
  und für die zweite Zeile
  \begin{flalign*}
    1 + \frac{4}{6} + r &= 8 && {\Big |} - \frac{10}{6} \\
    r &= \frac{38}{6}
  \end{flalign*}

  $\Rightarrow A_1 \cdot \qty(1, 1, \frac{1}{6}, \frac{38}{6})^T = \qty\big(5, 13, 8)^T$

\newpage
\item Geben Sie den Koordinatenvektor von $\omega$ bezüglich $C$ an, d.h.
  berechnen Sie $\omega_C$.

  \subparagraph{Lsg.} Es ist
  \[
    21 \cdot \begin{pmatrix}
      1 \\
      1 \\
      0 \\
    \end{pmatrix} + 8 \cdot \begin{pmatrix}
      -2 \\
      -1 \\
      1  \\
    \end{pmatrix} = \begin{pmatrix}
      5  \\
      13 \\
      8  \\
    \end{pmatrix}
  \]
  Somit ist $\omega_C = \begin{pmatrix}
    21 \\
    8  \\
  \end{pmatrix}$

\item Bestimmen Sie mit Hilfe von \hyperref[n_6_2_b]{(b)} und
  \hyperref[n_6_2_c]{(c)} die Lösungsmenge des linearen Gleichungssystemes
  $A_1x = \omega$ an.

  \subparagraph{Lsg.} Aus der Übung ist bereits bekannt, dass die Lösungsmenge
  des inhomogenen Gleichungssystems $A_1|\omega$ eine zur Lösungsmenge des
  homogenen Gleichungssystems $A_1|\qty\big(0, 0, 0)^T$ verschobene Ebene ist.
  Die Verschiebung selbst ist dabei eine beliebige Lösung des inhomogenen
  Gleichungssytems und in \hyperref[n_6_2_c]{Teilaufgabe (c)} wurde bereits
  $v$ als Lösung ermittelt.

  Weiter ist die Basis
  $B = \qty{\qty\big(-6, -4, 1, 0)^T, \qty\big(-3, -1, 0, 1)^T}$ vom Kern des
  homogenen Gleichungssystems $A_1|\qty\big(0, 0, 0)^T$ bereits aus
  \hyperref[n_6_2_b]{Teilaufgabe (b)} bekannt und somit
  \[
    L = \qty{
      \begin{pmatrix}
        1 \\
        1 \\
        \frac{1}{6} \\
        \frac{38}{6} \\
      \end{pmatrix} + \lambda \cdot \begin{pmatrix}
        -6 \\
        -4 \\
        1  \\
        0  \\
      \end{pmatrix} + \mu \cdot \begin{pmatrix}
        -3 \\
        -1 \\
        0  \\
        1  \\
      \end{pmatrix}
      \:\middle|\:
      \lambda, \mu \in \mathbb{R}
    }
  \]

\end{enumerate}
\end{document}
