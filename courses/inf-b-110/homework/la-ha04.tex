\documentclass{scrreprt}

\usepackage{aligned-overset}
\usepackage{amsmath}
\usepackage{amsthm}
\usepackage{amssymb}
\usepackage{bm}
\usepackage[shortlabels]{enumitem}
\usepackage{framed}
\usepackage{hyperref}
\usepackage[utf8]{inputenc}
\usepackage{multicol}
\usepackage{mathtools}
\usepackage{physics}
\usepackage{polynom}
\usepackage{tabularx}
\usepackage[table]{xcolor}
\usepackage{titling}
\usepackage{fancyhdr}
\usepackage{xfrac}
\usepackage{pgfplots}

\pgfplotsset{compat = newest}
\usepgfplotslibrary{fillbetween}
\usetikzlibrary{patterns}
\usetikzlibrary{through}


\author{Karsten Lehmann \\ 4935758}
\date{WiSe 2024/25}
\title{Nachbereitungsaufgaben 4\\INF-B-110, Lineare Algebra}

\setlength{\headheight}{26pt}
\pagestyle{fancy}
\fancyhf{}
\lhead{\thetitle}
\rhead{\theauthor}
\lfoot{\thedate}
\rfoot{Seite \thepage}

\begin{document}
\paragraph{N 4.2} Beweisen, bzw. widerlegen Sie, dass die folgenden Mengen
Untervektorräume der angegebenen $\mathbb{R}$-Vektorräume sind.
\begin{enumerate}[(a)]
  \item $\qty{f \in \text{Abb}\qty\big(\mathbb{R}, \mathbb{R}) \:\middle|\: f \text{ ist bijektiv}} \subseteq \text{Abb}\qty\big(\mathbb{R}, \mathbb{R})$
  \item $\qty{f \in \text{Abb}\qty\big(\mathbb{R}, \mathbb{R}) \:\middle|\: \forall\: x \in \mathbb{R} \colon f\qty\big(x) = -f\qty\big(-x)} \subseteq \text{Abb}\qty\big(\mathbb{R}, \mathbb{R})$
  \item $\qty{f \in \text{Abb}\qty\big(\mathbb{R}, \mathbb{R}) \:\middle|\: \exists\: y \in \mathbb{R} \: \forall\: x \in \mathbb{R} \colon f\qty\big(x) = y} \subseteq \text{Abb}\qty\big(\mathbb{R}, \mathbb{R})$
  \end{enumerate}

  \subparagraph{Lsg.}
  \begin{enumerate}[(a)]
  \item Sei $g \in \text{Abb}\qty\big(\mathbb{R}, \mathbb{R})$ mit
    $g\qty\big(x) = 0$ das Nullelement aus dem $\mathbb{R}$-Vektorraum
    $\text{Abb}\qty\big(\mathbb{R}, \mathbb{R})$.
    Dann ist $g$ nicht injektiv, da für $1, 2 \in \mathbb{R}$ gilt
    $g\qty\big(1) = 0 = g\qty\big(2)$.
    Somit ist $g$ auch nicht bijektiv und damit das Nullelement nicht in der
    gegebenen Menge enthalten.

    $\Rightarrow$ kein Untervektorraum.

  \item Sei $g \in \text{Abb}\qty\big(\mathbb{R}, \mathbb{R})$ mit
    $g\qty\big(x) = 0$ das Nullelement aus dem $\mathbb{R}$-Vektorraum
    $\text{Abb}\qty\big(\mathbb{R}, \mathbb{R})$.

    Dann gilt für alle $x \in \mathbb{R}$, dass
    $g\qty\big(x) = 0 = -g\qty\big(-x)$.

    $\Rightarrow$ das Nullelement ist in der gegebenen Menge enthalten.

    Seien nun $f, h \in \text{Abb}\qty\big(\mathbb{R}, \mathbb{R})$ beliebig mit
    $f$ ist bijektiv und $h$ ist bijektiv.
    Sei weiter auch $x \in \mathbb{R}$ beliebig.
    Dann ist
    \[
      \qty\big(f + h)\qty\big(x) = f\qty\big(x) + h\qty\big(x)
      = -f\qty\big(-x) - h\qty\big(-x)
      = -\qty(f\qty\big(-x) + h\qty\big(-x))
      = -\qty\big(f + h)\qty\big(-x)
    \]

    $\Rightarrow f + h$ ist in der Menge enthalten.

    $\Rightarrow$ die Menge ist bezüglich der Addition abgeschlossen.

    Seien nun $f \in \text{Abb}\qty\big(\mathbb{R}, \mathbb{R})$ beliebig mit
    $f$ ist injektiv und $\lambda \in \mathbb{R}$ beliebig.
    Dann ist
    \[
      \qty\big(\lambda \cdot f)\qty\big(x) = \lambda \cdot f\qty\big(x) = \lambda \cdot \qty(-f\qty\big(-x))
      = -\lambda \cdot f\qty\big(-x) = -\qty\big(\lambda \cdot f)\qty\big(-x)
    \]

    $\Rightarrow$ die Menge ist bezüglich der Skalarmultiplikation abgeschlossen.

    $\Rightarrow$ die Menge ist ein Untervektorraum von dem $\mathbb{R}$-Vektorraum
    $\text{Abb}\qty\big(\mathbb{R}, \mathbb{R})$.

  \item Sei $g \in \text{Abb}\qty\big(\mathbb{R}, \mathbb{R})$ mit
    $g\qty\big(x) = 0$ das Nullelement aus dem $\mathbb{R}$-Vektorraum
    $\text{Abb}\qty\big(\mathbb{R}, \mathbb{R})$.

    Dann existiert $0 \in \mathbb{R}$, so dass für alle $x \in \mathbb{R}$
    gilt $g\qty\big(x) = 0$.

    $\Rightarrow$ das Nullelement ist in der Menge enthalten.

    Seien weiter $f, h \in \text{Abb}\qty\big(\mathbb{R}, \mathbb{R})$ sowie
    $\lambda \in \mathbb{R}$ beliebig.
    Dann existieren $y_f, y_h \in \mathbb{R}$ mit $f\qty\big(x) = y_f$ sowie
    $h\qty\big(x) = y_h$ für alle $x \in \mathbb{R}$ mit
    $y_f + y_h \in \mathbb{R}$ als auch $\lambda \cdot y_f \in \mathbb{R}$.

    Sei noch $j, k \in \text{Abb}\qty\big(\mathbb{R}, \mathbb{R})$ mit
    $j\qty\big(x) = y_f + y_h$ und $k\qty\big(x) = \lambda \cdot y_f$.

    Schließlich ist
    \[
      \qty\big(f + h)\qty\big(x) = f\qty\big(x) + h\qty\big(x) = y_f + y_h
      = j\qty\big(x)
    \]
    und
    \[
      \qty\big(\lambda \cdot f)\qty\big(x) = \lambda \cdot f\qty\big(x)
      = \lambda \cdot y_f = k\qty\big(x)
    \]
    $\Rightarrow$ die Menge ist bezüglich der Addition und Skalarmultiplikation
    abgeschlossen.

    $\Rightarrow$ die Menge ist ein Untervektorraum von dem $\mathbb{R}$-Vektorraum
    $\text{Abb}\qty\big(\mathbb{R}, \mathbb{R})$.
  \end{enumerate}
\end{document}
