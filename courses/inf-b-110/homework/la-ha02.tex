\documentclass{scrreprt}

\usepackage{amsmath}
\usepackage{amsthm}
\usepackage{amssymb}
\usepackage{bm}
\usepackage[shortlabels]{enumitem}
\usepackage{framed}
\usepackage{hyperref}
\usepackage[utf8]{inputenc}
\usepackage{multicol}
\usepackage{mathtools}
\usepackage{physics}
\usepackage{polynom}
\usepackage{tabularx}
\usepackage[table]{xcolor}
\usepackage{titling}
\usepackage{fancyhdr}
\usepackage{xfrac}
\usepackage{pgfplots}

\pgfplotsset{compat = newest}
\usepgfplotslibrary{fillbetween}
\usetikzlibrary{patterns}
\usetikzlibrary{through}


\author{Karsten Lehmann \\ 4935758}
\date{WiSe 2024/25}
\title{Nachbereitungsaufgaben 2\\INF-B-110, Lineare Algebra}

\setlength{\headheight}{26pt}
\pagestyle{fancy}
\fancyhf{}
\lhead{\thetitle}
\rhead{\theauthor}
\lfoot{\thedate}
\rfoot{Seite \thepage}

\begin{document}
\paragraph{N 2.2} Es seien
\[
  x_1 \qquad x_2 \qquad x_3 \qquad x_4 \qquad x_5 \qquad x_6 \qquad x_7
\]
die Ziffern Ihrer Matrikelnummer.
Weiter sei $x \coloneqq x_6 + x_7i \in \mathbb{C}$.
Wir betrachten die Matrix
$A \coloneqq \begin{pmatrix} 1 & k \\ 0 & i \end{pmatrix} \in \mathbb{C}^{2 \times 2}$.

\begin{enumerate}[(a)]
\item Berechnen Sie $A^2$, $A^3$ und $A^4$.

  \subparagraph{Lsg.} Die einzelnen Ziffern der Matrikelnummer sind

  \begin{tabular}{|c|c|c|c|c|c|c|}
    \hline
    4 & 9 & 3 & 5 & 7 & 5 & 8 \\
    \hline
    $x_1$ & $x_2$ & $x_3$ & $x_4$ & $x_5$ & $x_6$ & $x_7$ \\
    \hline
  \end{tabular}

  Also ist $k = 5 + 8 \cdot i$ und
  $A = \begin{pmatrix} 1 & 5 + 8 \cdot i \\ 0 & i \end{pmatrix}$.
  Nun sind

  \begin{flalign*}
    A^2 &= \begin{pmatrix}
      1 \cdot 1 + \qty\big(5 + 8 \cdot i) \cdot 0 & 1 \cdot \qty\big(5 + 8 \cdot i) + \qty\big(5 + 8 \cdot i) \cdot i \\
      0 \cdot 1 + i \cdot 0                       & 0 \cdot \qty\big(5 + 8 \cdot i) + i \cdot i \\
    \end{pmatrix} & \\
        &= \begin{pmatrix}
          1 & -3 + 13i \\
          0 & -1 \\
        \end{pmatrix} \\
    A^3 = A^2 \cdot A &= \begin{pmatrix}
      1 \cdot 1 + \qty\big(-3 + 13 \cdot i) \cdot 0 & 1 \cdot \qty\big(5 + 8 \cdot i) + \qty\big(-3 + 13 \cdot i) \cdot i \\
      0 \cdot 1 + -1 \cdot 0                        & 0 \cdot \qty\big(5 + 8 \cdot i) + -1 \cdot i \\
    \end{pmatrix} \\
        &= \begin{pmatrix}
          1 & -8 + 5i \\
          0 & -i \\
        \end{pmatrix} \\
   A^4 = A^3 \cdot A &= \begin{pmatrix}
      1 \cdot 1 + \qty\big(-8 + 5 \cdot i) \cdot 0 & 1 \cdot \qty\big(5 + 8 \cdot i) + \qty\big(-8 + 5 \cdot i) \cdot i \\
      0 \cdot 1 + -i \cdot 0                       & 0 \cdot \qty\big(5 + 8 \cdot i) + -i \cdot i \\
    \end{pmatrix} \\
        &= \begin{pmatrix}
          1 & 0 \\
          0 & 1 \\
        \end{pmatrix} = E_2
  \end{flalign*}

\item Geben Sie (mit Begründung) eine Vermutung an, wie $A^n$ (für eine beliebige
  natürliche Zahl $n \in \mathbb{N}$) aussieht.

  \subparagraph{Lsg.} Da $A^4$ der Einheitsmatrix entspricht, wird
  $A^5 = A^4 \cdot A = E_2 \cdot A$ wieder der Matrix $A$ entsprechen,
  $A^6 = A^2$ und so weiter.
  Für ein beliebiges $n \in \mathbb{N}$ gilt somit
  $A^n = A^{n \mod 4}$ und $A^n$ nimmt eine der oben aufgezeigten Matrizen an.

\newpage
\item Bestimmen Sie eine komplexe Zahl $z \coloneqq x + yi$, so dass
  $A^3 + zA^2 + iA = 0_{2 \times 2}$ gilt.

  \subparagraph{Lsg.} Es ist
  \begin{flalign*}
    A^3 + zA^2 + iA &= \begin{pmatrix}
      \qty(A^3)_{11} + z \cdot \qty(A^2)_{11} + i \cdot \qty\big(A)_{11} & \qty(A^3)_{12} + z \cdot \qty(A^2)_{12} + i \cdot \qty\big(A)_{12} \\
      \qty(A^3)_{21} + z \cdot \qty(A^2)_{21} + i \cdot \qty\big(A)_{21} & \qty(A^3)_{22} + z \cdot \qty(A^2)_{22} + i \cdot \qty\big(A)_{22} \\
    \end{pmatrix} & \\
    &= \begin{pmatrix}
      1 + z \cdot 1 + i \cdot 1 & \qty\big(-8 + 5 \cdot i) + z \cdot \qty\big(-3 + 13 \cdot i) + i \cdot \qty\big(5 + 8 \cdot i) \\
      0 + z \cdot 0 + i \cdot 0 & -i + z \cdot -1 + i \cdot i \\
   \end{pmatrix} \\
    &= \begin{pmatrix}
      1 + z + i & \qty\big(-16 + 10 \cdot i) + z \cdot \qty\big(-3 + 13 \cdot i)  \\
      0  & -i - z - 1 \\
   \end{pmatrix}
  \end{flalign*}
  Also muss $z$ so sein, dass
  \begin{flalign*}
    1 + i + z &= 0 & \\
    \qty\big(-16 + 10 \cdot i) + z \cdot \qty\big(-3 + 13 \cdot i) &= 0 \\
    -z - (1 + i) &= 0 \\
  \end{flalign*}
  Aus der ersten und dritten Gleichung ist $z = -1 -i$ direkt ersichtlich.
  Zur Probe für die zweite Zeile:
  \begin{flalign*}
    \qty\big(-16 + 10 \cdot i) + z \cdot \qty\big(-3 + 13 \cdot i) &= 0 \\
    \qty\big(-16 + 10 \cdot i) + \qty\big(-1 -i) \cdot \qty\big(-3 + 13 \cdot i) &= 0 \\
    \qty\big(-16 + 10 \cdot i) + \qty\big((-1) \cdot (-3) -  (-1) \cdot 13) + \qty\big((-1) \cdot 13 + (-3) \cdot (-1)) \cdot i &= 0 \\
    \qty\big(-16 + 10 \cdot i) + \qty\big(3 + 13) + \qty\big(-13 + 3) \cdot i &= 0 \\
    \qty\big(-16 + 10 \cdot i) + \qty\big(16 - 10 \cdot i) &= 0 \\
    0 &= 0
  \end{flalign*}
  $\Rightarrow z = -1 -i$

\newpage
\item Bestimmen Sie eine Matrix $B \in \mathbb{C}^{2 \times 2}$, so dass
  $A \cdot B = E_2$ gilt.

  \subparagraph{Lsg.} Es ist
  \begin{flalign*}
    A \cdot B &= \begin{pmatrix}
      1 & 5 + 8 \cdot i \\
      0 & i \\
    \end{pmatrix} \cdot \begin{pmatrix}
      b_{11} & b_{12} \\
      b_{21} & b_{22} \\
    \end{pmatrix} &\\
   &= \begin{pmatrix}
      1 \cdot b_{11} + \qty\big(5 + 8 \cdot i) \cdot b_{21} & 1 \cdot b_{12} + \qty\big(5 + 8 \cdot i) \cdot b_{22} \\
      0 \cdot b_{11} + i \cdot b_{21}                       & 0 \cdot b_{12} + i \cdot b_{22} \\
    \end{pmatrix} &\\
  \end{flalign*}
  Aus $A \cdot B = \begin{pmatrix} 1 & 0 \\ 0 & 1 \end{pmatrix}$ folgt nun
  \[
    \begin{pmatrix}
      1 & 0 & 5 + 8 \cdot i & 0 \\
      0 & 1 & 0 & 5 + 8 \cdot i \\
      0 & 0 & i & 0 \\
      0 & 0 & 0 & i \\
    \end{pmatrix} \cdot \begin{pmatrix}
      b_{11} \\
      b_{12} \\
      b_{21} \\
      b_{22}
    \end{pmatrix} = \begin{pmatrix}
      1 \\
      0 \\
      0 \\
      1 \\
    \end{pmatrix}
  \]
  Die Lösung des linearen Gleichungssystemes:
  \begin{flalign*}
    \qty(\begin{array}{cccc|c}
      1 & 0 & 5 + 8 \cdot i & 0 & 1 \\
      0 & 1 & 0 & 5 + 8 \cdot i & 0 \\
      0 & 0 & i & 0 & 0 \\
      0 & 0 & 0 & i & 1 \\
    \end{array}) &\overset{\text{Zeile 4} \cdot (-i)}{\leadsto} \qty(\begin{array}{cccc|c}
      1 & 0 & 5 + 8 \cdot i & 0 & 1 \\
      0 & 1 & 0 & 5 + 8 \cdot i & 0 \\
      0 & 0 & i & 0 & 0 \\
      0 & 0 & 0 & 1 & -i \\
    \end{array}) & \\
    &\overset{\text{Zeile 3} \cdot (-i)}{\leadsto} \qty(\begin{array}{cccc|c}
      1 & 0 & 5 + 8 \cdot i & 0 & 1 \\
      0 & 1 & 0 & 5 + 8 \cdot i & 0 \\
      0 & 0 & 1 & 0 & 0 \\
      0 & 0 & 0 & 1 & -i \\
    \end{array}) \\
    &\overset{\text{Zeile 2 - Zeile 4} \cdot (5 + 8 \cdot i)}{\leadsto} \qty(\begin{array}{cccc|c}
      1 & 0 & 5 + 8 \cdot i & 0 & 1 \\
      0 & 1 & 0 & 0 & -8 + 5i \\
      0 & 0 & 1 & 0 & 0 \\
      0 & 0 & 0 & 1 & -i \\
    \end{array}) \\
    &\overset{\text{Zeile 1 - Zeile 3} \cdot (5 + 8 \cdot i)}{\leadsto} \qty(\begin{array}{cccc|c}
      1 & 0 & 0 & 0 & 1 \\
      0 & 1 & 0 & 0 & -8 + 5i \\
      0 & 0 & 1 & 0 & 0 \\
      0 & 0 & 0 & 1 & -i \\
    \end{array})
  \end{flalign*}
  $\Rightarrow B = \begin{pmatrix} 1 & -8 + 5i \\ 0 & -i \end{pmatrix}$
\end{enumerate}
\end{document}
