\documentclass{scrreprt}

\usepackage{amsmath}
\usepackage{amsthm}
\usepackage{amssymb}
\usepackage{bm}
\usepackage[shortlabels]{enumitem}
\usepackage{hyperref}
\usepackage[utf8]{inputenc}
\usepackage{multicol}
\usepackage{mathtools}
\usepackage{pdflscape}
\usepackage{physics}
\usepackage{polynom}
\usepackage{tabularx}
\usepackage[table]{xcolor}
\usepackage{titling}
\usepackage{fancyhdr}
\usepackage{xfrac}
\usepackage{pgfplots}

\pgfplotsset{compat = newest}
\usepgfplotslibrary{fillbetween}
\usetikzlibrary{arrows, arrows.meta}
\usetikzlibrary{patterns}

\author{Karsten Lehmann \\ 4935758}
\date{WiSe 2024/25}
\title{Nachbereitungsaufgaben 2\\INF-B-110, Diskrete Strukturen}

\setlength{\headheight}{26pt}
\pagestyle{fancy}
\fancyhf{}
\lhead{\thetitle}
\rhead{\theauthor}
\lfoot{\thedate}
\rfoot{Seite \thepage}

\begin{document}
\paragraph{N2}
\begin{enumerate}[(a)]
\item Auf der Menge der komplexen Zahlen $\mathbb{C}$ ist die Abbildung
  \[
    f \colon \mathbb{C} \to \mathbb{C} \text{ mit }
    f\qty\big(Z) = -1 + i \cdot \qty\big(z - 1)
  \]
  gegeben.
  Zeigen Sie, dass $f$ injektiv und surjektiv (d.h. bijektiv) ist.

  \subparagraph{Lsg.} Seien $z_1, z_2 \in \mathbb{C}$ beliebig mit
  $f\qty\big(z_1) = f\qty\big(z_2)$.
  Dann ist
  \begin{flalign*}
    -1 + i \cdot \qty\big(z_1 - 1) &= -1 + i \cdot \qty\big(z_2 - 1) && {\Big |} +1 && \\
    i \cdot \qty\big(z_1 - 1) &= i \cdot \qty\big(z_2 - 1) && {\Big |} \cdot \qty\big(-i) \\
    z_1 - 1 &= z_2 - 1  && {\Big |} +1\\
    z_1 &= z_2
  \end{flalign*}
  $\Rightarrow f$ ist injektiv.

  Sei nun weiter $y \in \mathbb{C}$ beliebig.
  Dann ist ebenfalls $1 - i \cdot \qty\big(y + 1) \in \mathbb{C}$.
  Nun ist
  \begin{flalign*}
    f\qty\big(1 - i \cdot \qty\big(y + 1)) &= -1 + i \cdot \qty\big((1 - i \cdot \qty\big(y + 1)) - 1) & \\
    &=  -1 + i \cdot \qty\big(- i \cdot \qty\big(y + 1)) \\
    &=  -1 + i \cdot \qty\big(-i \cdot y  -i)) \\
    &=  -1 + \qty\big(y + 1)) \\
    &= y
  \end{flalign*}
  $\Rightarrow$ für $y$ existiert
  $z = \qty\big(1 - i \cdot \qty\big(y + 1)) \in \mathbb{C}$
  mit $f\qty\big(z) = y$.

  $\Rightarrow$ da $y$ beliebig, ist $f$ surjektiv.

  $\Rightarrow f$ ist bijektiv.

\item Gegeben sind auf der Menge
  $X = \qty\big{1, 2, 3, 4, 5, 6, 7, 8, 9, 10}$ die zwei Permutationen:
  \[
    \alpha = \begin{pmatrix}
      1 & 2 & 3 & 4 & 5 & 6 & 7 & 8 & 9 & 10 \\
      3 & 8 & 1 & 2 & 10 & 9 & 7 & 6 & 4 & 5
    \end{pmatrix}, \qquad
    \beta = \begin{pmatrix} 1 & 5 & 4 \end{pmatrix} \circ
    \begin{pmatrix} 1 & 2 & 7 & 8 & 5 \end{pmatrix} \circ
    \begin{pmatrix} 2 & 3 & 10 \end{pmatrix}
  \]
  \begin{itemize}
  \item Geben sie $\alpha$ und $\beta$ in Zyklenschreibweise an.

    \subparagraph{Lsg.} Es sind
    \[
      \alpha = \begin{pmatrix} 1 & 3 \end{pmatrix}
      \begin{pmatrix} 2 & 8 & 6 & 9 & 4 \end{pmatrix}
      \begin{pmatrix} 5 & 10 \end{pmatrix}
      \begin{pmatrix} 7 \end{pmatrix}
      = \begin{pmatrix} 1 & 3 \end{pmatrix}
      \begin{pmatrix} 2 & 8 & 6 & 9 & 4 \end{pmatrix}
      \begin{pmatrix} 5 & 10 \end{pmatrix}
    \]
    und
    \[
      \beta = \begin{pmatrix} 1 & 2 & 3 & 10 & 7 & 8 & 4 \end{pmatrix}
      \begin{pmatrix} 5 \end{pmatrix}
      \begin{pmatrix} 6 \end{pmatrix}
      \begin{pmatrix} 9 \end{pmatrix}
      = \begin{pmatrix} 1 & 2 & 3 & 10 & 7 & 8 & 4 \end{pmatrix}
    \]

  \newpage
  \item Geben Sie für $\alpha$ und $\beta$ jeweils alle Fixpunkte an.

    \subparagraph{Lsg.} Die Permutation $\alpha$ hat 7 als einzigen Fixpunkt und
    die Permutation $\beta$ hat die Fixpunkte 5, 6 und 9.

  \item Stellen Sie $\alpha$ als Komposition von Transpositionen dar.

    \subparagraph{Lsg.} Es ist
    \[
      \alpha = \begin{pmatrix} 1 & 3 \end{pmatrix} \circ
      \begin{pmatrix} 2 & 8 \end{pmatrix} \circ
      \begin{pmatrix} 8 & 6 \end{pmatrix} \circ
      \begin{pmatrix} 6 & 9 \end{pmatrix} \circ
      \begin{pmatrix} 9 & 4 \end{pmatrix} \circ
      \begin{pmatrix} 5 & 10 \end{pmatrix}
    \]
  \end{itemize}
\end{enumerate}
\end{document}
