\documentclass{scrreprt}

\usepackage{amsmath}
\usepackage{amsthm}
\usepackage{amssymb}
\usepackage{bm}
\usepackage[shortlabels]{enumitem}
\usepackage{hyperref}
\usepackage[utf8]{inputenc}
\usepackage{multicol}
\usepackage{mathtools}
\usepackage{pdflscape}
\usepackage{physics}
\usepackage{polynom}
\usepackage{tabularx}
\usepackage[table]{xcolor}
\usepackage{titling}
\usepackage{fancyhdr}
\usepackage{xfrac}
\usepackage{pgfplots}

\pgfplotsset{compat = newest}
\usepgfplotslibrary{fillbetween}
\usetikzlibrary{arrows, arrows.meta}
\usetikzlibrary{patterns}

\author{Karsten Lehmann \\ 4935758}
\date{WiSe 2024/25}
\title{Nachbereitungsaufgaben 1\\INF-B-110, Diskrete Strukturen}

\setlength{\headheight}{26pt}
\pagestyle{fancy}
\fancyhf{}
\lhead{\thetitle}
\rhead{\theauthor}
\lfoot{\thedate}
\rfoot{Seite \thepage}

\begin{document}
\paragraph{N1}
\begin{enumerate}[(a)]
\item Beweisen Sie, dass für beliebige Teilmengen $A$, $B$ und $C$ einer Menge
  $U$ gilt:
  \[
    \qty\big(A \setminus B) \cap \qty\big(A \setminus C) = A \setminus \qty\big(B \cup C)
  \]

  Die in der Vorlesung aufgelisteten Rechenregeln für Mengenoperationen können
  Sie dazu verwenden.
  Geben Sie in den entsprechenden Beweisschritten dann mit an, welche Rechenregel
  Sie wo genau anwenden.

  \subparagraph{Lsg.} Es ist
  \begin{flalign*}
    \qty\big(A \setminus B) \cap \qty\big(A \setminus C)
    &= \qty\big{e \:{\big |}\: e \in \qty\big(A \setminus B) \text{ und } e \in \qty\big(A \setminus C)} && \text{(Def. Schnitt)} && \\
    &= \qty\big{e
        \:{\big |}\:
        \qty\big(e \in A \text{ und } e \notin B)
        \text{ und }
        \qty\big(e \in A \text{ und } e \notin C)
      } && \text{(Def. Differenz)} \\
    &= \qty\big{e
        \:{\big |}\:
        e \in A \text{ und } e \notin B \text{ und } e \notin C
      } && \text{(Assoziativität)} \\
    &= \qty\big{e
        \:{\big |}\:
        e \in A \text{ und } e \notin \qty\big{a \:{\big |}\: a \in B \text{ und } a \in C}
      }  \\
    &= \qty\big{e
        \:{\big |}\:
        e \in A \text{ und } e \notin \qty\big(B \cup C)
      } && \text{(Def. Vereinigung)} \\
    &= A \setminus \qty\big(B \cup C) && \text{(Def. Differenz)}
  \end{flalign*}

\item Gegeben sind die Mengen
  $A_m = \qty\big{n \in \mathbb{N} \:{\big |}\: m < n \leq 4m}$
  für $m = 0, 1, 2, \ldots$
  Geben Sie die Mengen $A_0$, $A_1$ und $A_2$, sowie die Potenzmengen
  $\mathcal{P}\qty\big(A_0)$ und $\mathcal{P}\qty\big(A_1)$ als Mengen
  konkreter Elemente an.

  Bestimmen Sie die Mächtigkeit $\abs{\mathcal{P}\qty\big(A_m)}$ für
  beliebiges $m \in \mathbb{N}$.

  \subparagraph{Lsg.} Es sind
  \[
    A_0 = \emptyset,
    A_1 = \qty\big{2, 3, 4},
    A_2 = \qty\big{3, 4, 5, 6, 7, 8}
  \]

  Weiter sind
  \[
    \mathcal{P}\qty\big(A_0) = \qty\big{\emptyset},
    \mathcal{P}\qty\big(A_1) = \qty\big{
      \emptyset, \qty\big{2}, \qty\big{3}, \qty\big{4},
      \qty\big{2, 3}, \qty\big{2, 4}, \qty\big{3, 4},
      \qty\big{2, 3, 4}
    }
  \]

  Schließlich ist $A_m = \qty\big{m + 1, \ldots, 4m}$ mit der Mächtigkeit
  \[
    \abs{\qty\big{m + 1, \ldots, 4m}} = 4m - m = 3m
  \]

  Aus der Vorlesung ist bereits bekannt, dass
  $\abs{\mathcal{P}\qty\big(A)} = 2^{\abs{A}}$.

  $\Rightarrow$ $\abs{\mathcal{P}\qty\big(A_m)} = 2^{3m}$
\end{enumerate}
\end{document}
