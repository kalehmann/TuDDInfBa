\documentclass{scrreprt}

\usepackage{aligned-overset}
\usepackage{amsmath}
\usepackage{amsthm}
\usepackage{amssymb}
\usepackage{bm}
\usepackage[shortlabels]{enumitem}
\usepackage{framed}
\usepackage{hyperref}
\usepackage[utf8]{inputenc}
\usepackage{multicol}
\usepackage{mathtools}
\usepackage{pdflscape}
\usepackage{physics}
\usepackage{polynom}
\usepackage{tabularx}
\usepackage[table]{xcolor}
\usepackage{titling}
\usepackage{fancyhdr}
\usepackage{xfrac}
\usepackage{pgfplots}

\pgfplotsset{compat = newest}
\usepgfplotslibrary{fillbetween}
\usetikzlibrary{calc}
\usetikzlibrary{patterns}
\usetikzlibrary{through}


\author{Karsten Lehmann \\ 4935758}
\date{WiSe 2024/25}
\title{Nachbereitungsaufgaben 10\\INF-B-110, Lineare Algebra}

\setlength{\headheight}{26pt}
\pagestyle{fancy}
\fancyhf{}
\lhead{\thetitle}
\rhead{\theauthor}
\lfoot{\thedate}
\rfoot{Seite \thepage}

\begin{document}
\paragraph{N 10.2}
\begin{enumerate}[(a)]
\item Es seien $x_4$ und $x_5$ die 4. und 5. Ziffer Ihrer Matrikelnummer.
  Bestimmen Sie die Eigenwerte $\lambda_i$ und eine Basis der jeweiligen
  Eigenräume $\text{Eig}\qty\big(A, \lambda_i)$ für die Matrix
  \[
    A = \begin{pmatrix}
      x_4 & 1   & 1   \\
      1   & x_4 & 1   \\
      1   & 1   & x_5 \\
    \end{pmatrix} \in \mathbb{R}^{3 \times 3}
  \]

  \subparagraph{Lsg.} Die Matrikelnummer ist 4935758, somit $x_4 = 5$ und
  $x_5 = 7$.
  Nun ist
  \begin{flalign*}
    \det\qty\big(A - \lambda \cdot E_3) =
    \det\begin{pmatrix}
      5 - \lambda & 1           & 1           \\
      1           & 5 - \lambda & 1           \\
      1           & 1           & 7 - \lambda \\
    \end{pmatrix}
    \overset{S_2 = S_2 - S_1}&\leadsto
    \det\begin{pmatrix}
      5 - \lambda & \lambda - 4  & 1           \\
      1           & 4 - \lambda & 1           \\
      1           & 0           & 7 - \lambda \\
    \end{pmatrix} \\
    \overset{S_3 = S_3 - S_1}&\leadsto
    \det\begin{pmatrix}
      5 - \lambda & \lambda - 4 & \lambda - 4 \\
      1           & 4 - \lambda & 0           \\
      1           & 0           & 6 - \lambda \\
    \end{pmatrix} \\
    \overset{Z_1 = Z_1 + Z_2 + Z_3}&\leadsto
    \det\begin{pmatrix}
      7 - \lambda & 0           & 2           \\
      1           & 4 - \lambda & 0           \\
      1           & 0           & 6 - \lambda \\
    \end{pmatrix} \\
    \overset{\text{Entwicklung nach der 2. Spalte/Zeile}}&=
    \qty\big(4 - \lambda) \cdot \det\begin{pmatrix}
      7 - \lambda & 2           \\
      1           & 6 - \lambda \\
    \end{pmatrix} \\
    &= \qty\big(4 - \lambda) \cdot \qty(\qty\big(7 - \lambda)\qty\big(6 - \lambda) - 2) \\
    &= \qty\big(4 - \lambda) \cdot \qty\big(\lambda^2 - 13\lambda + 40) \\
    &= \qty\big(4 - \lambda)\qty\big(\lambda - 5)\qty\big(\lambda - 8)
  \end{flalign*}
  Somit hat $A$ die Eigenwerte $\lambda_1 = 4$, $\lambda_2 = 5$ und
  $\lambda_3 = 8$.
  Nun sind die Eigenräume für
  \begin{itemize}
  \item[$\lambda_1 = 4$]
    \begin{flalign*}
      A - 4 \cdot E_3 &= \begin{pmatrix}
        1 & 1 & 1 \\
        1 & 1 & 1 \\
        1 & 1 & 3 \\
      \end{pmatrix}
      \overset{Z_2 = Z_2 - Z_1}\leadsto
      \begin{pmatrix}
        1 & 1 & 1 \\
        0 & 0 & 0 \\
        1 & 1 & 3 \\
      \end{pmatrix} \\
      \overset{Z_3 = \frac{1}{2} \cdot \qty\big(Z_3 - Z_1)}&\leadsto
      \begin{pmatrix}
        1 & 1 & 1 \\
        0 & 0 & 0 \\
        0 & 0 & 1 \\
      \end{pmatrix}
      \overset{Z_1 = Z_1 - Z_3}\leadsto
      \begin{pmatrix}
        1 & 1 & 0 \\
        0 & 0 & 0 \\
        0 & 0 & 1 \\
      \end{pmatrix}
    \end{flalign*}
    Somit ist $x_3 = 0$, $x_2 = -x_1$ die Lösung des homogenen LGS und
    \begin{flalign*}
      \text{Eig}\qty\big(A, \lambda_1)
      = \text{Ker}\qty\big(A - 4 \cdot E_3) =
      \qty{\begin{pmatrix}
        x_1  \\
        -x_1 \\
        0    \\
      \end{pmatrix}
      \:\middle|\:
      x_1 \in \mathbb{R}} =
      \text{Span}\qty{\begin{pmatrix} 1 \\ -1 \\ 0 \end{pmatrix}}
    \end{flalign*}

  \newpage
  \item[$\lambda_2 = 5$]
    \begin{flalign*}
      A - 5 \cdot E_3 &= \begin{pmatrix}
        0 & 1 & 1 \\
        1 & 0 & 1 \\
        1 & 1 & 2 \\
      \end{pmatrix}
      \overset{Z_3 = Z_3 - Z_1 - Z_2}\leadsto
      \begin{pmatrix}
        0 & 1 & 1 \\
        1 & 0 & 1 \\
        0 & 0 & 0 \\
      \end{pmatrix} \\
      \overset{Z_1 \leftrightarrow Z_2}&\leadsto
      \begin{pmatrix}
        1 & 0 & 1 \\
        0 & 1 & 1 \\
        0 & 0 & 0 \\
      \end{pmatrix}
    \end{flalign*}
    Somit ist $x_1 = -x_3$, $x_2 = -x_3$ die Lösung des homogenen LGS und
    \begin{flalign*}
      \text{Eig}\qty\big(A, \lambda_2)
      = \text{Ker}\qty\big(A - 5 \cdot E_3) =
      \qty{\begin{pmatrix}
        -x_3 \\
        -x_3 \\
        x_3  \\
      \end{pmatrix}
      \:\middle|\:
      x_3 \in \mathbb{R}} =
      \text{Span}\qty{\begin{pmatrix} -1 \\ -1 \\ 1 \end{pmatrix}}
    \end{flalign*}

    \item[$\lambda_3 = 8$]
    \begin{flalign*}
      A - 8 \cdot E_3 &= \begin{pmatrix}
        -3 & 1  & 1  \\
        1  & -3 & 1  \\
        1  & 1  & -1 \\
      \end{pmatrix}
      \overset{Z_3 = 2 \cdot Z_3 + Z_1 + Z_2}\leadsto
      \begin{pmatrix}
        -3 & 1  & 1 \\
        1  & -3 & 1 \\
        0  & 0  & 0 \\
      \end{pmatrix} \\
      \overset{Z_2 = -\frac{1}{8} \cdot \qty\big(3 \cdot Z_2 + Z_1)}&\leadsto
      \begin{pmatrix}
        -3 & 1 & 1            \\
        0  & 1 & -\frac{1}{2} \\
        0  & 0 & 0            \\
      \end{pmatrix}
      \overset{Z_1 = -\frac{1}{3} \cdot \qty\big(Z_1 - Z_2)}\leadsto
      \begin{pmatrix}
        1 & 0 & -\frac{1}{2}  \\
        0 & 1 & -\frac{1}{2} \\
        0 & 0 & 0            \\
      \end{pmatrix}
    \end{flalign*}
    Somit ist $x_1 = \frac{x_3}{2}$, $x_2 = \frac{x_3}{2}$ die Lösung des homogenen LGS und
    \begin{flalign*}
      \text{Eig}\qty\big(A, \lambda_3)
      = \text{Ker}\qty\big(A - 8 \cdot E_3) =
      \qty{\begin{pmatrix}
        \frac{x_3}{2} \\
        \frac{x_3}{2} \\
        x_3  \\
      \end{pmatrix}
      \:\middle|\:
      x_3 \in \mathbb{R}} =
      \text{Span}\qty{\begin{pmatrix} 1 \\ 1 \\ 2 \end{pmatrix}}
    \end{flalign*}
  \end{itemize}

\item Seien $K$ ein Körper und $V$ ein $K$-Vektorraum der Dimension
  $m \in \mathbb{N} \setminus \qty\big{0}$.
  Eine lineare Abbildung $f \colon V \to V$ heißt \emph{idempotent},
  wenn $\qty\big(f \circ f)\qty\big(v) = f\qty\big(v)$ für alle $v \in V$ gilt.
  \begin{enumerate}[(i)]
  \item Zeigen Sie: Ist $f$ idempotent, dann kann $f$ nur die Eigenwerte $0$ und
    $1$ haben.

    \subparagraph{Lsg.} Sei $f \colon V \to V$ eine beliebige idempotente
    Abbildung, $\lambda \notin \qty\big{0, 1}$ ein Eigenwert von $f$ und
    $v \in \text{Eig}\qty\big(f, \lambda)$.

    Dann ist $f\qty\big(v) = \lambda \cdot v$ und
    \[
      \qty\big(f \circ f)\qty\big(v) = f\qty(f\qty\big(v)) = f\qty\big(\lambda \cdot v)
      \overset{\text{Eigenraum ist Vektorraum}} \lambda^2 \cdot v
    \]
    Da $f$ idempotent gilt $\lambda^2 \cdot v = \lambda \cdot v$ und da der
    Nullraum kein Eigenraum ist, kann $v$ durchaus von 0 verschieden sein und es
    folgt $\lambda^2 - \lambda = 0$.

    Diese Gleichung hat genau zwei Lösungen, $\lambda_1 = 0$ und $\lambda_2 = 1$.

  \newpage
  \item Es sei $f$ eine lineare Abbildung, die die Eigenwerte $0$ und $1$ und
    keine weiteren Eigenwerte hat.
    Ist $f$ idempotent?

    \subparagraph{Lsg.}
    Angenommen es wäre $K = \mathbb{R}$ und $f$ durch die Abbildungsmatrix
    \[
      \begin{pmatrix}
        0 & -1 & 0 & 0 \\
        1 & 0  & 0 & 0 \\
        0 & 0  & 1 & 0 \\
        0 & 0  & 0 & 0 \\
      \end{pmatrix}
    \]
    gegeben.
    Dann hat $f$ das charakteristische Polynom
    $\chi_f\qty\big(\lambda) = \lambda\qty\big(1 - \lambda)\qty\big(\lambda^2 + 1)$
    und nur die zwei Eigenwerte 0 und 1 (da $i, -i \notin \mathbb{R}$).
    Nun ist $f$ nicht idempotent, da
    \[
      f\qty(\qty\big(0, 1, 0, 0)^T) = \qty\big(-1, 0, 0, 0) \ne
      \qty\big(0, -1, 0, 0)^T = \qty\big(f \circ f)\qty(\qty\big(0, 1, 0, 0)^T)
    \]
    Somit muss $f$ nicht zwangsläufig idempotent sein.
  \end{enumerate}
\end{enumerate}
\end{document}
