\documentclass{scrreprt}

\usepackage{aligned-overset}
\usepackage{amsmath}
\usepackage{amsthm}
\usepackage{amssymb}
\usepackage{bm}
\usepackage[shortlabels]{enumitem}
\usepackage{hyperref}
\usepackage[utf8]{inputenc}
\usepackage{multicol}
\usepackage{mathtools}
\usepackage{pdflscape}
\usepackage{pifont}
\usepackage{physics}
\usepackage{polynom}
\usepackage{tabularx}
\usepackage[table]{xcolor}
\usepackage{titling}
\usepackage{fancyhdr}
\usepackage{xfrac}
\usepackage{pgfplots}

\pgfplotsset{compat = newest}
\usetikzlibrary{arrows, arrows.meta}
\usetikzlibrary{calc}

\author{Karsten Lehmann \\ 4935758}
\date{WiSe 2024/25}
\title{Nachbereitungsaufgaben 12\\INF-B-110, Diskrete Strukturen}

\setlength{\headheight}{26pt}
\pagestyle{fancy}
\fancyhf{}
\lhead{\thetitle}
\rhead{\theauthor}
\lfoot{\thedate}
\rfoot{Seite \thepage}

\newcommand{\ggT}[0]{\text{ggT}}
\newcommand{\cmark}{\ding{51}}
\newcommand{\xmark}{\ding{55}}
\DeclarePairedDelimiter{\floor}{\lfloor}{\rfloor}

\begin{document}
\paragraph{N12} Auf der Menge $A \coloneqq \qty\big{1, 2, 3, 4, 5, 6}$ ist die
Relation
\[
  R \coloneqq \qty{
    \qty\big(1, 1),
    \qty\big(1, 4),
    \qty\big(2, 2),
    \qty\big(2, 5),
    \qty\big(3, 3),
    \qty\big(3, 1),
    \qty\big(3, 2),
    \qty\big(3, 4),
    \qty\big(3, 5),
    \qty\big(4, 4),
    \qty\big(4, 1),
    \qty\big(5, 5),
    \qty\big(5, 2),
    \qty\big(6, 6)
  }
\]
gegeben.
\begin{enumerate}[(a)]
\item Zeigen Sie, dass $R$ eine Quasiordnung ist, jedoch keine partielle Ordnung.
  (Zeigen Sie die Transitivität, indem Sie alle Fälle konkret aufschreiben.)

  \subparagraph{Lsg.} Die Relation $R$ ist eine Quasiordnung genau dann, wenn $R$
  \emph{reflexiv} und \emph{transitiv} ist.

  \begin{itemize}
  \item[\emph{reflexiv}:] In der obigen Mengendarstellung der Relation ist
    ersichtlich, dass $\qty\big(a, a) \in R$ für alle $a \in A$.

  \item[\emph{transitiv}:] Die Relation ist transitiv, falls für $a, b, c \in A$
    gilt $\qty\big(a, b) \in A \land \qty\big(b, c) \in A
    \Rightarrow \qty\big(a, c) \in A$.
    Da die Reflexivität von $R$ bereits gezeigt wurde, werden im Folgenden nur
    noch paarweise verschiedene $a, b, c$ betrachtet.

    % >>> A = [1,2,3,4,5,6]
    % >>> r = [(1,1),(1,4),(2,2),(2,5),(3,3),(3,1),(3,2),(3,4),(3,5),(4,4),(4,1),(5,5),(5,2),(6,6)]
    % >>> for a in A:
    % ...     for b in A:
    % ...         for c in A:
    % ...             if a==b or b==c or a==c:
    % ...                 continue
    % ...             print(f"      {a}   & {b}   & {c}   & {'\\cmark' if (a,b) in r else '\\xmark'}                 & {'\\cmark' if (b,c) in r else '\\xmark'}                 & {'\\cmark' if (a,c) in r else '\\xmark'}                 \\\\")
    \begin{tabular}{|c|c|c|c|c|c|}
      \hline
      $a$ & $b$ & $c$ & $\qty\big(a, b) \in A$ & $\qty\big(b, c) \in A$ & $\qty\big(a, c) \in A$ \\
      \hline
      3   & 1   & 4   & \cmark                 & \cmark                 & \cmark                 \\
      3   & 2   & 5   & \cmark                 & \cmark                 & \cmark                 \\
      3   & 4   & 1   & \cmark                 & \cmark                 & \cmark                 \\
      3   & 5   & 2   & \cmark                 & \cmark                 & \cmark                 \\
      \hline
    \end{tabular}

    $\Rightarrow R$ ist \emph{transitiv}.
  \end{itemize}

  Angenommen $R$ wäre nun eine partielle Ordnung, dann wäre $R$ zusätzlich noch
  \emph{antisymmetrisch}, dass heißt
  $\qty\big(a, b) \in R \land \qty\big(b, a) \in R \Rightarrow a = b$.
  Nun sind aber $\qty\big(1, 4) \in R$ und $\qty\big(4, 1) \in R$, allerdings
  ist $1 \ne 4$.
  Somit ist $R$ keine partielle Ordnung.

\item Bestimmen Sie die Elemente der Faktormenge $A/\sim$ für die
  Äquivalenzrelation
  \[
    \sim \coloneqq \qty{
      \qty\big(a, b) \in A^2
      \:\middle|\:
      \qty\big(a, b) \in R, \qty\big(b, a) \in R
    }
  \]

  \subparagraph{Lsg.} Es ist
  \[
    \sim = \qty{
      \qty\big(1, 1),
      \qty\big(1, 4),
      \qty\big(4, 1),
      \qty\big(2, 2),
      \qty\big(2, 5),
      \qty\big(5, 2),
      \qty\big(3, 3),
      \qty\big(4, 4),
      \qty\big(5, 5),
      \qty\big(6, 6)
    }
  \]
  und
  \begin{itemize}
  \item $1 / \sim = \qty\big{1, 4} = 4 / \sim$
  \item $2 / \sim = \qty\big{2, 5} = 5 / \sim$
  \item $3 / \sim = \qty\big{3}$
  \item $6 / \sim = \qty\big{6}$
  \end{itemize}

  Schließlich ist die Faktormenge $A / \sim = \qty\big{1 / \sim, 2 / \sim, 3 / \sim, 6 / \sim}
  = \qty{\qty\big{1, 4}, \qty\big{2, 5}, \qty\big{3}, \qty\big{6}}$.

\newpage
\item Geben Sie die zu $R$ gehörige Faktorordnung $\leq$ auf der Menge
  $A/\sim$ als Menge von Paaren an und zeichnen Sie ein Hassediagramm dieser
  Faktorordnung.

  \subparagraph{Lsg.} Die Relation $\leq$ auf $A / \sim$ ist definiert als
  $\qty\Big(\qty\big(a/\sim), \qty\big(b/\sim)) \in \leq \iff \qty\big(a, b) \in R$.

  Nun ist
  \[
    \leq = \qty{
      \qty\Big(\qty\big{1, 4}, \qty\big{1, 4}),
      \qty\Big(\qty\big{2, 5}, \qty\big{2, 5}),
      \qty\Big(\qty\big{3}, \qty\big{3}),
      \qty\Big(\qty\big{3}, \qty\big{1, 5}),
      \qty\Big(\qty\big{3}, \qty\big{1, 5}),
      \qty\Big(\qty\big{6}, \qty\big{6})
    }
  \]

  Und das Hasse-Diagramm:

  \begin{tikzpicture}
    \node[circle, draw, inner sep=0pt, minimum size=1mm,label=below:{$\qty\big{3}$}] (1) at (0,-1) {};
    \node[circle, draw, inner sep=0pt, minimum size=1mm,label=above left:{$\qty\big{1, 4}$}] (2) at (-0.5,0) {};
    \node[circle, draw, inner sep=0pt, minimum size=1mm,label=above right:{$\qty\big{2, 5}$}] (3) at (0.5,0) {};
    \node[circle, draw, inner sep=0pt, minimum size=1mm,label=below:{$\qty\big{6}$}] (4) at (1.5,-1) {};

    \draw (3) -- (1) -- (2);
  \end{tikzpicture}
\end{enumerate}
\end{document}
