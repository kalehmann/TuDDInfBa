\documentclass{scrreprt}

\usepackage{aligned-overset}
\usepackage{amsmath}
\usepackage{amsthm}
\usepackage{amssymb}
\usepackage{bm}
\usepackage[shortlabels]{enumitem}
\usepackage{hyperref}
\usepackage[utf8]{inputenc}
\usepackage{multicol}
\usepackage{mathtools}
\usepackage{pdflscape}
\usepackage{physics}
\usepackage{polynom}
\usepackage{tabularx}
\usepackage[table]{xcolor}
\usepackage{titling}
\usepackage{fancyhdr}
\usepackage{xfrac}
\usepackage{pgfplots}

\pgfplotsset{compat = newest}
\usetikzlibrary{arrows, arrows.meta}
\usetikzlibrary{calc}

\author{Karsten Lehmann \\ 4935758}
\date{WiSe 2024/25}
\title{Nachbereitungsaufgaben 8\\INF-B-110, Diskrete Strukturen}

\setlength{\headheight}{26pt}
\pagestyle{fancy}
\fancyhf{}
\lhead{\thetitle}
\rhead{\theauthor}
\lfoot{\thedate}
\rfoot{Seite \thepage}

\newcommand{\ggT}[0]{\text{ggT}}
\DeclarePairedDelimiter{\floor}{\lfloor}{\rfloor}

\begin{document}
\paragraph{N8} Betrachtet wird die Gruppe der Einheiten
$\qty\big(\mathbb{Z}_{20}^*; \cdot)$ mit der Multiplikation $\mod 20$.
\begin{enumerate}[(a)]
\item Geben Sie alle Elemente von $\mathbb{Z}_{20}^*$ an und stellen Sie die
  Verknüpfungstafel dieser Gruppe auf.
  \begin{scriptsize}
    (Tipp: Nutzen Sie die Kommutativität der Multiplikation und die Sudokuregel)
  \end{scriptsize}

  \subparagraph{Lsg.} Es ist $20 = 2^2 \cdot 5$ und $\phi\qty\big(20)
  = \qty\big(2 - 1) \cdot 2^{2 - 1} \cdot \qty\big(5 - 1) \cdot 5^{1 - 1} = 8$.
  Folglich hat $\mathbb{Z}_{20}^*$ insgesamt 8 Elemente und diese sind
  \[
    \mathbb{Z}_{20}^* = \qty\big{1, 3, 7, 9, 11, 13, 17, 19}
  \]
  mit der Verknüpfungstafel

  \begin{tabular}{|c|cccccccc|}
    \hline
    $\cdot$ & 1  & 3  & 7  & 9  & 11 & 13 & 17 & 19 \\
    \hline
    1       & 1  & 3  & 7  & 9  & 11 & 13 & 17 & 19 \\
    3       & 3  & 9  & 1  & 7  & 13 & 19 & 11 & 17 \\
    7       & 7  & 1  & 9  & 3  & 17 & 11 & 19 & 13 \\
    9       & 9  & 7  & 3  & 1  & 19 & 17 & 13 & 11 \\
    11      & 11 & 13 & 17 & 19 & 1  & 3  & 7  & 9  \\
    13      & 13 & 19 & 11 & 17 & 3  & 9  & 1  & 7  \\
    17      & 17 & 11 & 19 & 13 & 7  & 1  & 9  & 3  \\
    19      & 19 & 17 & 13 & 11 & 9  & 7  & 3  & 1  \\
    \hline
  \end{tabular}

\item Bestimmen Sie alle Untergruppen der Ordnung 2.
  Wie viele Linksnebenklassen hat jede dieser Untergruppen?

  \subparagraph{Lag} Alle Elemente mit Ordnung kleiner oder gleich 2 in
  $\mathbb{Z}_{20}^*$ lassen sich leicht aus der Verknüpfungstafel ablesen.
  Es sind
  \begin{itemize}
  \item $1^1 = 1$
  \item $9^2 = 1$
  \item $11^2 = 1$
  \item $19^2 = 1$
  \end{itemize}
  Jede Untergruppe muss das neutrale Element 1 enthalten, also gibt es die
  folgenden Untergruppen der Ordnung 2:
  \[
    \qty\big{1, 9}, \quad
    \qty\big{1, 11}, \quad
    \qty\big{1, 19}
  \]
  Nun hat jede dieser Untergruppen die Mächtigkeit 2 und nach dem Satz von
  Lagrange ist $\qty\big[\mathbb{Z}_{20}^* \colon U]
  = \frac{2}{\abs{\mathbb{Z}_{20}^*}}
  = \frac{2}{8} = 4$ für jede Untergruppe $U$ der Ordnung 2.

  $\Rightarrow$ jeder dieser Untergruppen hat 4 Linksnebenklassen.

\newpage
\item Begründen Sie kurz, dass $U \coloneq \qty\big{1, 3, 7, 9}$ eine Untergruppe
  von $\qty\big(\mathbb{Z}_{20}^*)$ ist.
  Geben Sie alle Linksnebenklasse von $U$ an.

  \subparagraph{Lsg.} Es ist $1$ als neutrales Element aus
  $\qty\big(\mathbb{Z}_{20}^*)$ in $U$ enthalten.
  Außerdem lässt sich aus der Verknüpfungstafel von $U$ ablesen, dass die
  Gruppe bezüglich der Multiplikation und $^{-1}$ abgeschlossen ist:

  \begin{tabular}{|c|cccc|}
    \hline
    $\cdot$ & 1 & 3 & 7 & 9 \\
    \hline
    1       & 1 & 3 & 7 & 9 \\
    3       & 3 & 9 & 1 & 7 \\
    7       & 7 & 1 & 9 & 3 \\
    7       & 9 & 7 & 3 & 1 \\
    \hline
  \end{tabular}

  $\Rightarrow U$ ist eine Untergruppe von $\qty\big(\mathbb{Z}_{20}^*)$.

  Nun hat $U$ genau $\qty\big[U \colon \qty\big(\mathbb{Z}_{20}^*)]
  = \frac{\abs{U}}{\abs{\mathbb{Z}_{20}^*}}
  = \frac{4}{8} = 2$ Linksnebenklassen und diese sind
  \begin{flalign*}
    \qty\big{1, 3, 7, 9} &= 1 \cdot U = 3 \cdot U = 7 \cdot U = 9 \cdot U \\
    \qty\big{11, 13, 17, 19} &= 11 \cdot U = 13 \cdot U = 17 \cdot U = 19 \cdot U
  \end{flalign*}

\item Berechnen Sie alle $x \in \mathbb{Z}_{20}$, die die Gleichung
  $9883^{8899} \cdot x \equiv 8 \mod 20$ erfüllen.
  Verwenden Sie dabei, wenn möglich, den Satz von Euler-Fermat.

  \subparagraph{Lsg.} Es ist $9883 \equiv 3 \qty\big(\mod 20)$ und da sich die
  modulo-Operation auch auf alle Zwischenschritte einer Rechnung anwenden lässt
  \[
    9883^{8899} \equiv 3 \qty\big(\mod 20)
  \]
  Nun ist $20 = 2^2 \cdot 5$ und $\phi\qty\big(20)
  = \qty\big(2 - 1) \cdot 2^{2 - 1} \cdot \qty\big(5 - 1) \cdot 5^{1 - 1}
  = 1 \cdot 2 \cdot 4 \cdot 1 = 8$.

  Da $\ggT\qty\big(3, 20) = 1$ ist nach dem Satz von Euler-Fermat
  $3^{\phi\qty\big(20)} = 3^8 \equiv 1 \qty\big(\mod 20)$.
  Schließlich folgt
  \[
    3^{8899} = \qty(3^8)^{1112} \cdot 3^3 \equiv 3^3 \equiv 7 \qty\big(\mod 20)
  \]
  Nun werden noch alle $x \in \mathbb{Z}_{20}$ mit
  $7 \cdot x \equiv 8 \qty\big(\mod 20)$ gesucht.
  Dabei muss für diese $x$ auch
  $7 \cdot \qty\big(x - 1) \equiv 1 \qty\big(\mod 20)$ gelten.
  Nun ist $7 \cdot 3 \equiv 1 \qty\big(\mod 20)$ und damit $x = 4$.

  (Die Aufgabenstellung spricht von ``allen $x \in \mathbb{Z}_{20}$''.
  Weitere Lösungen kann es jedoch nicht geben, da für ein $x' \ne x$, welches
  die Gleichung erfüllt auch $x' - 1$ invers zu 7 und damit in
  $\mathbb{Z}_{20}^*$ enthalten wäre - ein Widerspruch zur Eindeutigkeit des
  Inversen in Gruppen.
  Somit ist $x = 4$ auch wirklich die einzige Lösung.)
\end{enumerate}
\end{document}
