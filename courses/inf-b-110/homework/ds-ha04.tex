\documentclass{scrreprt}

\usepackage{aligned-overset}
\usepackage{amsmath}
\usepackage{amsthm}
\usepackage{amssymb}
\usepackage{bm}
\usepackage[shortlabels]{enumitem}
\usepackage{hyperref}
\usepackage[utf8]{inputenc}
\usepackage{multicol}
\usepackage{mathtools}
\usepackage{pdflscape}
\usepackage{physics}
\usepackage{polynom}
\usepackage{tabularx}
\usepackage[table]{xcolor}
\usepackage{titling}
\usepackage{fancyhdr}
\usepackage{xfrac}
\usepackage{pgfplots}

\pgfplotsset{compat = newest}
\usepgfplotslibrary{fillbetween}
\usetikzlibrary{arrows, arrows.meta}
\usetikzlibrary{patterns}

\author{Karsten Lehmann \\ 4935758}
\date{WiSe 2024/25}
\title{Nachbereitungsaufgaben 4\\INF-B-110, Diskrete Strukturen}

\setlength{\headheight}{26pt}
\pagestyle{fancy}
\fancyhf{}
\lhead{\thetitle}
\rhead{\theauthor}
\lfoot{\thedate}
\rfoot{Seite \thepage}

\begin{document}
\paragraph{N4}
\begin{enumerate}[(a)]
\item Gegeben ist der Ausdruck
  \[
    A \coloneqq \qty\big(\neg x_1 \lor \neg x_2 \lor \neg x_3) \land
    x_2 \land
    \qty\big(x_1 \lor \neg x_2 \lor \neg x_3) \land
    \qty\big(\neg x_2 \lor x_3) \land
    \qty\big(\neg x_1 \lor x_2 \lor \neg x_3)
  \]
  in den drei Variablen $x_1$, $x_2$ und $x_3$.
  Verwenden Sie den Algorithmus aus der Vorlesung, um zu entscheiden, ob $A$
  erfüllbar ist.
  Falls ja, so geben Sie eine erfüllende Belegung an, falls nein, begründen Sie,
  warum $A$ nicht erfüllbar ist.
  Prüfen Sie zuerst, ob die Voraussetzung für die Anwendung des Algorithmus
  gegeben ist.

  \subparagraph{Lsg.} Der Ausdruck $A$ liegt in konjunktiver Normalform vor und
  hat in jeder Klausel maximal ein positives Literal.
  Daher ist $A$ eine Horn-Formel und die Erfüllbarkeit von $A$ lässt sich mittels
  \emph{positiver 1-Resolution} bestimmen.

  Suchen wir nun die erste \emph{positive 1-Klausel} in $A$ und markieren diese
  \[
    A \coloneqq \qty\big(\neg x_1 \lor \neg x_2 \lor \neg x_3) \land
    \qty\big(\colorbox{purple!20}{$x_2$}) \land
    \qty\big(x_1 \lor \neg x_2 \lor \neg x_3) \land
    \qty\big(\neg x_2 \lor x_3) \land
    \qty\big(\neg x_1 \lor x_2 \lor \neg x_3)
  \]
  Im nächsten Schritt streichen wir jedes Auftreten von $\neg x_2$ aus den
  weiteren Klauseln.
  Also wird aus
  \[
    A \coloneqq \qty\big(\neg x_1 \lor \colorbox{purple!20}{$\neg x_2$} \lor \neg x_3) \land
    \qty\big(\colorbox{purple!20}{$x_2$}) \land
    \qty\big(x_1 \lor \colorbox{purple!20}{$\neg x_2$} \lor \neg x_3) \land
    \qty\big(\colorbox{purple!20}{$\neg x_2$} \lor x_3) \land
    \qty\big(\neg x_1 \lor x_2 \lor \neg x_3)
  \]
  nun
  \[
    A \coloneqq \qty\big(\neg x_1 \lor \neg x_3) \land
    \qty\big(\colorbox{purple!20}{$x_2$}) \land
    \qty\big(x_1 \lor \neg x_3) \land
    \qty\big(x_3) \land
    \qty\big(\neg x_1 \lor x_2 \lor \neg x_3)
  \]
  Nun suchen wir die nächste \emph{positive 1-Klausel} und markieren auch diese
  \[
    A \coloneqq \qty\big(\neg x_1 \lor \neg x_3) \land
    \qty\big(\colorbox{purple!20}{$x_2$}) \land
    \qty\big(x_1 \lor \neg x_3) \land
    \qty\big(\colorbox{yellow}{$x_3$}) \land
    \qty\big(\neg x_1 \lor x_2 \lor \neg x_3)
  \]
  Im nächsten Schritt streichen wir jedes Auftreten von $\neg x_3$ aus den
  weiteren Klauseln.
  Also wird aus
  \[
    A \coloneqq \qty\big(\neg x_1 \lor \colorbox{yellow}{$\neg x_3$}) \land
    \qty\big(\colorbox{purple!20}{$x_2$}) \land
    \qty\big(x_1 \lor \colorbox{yellow}{$\neg x_3$}) \land
    \qty\big(\colorbox{yellow}{$x_3$}) \land
    \qty\big(\neg x_1 \lor x_2 \lor \colorbox{yellow}{$\neg x_3$})
  \]
  nun
  \[
    A \coloneqq \qty\big(\neg x_1) \land
    \qty\big(\colorbox{purple!20}{$x_2$}) \land
    \qty\big(x_1) \land
    \qty\big(\colorbox{yellow}{$x_3$}) \land
    \qty\big(\neg x_1 \lor x_2)
  \]
  Nun suchen wir erneut eine \emph{positive 1-Klausel} und markieren auch diese
  \[
    A \coloneqq \qty\big(\neg x_1) \land
    \qty\big(\colorbox{purple!20}{$x_2$}) \land
    \qty\big(\colorbox{teal!30}{$x_1$}) \land
    \qty\big(\colorbox{yellow}{$x_3$}) \land
    \qty\big(\neg x_1 \lor x_2)
  \]
  Im nächsten Schritt streichen wir jedes Auftreten von $\neg x_1$ aus den
  weiteren Klauseln.
  Also wird aus
  \[
    A \coloneqq \qty\big(\colorbox{teal!30}{$\neg x_1$}) \land
    \qty\big(\colorbox{purple!20}{$x_2$}) \land
    \qty\big(\colorbox{teal!30}{$x_1$}) \land
    \qty\big(\colorbox{yellow}{$x_3$}) \land
    \qty\big(\colorbox{teal!30}{$\neg x_1$} \lor x_2)
  \]
  nun
  \[
    A \coloneqq \qty\big() \land
    \qty\big(\colorbox{purple!20}{$x_2$}) \land
    \qty\big(\colorbox{teal!30}{$x_1$}) \land
    \qty\big(\colorbox{yellow}{$x_3$}) \land
    \qty\big(x_2)
  \]
  Jetzt haben wir eine leere Klausel und sehen somit, dass der Ausdruck
  unerfüllbar ist.

\newpage
\item Beweisen Sie mit der Methode der vollständigen Induktion (streng nach dem
  in der Vorlesung angegebenem Vorgehen):
  \[
    \text{Für alle } n \in \mathbb{N}, n \geq 1,
    \text{ gilt: }\;
    \sum_{k = 1}^n \qty\big(3k - 2) = \frac{1}{2}n\qty\big{3n - 1}.
  \]
  \begin{small}
    (Wer möchte, kann sich zusätzlich auch einen Beweis ohne vollständige
    Induktion überlegen.)
  \end{small}

  \subparagraph{Lsg.} Sei die Aussage
  \[
    A\qty\big(n) \colon \sum_{k = 1}^n \qty\big(3k - 2) = \frac{1}{2}n\qty\big(3n - 1)
  \]
  und die Behauptung $A\qty\big(n)$ sei wahr für alle
  $n \in \mathbb{N}, n \geq 1$.

  \textbf{Induktionsanfang:} Für $n = 1$ ist
  \[
    \sum_{k = 1}^1 \qty\big(3k - 2) = \qty\big(3 \cdot 1 - 2) = 1 = \frac{1}{2} \cdot \qty\big(3 - 1)
  \]
  Somit ist $A\qty\big(1)$ wahr.

  \textbf{Induktionsschritt:} Angenommen es wäre $A\qty\big(n)$ für ein
  beliebiges $n \in \mathbb{N}, n \geq 1$ wahr
  (\emph{Induktionsvoraussetzung} (IV)).
  Dann ist zu zeigen, dass $A\qty\big(n) \Rightarrow A\qty\big(n + 1)$.
  Nun ist
  \begin{flalign*}
    A\qty\big(n + 1) \colon \sum_{k = 1}^{n + 1} \qty\big(3k - 2)
    &= \sum_{k = 1}^n \qty\big(3k - 2) + \qty\big(3\qty\big(n + 1) - 2) & \\
    \overset{\text{(IV)}}&= \frac{1}{2}n\qty\big(3n - 1) + \qty\big(3n + 3 - 2) \\
    &= \frac{1}{2}n\qty\big(3n - 1) + \qty\big(3n + 1) \\
    &= \frac{1}{2}n\qty\big(3n - 1) + \frac{1}{2}\qty\big(6n + 2) \\
    &= \frac{1}{2}n\qty\big(3n - 1) + \frac{1}{2}\qty\big(3n - 1) + \frac{1}{2}\qty\big(3n + 3) \\
    &= \frac{1}{2}n\qty\big(3n - 1) + \frac{1}{2}\qty\big(3n - 1) + \frac{3}{2}\qty\big(n + 1) \\
    &= \frac{1}{2}\qty\big(n + 1)\qty\big(3n - 1) + \frac{3}{2}\qty\big(n + 1) \\
    &= \frac{1}{2}\qty\big(n + 1)\qty\big(3n + 3 - 1) \\
    &= \frac{1}{2}\qty\big(n + 1)\qty(3\qty\big(n + 1) - 1)
  \end{flalign*}
  Somit ist $A\qty\big(n) \Rightarrow A\qty\big(n + 1)$ und aus dem Satz über
  die vollständige Induktion folgt die Behauptung.

  \newpage
  \begin{landscape}
    \textbf{Beweis ohne Induktion:} Sei $n \in \mathbb{N}$ beliebig mit
    $2 | n$.
    \begin{flalign*}
      \sum_{k = 1}^n \qty\big(3k - 2) &= \qty(\qty\big(3 \cdot 1 - 2) + \qty\big(3n - 2)) +
                                        \qty(\qty\big(3 \cdot 2 - 2) + \qty\big(3\qty\big(n - 1) - 2))
                                        + \ldots + \qty(\qty(3 \cdot \frac{n}{2} - 2) + \qty(3 \frac{n + 1}{2} - 2)) \\
                                      &= \qty(\qty\big(3 \cdot 1 - 2) + \qty\big(3n - 2)) +
                                        \qty(\qty\big(3 \cdot 2 - 2) + \qty\big(3\qty\big(n - 1) - 2))
                                        + \ldots + \qty(\qty(3 \cdot \frac{n}{2} - 2) + \qty(3 \frac{n}{2} + \frac{6}{2} - 2)) \\
                                      &= \qty\big(3n + 3 - 4) + \qty\big(3n + 3 - 4) + \ldots + \qty\big(3n + 3 - 4) \\
                                      &= \qty\big(3n - 1) + \qty\big(3n - 1) + \ldots + \qty\big(3n - 1) \\
                                      &= \frac{1}{2}n\qty\big(3n - 1)
    \end{flalign*}
    Sei nun erneut $n \in \mathbb{N}$ beliebig, allerdings mit nicht $2 | n$.
    Dann ist
    \begin{flalign*}
      \sum_{k = 1}^n \qty\big(3k - 2) &= \qty(\qty\big(3 \cdot 1 - 2) + \qty\big(3n - 2)) +
                                        \qty(\qty\big(3 \cdot 2 - 2) + \qty\big(3\qty\big(n - 1) - 2))
                                        + \ldots + \qty(\qty(3 \cdot \frac{n - 1}{2} - 2) + \qty(3 \frac{n + 3}{2} - 2))
                                        + \qty(3\frac{n + 1}{2} - 2) \\
                                      &= \qty\big(3 \cdot 1 - 2) + \qty\big(3n - 2) +
                                        \qty\big(3 \cdot 2 - 2) + \qty\big(3\qty\big(n - 1) - 2)
                                        + \ldots + \qty(\qty(3 \cdot \frac{n}{2} - \frac{3}{2} - 2) + \qty(3 \frac{n}{2} + \frac{9}{2} - 2))
                                        + \qty(3\frac{n}{2} + \frac{3}{2} - 2)\\
                                      &= \qty\big(3n + 3 - 4) + \qty\big(3n + 3 - 4) + \ldots + \qty\big(3n + 3 - 4) + \frac{1}{2}\qty\big(3n + 3 - 4) \\
                                      &= \qty\big(3n - 1) + \qty\big(3n - 1) + \ldots + \qty\big(3n - 1) + \frac{1}{2}\qty\big(3n - 1) \\
                                      &= \frac{n - 1}{2}\qty\big(3n - 1) + \frac{1}{2}\qty\big(3n - 1) \\
                                      &= \frac{1}{2}n\qty\big(3n - 1)
    \end{flalign*}
  \end{landscape}
\end{enumerate}
\end{document}
