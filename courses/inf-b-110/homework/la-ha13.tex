\documentclass{scrreprt}

\usepackage{aligned-overset}
\usepackage{amsmath}
\usepackage{amsthm}
\usepackage{amssymb}
\usepackage{bm}
\usepackage[shortlabels]{enumitem}
\usepackage{framed}
\usepackage{hyperref}
\usepackage[utf8]{inputenc}
\usepackage{multicol}
\usepackage{mathtools}
\usepackage{pdflscape}
\usepackage{physics}
\usepackage{polynom}
\usepackage{tabularx}
\usepackage[table]{xcolor}
\usepackage{titling}
\usepackage{fancyhdr}
\usepackage{xfrac}
\usepackage{pgfplots}

\pgfplotsset{compat = newest}
\usepgfplotslibrary{fillbetween}
\usetikzlibrary{calc}
\usetikzlibrary{patterns}
\usetikzlibrary{through}


\author{Karsten Lehmann \\ 4935758}
\date{WiSe 2024/25}
\title{Nachbereitungsaufgaben 13\\INF-B-110, Lineare Algebra}

\setlength{\headheight}{26pt}
\pagestyle{fancy}
\fancyhf{}
\lhead{\thetitle}
\rhead{\theauthor}
\lfoot{\thedate}
\rfoot{Seite \thepage}

\begin{document}
\paragraph{N 13.2} Die Vektoren $v_1 = \begin{pmatrix}
  2  \\
  -2 \\
  1  \\
\end{pmatrix}$ und $v_2 = \begin{pmatrix}
  3 \\
  0 \\
  3 \\
\end{pmatrix}$ spannen einen zweidimensionalen Untervektorraum $U$ des
(mit dem kanonischen Skalarprodukt ausgestatteten) Raumes $\mathbb{R}^3$ auf.
\begin{enumerate}[(a)]
\item Bestimmen Sie eine Orthonormalbasis von $U$.

  \subparagraph{Lsg.} Es ist
  \begin{flalign*}
    \bar{b_1} &= v_1 \\
    b_1 &= \frac{1}{\norm{\bar{b_1}}}\bar{b_1}
         = \frac{1}{\norm{\qty\big(2, -2, 1)^T}} \begin{pmatrix}
           2  \\
           -2 \\
           1  \\
         \end{pmatrix}
         = \frac{1}{\sqrt{9}} \begin{pmatrix}
           2  \\
           -2 \\
           1  \\
         \end{pmatrix}
         = \frac{1}{3} \begin{pmatrix}
           2  \\
           -2 \\
           1  \\
         \end{pmatrix} \\
    \bar{b_2} &= v_2 - \qty\big(v_2 \bullet b_1)b_1
               = \begin{pmatrix}
                 3 \\
                 0 \\
                 3 \\
               \end{pmatrix} - \qty(
                \begin{pmatrix}
                  3 \\
                  0 \\
                  3 \\
                \end{pmatrix} \bullet\frac{1}{3} \begin{pmatrix}
                  2  \\
                  -2 \\
                  1  \\
                \end{pmatrix}
              ) \frac{1}{3} \begin{pmatrix}
                2  \\
                -2 \\
                1  \\
              \end{pmatrix}
              = \begin{pmatrix}
                 3 \\
                 0 \\
                 3 \\
               \end{pmatrix} - \begin{pmatrix}
                2  \\
                -2 \\
                1  \\
               \end{pmatrix}
              = \begin{pmatrix}
                1 \\
                2 \\
                2 \\
              \end{pmatrix} \\
    b_2 &= \frac{1}{\norm{\bar{b_2}}}\bar{b_2}
         = \frac{1}{\norm{\qty\big(1, 2, 2)^T}} \begin{pmatrix}
           1 \\
           2 \\
           2 \\
         \end{pmatrix}
         = \frac{1}{\sqrt{9}} \begin{pmatrix}
           1 \\
           2 \\
           2 \\
         \end{pmatrix}
         = \frac{1}{3} \begin{pmatrix}
           1 \\
           2 \\
           2 \\
         \end{pmatrix}
  \end{flalign*}
  Und die orthonormierte Basis ist $B_U = \qty\big(b_1, b_2)$.

\item Es sei $p_U \colon \mathbb{R}^3 \to \mathbb{R}^3$ die Orthogonalprojektion
  auf $U$.
  Bestimmen Sie die Darstellungsmatrix $A_{B,B}$ der linearen Abbildung
  $\text{proj}_U$ bezüglich der Standardbasis $B$.

  \subparagraph{Lsg.} Betrachtet wird die Orthogonalprojektion der Standardbasis:
  \begin{flalign*}
    \text{proj}_U\qty(\qty\big(1, 0, 0)^T)
    &= \qty(\qty\big(1, 0, 0)^T \bullet b_1)b_1 + \qty(\qty\big(1, 0, 0)^T \bullet b_2)b_2 \\
    &= \frac{2}{3}b_1 + \frac{1}{3}b_2 \\
    &= \frac{2}{9}\begin{pmatrix}
      2  \\
      -2 \\
      1  \\
    \end{pmatrix} + \frac{1}{9}\begin{pmatrix}
      1 \\
      2 \\
      2 \\
    \end{pmatrix} = \frac{1}{9}\begin{pmatrix}
      5  \\
      -2 \\
      4  \\
    \end{pmatrix} \\
    \text{proj}_U\qty(\qty\big(0, 1, 0)^T)
    &= \qty(\qty\big(0, 1, 0)^T \bullet b_1)b_1 + \qty(\qty\big(0, 1, 0)^T \bullet b_2)b_2 \\
    &= -\frac{2}{3}b_1 + \frac{2}{3}b_2 \\
    &= -\frac{2}{9}\begin{pmatrix}
      2  \\
      -2 \\
      1  \\
    \end{pmatrix} + \frac{2}{9}\begin{pmatrix}
      1 \\
      2 \\
      2 \\
    \end{pmatrix} = \frac{1}{9}\begin{pmatrix}
      -2 \\
      8  \\
      2  \\
    \end{pmatrix} \\
  \end{flalign*}
  \begin{flalign*}
    \text{proj}_U\qty(\qty\big(0, 0, 1)^T)
    &= \qty(\qty\big(0, 0, 1)^T \bullet b_1)b_1 + \qty(\qty\big(0, 0, 1)^T \bullet b_2)b_2 \\
    &= \frac{1}{3}b_1 + \frac{2}{3}b_2 \\
    &= \frac{1}{9}\begin{pmatrix}
      2  \\
      -2 \\
      1  \\
    \end{pmatrix} + \frac{2}{9}\begin{pmatrix}
      1 \\
      2 \\
      2 \\
    \end{pmatrix} = \frac{1}{9}\begin{pmatrix}
      4 \\
      2 \\
      5 \\
    \end{pmatrix}
  \end{flalign*}
  Nun sind die Spalten der Abbildungsmatrix $A_{B, B}$ die Abbildungen der
  einzelnen Basisvektoren und
  \[
    A_{B, B} = \frac{1}{9}\begin{pmatrix}
      5  & -2 & 4 \\
      -2 & 8  & 2 \\
      4  & 2  & 5 \\
    \end{pmatrix}
  \]

\item Begründen Sie, dass der Orthogonalraum $U^{\bot}$ die Dimension 1 hat.
  Finden Sie einen Vektor $v_3 \in \mathbb{R}^3$ mit
  $U^{\bot} = \text{Span}\qty\big{v_3}$.

  \subparagraph{Lsg.} Der Raum $U$ wird von den beiden offensichtlich linear
  unabhängigen Vektoren $v_1$ und $v_2$ aufgespannt.
  Somit ist $\text{dim}\qty\big(U) = 2$.
  Nun ist $\text{dim}\qty\big(U) + \text{dim}\qty\big(U^{\bot})
  = \text{dim}\qty\big(\mathbb{R}^3)$ und damit
  $\text{dim}\qty\big(U^{\bot}) = 1$.

  Weiter ist
  \begin{flalign*}
    U^{\bot} = \text{Ker} \begin{pmatrix}
      2 & -2 & 1 \\
      3 & 0  & 3 \\
    \end{pmatrix}
    \overset{Z_1 \leftrightarrow Z_2}&= \text{Ker} \begin{pmatrix}
      3 & 0  & 3 \\
      2 & -2 & 1 \\
    \end{pmatrix} \\
    \overset{Z_1 = \frac{1}{3}Z_1}&= \text{Ker} \begin{pmatrix}
      1 & 0  & 1 \\
      2 & -2 & 1 \\
    \end{pmatrix} \\
    \overset{Z_2 = -\frac{1}{2}\qty\big(Z_2 - 2 \cdot Z_1)}&= \text{Ker} \begin{pmatrix}
      1 & 0 & 1           \\
      0 & 1 & \frac{1}{2} \\
    \end{pmatrix} \\
    &= \qty{
      \begin{pmatrix}
        -\lambda            \\
        -\frac{1}{2}\lambda \\
        \lambda             \\
      \end{pmatrix}
      \:\middle|\:
      \lambda \in \mathbb{R}
    } \\
    &= \text{Span}\qty{
      \begin{pmatrix}
        -2 \\
        -1 \\
        2  \\
      \end{pmatrix}
    }
  \end{flalign*}
  Somit ist $v_3 = \qty\big(2, 1, -2)^T$

\newpage
\item Bestimmen Sie die Darstellungsmatrix $A_{C, C}$ von $\text{proj}_U$
  bezüglich der Basis $C = \qty\big(v_1, v_2, v_3)$.

  \subparagraph{Lsg.} Es ist $M\qty\big(C) = \begin{pmatrix}
    2  & 3 & -2 \\
    -2 & 0 & -1 \\
    1  & 3 & 2  \\
  \end{pmatrix}$
  die Basiswechselmatrix von $C$ zu $B$ und
  \begin{flalign*}
    \qty(\begin{array}{ccc|ccc}
      2  & 3 & -2 & 1 & 0 & 0 \\
      -2 & 0 & -1 & 0 & 1 & 0 \\
      1  & 3 & 2  & 0 & 0 & 1 \\
    \end{array})
    \overset{Z_3 = Z_3 + \frac{1}{2} \cdot Z_2}&\leadsto
    \qty(\begin{array}{ccc|ccc}
      2  & 3 & -2           & 1 & 0           & 0 \\
      -2 & 0 & -1           & 0 & 1           & 0 \\
      0  & 3 & \frac{3}{2}  & 0 & \frac{1}{2} & 1 \\
    \end{array}) \\
    \overset{Z_2 = Z_2 + Z_1}&\leadsto
    \qty(\begin{array}{ccc|ccc}
      2  & 3 & -2           & 1 & 0           & 0 \\
      0  & 3 & -3           & 1 & 1           & 0 \\
      0  & 3 & \frac{3}{2}  & 0 & \frac{1}{2} & 1 \\
    \end{array}) \\
    \overset{Z_3 = Z_3 - Z_2}&\leadsto
    \qty(\begin{array}{ccc|ccc}
      2  & 3 & -2           & 1  & 0            & 0 \\
      0  & 3 & -3           & 1  & 1            & 0 \\
      0  & 0 & \frac{9}{2}  & -1 & -\frac{1}{2} & 1 \\
    \end{array}) \\
    \overset{Z_3 = \frac{2}{9} \cdot Z_3}&\leadsto
    \qty(\begin{array}{ccc|ccc}
      2  & 3 & -2 & 1            & 0            & 0           \\
      0  & 3 & -3 & 1            & 1            & 0           \\
      0  & 0 & 1  & -\frac{2}{9} & -\frac{1}{9} & \frac{2}{9} \\
    \end{array}) \\
    \overset{Z_2 = Z_2 + 3 \cdot Z_3}&\leadsto
    \qty(\begin{array}{ccc|ccc}
      2  & 3 & -2 & 1            & 0            & 0           \\
      0  & 3 & 0  & \frac{3}{9}  & \frac{6}{9}  & \frac{6}{9} \\
      0  & 0 & 1  & -\frac{2}{9} & -\frac{1}{9} & \frac{2}{9} \\
    \end{array}) \\
    \overset{Z_2 = \frac{1}{3} \cdot Z_2}&\leadsto
    \qty(\begin{array}{ccc|ccc}
      2  & 3 & -2 & 1            & 0            & 0           \\
      0  & 1 & 0  & \frac{1}{9}  & \frac{2}{9}  & \frac{2}{9} \\
      0  & 0 & 1  & -\frac{2}{9} & -\frac{1}{9} & \frac{2}{9} \\
    \end{array}) \\
    \overset{Z_1 = Z_1 - 3 \cdot Z_2}&\leadsto
    \qty(\begin{array}{ccc|ccc}
      2  & 0 & -2 & \frac{6}{9}  & -\frac{6}{9} & -\frac{6}{9} \\
      0  & 1 & 0  & \frac{1}{9}  & \frac{2}{9}  & \frac{2}{9}  \\
      0  & 0 & 1  & -\frac{2}{9} & -\frac{1}{9} & \frac{2}{9}  \\
    \end{array}) \\
    \overset{Z_1 = Z_1 + 2 \cdot Z_3}&\leadsto
    \qty(\begin{array}{ccc|ccc}
      2  & 0 & 0 & \frac{2}{9}  & -\frac{8}{9} & -\frac{2}{9} \\
      0  & 1 & 0 & \frac{1}{9}  & \frac{2}{9}  & \frac{2}{9}  \\
      0  & 0 & 1 & -\frac{2}{9} & -\frac{1}{9} & \frac{2}{9}  \\
    \end{array}) \\
    \overset{Z_1 = \frac{1}{2} \cdot Z_1}&\leadsto
    \qty(\begin{array}{ccc|ccc}
      1  & 0 & 0 & \frac{1}{9}  & -\frac{4}{9} & -\frac{1}{9} \\
      0  & 1 & 0 & \frac{1}{9}  & \frac{2}{9}  & \frac{2}{9}  \\
      0  & 0 & 1 & -\frac{2}{9} & -\frac{1}{9} & \frac{2}{9}  \\
    \end{array})
  \end{flalign*}
  $M\qty\big(C)^{-1} = \frac{1}{9}\begin{pmatrix}
    1  & -4 & -1 \\
    1  & 2  & 2  \\
    -2 & -1 & 2  \\
  \end{pmatrix}$ die Basiswechselmatrix von $B$ zu $C$.

  Nun ist
  \begin{flalign*}
    A_{C, C} &= M\qty\big(C)^{-1} \cdot A_{B, B} \cdot M\qty\big(C) \\
    &= \frac{1}{9}\begin{pmatrix}
      1  & -4 & -1 \\
      1  & 2  & 2  \\
      -2 & -1 & 2  \\
    \end{pmatrix} \cdot \frac{1}{9}\begin{pmatrix}
      5  & -2 & 4 \\
      -2 & 8  & 2 \\
      4  & 2  & 5 \\
    \end{pmatrix} \cdot \begin{pmatrix}
      2  & 3 & -2 \\
      -2 & 0 & -1 \\
      1  & 3 & 2  \\
    \end{pmatrix}\\
    &= \frac{1}{9} \cdot \begin{pmatrix}
      1 & -4 & -1 \\
      1 & 2  & 2  \\
      0 & 0  & 0  \\
    \end{pmatrix} \cdot \begin{pmatrix}
      1  & -4 & -1 \\
      1  & 2  & 2  \\
      -2 & -1 & 2  \\
    \end{pmatrix} \\
    &= \begin{pmatrix}
      1 & 0 & 0 \\
      0 & 1 & 0 \\
      0 & 0 & 0 \\
    \end{pmatrix}
  \end{flalign*}
  und mit ein wenig mehr Bedenkzeit darüber was die Matrix eigentlich abbildet,
  hätte man den Lösungsweg deutlich kürzer halten können $\ldots$
\end{enumerate}
\end{document}
