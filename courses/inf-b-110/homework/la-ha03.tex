\documentclass{scrreprt}

\usepackage{aligned-overset}
\usepackage{amsmath}
\usepackage{amsthm}
\usepackage{amssymb}
\usepackage{bm}
\usepackage[shortlabels]{enumitem}
\usepackage{framed}
\usepackage{hyperref}
\usepackage[utf8]{inputenc}
\usepackage{multicol}
\usepackage{mathtools}
\usepackage{physics}
\usepackage{polynom}
\usepackage{tabularx}
\usepackage[table]{xcolor}
\usepackage{titling}
\usepackage{fancyhdr}
\usepackage{xfrac}
\usepackage{pgfplots}

\pgfplotsset{compat = newest}
\usepgfplotslibrary{fillbetween}
\usetikzlibrary{patterns}
\usetikzlibrary{through}


\author{Karsten Lehmann \\ 4935758}
\date{WiSe 2024/25}
\title{Nachbereitungsaufgaben 3\\INF-B-110, Lineare Algebra}

\setlength{\headheight}{26pt}
\pagestyle{fancy}
\fancyhf{}
\lhead{\thetitle}
\rhead{\theauthor}
\lfoot{\thedate}
\rfoot{Seite \thepage}

\begin{document}
\paragraph{N 3.2} Es seien
$\quad m_1 \quad m_2 \quad m_3 \quad m_4 \quad m_5 \quad m_6 \quad m_7 \quad$
die Ziffer Ihrer Matrikelnummer.
Wir betrachten, für die Parameter $a, x, y, z \in \mathbb{R}$ das lineare
Gleichungssystem, welches durch die folgende erweiterte Koeffizientenmatrix
gegeben ist:
\[
  \qty\big(A {\big|} b) =
  \qty(\begin{array}{cccc|c}
    5  & -4 & 2 & 3  & x \\
    -1 & 1  & 1 & -1 & y \\
    3  & -2 & 4 & a  & z \\
  \end{array})
\]

\begin{enumerate}[(a)]
\item Bestimmen Sie die Lösungsmenge des durch $a \coloneqq 1$, $x \coloneqq 0$,
  $y \coloneqq 0$, $z \coloneqq 0$ definierten \emph{homogenen} linearen
  Gleichungssystemes.

\item Bestimmen Sie die Lösungsmenge des durch $a \coloneqq 1$,
  $x \coloneqq m_4$, $y \coloneqq m_5$, $z \coloneqq -1$ definierten
  \emph{inhomogenen} linearen Gleichungssystemes.

\item Bestimmen Sie die Lösungsmenge des durch $a \coloneqq 1$,
  $x \coloneqq m_4$, $y \coloneqq m_5$, $z \coloneqq m_6$ definierten
  \emph{inhomogenen} linearen Gleichungssystemes.
\end{enumerate}
Machen Sie jeweils eine Probe, falls möglich!

\subparagraph{Lsg.} Die einzelnen Ziffern der Matrikelnummer sind

\begin{tabular}{|c|c|c|c|c|c|c|}
  \hline
  4 & 9 & 3 & 5 & 7 & 5 & 8 \\
  \hline
  $m_1$ & $m_2$ & $m_3$ & $m_4$ & $m_5$ & $m_6$ & $m_7$ \\
  \hline
\end{tabular}
\begin{enumerate}[(a)]
\item Es ist
  \begin{flalign*}
    \qty(\begin{array}{cccc|c}
      5  & -4 & 2 & 3  & 0 \\
      -1 & 1  & 1 & -1 & 0 \\
      3  & -2 & 4 & 1  & 0 \\
    \end{array})
    \overset{3 \cdot Z_2 + Z_3}&\leadsto
    \qty(\begin{array}{cccc|c}
      5 & -4 & 2 & 3  & 0 \\
      0 & 1  & 7 & -2 & 0 \\
      3 & -2 & 4 & 1  & 0 \\
    \end{array}) \\
    \overset{5 \cdot Z_3 - 3 \cdot Z_1}&\leadsto
    \qty(\begin{array}{cccc|c}
      5 & -4 & 2  & 3  & 0 \\
      0 & 1  & 7  & -2 & 0 \\
      0 & 2  & 14 & -4 & 0 \\
    \end{array}) \\
    \overset{Z_3 - 2 \cdot Z_1}&\leadsto
    \qty(\begin{array}{cccc|c}
      5 & -4 & 2 & 3  & 0 \\
      0 & 1  & 7 & -2 & 0 \\
      0 & 0  & 0 & 0  & 0 \\
    \end{array}) \\
    \overset{Z_1 + 4 \cdot Z_2}&\leadsto
    \qty(\begin{array}{cccc|c}
      5 & 0 & 30 & -5 & 0 \\
      0 & 1 & 7  & -2 & 0 \\
      0 & 0 & 0  & 0  & 0 \\
    \end{array}) \\
    \overset{\frac{1}{5} \cdot Z_1}&\leadsto
    \qty(\begin{array}{cccc|c}
      1 & 0 & 6 & -1 & 0 \\
      0 & 1 & 7 & -2 & 0 \\
      0 & 0 & 0 & 0  & 0 \\
    \end{array})
  \end{flalign*}
  Somit folgt
  \[
    \mathbb{L} = \qty{
      \begin{pmatrix}
        x_4 - 6x_3 \\
        2x_4 - 7x_3 \\
        x_3 \\
        x_4
      \end{pmatrix}
      \:\middle|\:
      x_3, x_4 \in \mathbb{R}
    }
  \]
  \newpage
  \textbf{Probe:}
  Es sind
  \begin{flalign*}
    5 \cdot \qty\big(x_4 - 6x_3) - 4 \cdot \qty\big(2x_4 - 7x_3) + 2x_3 + 3x_4
    &= 5x_4 - 30x_3 - 8x_4 + 28x_3 + 2x_3 + 3x_4 \\
    &= \qty\big(-30 + 28 + 2) \cdot x_3 + \qty\big(5 - 8 + 3) \cdot x_4 \\
    &= 0 \cdot x_3 + 0 \cdot x_4 \\
    &= 0
  \end{flalign*}
  und
  \begin{flalign*}
    0 \cdot \qty\big(x_4 - 6x_3) + 1 \cdot \qty\big(2x_4 - 7x_3) + 7x_3 - 2x_4
    &= 2x_4 - 7x_3 + 7x_3 - 2x_4 \\
    &= \qty\big(-7 + 7) \cdot x_3 + \qty\big(2 - 2) \cdot x_4 \\
    &= 0 \cdot x_3 + 0 \cdot x_4 \\
    &= 0
  \end{flalign*}
  sowie
  \begin{flalign*}
    3 \cdot \qty\big(x_4 - 6x_3) - 2 \cdot \qty\big(2x_4 - 7x_3) + 4x_3 + x_4
    &= 3x_4 - 18x_3 - 4x_4 + 14x_3 + 4x_3 + x_4 \\
    &= \qty\big(-18 + 14 + 4) \cdot x_3 + \qty\big(3 - 4 + 1) \cdot x_4 \\
    &= 0 \cdot x_3 + 0 \cdot x_4 \\
    &= 0
  \end{flalign*}

\item Es ist
  \begin{flalign*}
    \qty(\begin{array}{cccc|c}
      5  & -4 & 2 & 3  & 5  \\
      -1 & 1  & 1 & -1 & 7  \\
      3  & -2 & 4 & 1  & -1 \\
    \end{array})
    \overset{3 \cdot Z_2 + Z_3}&\leadsto
    \qty(\begin{array}{cccc|c}
      5 & -4 & 2 & 3  & 5  \\
      0 & 1  & 7 & -2 & 20 \\
      3 & -2 & 4 & 1  & -1 \\
    \end{array}) \\
    \overset{5 \cdot Z_3 - 3 \cdot Z_1}&\leadsto
    \qty(\begin{array}{cccc|c}
      5 & -4 & 2  & 3  & 5   \\
      0 & 1  & 7  & -2 & 20  \\
      0 & 2  & 14 & -4 & -20 \\
    \end{array}) \\
    \overset{Z_3 - 2 \cdot Z_2}&\leadsto
    \qty(\begin{array}{cccc|c}
      5 & -4 & 2 & 3  & 5   \\
      0 & 1  & 7 & -2 & 20  \\
      0 & 0  & 0 & 0  & -60 \\
    \end{array})
  \end{flalign*}
  Somit folgt $\mathbb{L} = \emptyset$

\newpage
\item Es ist
  \begin{flalign*}
    \qty(\begin{array}{cccc|c}
      5  & -4 & 2 & 3  & 5 \\
      -1 & 1  & 1 & -1 & 7 \\
      3  & -2 & 4 & 1  & 5 \\
    \end{array})
    \overset{3 \cdot Z_2 + Z_3}&\leadsto
    \qty(\begin{array}{cccc|c}
      5 & -4 & 2 & 3  & 5  \\
      0 & 1  & 7 & -2 & 26 \\
      3 & -2 & 4 & 1  & 5  \\
    \end{array}) \\
    \overset{5 \cdot Z_3 - 3 \cdot Z_1}&\leadsto
    \qty(\begin{array}{cccc|c}
      5 & -4 & 2  & 3  & 5  \\
      0 & 1  & 7  & -2 & 20 \\
      0 & 2  & 14 & -4 & 10 \\
    \end{array}) \\
    \overset{Z_3 - 2 \cdot Z_2}&\leadsto
    \qty(\begin{array}{cccc|c}
      5 & -4 & 2 & 3  & 5   \\
      0 & 1  & 7 & -2 & 20  \\
      0 & 0  & 0 & 0  & -30 \\
    \end{array}) \\
  \end{flalign*}
  Somit folgt ebenfalls $\mathbb{L} = \emptyset$

\end{enumerate}
\end{document}
