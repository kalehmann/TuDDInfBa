\documentclass{scrreprt}

\usepackage{amsmath}
\usepackage{amsthm}
\usepackage{amssymb}
\usepackage{bm}
\usepackage[shortlabels]{enumitem}
\usepackage{framed}
\usepackage{hyperref}
\usepackage[utf8]{inputenc}
\usepackage{multicol}
\usepackage{mathtools}
\usepackage{physics}
\usepackage{polynom}
\usepackage{tabularx}
\usepackage[table]{xcolor}
\usepackage{titling}
\usepackage{fancyhdr}
\usepackage{xfrac}
\usepackage{pgfplots}

\pgfplotsset{compat = newest}
\usepgfplotslibrary{fillbetween}
\usetikzlibrary{patterns}
\usetikzlibrary{through}


\author{Karsten Lehmann \\ 4935758}
\date{WiSe 2024/25}
\title{Nachbereitungsaufgaben 1\\INF-B-110, Lineare Algebra}

\setlength{\headheight}{26pt}
\pagestyle{fancy}
\fancyhf{}
\lhead{\thetitle}
\rhead{\theauthor}
\lfoot{\thedate}
\rfoot{Seite \thepage}

\begin{document}
\paragraph{N 1.2} Es seien $m_1$ $m_2$ $m_3$ $m_4$ $m_5$ $m_6$ $m_7$ die Ziffern
Ihrer Matrikelnummer.

\begin{enumerate}[(a)]
\item Berechnen Sie $x \coloneqq \qty\big(-1)^{m_6} + \qty\big(-1)^{m_7} \cdot i$.
  Bestimmen Sie $z \coloneqq x^{m_5 + 5}$ in Eulerscher Darstellung auf 2 Arten:
  \begin{itemize}
  \item Berechnen Sie zuerst $z$ in arithmetischer Darstellung und rechnen Sie
    dann in Polarkoordinaten um.
  \item Bestimmen Sie zuerst die Eulersche Darstellung von $x$ und berechnen Sie dann
    die $\qty\big(m_5 + 5)$-te Potenz.
  \end{itemize}

  \subparagraph{Lsg.} Die einzelnen Ziffern der Matrikelnummer sind

  \begin{tabular}{|c|c|c|c|c|c|c|}
    \hline
    4 & 9 & 3 & 5 & 7 & 5 & 8 \\
    \hline
    $m_1$ & $m_2$ & $m_3$ & $m_4$ & $m_5$ & $m_6$ & $m_7$ \\
    \hline
  \end{tabular}

  Nun ist $x \coloneqq \qty\big(-1)^5 + \qty\big(-1)^8 \cdot i = -1 + i$ und
  \begin{flalign*}
    \qty\big(-1 + i)^2
    &= \qty\big(\qty(-1) \cdot \qty(-1) - 1 \cdot 1) + \qty\big(\qty(-1) \cdot 1 + 1 \cdot \qty(-1)) \cdot i
      = -2i & \\
    \qty\big(-1 + i)^4
    &= \qty\big(-2i)^2 = \qty\big(0 \cdot 0 - \qty(-2) \cdot \qty(-2)) + \qty\big(0 \cdot \qty(-2) + \qty(-2) \cdot 0) \cdot i
      = -4 \\
    z \coloneqq \qty\big(-1 + i)^{m_5 + 5} &= \qty\big(-1 + i)^{12} = \qty\big(-4)^{3} = -64
  \end{flalign*}

  In Polardarstellung von $z$ ist nun $r = \sqrt{\qty\big(-64)^2 + 0^2} = 64$,
  $\sin\qty\big(\varphi) = \frac{0}{64} = 0$ und
  $\cos\qty\big{\varphi} = \frac{-64}{64} = -1$.

  $\Rightarrow \varphi = \pi, z = 64 \cdot \qty(\cos\qty\big(\pi) + \sin\qty\big(\pi) \cdot i)$

  Weiter ist für die Eulersche Darstellung von $x$ nun
  $r = \sqrt{\qty\big(-1)^2 + 1^2} = \sqrt{2}$,
  $\sin\qty\big(\varphi) = \frac{1}{\sqrt{2}}$ und
  $\cos\qty\big{\varphi} = -\frac{1}{\sqrt{2}}$.

  $\Rightarrow \varphi = \frac{3}{4}\pi, x = \sqrt{2} \cdot e^{i\frac{3}{4}\pi}$

  Schließlich ist $x^{m_5 + 5} = x^{12} = \qty(\sqrt{2} \cdot e^{i\frac{3}{4}\pi})^{12} = 64 \cdot e^{i9\pi} = 64 \cdot e^{i\pi}$

\item Bestimmen Sie $y \coloneqq \qty\big(m_1 + m_4) + \qty\big(m_2 + m_5) \cdot i$.
  Bestimmen Sie $\qty\big(\overline{y})^{-1}$.

  \subparagraph{Lsg.} Es ist $y \coloneqq 9 + 16 \cdot i$ und
  $\overline{y} = 9 - 16 \cdot i$.
  Weiter ist
  \[
    \qty\big(9 - 16 \cdot i)^{-1}
    = \frac{9}{9^2 + 16^2} + \frac{16}{9^2 + 16^2} \cdot i
    = \frac{9}{337} + \frac{16}{337} \cdot i
  \]

\newpage
\item Beweisen Sie, dass für alle komplexen Zahlen $x \ne 0$ die Gleichung
  $\overline{x}^{-1} = \overline{x^{-1}}$ gilt.

  Hinweise: Formulieren Sie den Beweis sorgfältig!
  Es gibt (mindestens) 2 Beweismöglichkeiten.

  \subparagraph{Lsg.} Sei $x = a + b \cdot i$, $a, b \ne 0$.
  In der Vorlesung wurde bereits gezeigt, dass
  \[
    x^{-1} = \frac{a}{a^2 + b^2} + \frac{-b}{a^2 + b^2} \cdot i
  \]
  Somit ist
  \[
    \overline{x^{-1}} = \frac{a}{a^2 + b^2} + \frac{b}{a^2 + b^2} \cdot i
  \]

  Sein nun $c = -b$.
  Dann ist $a + c \cdot i = \overline{x}$ und
  \begin{flalign*}
    \qty\big(a + c)^{-1}
    &= \frac{a}{a^2 + c^2} + \frac{-c}{a^2 + c^2} \cdot i && {\Big |} c = (-b) & \\
    &= \frac{a}{a^2 + \qty\big(-b)^2} + \frac{-\qty\big(-b)}{a^2 + \qty\big(-b)^2} \cdot i && {\Big |} (-b)^2 = b^2 & \\
    &= \frac{a}{a^2 + b^2} + \frac{-\qty\big(-b)}{a^2 + b^2} \cdot i && {\Big |} -(-b) = b\\
    &= \frac{a}{a^2 + b^2} + \frac{b}{a^2 + b^2} \cdot i \\
    &= \overline{x}^{-1} && \text{(siehe oben)}
  \end{flalign*}
\end{enumerate}
\end{document}
