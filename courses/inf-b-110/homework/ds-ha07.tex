\documentclass{scrreprt}

\usepackage{aligned-overset}
\usepackage{amsmath}
\usepackage{amsthm}
\usepackage{amssymb}
\usepackage{bm}
\usepackage[shortlabels]{enumitem}
\usepackage{hyperref}
\usepackage[utf8]{inputenc}
\usepackage{multicol}
\usepackage{mathtools}
\usepackage{pdflscape}
\usepackage{physics}
\usepackage{polynom}
\usepackage{tabularx}
\usepackage[table]{xcolor}
\usepackage{titling}
\usepackage{fancyhdr}
\usepackage{xfrac}
\usepackage{pgfplots}

\pgfplotsset{compat = newest}
\usetikzlibrary{arrows, arrows.meta}
\usetikzlibrary{calc}

\author{Karsten Lehmann \\ 4935758}
\date{WiSe 2024/25}
\title{Nachbereitungsaufgaben 7\\INF-B-110, Diskrete Strukturen}

\setlength{\headheight}{26pt}
\pagestyle{fancy}
\fancyhf{}
\lhead{\thetitle}
\rhead{\theauthor}
\lfoot{\thedate}
\rfoot{Seite \thepage}

\newcommand{\ggT}[0]{\text{ggT}}
\DeclarePairedDelimiter{\floor}{\lfloor}{\rfloor}

\begin{document}
\paragraph{N7}
\begin{enumerate}[(a)]
\item Geben Sie alle Elemente der Gruppe $\qty(\mathbb{Z}_{12}^*; \cdot)$ an und
  stellen Sie die Verknüpfungstafel dieser Gruppe auf.

  Ist $\qty(\mathbb{Z}_{12}^*; \cdot)$ zyklisch?
  Begründen Sie Ihre Antwort!

  \subparagraph{Lsg.} Die Gruppe $\qty(\mathbb{Z}_{12}^*; \cdot)$ enthält alle
  Einheiten aus $\mathbb{Z}_{12}$ und hat $\phi\qty\big(12) = 4$ Elemente.

  \begin{tabular}{|c|cccc|}
    \hline
    $\cdot$ & 1  & 5  & 7  & 11 \\
    \hline
    1       & 1  & 5  & 7  & 11 \\
    5       & 5  & 1  & 11 & 7  \\
    7       & 7  & 11 & 1  & 5  \\
    11      & 11 & 7  & 5  & 1  \\
    \hline
  \end{tabular}

  Nun ist die Gruppe nicht zyklisch, da jedes Element invers zu sich selbst ist.
  So sind für jedes $a \in \qty(\mathbb{Z}_{12}^*; \cdot)$ die Verknüpfungen
  $a^2 = 1$, $a^3 = a$ und $a^n = \begin{cases}
    1 & 2|n \\
    a & 2|(n + 1)
  \end{cases}$.
  Somit werden durch beliebige Verkettung eines Elementes maximal zwei der vier
  Elemente der Gruppe erreicht.

\item Es wird die Gruppe $\qty(\mathbb{Z}_{17}^*; \cdot)$ betrachtet.
  \begin{enumerate}[(1)]
  \item Zeigen Sie, dass 5 eine Primitivwurzel in
    $\qty(\mathbb{Z}_{17}^*; \cdot)$ ist.

    \subparagraph{Lsg.} Es ist

    \begin{tabular}{c|c}
      n & $5^n \mod 17$ \\
      \hline
       1 &  5 \\
       2 &  8 \\
       3 &  6 \\
       4 & 13 \\
       5 & 14 \\
       6 &  2 \\
       7 & 10 \\
       8 & 16 \\
       9 & 12 \\
      10 &  9 \\
      11 & 11 \\
      12 &  4 \\
      13 &  3 \\
      14 & 15 \\
      15 &  7 \\
      16 &  1 \\
    \end{tabular}

    Somit lässt sich jedes Element der Gruppe auch durch mehrfache Verknüpfung
    der 5 darstellen.

  \item Nutzen Sie die Primitivwurzel 5, um alle $x \in \mathbb{Z}_{17}$ zu
    berechnen, die
    \[
      14^{10} \cdot 9^{-1} \cdot x \equiv 13^7 \qty\big(\mod 17)
    \]
    erfüllen.

    \subparagraph{Lsg.} In der vorherigen Teilaufgabe wurde bereits dargelegt,
    dass $14 \equiv 5^5 \qty\big(\mod 17)$, $9 \equiv 5^{10} \qty\big(\mod 17)$
    und $13 \equiv 5^4 \qty\big(\mod 17)$.
    Außerdem ist
    $9^{-1} \equiv \qty(5^{10})^{-1} \equiv 5^6 \equiv 2 \qty\big(\mod 17)$.
    Somit ist
    \begin{flalign*}
      14^{10} \cdot 9^{-1} \cdot x \equiv 13^7 \qty\big(\mod 17)
      &\iff 14^{10} \cdot 2 \cdot x \equiv 13^7 \qty\big(\mod 17) \\
      &\iff \qty\big(5^5)^{10} \cdot 5^6 \cdot x \equiv \qty\big(5^4)^7 \qty\big(\mod 17) \\
      &\iff 5^{50} \cdot 5^6 \cdot x \equiv 5^{28} \qty\big(\mod 17) \\
      &\iff 5^{56} \cdot x \equiv 5^{28} \qty\big(\mod 17) \\
      &\iff 3 \cdot 5^{17 - 1} \cdot 5^{8} \cdot x \equiv 5^{17 - 1} \cdot 5^{12} \qty\big(\mod 17) \\
      \overset{\text{Lemma von Euler-Fermat}}&\iff  5^{8} \cdot x \equiv 5^{12} \qty\big(\mod 17)
    \end{flalign*}
    Somit ist $x = 5^4 \equiv 13 \qty\big(\mod 17)$
  \item Bestimmen Sie die Anzahl der Erzeuger der Gruppe
    $\qty(\mathbb{Z}_{17}^*; \cdot)$.

    \subparagraph{Lsg.} 17 ist eine Primzahl, also ist die Anzahl der Erzeuger
    von $\qty(\mathbb{Z}_{17}^*; \cdot)$ gleich $\phi\qty(\phi\qty\big(17))
    = \phi\qty\big(16) = \phi\qty\big(2^4) = \qty\big(2 - 1) \cdot 2^3 = 8$.
  \end{enumerate}
\end{enumerate}
\end{document}
