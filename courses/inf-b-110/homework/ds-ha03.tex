\documentclass{scrreprt}

\usepackage{aligned-overset}
\usepackage{amsmath}
\usepackage{amsthm}
\usepackage{amssymb}
\usepackage{bm}
\usepackage[shortlabels]{enumitem}
\usepackage{hyperref}
\usepackage[utf8]{inputenc}
\usepackage{multicol}
\usepackage{mathtools}
\usepackage{pdflscape}
\usepackage{physics}
\usepackage{polynom}
\usepackage{tabularx}
\usepackage[table]{xcolor}
\usepackage{titling}
\usepackage{fancyhdr}
\usepackage{xfrac}
\usepackage{pgfplots}

\pgfplotsset{compat = newest}
\usepgfplotslibrary{fillbetween}
\usetikzlibrary{arrows, arrows.meta}
\usetikzlibrary{patterns}

\author{Karsten Lehmann \\ 4935758}
\date{WiSe 2024/25}
\title{Nachbereitungsaufgaben 3\\INF-B-110, Diskrete Strukturen}

\setlength{\headheight}{26pt}
\pagestyle{fancy}
\fancyhf{}
\lhead{\thetitle}
\rhead{\theauthor}
\lfoot{\thedate}
\rfoot{Seite \thepage}

\begin{document}
\begin{landscape}
\paragraph{N3}
\begin{enumerate}[(a)]
\item Zeigen Sie, dass für alle aussagenlogischen Ausdrücke $A$, $B$, $C$ die
  Ausdrücke
  \[
    \qty\big(A \lor C) \land \qty\big(C \Rightarrow \qty\big(A \land \neg B))
    \text{ und }
    A \land \neg \qty\big(B \land C)
  \]
  äquivalent sind.
  Beweisen Sie das auf zwei verschiedenen Wegen:
  \begin{enumerate}[(1)]
  \item Verwenden Sie eine Wertetabelle.
    Geben Sie darin auch Teilausdrücke an, so dass die Ergebnisse gut
    nachvollziehbar sind.

    \subparagraph{Lsg.}\;\\
    \begin{small}
      \begin{tabular}{|c|c|c|c|c|c|c|c|c|c|}
        \hline
        $A$ & $B$ & $C$ & $A \lor C$ & $A \land \neg B$ & $C \Rightarrow \qty\big(A \land \neg B)$ & $\qty\big(A \lor C) \land \qty\big(C \Rightarrow \qty\big(A \land \neg B))$ & $B \land C$ & $\neg \qty\big(B \land C)$ & $A \land \neg \qty\big(B \land C)$ \\
        \hline
        0 & 0 & 0 & 0 & 0 & 1 & 0 & 0 & 1 & 0 \\
        0 & 0 & 1 & 1 & 0 & 0 & 0 & 0 & 1 & 0 \\
        0 & 1 & 0 & 0 & 0 & 1 & 0 & 0 & 1 & 0 \\
        0 & 1 & 1 & 1 & 0 & 0 & 0 & 1 & 0 & 0 \\
        1 & 0 & 0 & 1 & 1 & 1 & 1 & 0 & 1 & 1 \\
        1 & 0 & 1 & 1 & 1 & 1 & 1 & 0 & 1 & 1 \\
        1 & 1 & 0 & 0 & 1 & 1 & 1 & 0 & 1 & 1 \\
        1 & 1 & 1 & 1 & 0 & 0 & 0 & 1 & 0 & 0 \\
        \hline
      \end{tabular}
    \end{small}

  \item Formen Sie den einen Ausdruck anhand der in der Vorlesung angegebenen
    Rechenregeln äquivalent um, bis Sie den anderen Ausdruck erhalten.
    Geben Sie dabei genau an, welche Regel Sie in welchem Schritt anwenden.

    \subparagraph{Lsg.} Es ist
    \begin{small}
    \begin{flalign*}
      \qty\big(A \lor C) \land \qty\big(C \Rightarrow \qty\big(A \land \neg B))
      \overset{\text{Def. Implikation}}&\iff
      \qty\big(A \lor C) \land \qty\big(\neg C \lor \qty\big(A \land \neg B)) & \\
      \overset{\text{Distributivität}}&\iff
      \qty\big(A \lor C) \land \qty\big(\qty\big(\neg C \lor A) \land \qty\big(\neg C \lor \neg B)) \\
      \overset{\text{De Morgan}}&\iff
      \qty\big(A \lor C) \land \qty\big(\qty\big(\neg C \lor A) \land \neg \qty\big(C \land  B)) \\
      \overset{\text{Assoziativität und Kommutativität}}&\iff
      \qty\big(\qty\big(A \lor C) \land \qty\big(A \lor \neg C)) \land \neg \qty\big(C \land  B) \\
      \overset{\text{Distributivität}}&\iff
      \qty\big(A \lor \underset{\text{Immer Falsch!}}{\underbrace{\qty\big(C \land \neg C)}}) \land \neg \qty\big(C \land  B)
      \iff
      A \land \neg \qty\big(C \land  B)
    \end{flalign*}
    \end{small}

  \end{enumerate}
\end{enumerate}
\end{landscape}
\begin{enumerate}[(a)]
\setcounter{enumi}{1}
\item Durch untenstehende Wertetabelle ist eine boolsche Funktion
  $f \colon \qty\big{0, 1}^3 \to \qty\big{0, 1}$ gegeben.

  \begin{tabular}{|c|c|c|c|}
    \hline
    $x_1$ & $x_2$ & $x_3$ & $f\qty\big(x_1, x_2, x_3)$ \\
    \hline
    0 & 0 & 0 & 1 \\
    0 & 0 & 1 & 0 \\
    0 & 1 & 0 & 1 \\
    0 & 1 & 1 & 0 \\
    1 & 0 & 0 & 0 \\
    1 & 0 & 1 & 1 \\
    1 & 1 & 0 & 0 \\
    1 & 1 & 1 & 0 \\
    \hline
  \end{tabular}

  Geben Sie eine Darstellung der Funktion in disjunktiver Normalform (DNF) an.

  \subparagraph{Lsg.} Aus der Wertetabelle lässt sich ablesen, dass
  $f\qty\big(x_1, x_2, x_3) = 1$ für
  \begin{itemize}
  \item $\neg x_1 \land \neg x_2 \land \neg x_3$
  \item $\neg x_1 \land x_2 \land \neg x_3$
  \item $x_1 \land \neg x_2 \land x_3$
  \end{itemize}

  Dementsprechend ist die disjunktive Normalform
  \[
    \qty\big(\neg x_1 \land \neg x_2 \land \neg x_3) \lor
    \qty\big(\neg x_1 \land x_2 \land \neg x_3) \lor
    \qty\big(x_1 \land \neg x_2 \land x_3)
  \]
\end{enumerate}
\end{document}
