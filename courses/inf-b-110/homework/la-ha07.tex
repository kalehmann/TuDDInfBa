\documentclass{scrreprt}

\usepackage{aligned-overset}
\usepackage{amsmath}
\usepackage{amsthm}
\usepackage{amssymb}
\usepackage{bm}
\usepackage[shortlabels]{enumitem}
\usepackage{framed}
\usepackage{hyperref}
\usepackage[utf8]{inputenc}
\usepackage{multicol}
\usepackage{mathtools}
\usepackage{pdflscape}
\usepackage{physics}
\usepackage{polynom}
\usepackage{tabularx}
\usepackage[table]{xcolor}
\usepackage{titling}
\usepackage{fancyhdr}
\usepackage{xfrac}
\usepackage{pgfplots}

\pgfplotsset{compat = newest}
\usepgfplotslibrary{fillbetween}
\usetikzlibrary{patterns}
\usetikzlibrary{through}


\author{Karsten Lehmann \\ 4935758}
\date{WiSe 2024/25}
\title{Nachbereitungsaufgaben 7\\INF-B-110, Lineare Algebra}

\setlength{\headheight}{26pt}
\pagestyle{fancy}
\fancyhf{}
\lhead{\thetitle}
\rhead{\theauthor}
\lfoot{\thedate}
\rfoot{Seite \thepage}

\begin{document}
\paragraph{N 7.2}
Es seien $m_1 \quad m_2 \quad m_3 \quad m_4 \quad m_5 \quad m_6 \quad m_7$ die
Ziffern Ihrer Matrikelnummer.
Für alle $a \in \mathbb{R}$ ist durch $M_a \coloneqq \begin{pmatrix}
  1   & m_3 & m_6 \\
  m_1 & 1   & m_7 \\
  m_2 & m_5 & a   \\
\end{pmatrix} \in \mathbb{R}^{3 \times 3}$ eine Abbildung
$f_a \colon \mathbb{R}^3 \to \mathbb{R}^3, x \mapsto M_a x$ definiert.
\begin{enumerate}[(a)]
\item Begründen Sie, dass $f_a$ für alle $a \in \mathbb{R}$ linear ist.

  \subparagraph{Lsg.} Für jedes $a \in \mathbb{R}$ ist
  $M_a \in \mathbb{R}^{3 \times 3}$ und in der Vorlesung wurde bereits gezeigt,
  dass jede Abbildung $g \colon \mathbb{R}^n \to \mathbb{R}^m, x \mapsto B \cdot x$
  mit $B \in \mathbb{R}^{m \times n}$ linear ist.

\item Bestimmen Sie $f_1\qty(\qty\big(1, 1, 1)^T)$ sowie alle
  $x \in \mathbb{R}^3$ mit $f_3\qty\big(x) = \qty\big(2, 2, 2)^T$.

  \subparagraph{Lsg.} Die einzelnen Ziffern der Matrikelnummer sind

  \begin{tabular}{|c|c|c|c|c|c|c|}
    \hline
    4 & 9 & 3 & 5 & 7 & 5 & 8 \\
    \hline
    $m_1$ & $m_2$ & $m_3$ & $m_4$ & $m_5$ & $m_6$ & $m_7$ \\
    \hline
  \end{tabular}

  Somit ist $M_a \coloneqq \begin{pmatrix}
    1 & 3 & 5 \\
    4 & 1 & 8 \\
    9 & 7 & a \\
  \end{pmatrix}$ und
  \[
    f_1 \qty(\qty\big(1, 1, 1)^T) = \begin{pmatrix}
      1 & 3 & 5 \\
      4 & 1 & 8 \\
      9 & 7 & 1 \\
    \end{pmatrix} \cdot \begin{pmatrix}
      1 \\
      1 \\
      1 \\
    \end{pmatrix} = \begin{pmatrix}
      1 + 3 + 5 \\
      4 + 1 + 8 \\
      9 + 7 + 1 \\
    \end{pmatrix} = \begin{pmatrix}
      9 \\
      13 \\
      17 \\
    \end{pmatrix}
  \]
  Weiter ist
  \begin{flalign*}
    \qty(\begin{array}{ccc|c}
      1 & 3 & 5 & 2 \\
      4 & 1 & 8 & 2 \\
      9 & 7 & 3 & 2 \\
    \end{array})
    \overset{Z_3 = Z_3 - 2 \cdot Z_2 - Z_1}&\leadsto
    \qty(\begin{array}{ccc|c}
      1 & 3 & 5   & 2  \\
      4 & 1 & 8   & 2  \\
      0 & 2 & -18 & -4 \\
    \end{array}) \\
    \overset{Z_2 = Z_2 - 4 \cdot Z_1}&\leadsto
    \qty(\begin{array}{ccc|c}
      1 & 3   & 5   & 2  \\
      0 & -11 & -12 & -6 \\
      0 & 2   & -18 & -4 \\
    \end{array}) \\
    \overset{Z_3 = 11 \cdot Z_3 + 2 \cdot Z_2}&\leadsto
    \qty(\begin{array}{ccc|c}
      1 & 3   & 5    & 2   \\
      0 & -11 & -12  & -6  \\
      0 & 0   & -222 & -56 \\
    \end{array}) \\
    \overset{Z_3 = -\frac{1}{222} \cdot Z_3}&\leadsto
    \qty(\begin{array}{ccc|c}
      1 & 3   & 5   & 2              \\
      0 & -11 & -12 & -6             \\
      0 & 0   & 1   & \frac{28}{111} \\
    \end{array}) \\
    \overset{Z_2 = Z_2 + 12 \cdot Z_3}&\leadsto
    \qty(\begin{array}{ccc|c}
      1 & 3   & 5 & 2               \\
      0 & -11 & 0 & -\frac{110}{37} \\
      0 & 0   & 1 & \frac{28}{111}  \\
    \end{array})
  \end{flalign*}
  \begin{flalign*}
    \qty(\begin{array}{ccc|c}
      1 & 3   & 5 & 2               \\
      0 & -11 & 0 & -\frac{110}{37} \\
      0 & 0   & 1 & \frac{28}{111}  \\
    \end{array})
    \overset{Z_2 = -\frac{1}{11} \cdot Z_2}&\leadsto
    \qty(\begin{array}{ccc|c}
      1 & 3 & 5 & 2              \\
      0 & 1 & 0 & \frac{10}{37}  \\
      0 & 0 & 1 & \frac{28}{111} \\
    \end{array}) \\
    \overset{Z_1 = Z_1 - 3 \cdot Z_2 - 5 \cdot Z_3}&\leadsto
    \qty(\begin{array}{ccc|c}
      1 & 0 & 0 & -\frac{8}{111} \\
      0 & 1 & 0 & \frac{10}{37}  \\
      0 & 0 & 1 & \frac{28}{111} \\
    \end{array})
  \end{flalign*}
  Somit ist $L = \qty{\begin{pmatrix}
    -\frac{8}{111} \\
    \frac{10}{37}  \\
    \frac{28}{111} \\
  \end{pmatrix}}$.

\item \label{n7_c} Bestimmen Sie ein $a \in \mathbb{R}$, so dass $f_a$ nicht
  injektiv ist.
  Bestimmen Sie für dieses $f_a$ eine Basis von $\text{Ker}\qty\big(f_a)$ und
  eine Basis von $\text{Im}\qty\big(f_a)$.

  \subparagraph{Lsg.} Es ist $f_a$ nicht injektiv, wenn für eine Basis
  $\qty\big(b_1, b_2, b_3)$ von $\mathbb{R}^3$ die Vektoren
  $f_a\qty\big(b_1), f_a\qty\big(b_2)$ und $f_a\qty\big(b_3)$ linear abhängig.

  Mit der Standardbasis
  $\qty(\qty\big(1, 0, 0)^T, \qty\big(0, 1, 0)^T, \qty\big(0, 0, 1)^T)$ genügt
  es somit zu zeigen, dass die Einzelnen Spalten von $M_a$ linear unabhängig sind.

  \begin{flalign*}
    \begin{pmatrix}
      1 & 3 & 5 \\
      4 & 1 & 8 \\
      9 & 7 & a \\
    \end{pmatrix}
    \overset{Z_3 = Z_3 - 2 \cdot Z_2 - Z_1}&\leadsto
    \begin{pmatrix}
      1 & 3 & 5      \\
      4 & 1 & 8      \\
      0 & 2 & a - 21 \\
    \end{pmatrix} \\
    \overset{Z_2 = Z_2 - 4 \cdot Z_1}&\leadsto
    \begin{pmatrix}
      1 & 3   & 5      \\
      0 & -11 & -12    \\
      0 & 2   & a - 21 \\
    \end{pmatrix} \\
    \overset{Z_3 = 11 \cdot Z_3 + 2 \cdot Z_2}&\leadsto
    \begin{pmatrix}
      1 & 3   & 5          \\
      0 & -11 & -12        \\
      0 & 0   & 11a - 255 \\
    \end{pmatrix} \\
    \overset{Z_3 = \frac{1}{11} \cdot Z_3}&\leadsto
    \begin{pmatrix}
      1 & 3   & 5                  \\
      0 & -11 & -12                \\
      0 & 0   & a - \frac{255}{11} \\
    \end{pmatrix} \\
    \overset{\text{Sei } a = \frac{255}{11}}&\leadsto
    \begin{pmatrix}
      1 & 3   & 5   \\
      0 & -11 & -12 \\
      0 & 0   & 0   \\
    \end{pmatrix} \\
    \overset{Z_2 = -\frac{1}{11} \cdot Z_2}&\leadsto
    \begin{pmatrix}
      1 & 3 & 5             \\
      0 & 1 & \frac{12}{11} \\
      0 & 0 & 0             \\
    \end{pmatrix} \\
    \overset{Z_1 = Z_1 - 3 \cdot Z_2}&\leadsto
    \begin{pmatrix}
      1 & 0 & \frac{19}{11} \\
      0 & 1 & \frac{12}{11} \\
      0 & 0 & 0             \\
    \end{pmatrix}
  \end{flalign*}

  Nun ist schön zu sehen, dass für $a = \frac{255}{11}$ die ersten beiden Spalten
  linear unabhängig sind und die dritte Spalte von $M_{21}$ von den ersten beiden
  Spalten linear abhängig ist:
  \[
    \frac{19}{11} \cdot \begin{pmatrix}
      1 \\
      4 \\
      9 \\
    \end{pmatrix} + \frac{12}{11} \cdot \begin{pmatrix}
      3 \\
      1 \\
      7 \\
    \end{pmatrix} = \begin{pmatrix}
      5  \\
      8  \\
      \frac{255}{11} \\
    \end{pmatrix}
  \]
  Folglich bilden die ersten beiden Spalten, also
  $\qty{\qty\big(1, 4, 9)^T, \qty\big(3, 1, 7)^T}$ eine Basis von
  $\text{Im}\qty(f_{\frac{255}{11}})$.

  Weiter sind für jeden Vektor $\qty\big(x_1, x_2, x_3)^T$ der Lösungsmenge des
  homogenen Gleichungssystemes $x_1 = -\frac{19}{11}x_3$ sowie
  $x_2 = -\frac{12}{11}x_3$.
  Somit ist
  \[
    \text{Ker}\qty(f_{\frac{255}{11}}) = \qty{
      \begin{pmatrix}
        -\frac{19}{11}x \\
        -\frac{12}{11}x \\
        x
      \end{pmatrix}
      \:\middle|\:
      x \in \mathbb{R}
    } = \text{Span}\qty(\qty{
      \begin{pmatrix}
        -\frac{19}{11} \\
        -\frac{12}{11} \\
        1
      \end{pmatrix}
    })
  \]
  und $\qty{\qty(-\frac{19}{11}, -\frac{12}{11}, 1)^T}$ eine Basis von
  $\text{Ker}\qty(f_{\frac{255}{11}})$.

\item Für welche $a \in \mathbb{R}$ ist $f_a$ surjektiv?
  Begründen Sie mit Hilfe der Dimensionsformel für lineare Abbildungen.

  \subparagraph{Lsg.} Die Abbildung $f_a$ ist surjektiv, wenn Sie auf jedes
  Element der Zielmenge abbildet, oder anders ausgedrückt, wenn
  $\text{Im}\qty\big(f_a) = \mathbb{R}^3$.
  Nun folgt aus der Dimensionsformel für lineare Abbildungen, dass
  $\text{Dim}\qty(\text{Ker}\qty\big(f_a)) = 0$, also muss das homogene
  Gleichungssystem $M_a$ lediglich die triviale Lösung besitzen, dass heißt
  alle Zeilen von $M_a$ linear unabhängig sein.
  In \hyperref[n7_c]{Teilaufgabe (c)} wurde bereits gezeigt, dass dies für alle
  $a \ne \frac{255}{11} \in \mathbb{R}$ der Fall ist.

\item Berechnen Sie $\qty(M_{-5})^{-1}$ (d.h. wählen Sie $a = -5$) mit Hilfe des
  Algorithmus aus der Vorlesung, falls möglich.

  \subparagraph{Lsg.} Es ist
  \begin{flalign*}
    \qty(\begin{array}{ccc|ccc}
      1 & 3 & 5  & 1 & 0 & 0 \\
      4 & 1 & 8  & 0 & 1 & 0 \\
      9 & 7 & -5 & 0 & 0 & 1 \\
    \end{array})
    \overset{Z_3 = Z_3 - 2 \cdot Z_2 - Z_1}&\leadsto
    \qty(\begin{array}{ccc|ccc}
      1 & 3 & 5   & 1  & 0  & 0 \\
      4 & 1 & 8   & 0  & 1  & 0 \\
      0 & 2 & -26 & -1 & -2 & 1 \\
    \end{array}) \\
    \overset{Z_2 = Z_2 - 4 \cdot Z_1}&\leadsto
    \qty(\begin{array}{ccc|ccc}
      1 & 3   & 5   & 1  & 0  & 0 \\
      0 & -11 & -12 & -4 & 1  & 0 \\
      0 & 2   & -26 & -1 & -2 & 1 \\
    \end{array}) \\
    \overset{Z_3 = 11 \cdot Z_3 + 2 \cdot Z_2}&\leadsto
    \qty(\begin{array}{ccc|ccc}
      1 & 3   & 5    & 1   & 0   & 0  \\
      0 & -11 & -12  & -4  & 1   & 0  \\
      0 & 0   & -310 & -19 & -20 & 11 \\
    \end{array})
  \end{flalign*}
  \begin{flalign*}
    \qty(\begin{array}{ccc|ccc}
      1 & 3   & 5    & 1   & 0   & 0  \\
      0 & -11 & -12  & -4  & 1   & 0  \\
      0 & 0   & -310 & -19 & -20 & 11 \\
    \end{array})
    \overset{Z_3 = -\frac{1}{310} \cdot Z_3}&\leadsto
    \qty(\begin{array}{ccc|ccc}
      1 & 3   & 5   & 1              & 0            & 0               \\
      0 & -11 & -12 & -4             & 1            & 0               \\
      0 & 0   & 1   & \frac{19}{310} & \frac{2}{31} & -\frac{11}{310} \\
    \end{array}) \\
    \overset{Z_2 = Z_2 + 12 \cdot Z_3}&\leadsto
    \qty(\begin{array}{ccc|ccc}
      1 & 3   & 5 & 1                & 0             & 0               \\
      0 & -11 & 0 & -\frac{506}{155} & \frac{55}{31} & -\frac{66}{155} \\
      0 & 0   & 1 & \frac{19}{310}   & \frac{2}{31}  & -\frac{11}{310} \\
    \end{array}) \\
    \overset{Z_2 = -\frac{1}{11} \cdot Z_2}&\leadsto
    \qty(\begin{array}{ccc|ccc}
      1 & 3 & 5 & 1              & 0             & 0               \\
      0 & 1 & 0 & \frac{46}{155} & -\frac{5}{31} & \frac{6}{155}   \\
      0 & 0 & 1 & \frac{19}{310} & \frac{2}{31}  & -\frac{11}{310} \\
    \end{array}) \\
    \overset{Z_1 = Z_1 - 3 \cdot Z_2 - 5 \cdot Z_3}&\leadsto
    \qty(\begin{array}{ccc|ccc}
      1 & 0 & 0 & -\frac{61}{310} & \frac{5}{31}  & \frac{19}{310}  \\
      0 & 1 & 0 & \frac{46}{155}  & -\frac{5}{31} & \frac{6}{155}   \\
      0 & 0 & 1 & \frac{19}{310}  & \frac{2}{31}  & -\frac{11}{310} \\
    \end{array})
  \end{flalign*}
  \[
    \qty(M_{-5})^{-1} = \begin{pmatrix}
      -\frac{61}{310} & \frac{5}{31}  & \frac{19}{310}  \\
      \frac{46}{155}  & -\frac{5}{31} & \frac{6}{155}   \\
      \frac{19}{310}  & \frac{2}{31}  & -\frac{11}{310} \\
    \end{pmatrix}
  \]
\item Bestimmen Sie (mit Begründung) ein $a \in \mathbb{R}$, so dass $M_a$ nicht
  invertierbar ist.

  \subparagraph{Lsg.} In Ü 7.5 wurde Bewiesen, dass $M_a$ genau dann invertierbar
  ist, wenn $\text{Ker}\qty\big(M_a) = \qty{0_{\mathbb{R}^3}}$.
  Nun ist $\text{Ker}\qty\big(M_{\frac{255}{11}})$ in
  \hyperref[n7_c]{Teilaufgabe (c)} bereits als $\ne  \qty{0_{\mathbb{R}^3}}$
  bestimmt wurden und somit ist $M_a$ für $a = \frac{255}{11}$ nicht
  invertierbar.
\end{enumerate}
\end{document}
