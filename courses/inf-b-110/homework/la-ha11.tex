\documentclass{scrreprt}

\usepackage{aligned-overset}
\usepackage{amsmath}
\usepackage{amsthm}
\usepackage{amssymb}
\usepackage{bm}
\usepackage[shortlabels]{enumitem}
\usepackage{framed}
\usepackage{hyperref}
\usepackage[utf8]{inputenc}
\usepackage{multicol}
\usepackage{mathtools}
\usepackage{pdflscape}
\usepackage{physics}
\usepackage{polynom}
\usepackage{tabularx}
\usepackage[table]{xcolor}
\usepackage{titling}
\usepackage{fancyhdr}
\usepackage{xfrac}
\usepackage{pgfplots}

\pgfplotsset{compat = newest}
\usepgfplotslibrary{fillbetween}
\usetikzlibrary{calc}
\usetikzlibrary{patterns}
\usetikzlibrary{through}


\author{Karsten Lehmann \\ 4935758}
\date{WiSe 2024/25}
\title{Nachbereitungsaufgaben 11\\INF-B-110, Lineare Algebra}

\setlength{\headheight}{26pt}
\pagestyle{fancy}
\fancyhf{}
\lhead{\thetitle}
\rhead{\theauthor}
\lfoot{\thedate}
\rfoot{Seite \thepage}

\begin{document}
\paragraph{N 11.2}
Wir betrachten, in Abhängigkeit von den Parametern $m, n \in \mathbb{R}$, die
Matrix
\[
  A_{m, n} = \begin{pmatrix}
    2 & m & 0 \\
    0 & 1 & n \\
    0 & n & 1 \\
  \end{pmatrix} \in \mathbb{R}^{3 \times 3}
\]
\begin{enumerate}[(a)]
\item Bestimmen Sie die Eigenwerte von $A_{m, n}$

  \subparagraph{Lsg.} Vorbetrachung: Seien
  \begin{flalign*}
    x + y &= -2 \\
    x \cdot y &= 1 - n^2
  \end{flalign*}
  Dann ist $x = -2 - y$ und
  \begin{flalign*}
    \qty\big(-2 - y) \cdot y &= 1 - n^2 \\
    -2y - y^2 &= 1 - n^2 \\
    -y^2 - 2y - 1 &= -n^2 \\
    y^2 + 2y + 1 &= n^2 \\
    \qty\big(y + 1)^2 &= n^2 \\
    y + 1 &= n \\
    y &= n - 1, x = -n + 1 && (1)
  \end{flalign*}

  Es ist
  \begin{flalign*}
    \det\qty(A_{m, n} - \lambda \cdot E_3)
    &= \det\begin{pmatrix}
      2 - \lambda & m           & 0           \\
      0           & 1 - \lambda & n           \\
      0           & n           & 1 - \lambda \\
    \end{pmatrix} \\
    \overset{\text{Entwicklung nach 1. Spalte}}&=
    \qty\big(2 - \lambda) \cdot \det\begin{pmatrix}
      1 - \lambda & n           \\
      n           & 1 - \lambda \\
    \end{pmatrix} \\
    &= \qty\big(2 - \lambda)\qty(\qty\big(1 - \lambda)^2 - n^2) \\
    &= \qty\big(2 - \lambda)\qty(\lambda^2 - 2\lambda + 1 - n^2) \\
    \overset{(1)}&= \qty\big(2 - \lambda)\qty\big(\lambda + n - 1)\qty\big(\lambda - n + 1)
  \end{flalign*}
  Nun sind $\lambda_1 = 2$, $\lambda_2 = 1 + n$ und $\lambda_3 = 1 - n$.

\newpage
\item Für welche Wertepaare $\qty\big(m, n)$ ist
  $v = \qty\big(0, 1, 1)^T \in \mathbb{R}^3$ ein Eigenvektor von $A_{m, n}$?
  Bestimmen Sie den zugehörigen Eigenwert $\lambda$ und den Eigenraum
  $\text{Eig}\qty\big(A_{m, n}, \lambda)$.

  \begin{small}
    Hinweis: Zur Berechnung des Eigenraums sollten Sie eine Fallunterscheidung
    machen mit den drei Fällen: $n = 0, n = 1$ und $n \notin \qty\big{0, 1}$.
  \end{small}

  \subparagraph{Lsg.} Es ist
  \[
    \begin{pmatrix}
      2 & m & 0 \\
      0 & 1 & n \\
      0 & n & 1 \\
    \end{pmatrix} \cdot \begin{pmatrix}
      0 \\
      1 \\
      1 \\
    \end{pmatrix} = \begin{pmatrix}
      m \\
      1 + n \\
      1 + n \\
    \end{pmatrix}
  \]
  Somit ist $\qty\big(0, 1, 1)^T$ ein Eigenvektor aller Matrizen
  $A_{0, n}, n \in \mathbb{R}$ mit dem Eigenwert $\lambda_2 = 1 + n$.

  Weiter ist
  \begin{flalign*}
    \text{Ker}\qty\big(A_{m, n} - \lambda_2 \cdot E_3)
    &= \text{Ker}\begin{pmatrix}
      1 - n & 0  & 0  \\
      0     & -n & n  \\
      0     & n  & -n \\
    \end{pmatrix} \\
    \overset{Z_3 = Z_3 + Z_2}&= \text{Ker}\begin{pmatrix}
      1 - n & 0  & 0 \\
      0     & -n & n \\
      0     & 0  & 0 \\
    \end{pmatrix}
  \end{flalign*}
  und für
  \begin{itemize}
  \item[$n = 0$:] $\text{Eig}\qty\big(A_{0, 0}, \lambda_2) = \qty{
      \begin{pmatrix}
        0   \\
        x_2 \\
        x_3 \\
      \end{pmatrix}
      \:\middle|\:
      x_2, x_3 \in \mathbb{R}
    } = \text{Span}\qty{
      \begin{pmatrix}
        0 \\
        1 \\
        0 \\
      \end{pmatrix}, \begin{pmatrix}
        0 \\
        0 \\
        1 \\
      \end{pmatrix}
    }$

  \item[$n = 1$:] $\text{Eig}\qty\big(A_{0, 0}, \lambda_2) = \qty{
      \begin{pmatrix}
        x_1 \\
        x_2 \\
        x_2 \\
      \end{pmatrix}
      \:\middle|\:
      x_1, x_2 \in \mathbb{R}
    } = \text{Span}\qty{
      \begin{pmatrix}
        1 \\
        0 \\
        0 \\
      \end{pmatrix}, \begin{pmatrix}
        0 \\
        1 \\
        1 \\
      \end{pmatrix}
    }$
  \item[$n \notin \qty\big{0, 1}$:]
    $\text{Eig}\qty\big(A_{0, 0}, \lambda_2) = \qty{
      \begin{pmatrix}
        0  \\
        x_2 \\
        x_2 \\
      \end{pmatrix}
      \:\middle|\:
      x_2 \in \mathbb{R}
    } = \text{Span}\qty{
      \begin{pmatrix}
        0 \\
        1 \\
        1 \\
      \end{pmatrix}
    }$
  \end{itemize}

\newpage
\item Wählen Sie $m$ gleich der vorletzten und $n$ gleich der letzten Ziffer
  Ihrer Matrikelnummer.
  Untersuchen Sie, ob die Matrix $A_{m, n}$ für diese konkreten Werte von $m$ und
  $n$ diagonalisierbar ist.
  Geben Sie in diesem Fall eine Diagonalmatrix $D$ sowie eine invertierbare Matrix
  $S$ an, für die $D = S^{-1}A_{m, n}S$ gelten.
  Machen Sie eine Probe, indem Sie $S^{-1}A_{m, n}S$ ausrechnen.

  \subparagraph{Lsg.} Die Matrikelnummer ist 4935758.
  Somit sind $m = 5$ und $n = 8$.

  Wie in Teilaufgabe (a) bereits gezeigt, hat $A_{5, 8}$ die Eigenwerte
  $\lambda_1 = 2$, $\lambda_2 = 9$ und $\lambda_3 = -7$.
  Nun sind
  \begin{flalign*}
    \text{Eig}\qty\big(A_{5, 8}, \lambda_1)
    = \text{Ker}\qty\big(A_{5, 8} - \lambda_1 \cdot E_3)
    &= \text{Ker}\begin{pmatrix}
      0 & 5  & 0  \\
      0 & -1 & 8  \\
      0 & 8  & -1 \\
    \end{pmatrix} \\
    \overset{Z_3 = \frac{1}{63}\qty\big(Z_3 + 8 \cdot Z_2)}&= \text{Ker}\begin{pmatrix}
      0 & 5  & 0 \\
      0 & -1 & 8 \\
      0 & 0  & 1 \\
    \end{pmatrix} \\
    \overset{Z_2 = -1 \cdot \qty\big(Z_2 - 8 \cdot Z_3)}&= \text{Ker}\begin{pmatrix}
      0 & 5 & 0 \\
      0 & 1 & 0 \\
      0 & 0 & 1 \\
    \end{pmatrix} \\
    \overset{Z_1 = Z_1 - 5 \cdot Z_2}&= \text{Ker}\begin{pmatrix}
      0 & 5 & 0 \\
      0 & 1 & 0 \\
      0 & 0 & 1 \\
    \end{pmatrix} \\
    &= \qty{
      \begin{pmatrix}
        x_1 \\
        0   \\
        0   \\
      \end{pmatrix}
      \:\middle|\:
      x_1 \in \mathbb{R}
    } \\
    &= \text{Span}\qty{
      \begin{pmatrix}
        1 \\
        0 \\
        0 \\
      \end{pmatrix}
    }
  \end{flalign*}
  \begin{flalign*}
    \text{Eig}\qty\big(A_{5, 8}, \lambda_2)
    = \text{Ker}\qty\big(A_{5, 8} - \lambda_2 \cdot E_3)
    &= \text{Ker}\begin{pmatrix}
      -7 & 5  & 0  \\
      0  & -8 & 8  \\
      0  & 8  & -8 \\
    \end{pmatrix} \\
    \overset{Z_3 = Z_3 + Z_2}&= \text{Ker}\begin{pmatrix}
      -7 & 5  & 0 \\
      0  & -8 & 8 \\
      0  & 0  & 0 \\
    \end{pmatrix} \\
    \overset{Z_2 = -\frac{1}{8} \cdot Z_2}&= \text{Ker}\begin{pmatrix}
      -7 & 5 & 0  \\
      0  & 1 & -1 \\
      0  & 0 & 0  \\
    \end{pmatrix} \\
    \overset{Z_1 = -\frac{1}{7} \cdot \qty\big(Z_1 - 5 \cdot Z_2)}&= \text{Ker}\begin{pmatrix}
      1 & 0 & -\frac{5}{7} \\
      0 & 1 & -1           \\
      0 & 0 & 0            \\
    \end{pmatrix} \\
    &= \qty{
      \begin{pmatrix}
        \frac{5}{7}x_3 \\
        x_3            \\
        x_3            \\
      \end{pmatrix}
      \:\middle|\:
      x_3 \in \mathbb{R}
    } \\
    &= \text{Span}\qty{
      \begin{pmatrix}
        5 \\
        7 \\
        7 \\
      \end{pmatrix}
    }
  \end{flalign*}
  \begin{flalign*}
    \text{Eig}\qty\big(A_{5, 8}, \lambda_3)
    = \text{Ker}\qty\big(A_{5, 8} - \lambda_3 \cdot E_3)
    &= \text{Ker}\begin{pmatrix}
      9 & 5 & 0 \\
      0 & 8 & 8 \\
      0 & 8 & 8 \\
    \end{pmatrix} \\
    \overset{Z_3 = Z_3 - Z_2}&= \text{Ker}\begin{pmatrix}
      9 & 5 & 0 \\
      0 & 8 & 8 \\
      0 & 8 & 8 \\
    \end{pmatrix} \\
    \overset{Z_2 = \frac{1}{8} \cdot Z_2}&= \text{Ker}\begin{pmatrix}
      9 & 5 & 0 \\
      0 & 1 & 1 \\
      0 & 0 & 0 \\
    \end{pmatrix} \\
    \overset{Z_1 = \frac{1}{9} \cdot \qty\big(Z_1 - 5 \cdot Z_2)}&= \text{Ker}\begin{pmatrix}
      1 & 0 & -\frac{5}{9} \\
      0 & 1 & 1            \\
      0 & 0 & 0            \\
    \end{pmatrix} \\
    &= \qty{
      \begin{pmatrix}
        \frac{5}{9}x_3 \\
        -x_3           \\
        x_3            \\
      \end{pmatrix}
      \:\middle|\:
      x_3 \in \mathbb{R}
    } \\
    &= \text{Span}\qty{
      \begin{pmatrix}
        5  \\
        -9 \\
        9  \\
      \end{pmatrix}
    }
  \end{flalign*}

  \newpage
  Nun sind die 3 Vektoren
  \[
    v_1 = \begin{pmatrix}
      1 \\
      0 \\
      0 \\
    \end{pmatrix}, v_2 = \begin{pmatrix}
      5 \\
      7 \\
      7 \\
    \end{pmatrix}, v_3 = \begin{pmatrix}
      5  \\
      -9 \\
      9  \\
    \end{pmatrix}
  \]
  augenscheinlich linear unabhängig und somit eine Eigenvektorbasis von
  $A_{5, 8}$ in $\mathbb{R}^3$.
  (Der Beweis für die lineare Unabhängigkeit wird durch die Diagonalisierung
  gegeben)

  Sei nun
  \[
    D = \begin{pmatrix}
      \lambda_1 & 0         & 0         \\
      0 &       & \lambda_2 & 0         \\
      0 &       & 0         & \lambda_3 \\
    \end{pmatrix} = \begin{pmatrix}
      2 & 0 & 0  \\
      0 & 9 & 0  \\
      0 & 0 & -7 \\
    \end{pmatrix}, \qquad
    S = \begin{pmatrix}
      v_1 & v_2 & v_3
    \end{pmatrix} = \begin{pmatrix}
      1 & 5 & 5  \\
      0 & 7 & -9 \\
      0 & 7 & 9  \\
    \end{pmatrix}
  \]
  Dann ist
  \begin{flalign*}
    \qty(\begin{array}{ccc|ccc}
      1 & 5 & 5  & 1 & 0 & 0 \\
      0 & 7 & -9 & 0 & 1 & 0 \\
      0 & 7 & 9  & 0 & 0 & 1 \\
    \end{array})
    \overset{Z_3 = \frac{1}{18}\qty\big(Z_3 - Z_2)}&\leadsto
    \qty(\begin{array}{ccc|ccc}
      1 & 5 & 5  & 1 & 0             & 0            \\
      0 & 7 & -9 & 0 & 1             & 0            \\
      0 & 0 & 1  & 0 & -\frac{1}{18} & \frac{1}{18} \\
    \end{array}) \\
    \overset{Z_2 = \frac{1}{7}\qty\big(Z_2 + 9 \cdot Z_3)}&\leadsto
    \qty(\begin{array}{ccc|ccc}
      1 & 5 & 5  & 1 & 0             & 0            \\
      0 & 1 & 0  & 0 & \frac{1}{14}  & \frac{1}{14} \\
      0 & 0 & 1  & 0 & -\frac{1}{18} & \frac{1}{18} \\
    \end{array}) \\
    \overset{Z_1 = Z_1 - 5 \cdot Z_2)}&\leadsto
    \qty(\begin{array}{ccc|ccc}
      1 & 0 & 5  & 1 & -\frac{5}{14} & -\frac{5}{14} \\
      0 & 1 & 0  & 0 & \frac{1}{14}  & \frac{1}{14}  \\
      0 & 0 & 1  & 0 & -\frac{1}{18} & \frac{1}{18}  \\
    \end{array}) \\
    \overset{Z_1 = Z_1 - 5 \cdot Z_3)}&\leadsto
    \qty(\begin{array}{ccc|ccc}
      1 & 0 & 0 & 1 & -\frac{20}{252}  & -\frac{160}{252} \\
      0 & 1 & 0 & 0 & \frac{1}{14}     & \frac{1}{14}    \\
      0 & 0 & 1 & 0 & -\frac{1}{18}    & \frac{1}{18}    \\
    \end{array})
  \end{flalign*}
  $\Rightarrow S^{-1} = \frac{1}{252}\begin{pmatrix}
    252 & -20 & -160 \\
    0   & 18  & 18   \\
    0   & -14 & 14   \\
  \end{pmatrix} = \frac{1}{126}\begin{pmatrix}
    126 & -10 & -80 \\
    0   & 9   & 9   \\
    0   & -7  & 7   \\
  \end{pmatrix}$

  Und schlussendlich die Probe:
  \begin{flalign*}
    S^{-1}A_{5, 8}S &=
    \frac{1}{126}\begin{pmatrix}
      126 & -10 & -80 \\
      0   & 9   & 9   \\
      0   & -7  & 7   \\
    \end{pmatrix} \cdot \begin{pmatrix}
      2 & 5 & 0 \\
      0 & 1 & 8 \\
      0 & 8 & 1 \\
    \end{pmatrix} \cdot \begin{pmatrix}
      1 & 5 & 5  \\
      0 & 7 & -9 \\
      0 & 7 & 9  \\
    \end{pmatrix} \\
    &= \frac{1}{126}\begin{pmatrix}
      252 & -20 & -160 \\
      0   & 81  & 81   \\
      0   & 49  & -49  \\
    \end{pmatrix} \cdot \begin{pmatrix}
      1 & 5 & 5  \\
      0 & 7 & -9 \\
      0 & 7 & 9  \\
    \end{pmatrix} \\
    &= \frac{1}{126}\begin{pmatrix}
      252 & 0    & 0    \\
      0   & 1134 & 0    \\
      0   & 0    & -882 \\
    \end{pmatrix} \\
    &= \begin{pmatrix}
      2 & 0 & 0  \\
      0 & 9 & 0  \\
      0 & 0 & -7 \\
    \end{pmatrix} = D
  \end{flalign*}
\end{enumerate}
\end{document}
