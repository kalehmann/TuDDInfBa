\documentclass{scrreprt}

\usepackage{aligned-overset}
\usepackage{amsmath}
\usepackage{amsthm}
\usepackage{amssymb}
\usepackage{bm}
\usepackage[shortlabels]{enumitem}
\usepackage{framed}
\usepackage{hyperref}
\usepackage[utf8]{inputenc}
\usepackage{multicol}
\usepackage{mathtools}
\usepackage{pdflscape}
\usepackage{physics}
\usepackage{polynom}
\usepackage{tabularx}
\usepackage[table]{xcolor}
\usepackage{titling}
\usepackage{fancyhdr}
\usepackage{xfrac}
\usepackage{pgfplots}

\pgfplotsset{compat = newest}
\usepgfplotslibrary{fillbetween}
\usetikzlibrary{calc}
\usetikzlibrary{patterns}
\usetikzlibrary{through}


\author{Karsten Lehmann \\ 4935758}
\date{WiSe 2024/25}
\title{Nachbereitungsaufgaben 9\\INF-B-110, Lineare Algebra}

\setlength{\headheight}{26pt}
\pagestyle{fancy}
\fancyhf{}
\lhead{\thetitle}
\rhead{\theauthor}
\lfoot{\thedate}
\rfoot{Seite \thepage}

\begin{document}
\paragraph{N 9.2}
\begin{enumerate}[(a)]
\item Es seien $m_1 \quad m_2 \quad m_3 \quad m_4 \quad m_5 \quad m_6 \quad m_7$
  die Ziffern Ihrer Matrikelnummer.
  Bestimmen Sie die Determinante $\det\qty\big(AB)$ des Produkts der
  $\mathbb{R}^{4 \times 4}$-Matrizen
  \[
    A = \begin{pmatrix}
      1 & 0 & 0       & 1       \\
      0 & 2 & 0       & 1       \\
      0 & 0 & m_3 + 1 & m_4 + 1 \\
      0 & 0 & m_5 + 1 & m_6 + 1 \\
    \end{pmatrix}
    \text{ und }
    B = \begin{pmatrix}
      1 & -1 & 3 & 4       \\
      0 & 1  & 2 & 1       \\
      0 & 0  & 1 & 5       \\
      0 & 0  & 0 & m_7 + 1 \\
    \end{pmatrix}
  \]

  \subparagraph{Lsg.} Die einzelnen Ziffern der Matrikelnummer sind

  \begin{tabular}{|c|c|c|c|c|c|c|}
    \hline
    4 & 9 & 3 & 5 & 7 & 5 & 8 \\
    \hline
    $m_1$ & $m_2$ & $m_3$ & $m_4$ & $m_5$ & $m_6$ & $m_7$ \\
    \hline
  \end{tabular}

  und somit
  \[
    A = \begin{pmatrix}
      1 & 0 & 0 & 1 \\
      0 & 2 & 0 & 1 \\
      0 & 0 & 4 & 6 \\
      0 & 0 & 8 & 6 \\
    \end{pmatrix}
    \text{ und }
    B = \begin{pmatrix}
      1 & -1 & 3 & 4 \\
      0 & 1  & 2 & 1 \\
      0 & 0  & 1 & 5 \\
      0 & 0  & 0 & 9 \\
    \end{pmatrix}
  \]
  Weiter gilt nach dem Multiplikationssatz
  $\det\qty\big(AB) = \det\qty\big(A) \cdot \det\qty\big(B)$
  und
  \begin{flalign*}
    \det\qty\big(A)
    &= \det\qty(\begin{pmatrix}
      1 & 0 & 0 & 1 \\
      0 & 2 & 0 & 1 \\
      0 & 0 & 4 & 6 \\
      0 & 0 & 8 & 6 \\
    \end{pmatrix}) \\
    \overset{\text{Entwicklung nach der ersten Spalte}}&= \det\qty(\begin{pmatrix}
      2 & 0 & 1 \\
      0 & 4 & 6 \\
      0 & 8 & 6 \\
    \end{pmatrix}) \\
    \overset{\text{Entwicklung nach der ersten Spalte}}&= 2 \cdot \det\qty(\begin{pmatrix}
      4 & 6 \\
      8 & 6 \\
    \end{pmatrix}) \\
    &= 2 \cdot \qty\big(4 \cdot 6 - 6 \cdot 8) \\
    &= -48
  \end{flalign*}
  sowie
  \begin{flalign*}
    \det\qty\big(B) &= \det\qty(\begin{pmatrix}
      1 & -1 & 3 & 4 \\
      0 & 1  & 2 & 1 \\
      0 & 0  & 1 & 5 \\
      0 & 0  & 0 & 9 \\
    \end{pmatrix}) \\
    \overset{\text{Ü 9.6 (a)}}&= 1 \cdot 1 \cdot 1 \cdot 9 \\
    &= 9
  \end{flalign*}
  Schließlich ist $\det\qty\big(AB) = -48 \cdot 9 = -432$

\newpage
\item Es seien $n \in \mathbb{N} \setminus \qty\big{0}, \lambda \in \mathbb{R}$
  und $A\qty\big(n, \lambda) = \qty\big(A_{ij}) \in \mathbb{R}^{n \times n}$ mit
  $a_{ij} \coloneqq \begin{cases}
    \lambda, & i = j \\
    1,       & \abs{i - j} = 1 \\
    0,       & \text{sonst.}
  \end{cases}$
  \begin{enumerate}[(i)]
  \item Bestimmen Sie $\det\qty(A\qty\big(2, \lambda))$ und
    $\det\qty(A\qty\big(3, \lambda))$.

    \subparagraph{Lsg.} Es ist
    \[
      A\qty\big(2, \lambda) = \begin{pmatrix}
        \lambda & 1       \\
        1       & \lambda \\
      \end{pmatrix}
    \]
    mit $\det\qty(A\qty\big(2, \lambda)) = \lambda^2 - 1$ sowie
    \[
      A\qty\big(3, \lambda) = \begin{pmatrix}
        \lambda & 1       & 0       \\
        1       & \lambda & 1       \\
        0       & 1       & \lambda \\
      \end{pmatrix}
    \]
    mit
    \begin{flalign*}
      \det\qty(A\qty\big(3, \lambda))
      &= \begin{pmatrix}
        \lambda & 1       & 0       \\
        1       & \lambda & 1       \\
        0       & 1       & \lambda \\
      \end{pmatrix} \\
      \overset{\text{Entwicklung nach 1. Zeile}}&=
      \lambda \cdot \det\qty(A\qty\big(2, \lambda))
      -1 \cdot \det\qty(\begin{pmatrix}
        1 & 0       \\
        1 & \lambda \\
      \end{pmatrix}) \\
      & \lambda \cdot \qty(\lambda^2 - 1) - \lambda \\
      &= \lambda^3 - 2\lambda
    \end{flalign*}

  \item Bestimmen Sie $\det\qty(A\qty\big(2k + 1, 0))$ für jedes
    $k \in \mathbb{N}$.

    \subparagraph{Lsg.} Der Fall $\det\qty(A\qty\big(2 \cdot 0 + 1, 0)) = 0$
    ist trivial und $\det\qty(A\qty\big(2 \cdot  + 1, 0))= 0^3 - 2 \cdot 0 = 0$
    wurde bereits in Teilaufgabe (i) gezeigt.
    Sei nun $k \in \mathbb{N}$ mit $k > 1$ beliebig und $S_i$ die $i$-te Zeile
    von $A\qty\big(2k + 1, 0)$.
    Dann ist der $j$-te Eintrag $\qty\big(S_i)_j = \begin{cases}
      1,       & \abs{i - j} = 1 \\
      0,       & \text{sonst.}
    \end{cases}$

    Für $1 < i < 2k + 1$ hat $S_i$ somit zweimal den Eintrag 1, einmal an dem
    Index $i - 1$ und einmal an dem Index $i + 1$.
    Die restlichen Einträge sind jeweils Null.

    Die Zeile $S_1$ hat dagegen nur eine 1 an dem Index $2$ und die Zeile
    $S_{2k + 1}$ ebenfalls nur eine 1 an dem Index $2k$.

    \newpage
    Nehme nun die 3. Zeile.
    Diese hat eine 1 am Index 2 und eine 1 am Index 4.
    Alle anderen Einträge sind Null.
    Ziehen wir von der 3. Zeile die erste Zeile ab verbleibt eine 1 am Index 4.
    Durch das Abziehen einer Zeile von einer anderen ändert sich dabei die
    Determinante der Matrix nicht.

    Nun hat die 5. Zeile eine 1 am Index 4 und eine 1 am Index 6.
    Die anderen Einträge sind Nullen.
    Zieht man die 5. Zeile von der 3. Zeile ab, verbleibt in der 3. Zeile eine
    -1 am Index 6 und alle anderen Einträge sind Nullen.

    Weiter hat die 7. Zeile eine 1 am Index 6 und eine 1 am Index 8.
    Die anderen Einträge sind Nullen.
    Addiert man die 7. Zeile nun auf die 3. Zeile, dann verbleibt in der 3. Zeile
    eine 1 am Index 8 und alle anderen Einträge sind Nullen.

    Durch beliebige Fortführung des beschriebenen Vorgangs lässt sich für jeden
    geraden Index die 3. Zeile so umformen, dass an diesem Index eine 1 oder -1
    steht und alle anderen Einträge Nullen sind.

    Dies führt man durch bis in der 3. Zeile lediglich am Index $2k$ eine 1 oder
    -1 steht und addiert schließlich die $\qty\big(2k + 1)$-Zeile auf oder zieht
    sie ab, um eine Nullzeile zu erhalten.

    Da die Nullzeile durch äquivalente Zeilenumformungen entstanden ist, hat
    $A\qty\big(2k + 1, 0)$ keinen vollen Rang und
    $\det\qty(A\qty\big(2k + 1, 0)) = 0$.
  \end{enumerate}
\end{enumerate}
\end{document}
