\documentclass{scrreprt}

\usepackage{aligned-overset}
\usepackage{amsmath}
\usepackage{amsthm}
\usepackage{amssymb}
\usepackage{bm}
\usepackage[inline,shortlabels]{enumitem}
\usepackage{hyperref}
\usepackage[utf8]{inputenc}
\usepackage{multicol}
\usepackage{mathtools}
\usepackage{pdflscape}
\usepackage{physics}
\usepackage{tabularx}
\usepackage[table]{xcolor}
\usepackage{titling}
\usepackage{fancyhdr}
\usepackage{xfrac}
\usepackage{pgfplots}

\pgfplotsset{compat = newest}
\usepgfplotslibrary{fillbetween}
\usetikzlibrary{arrows.meta}
\usetikzlibrary{calc}


\author{Karsten Lehmann}
\date{SoSe 2025}
\title{Übungsblatt 02\\INF-B-370, Rechnernetze}

\setlength{\parindent}{0pt}

\setlength{\headheight}{26pt}
\pagestyle{fancy}
\fancyhf{}
\lhead{\thetitle}
\rhead{\theauthor}
\lfoot{\thedate}
\rfoot{Seite \thepage}

\begin{document}
\paragraph{Ü 2.1 Circuit Switching vs Packet Switching}
\begin{enumerate}[(a)]
\item Diskutieren Sie kurz die Vor- und Nachteile der beiden Implementierungen
  für ein Netzwerk.

  \subparagraph{Lsg.} \phantom{\null}

  \begin{minipage}{0.45\textwidth}
    \textbf{Circuit Switching}

    \begin{itemize}
    \item[$\oplus$] Kein Overhead
    \item[$\oplus$] Geringe Latenz
    \item[$\oplus$] Garantierte Leistung
    \item[$\ominus$] Komplexer Aufbau
    \item[$\ominus$] Nicht adaptiv bei Fehler
    \item[$\ominus$] Ineffizient für kurze Verbindungen
    \end{itemize}
  \end{minipage}
  \begin{minipage}{0.45\textwidth}
    \textbf{Packet Switching}

    \begin{itemize}
    \item[$\oplus$] Flexibel
    \item[$\oplus$] \ldots
    \item[$\ominus$] Eventuell höhere Latenz
    \item[$\ominus$] Overhead für jedes Paket
    \item[$\ominus$] Puffer benötigt
    \item[$\ominus$] Geringere Bandbreite
    \end{itemize}
  \end{minipage}

\item Sie möchten zuverlässig Echtzeitdaten liefern.
  Für welchen Ansatz entscheiden Sie sich?
  Warum?

  \subparagraph{Lsg.} Es sollte Circuit Switching verwendet werden für eine
  geringe Latenz und garantierte Bandbreite.
\end{enumerate}

\paragraph{Ü 2.2 Dienste, Protokolle und Schichtenmodell}
\begin{enumerate}[(a)]
\item Grenzen Sie die Begriffe \emph{Dienst} und \emph{Protokoll} gegeneinander ab.

  \subparagraph{Lsg.} \textbf{Dienst:} Kommunikation zwischen zwei verschiedenen
  Schichten im OSI Modell.

  \textbf{Protokoll:} Kommunikation zwischen zwei Systemen auf einer Schicht
  des OSI-Modells.
\setcounter{enumi}{2}
\item Finden Sie die am besten passende Schicht für die folgenden
  Aufgaben/Geräte:

  \subparagraph{Lsg. } \phantom{\null}

  \begin{tabular}{l|l}
    \textbf{Aufgabe/Geräte} & \textbf{Schicht} \\
    \hline
    Verschlüsselung von Nachrichten & Application Layer / Transportlayer (TLS) \\
    Switch in einem Netzwerk & Link Layer \\
    Wegewahl & Network Layer \\
    Ende-zu-Ende Kommunikation & Application Layer \\
    Hinzufügen einer Sequenznummer zu einem Paket & Transport Layer \\
    Router in einem Netzwerk & Network Layer \\
    Deep-Packet-Inpsection für Mailware-Erkennung & Application Layer \\
  \end{tabular}
\end{enumerate}

\newpage
\paragraph{Ü 2.3 Protokolle und Standards}

Protokolle und Standards sind wichtig in der Netzwerkkommunikation.
Warum?


\subparagraph{Lsg.} Standards ermöglichen Interoperabilität und Protokolle
definieren Syntax und Semantik.
Damit Protokolle universell nutzbar sind, müssen sie standardisiert werden.

\end{document}
